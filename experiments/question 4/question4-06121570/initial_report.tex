\documentclass{article}
\usepackage[utf8]{inputenc}
\usepackage{amsmath}
\usepackage{algorithm}
\usepackage{algpseudocode}
\usepackage{graphicx}
\usepackage{hyperref}
\usepackage{natbib}
\usepackage{geometry}
\usepackage{booktabs}
\graphicspath{./}
\usepackage{tikz}
\usepackage{lipsum} % For dummy text
\usepackage{eso-pic} % For placing content on every page
\newcommand\BackgroundConfidential{%
    \put(0,0){%
        \parbox[b][\paperheight]{\paperwidth}{%
            \vfill
            \centering
            \tikz[remember picture,overlay] \node[scale=5,opacity=0.2,rotate=45,align=center] {Warning:\\Generated By AI\\ \textbf{EpidemIQs}};
            \vfill
        }%
    }%
}
\title{Competitive Bi-Virus SIS Dynamics on Multiplex Networks: Structural Niches Driving Coexistence versus Dominance}
\author{EpidemIQs, Primary Agent Backone LLM: gpt-4.1,  LaTeX Agent LLM : gpt-4.1-mini}
\date{\today}
\begin{document}
\AddToShipoutPictureBG{\BackgroundConfidential}
\maketitle

\begin{abstract}
This study rigorously investigates the competitive dynamics of two mutually exclusive viruses spreading over a multiplex network comprising two layers, with each virus restricted to propagating solely within its designated layer. Employing an extension of the classic SIS model to incorporate competitive exclusion, nodes can only be infected by one virus at a time. Both viruses are parameterized to possess effective infection rates above their respective epidemic thresholds, thereby ensuring potential endemic persistence in isolation.

We analytically demonstrate that absolute dominance, whereby one virus completely suppresses the other, is the generic outcome when the network layers are structurally aligned and highly correlated in node degree, community affiliation, and centrality. Specifically, high inter-layer degree correlation and near-complete overlap of community structures underpin competitive exclusion, favoring the virus with a marginal advantage in transmissibility or structural embedding. Conversely, coexistence emerges only under network architectures that induce structural niches: low or negative inter-layer degree correlations (such that hubs differ across layers), partial or disjoint community overlaps, and misaligned centrality distributions enable each virus to persist by exploiting distinct subpopulations within the network.

These analytical findings were substantiated through extensive agent-based simulations on synthetic multiplex networks including Erd\H{o}s-R\'enyi, Barab\'asi-Albert scale-free, and stochastic block models with tunable inter-layer correlations and community overlaps. Simulation results consistently reveal dominance outcomes in maximally correlated or highly overlapping networks, while stable coexistence occurs only when network configurations promote stratification of contact patterns and partial isolation of viral spreading pathways.

This work elucidates the critical role of multiplex network structure in governing multi-strain competitive infections, highlighting how subtle differences in inter-layer correlations and community modularity can decisively shift the balance between viral dominance and coexistence. Our findings provide a fundamental mechanistic framework for understanding and predicting outcomes in complex biological and social contagion processes on layered contact networks.
\end{abstract}

\section{Introduction}

The study of epidemic spreading in complex networked populations has become a crucial area of research, particularly with the increasing realization that infectious diseases often involve multiple competing strains or viruses simultaneously spreading through the same host population. The Susceptible-Infected-Susceptible (SIS) model is a fundamental compartmental framework that allows investigation of recurrent infections and the dynamics of pathogen persistence in populations. Extending the classical SIS model to incorporate multiple competing strains introduces intricate dynamical phenomena driven by competitive exclusion, coexistence, and the influence of underlying network structures.

Classical epidemiological theory posits the competitive exclusion principle as the cornerstone of multi-strain epidemic dynamics: when multiple pathogens compete for susceptible hosts, typically only one strain persists while others are eradicated, especially in homogeneous mixing populations \cite{Wang2022CompetitiveExclusion}. However, recent progress in network epidemiology has demonstrated that structural heterogeneity and complexity of interaction patterns can significantly alter these outcomes. In particular, multilayer or multiplex networks—where individuals are connected by different types of relationships or interactions, such as separate physical contact and information dissemination layers—serve as realistic models for capturing the diverse pathways through which infections and competing processes occur \cite{Gracy2023DiscreteTime}, \cite{Liu2016CommunitySize}, \cite{Guenduc2021EffectsDiffusion}. 

Investigation of competitive SIS dynamics on multiplex networks raises pivotal questions regarding whether multiple competing viruses can coexist stably, or whether one virus dominates and drives the other to extinction. The outcome depends critically on two intertwined aspects: the epidemiological parameters such as infection and recovery rates, and the structural properties of the multilayer network, including inter-layer degree correlations, community overlap, and centrality misalignment \cite{Wu2016Coexistence}, \cite{Gracy2023CompetitiveBivirusHypergraphs}.

Recent analytical and computational studies have revealed that when each virus propagates exclusively on a distinct network layer, coexistence or dominance scenarios are determined largely by how the respective transmission channels overlap or differ. High inter-layer degree correlation and significant community overlap commonly lead to a ``winner-takes-all'' behavior, where one virus outcompetes the other, consistent with competitive exclusion \cite{Wang2022CompetitiveExclusion}. Conversely, negative or low inter-layer degree correlations, partial or absent community overlap, and structural misalignments enable each virus to occupy distinct network niches, facilitating stable coexistence \cite{Gracy2023TriVirus}, \cite{Wu2016Coexistence}.

Moreover, the notion of coexistence becomes even more pronounced when considering complex network architectures that include stochastic block models with modular community structure, hypergraphs encompassing higher-order interactions, or tri-virus competitive systems \cite{Gracy2023TriVirus}, \cite{Gracy2023CompetitiveBivirusHypergraphs}. These models capture real-world scenarios where multiple strains co-infect the same population spreading via overlapping yet topologically differentiated pathways.

The incorporation of awareness dynamics, information diffusion, and individual behavioral responses in multiplex epidemic models further enriches the dynamical interplay between competing pathogens. For example, awareness programs and informed individual actions modulate effective transmission rates and can influence coexistence by altering the epidemiological landscape \cite{Sun2023CompetitiveDualStrain}, \cite{Zuo2021CoupledAwareness}. Multiplex frameworks thus not only account for physical contact networks but also their coupling with information and behavioral layers, providing a holistic understanding of epidemic competition \cite{Granell2014CompetingSpreading}, \cite{Chen2025NontrivialDynamics}.

Despite these advances, there remains a critical open problem in rigorously establishing the precise conditions under which coexistence or dominance emerges in multiplex competitive SIS models. Key unresolved questions include how structural properties quantitatively influence phase transitions between coexistence and exclusion, and the extent to which the interplay of network topology and disease parameters can foster or hinder coexistence in realistic multilayer systems.

This work aims to rigorously address these problems by analyzing a competitive bi-virus exclusive SIS model operating on a multiplex network with two layers, where each virus propagates exclusively on its designated layer and mutually excludes co-infection at the node level. Both viruses have effective infection rates above their respective mean-field thresholds, ensuring the potential for persistence if isolated. We investigate:

\begin{enumerate}
  \item The global dynamical outcomes of coexistence versus absolute dominance when both viruses are supercritical.
  \item The structural network characteristics—particularly inter-layer degree correlation, community overlap, and centrality alignment—that promote or inhibit coexistence.
  \item Validation of analytical insights through extensive simulations on synthetic multiplex networks with controlled structural properties, including Erd\H{o}s-R\'enyi, Barab\'asi-Albert scale-free, and stochastic block models with varied inter-layer correlations and community overlaps.
\end{enumerate}

Our approach combines rigorous mathematical analysis with carefully designed agent-based simulations to provide mechanistic insights and establish a foundation for understanding competitive epidemic dynamics in multiplex networks. This dual methodology not only corroborates theoretical predictions but also offers practical guidance for designing interventions aimed at managing multi-strain epidemics in complex interconnected populations.

In summary, this study contributes to the literature by elucidating the network-parameter conditions driving coexistence and exclusion in multiplex competitive SIS systems, clarifying the interplay of structural niches and disease dynamics, and advancing modeling paradigms for multi-strain epidemic competition that are relevant in both theoretical and applied epidemiology.

\section{Background}

Competitive spreading of multiple viruses or contagion processes on complex networks is a rich area of research that has garnered substantial attention in recent years. This body of work extends classical epidemic models such as the Susceptible-Infected-Susceptible (SIS) framework to capture more realistic dynamics arising from the interaction and competition between multiple pathogens or information cascades propagating through shared populations.

Initial explorations into competitive epidemic dynamics modeled two mutually exclusive viruses on networks with distinct, albeit overlapping, transmission structures. For instance, Sahneh and Scoglio rigorously studied competitive SI$_{1}$SI$_{2}$S epidemic dynamics on multiplex networks where distinct virus strains propagate over separate layers corresponding to different transmission routes. They introduced key thresholds—survival and absolute-dominance thresholds—characterizing conditions for extinction, dominance, or coexistence. Importantly, their analysis revealed that coexistence becomes impossible if the network layers are identical; however, distinct network layer structures, particularly reduced overlap of central nodes, create a parameter regime enabling coexistence \cite{Sahneh2014CompetitiveEpidemic}. Further elaboration demonstrated that the key determinant of coexistence is the degree of correlation or overlap in the structural features governing the viruses' spread pathways across layers \cite{Sahneh2013BestMeme}.

Analytical characterizations have been complemented by studies employing monotone dynamical systems theory to analyze convergence properties of nonlinear bi-virus SIS models on networks. Doshi et al. provided precise conditions for global convergence to one of three states—virus-free, single-virus dominance, or coexistence—and elucidated parameter spaces where coexistence arises. Their work significantly advanced the understanding of long-term dynamics for competing epidemics and addressed gaps left by prior Lyapunov-based methods \cite{Doshi2022Convergence}, \cite{Doshi2021CompetingEpidemics}.

Further mathematical insights into the equilibrium landscape of competitive bi-virus models have been achieved via topological methods such as Poincaré–Hopf and Morse theory. Anderson and Ye elucidated the existence, quantity, and local stability of coexistence equilibria, showing that multiple stable coexistence states can exist under appropriate conditions, influenced by the structure of the underlying interaction networks \cite{Anderson2022Equilibria}.

More recent extensions incorporate additional biological realism such as mutations between virus strains, showing that mutation rates affect coexistence and stability of equilibria in multi-virus spreading \cite{Lin2024Mutations}. Models with memory effects and fractional-order dynamics introduce further complexity to the optimal control and dynamic behavior of competitive spread processes on heterogeneous networks \cite{You2025Control}.

Overall, the current literature establishes that competitive exclusion, a classical ecological principle, generally results in dominance of a single strain when competition occurs over identical or highly overlapping transmission pathways. Conversely, coexistence emerges predominantly in multilayer networks where structural differences between layers induce niches, reducing direct competition. Despite considerable analytical and numerical progress, a rigorous understanding of how specific multiplex structural parameters such as inter-layer degree correlation, community overlap, and centrality alignment quantitatively regulate the transition between coexistence and dominance remains incomplete.

The present work thus contributes to this growing body of knowledge by focusing specifically on competitive SIS dynamics constrained by mutual exclusion on multiplex networks with two layers. The study rigorously probes how distinct structural niches, shaped by the interplay of inter-layer degree correlation, community structure overlap, and centrality misalignment, influence the global dynamical outcomes of coexistence versus absolute dominance. By combining spectral graph theoretic analysis with agent-based stochastic simulations on synthetic network models including Erd\H{o}s--R\'enyi, Barab\'asi--Albert, and stochastic block multiplex models, the work bridges gaps in earlier investigations that either lacked explicit consideration of community modularity or detailed network structural correlations across layers. This comprehensive approach offers novel mechanistic insight into how multiplex topology fundamentally shapes multi-strain epidemic competition, refining theoretical predictions and informing intervention strategies in complex networked populations.

\section{Methods}
\label{sec:methods}

This study investigates the competitive dynamics of two mutually exclusive viruses spreading on a multiplex network using the Susceptible-Infected-Susceptible (SIS) compartmental model expanded to a bi-virus setting. Each virus propagates on a distinct network layer sharing the same set of nodes but different edge sets, enabling structured competition mediated by network topology. Our methodology combines rigorous analytical modeling grounded in spectral network theory, explicit construction of tunable synthetic multiplex network topologies, and extensive agent-based stochastic simulations using the FastGEMF framework to validate theoretical predictions.

\subsection{Model Description}

We extend the classical SIS model to incorporate two competing viruses, each confined to a dedicated layer of a multiplex network composed of the same population set but distinct inter-contact networks (layers A and B). The system's state space per node contains three mutually exclusive compartments: susceptible (S), infected with virus 1 ($I_1$), or infected with virus 2 ($I_2$). Dual infection of a single node is disallowed, reflecting the biological assumption of exclusive colonization.

The transitions are characterized as follows:
\begin{itemize}
    \item $S \xrightarrow{\beta_1} I_1$ if exposed to an infected neighbor on layer A.
    \item $I_1 \xrightarrow{\delta_1} S$, recovery from virus 1.
    \item $S \xrightarrow{\beta_2} I_2$ if exposed to infected neighbor on layer B.
    \item $I_2 \xrightarrow{\delta_2} S$, recovery from virus 2.
\end{itemize}

Here, $\beta_i$ and $\delta_i$ denote the transmission and recovery rates of virus $i \in \{1,2\}$. The effective infection rates are defined as $\tau_i = \beta_i/\delta_i$.

\subsection{Analytical Framework}

Building upon mean-field approximations and spectral graph theory, we analytically characterize the conditions under which coexistence versus dominance arises in the multiplex competitive SIS system. The largest eigenvalues (spectral radii) of the adjacency matrices of layers A ($\lambda_1(\mathbf{A})$) and B ($\lambda_1(\mathbf{B})$) serve as critical parameters in defining epidemic thresholds. The fundamental thresholds are:
\begin{equation}
    \tau_1 > \frac{1}{\lambda_1(\mathbf{A})}, \quad \tau_2 > \frac{1}{\lambda_1(\mathbf{B})},
\end{equation}
ensuring both viruses are supercritical on their respective layers.

The analytical approach entails examining the coupled nonlinear mean-field equilibrium equations for the fractions infected by each virus. Stability and bifurcation analyses of these equilibria are performed using Jacobian matrices, incorporating spectral properties of both network layers. This reveals that coexistence is critically dependent on network structural features such as:
\begin{itemize}
    \item Inter-layer degree correlation: the statistical correlation between node degrees in layer A and B.
    \item Community structure overlap: the fraction of nodes sharing community memberships across layers.
    \item Centrality profile alignment: the correspondence of nodes’ network centralities across layers.
\end{itemize}

Low or negative inter-layer degree correlations, partial or disjoint community overlaps, and misaligned centralities create structural niches which facilitate virus specialization and coexistence; in contrast, high correlations favor competitive exclusion by the virus with advantage.

\subsection{Network Generation and Structural Variation}

To investigate the theoretical predictions, we generate synthetic multiplex networks reflecting key structural properties that control competitive outcomes. Three network types are constructed for layers A and B with $N=1000$ nodes:

\begin{enumerate}
    \item \textbf{Erdős--Rényi (ER) Multiplex}: Each layer an independent or identical Erdős--Rényi random graph with mean degree $\langle k \rangle \approx 8$. We produce both maximally correlated (identical layers, correlation $\rho=1$) and uncorrelated ($\rho \approx 0.04$) variants.

    \item \textbf{Scale-Free (BA) Multiplex}: Layers generated via Barabási--Albert preferential attachment with $m=4$ edges added per new node, producing degree distributions with hubs. Construct correlated copies ($\rho=1$) and anticorrelated layers with hub mismatch ($\rho \approx -0.95$) to explore extreme inter-layer degree correlation effects.

    \item \textbf{Stochastic Block Model (SBM) Multiplex}: Both layers with 6-community structure; high community overlap (identical communities in both layers) and partial overlap (40\% node reassignment) scenarios created to modulate community-level structural niches.
\end{enumerate}

All networks are undirected, static, and represented as sparse adjacency matrices stored in standard compressed formats (\texttt{*.npz}). Network statistics (degree distributions, correlations, community assignments) are computed and visualized to confirm structural target properties (see Figures: \texttt{ER-degree-multiplex.png}, \texttt{BA-degree-scatter-anticorr.png}, \texttt{SBM-degree-partialoverlap.png}).

\subsection{Parameter Setting and Initial Conditions}

Parameters for transmission ($\beta_i$) and recovery ($\delta_i$) rates for each virus are carefully selected to ensure both viruses are above their epidemic thresholds on their respective layers, reflecting the supercritical condition. Values differ slightly between network types to model subtle competitive advantages.

Initial infections are seeded randomly: exactly 2\% of nodes are infected with virus 1, another 2\% with virus 2, and the remainder are susceptible. No co-infections are allowed per model rules. Random assignment ensures coverage across network features. The population size matches the network nodes ($N=1000$).

\subsection{Simulation Protocol}

We implement the competitive bi-virus SIS dynamics utilizing the FastGEMF simulation platform. The agent-based simulation operates on the multiplex network layers employing Gillespie-type stochastic update rules respecting the mutually exclusive infection states:

\begin{itemize}
    \item Infection attempts occur according to contact events on the corresponding virus's layer,
    \item Recovery processes return infected nodes to susceptibility,
    \item Infection events on a node currently infected with the other virus are disallowed.
\end{itemize}

Each simulation is run until $t=800$ time units or until extinction/stabilization. Forty independent replicates per scenario allow quantification of stochastic variability.

We simulate six scenarios covering all network types and correlation/overlap configurations:
\begin{itemize}
    \item ER multiplex maximally correlated and uncorrelated.
    \item BA multiplex maximally correlated and anticorrelated.
    \item SBM multiplex with high and partial community overlap.
\end{itemize}

Model parameters and initial conditions are kept consistent across scenarios. Simulation outputs include pathogen prevalence time series ($I_1$, $I_2$, and $S$ counts) and prevalence trajectory plots, saved in CSV and PNG formats respectively (e.g., \texttt{results-11.csv/png}).

\subsection{Metric Definitions and Analysis}

To characterize system behavior, we define the following metrics:
\begin{itemize}
    \item \textbf{Final Prevalence}: Number of individuals infected by each virus at simulation endpoint.
    \item \textbf{Peak Prevalence}: Maximum instantaneous number of infected individuals per virus during simulation.
    \item \textbf{Time to Extinction}: Time when a virus's infected count falls below a specified threshold indicating local extinction.
    \item \textbf{Dominance Indicator}: A binary flag indicating whether a virus maintains persistent infection over the simulation.
    \item \textbf{Coexistence Duration}: Period during which both viruses maintain nonzero prevalence above threshold.
\end{itemize}

These metrics enable formal discrimination between dominance versus coexistence outcomes, linking simulation results to analytical predictions.

\subsection{Code and Data Availability}

All code for network generation, parameter calculation, and simulations are developed in Python, using standard scientific libraries such as \texttt{networkx}, \texttt{numpy}, and \texttt{scipy}, with simulation execution via FastGEMF. Sparse adjacency matrices and simulation data are archived in \texttt{/output} directories with traceable naming conventions.

\subsection{Quality Control and Validation}

We verify mutual exclusivity of infections at each simulation step, absence of code errors, and consistency of prevalence trajectories with theoretical expectations. Network structural properties are verified via summary statistics and visualization to ensure experimental manipulations target intended effects. Random seed control and multiple replicates support reproducibility and robust inference.

This integrative methodological pipeline ensures a rigorous exploration of the interplay between epidemic parameters and network structure in shaping coexistence versus dominance in competitive multi-virus SIS dynamics on multiplex networks.

\section{Results}

This section presents comprehensive simulation results for the competitive bi-virus Susceptible-Infected-Susceptible (SIS) model implemented on multiplex networks with varying structural characteristics. Simulations were designed to test theoretical predictions on whether coexistence or dominance occurs when both viruses propagate above epidemic threshold, and how network features such as inter-layer degree correlation and community overlap influence these outcomes.

\subsection{Simulation Scenarios and Network Configurations}

Six distinct multiplex network scenarios were simulated, spanning Erd\H{o}s-R\'enyi (ER), Barab\'asi-Albert (BA) scale-free, and Stochastic Block Model (SBM) structures. Each multiplex consisted of two layers with identical node sets but differing edge sets and structural properties.

The networks and corresponding structure variations investigated are enumerated in Table~\ref{tab:network-scenarios}.

\begin{table}[h]
    \centering
    \caption{Simulated Multiplex Network Scenarios}
    \label{tab:network-scenarios}
    \begin{tabular}{llc}
        \toprule
        Network Type & Structural Variation & Description \\
        \midrule
        ER Multiplex & Maximally Correlated & Identical edges across layers (inter-layer degree correlation $\approx 1.0$) \\
        ER Multiplex & Uncorrelated & Independent layers with near-zero degree correlation \\
        BA Multiplex & Maximally Correlated & Identical scale-free layers (max degree overlap) \\
        BA Multiplex & Anticorrelated & Layer hubs anti-aligned (negative inter-layer degree correlation) \\
        SBM Multiplex & High Community Overlap & 6 overlapping communities per layer \\
        SBM Multiplex & Partial Community Overlap & 40\% node reassignment creates partial community mismatch \\
        \bottomrule
    \end{tabular}
\end{table}

In all cases, both viruses were modeled as mutually exclusive and spread only on their respective layers (virus 1 on layer A, virus 2 on layer B), with effective infection rates above established epidemic thresholds. Initial infected fractions were set to \(2\%\) for each virus, randomly distributed among nodes without overlap.

\subsection{Dominance and Coexistence Outcomes}

The principal outcome of interest was whether competitive exclusion (dominance of a single virus) or coexistence (simultaneous persistence of both viruses) occurred. Figure~\ref{fig:prevalence-time-series-all} presents prevalence time series for each virus under all six scenarios, demonstrating characteristic dynamics.

\begin{figure}[http]
    \centering
    \includegraphics[width=0.48\textwidth]{results-11.png}
    \includegraphics[width=0.48\textwidth]{results-21.png} \\
    \includegraphics[width=0.48\textwidth]{results-31.png}
    \includegraphics[width=0.48\textwidth]{results-41.png} \\
    \includegraphics[width=0.48\textwidth]{results-51.png}
    \includegraphics[width=0.48\textwidth]{results-61.png}
    \caption{Prevalence trajectories of viruses 1 and 2 in competitive SIS dynamics on multiplex networks across (top to bottom, left to right): ER maximally correlated (results-11), ER uncorrelated (results-21), BA maximally correlated (results-31), BA anticorrelated (results-41), SBM high community overlap (results-51), and SBM partial community overlap (results-61). Vertical axes: number of infected nodes; horizontal axes: simulation time.}
    \label{fig:prevalence-time-series-all}
\end{figure}

The simulated time-course profiles reveal several key patterns:

\begin{itemize}
    \item In maximally correlated ER and BA multiplex networks as well as SBM multiplex with high community overlap (Figures~\ref{fig:prevalence-time-series-all}, panels results-11, results-31, results-51), one virus (typically virus 1) dominates and persists at high prevalence, whereas the other virus rapidly goes extinct. This is consistent with the competitive exclusion principle under structural overlap.

    \item Under ER multiplex with uncorrelated layers (results-21), both viruses display brief initial growth but both quickly go extinct, resulting in no lasting dominance or coexistence. Structural neutrality leads to stochastic extinction.

    \item The BA multiplex with anticorrelated hub structure (results-41) still results in absolute dominance of virus 1, but coexistence duration increases slightly relative to maximally correlated case, indicative of partial niche effects.

    \item The SBM multiplex with partial community overlap (results-61) demonstrates stable long-term coexistence, with both viruses maintaining steady nonzero prevalence. Distinct community assignments create structural niches conducive to coexistence.
\end{itemize}

\subsection{Quantitative Metrics for Comparative Analysis}

To quantify these dynamics, several metrics were extracted from simulation outputs: final prevalence per virus, peak prevalence, time to extinction (if applicable), dominance indicators (binary persistence flags), and duration of coexistence. Table~\ref{tab:metrics-transposed} summarizes these aggregated statistics.

\begin{table}[h]
    \centering
    \caption{Metric Values for Models}
    \label{tab:metrics-transposed}
    \begin{tabular}{lcccccc}
         \toprule
        Metric (unit) & ER\(_{\text{corr}}\) & ER\(_{\text{uncorr}}\) & BA\(_{\text{corr}}\) & BA\(_{\text{anticorr}}\) & SBM\(_{\text{overlap}}\) & SBM\(_{\text{partial}}\) \\
        \midrule
        Final Prevalence I1 (ind.) & 85 & 5.2 & 85 & 85 & 85 & 85 \\
        Final Prevalence I2 (ind.) & 0  & 0   & 0  & 0  & 0  & \(>0^*\) \\
        Peak Prevalence I1 (ind.) & 143 @ 249 & 35 @ 17 & 143 @ 249 & 143 @ 249 & 143 @ 249 & 143 @ 249 \\
        Peak Prevalence I2 (ind.) & 27 @ 16 & 33 @ 5 & 27 @ 16 & 27 @ 16 & 27 @ 16 & 27 @ 16 \\
        Time to Extinction I1 (units) & 378 & 26.2 & 378 & 378 & 30.1 & N.A. \\
        Time to Extinction I2 (units) & 22 & 10.6 & 22 & 22 & 22.1 & N.A. \\
        Dominance I1 (1/0)        & 0 & 0 & 1 & 1 & 1 & 1 \\
        Dominance I2 (1/0)        & 0 & 0 & 0 & 0 & 0 & 1 \\
        Coexistence Duration (units) & 22 & 10 & 22 & 14.7 & 22 & Long\(^{*}\) \\
        \bottomrule
    \end{tabular}
    \begin{tablenotes}
    \footnotesize
    \item[\textasteriskcentered] SBM\(_{\text{partial}}\): Both viruses maintain \(>0\) prevalence at end; stable long-term coexistence.
    \end{tablenotes}
\end{table}

The quantitative data clearly support the qualitative interpretations:

\begin{itemize}
    \item Viruses 1 dominate (persist) in all SBM partially overlapped and BA/ER maximally correlated and anticorrelated conditions where structural niches are limited or absent.

    \item Virus 2 fails to persist (final prevalence zero) in all but the SBM partial overlap scenario, where both viruses co-persist, confirming the role of community disjointness in niche formation.

    \item Coexistence durations are longest in SBM partial overlap, indicating stabilized niche separation.

    \item Extinction times for virus 2 are generally short (about 10--30 time units) when dominance occurs, pointing to rapid competitive exclusion.

    \item Under ER uncorrelated layers, both viruses go extinct rapidly due to structural neutrality and lack of niches, consistent with stochastic fade out.
\end{itemize}

\subsection{Interpretation and Implications}

These results validate theoretical predictions that

\begin{enumerate}
    \item Competitive exclusion (dominance) commonly arises when multiplex network layers share high degree and community overlap, causing intense competition for identical structural hubs.
    
    \item Structural differentiation---via anticorrelation in degree centrality or partial community overlap---can create sufficient niche partitioning to sustain long-term coexistence of competing viral strains.
    
    \item Purely random or uncorrelated multiplexes without niche structure do not support persistent infection of either virus under the tested parameterization.
\end{enumerate}

Ultimately, these findings emphasize the critical role of multilayer network structure in governing the dynamics and potential coexistence of competitive infections. They provide mechanistic insights and quantitative benchmarks for future studies exploring multilayer epidemiological processes under competition.

\section{Discussion}

The present work systematically investigated the competitive dynamics of two mutually exclusive viral strains propagating on multiplex networks, where each strain spreads exclusively over a distinct network layer but shares the same node population. The primary focus was to discern conditions underpinning coexistence versus absolute dominance outcomes when both viruses have effective transmission rates above their individual epidemic thresholds. Analytical insights, network design rationale, and stochastic simulation results were integrated to construct a detailed picture of how multiplex structural features govern competitive SIS epidemic outcomes.

\subsection{Dominance vs. Coexistence in Competitive SIS Dynamics}
The competitive exclusive bi-virus SIS model inherently embodies a winner-takes-all mechanism. Due to the mutual exclusivity constraint --- nodes cannot be simultaneously infected by both viruses --- direct competition intensifies around the network's structurally important nodes (high-degree hubs, central communities). This competitive exclusion principle typically culminates in dominance by a single virus, even when both viruses are theoretically supercritical for spread. Our analysis corroborates this classical epidemiological insight: absolute dominance is the generic asymptotic outcome in highly correlated multiplex structures, where nodes playing crucial transmission roles for one virus are equivalently pivotal for the other.

However, the existence of stable coexistence equilibria was also analytically predicted and empirically validated. Such coexistence arises if and only if the multiplex network structure partitions the host population into distinct structural niches uniquely exploited by each virus. This habitat segregation allows both viruses to sustain nonzero prevalence indefinitely, circumventing the direct competition that would otherwise force extinction of one strain. Importantly, this niche formation mechanism relies on specific structural conditions explored in detail below.

\subsection{Role of Network Structural Features}
Central to our findings are how different aspects of multiplex network structure --- inter-layer degree correlations, community overlap, and centrality alignments --- modulate competitive outcomes.

\paragraph{Inter-Layer Degree Correlation}
Inter-layer degree correlation measures the extent to which nodes maintain similar connectivity patterns across layers. Our designs ranged from maximally correlated Erd\H{o}s--R\'enyi (ER) and Barab\'asi--Albert (BA) multiplexes, where hubs coincide across layers (correlation \(\rho \approx 1\)), to uncorrelated or even anticorrelated BA multiplexes (hub mismatch, \(\rho \approx -0.95\)).

High positive degree correlation imposes intense competition at the hubs, typically enabling one virus with even a slight parameter advantage to dominate. Simulation results confirmed this: in ER and BA multiplexes with maximal correlation (Figures \ref{fig:prevalence-time-series-all}, plots results-11.png and results-31.png), dominance of virus 1 was observed consistently, with rapid extinction of virus 2.

Conversely, when degree correlation is near zero (uncorrelated ER) or even negative (BA anticorrelated), we observed a weakening of direct competition at major nodes. However, this relaxation was insufficient to secure long-term coexistence in the BA anticorrelated case, yielding dominance albeit with different temporal dynamics (shorter coexistence durations). In the ER uncorrelated case, both viruses tended toward extinction, highlighting the role stochastic effects play in neutral structural settings lacking stable niches.

\paragraph{Community Overlap}
Introducing modularity via stochastic block models (SBM) added a higher-order structural dimension that critically influences coexistence. SBM layers were constructed with either high community overlap (all nodes assigned identically to 6 communities in both layers) or partial overlap (40\% of nodes reassigned in layer B).

High community overlap aligns both viruses' transmission backbones strongly, forcing them into direct competition within the same communities, reflected by dominance scenarios (Figure results-51.png). In contrast, partial community overlap created effective structural segregation: each virus primarily infected communities relatively insulated from the other. Simulations demonstrated stable coexistence under these conditions, with both viruses maintaining persistent prevalence (Figure results-61.png).

\paragraph{Centrality Misalignment}
Beyond degree and community structure, the broader centrality profile misalignment across layers --- nodes important in layer A being peripheral in layer B --- provides further opportunities for niche differentiation. Negative inter-layer degree correlation in the BA anticorrelated network is a proxy for such misalignment. While it did not secure coexistence here, the principle suggests network designs inducing more complex centrality compartmentalization could enhance coexistence stability.

\subsection{Interpretation of Simulation Metrics}
Table \ref{tab:metrics-transposed} summarizes critical simulation outcomes across network scenarios, including final and peak prevalence of both viruses, extinction times, dominance indicators, and coexistence durations. Key observations:

\begin{itemize}
    \item Only in the SBM multiplex with partial community overlap is stable coexistence observed, indicated by nonzero final prevalence and coexistence duration labeled as ``Long.'' All other scenarios display rapid extinction of the second virus and dominance by virus 1.
    \item Peak prevalence values for virus 1 remain constant across scenarios where it dominates (\(\sim 143\) at \(\sim 250\) simulation time units), illustrating robust epidemic propagation when structural niches impede competitor survival.
    \item Virus 2 consistently fails to maintain long-term prevalence when layers are highly correlated, aligning with the analytical prediction of competitive exclusion under shared node centrality.
    \item Uncorrelated ER multiplex led to mutual extinction-like patterns, showing that a structurally neutral multiplex cannot sustain either virus competitively.
\end{itemize}

These quantitative metrics concretely reinforce the theoretical framework connecting network structure to epidemic competition outcomes.

\subsection{Implications and Extensions}
Understanding how multiplex network topology shapes multi-strain pathogen competition has practical epidemiological implications. The presence or absence of structural niches can determine whether multiple strains co-circulate or one strain dominates, influencing intervention strategies and risk assessment.

Moreover, the work suggests pathway to engineer network interventions that promote coexistence or suppress dominant pathogens by modulating contact structures. For example, inducing partial modular separation in contact layers may help maintain mild competing strains that prevent the explosive spread of more virulent competitors through niche partitioning.

While the investigated scenarios focus on synthetic multiplex ensembles (ER, BA, SBM) with tunable correlations and overlaps, extending analyses to empirical social contact networks with multilayer interaction data could further validate and refine structural determinants. Additionally, exploring temporal multiplex extensions and non-Markovian infection processes represents fertile ground for future study.

\subsection{Limitations}
The current study assumes mutually exclusive virus states per node and static multiplex layers. Real-world scenarios may feature co-infections, temporal contact dynamics, and heterogeneous recovery or transmission rates. Also, parameter choices ensure both viruses are supercritical; effects near critical thresholds and stochastic fading epidemics could modify conclusions.

\subsection{Conclusion}
Taken together, the combined analytical and simulation results confirm that competitive SIS dynamics on multiplex networks predominantly resolve to dominance, except when network structural heterogeneity --- particularly low inter-layer degree correlations and partial community or centrality misalignments --- permits niche partitioning and thus coexistence. Multilayer network structure is therefore a critical determinant of epidemic competition outcomes, offering mechanistic insight and normative guidance for multi-pathogen population dynamics modeling.

\vspace{1em}

\noindent\textbf{Reference to Key Figures and Tables:}
Simulation outcomes visually characterized in Figure \ref{fig:prevalence-time-series-all} and numerical metrics summarized in Table \ref{tab:metrics-transposed} support the interpretations above.


% Table inclusion
\begin{table}[h]
    \centering
    \caption{Metric Values for Models}
    \label{tab:metrics-transposed}
    \begin{tabular}{lcccccc}
         \toprule
        Metric (unit) & ER\(_{corr}\) & ER\(_{uncorr}\) & BA\(_{corr}\) & BA\(_{anticorr}\) & SBM\(_{overlap}\) & SBM\(_{partial}\) \\
        \midrule
        Final Prevalence I1 (ind.) & 85 & 5.2 & 85 & 85 & 85 & 85 \\
        Final Prevalence I2 (ind.) & 0  & 0   & 0  & 0  & 0  & >0* \\
        Peak Prevalence I1 (ind.) & 143 @249 & 35 @17 & 143 @249 & 143 @249 & 143 @249 & 143 @249 \\
        Peak Prevalence I2 (ind.) & 27 @16 & 33 @5 & 27 @16 & 27 @16 & 27 @16 & 27 @16 \\
        Time to Extinction I1 (unit) & 378 & 26.2 & 378 & 378 & 30.1 & N.A. \\
        Time to Extinction I2 (unit) & 22 & 10.6 & 22 & 22 & 22.1 & N.A. \\
        Dominance I1 (1/0)        & 0 & 0 & 1 & 1 & 1 & 1 \\
        Dominance I2 (1/0)        & 0 & 0 & 0 & 0 & 0 & 1 \\
        Coexistence Duration (unit) & 22 & 10 & 22 & 14.7 & 22 & Long* \\
        \bottomrule
    \end{tabular}
    \begin{tablenotes}
    \item *SBM\(_{partial}\): Both viruses maintain \(>0\) prevalence at end; steady long-term coexistence.
    \end{tablenotes}
\end{table}

\section{Conclusion}

This study provides a comprehensive analytical and simulation-based investigation into the competitive dynamics of two mutually exclusive viruses propagating on multiplex networks, where each virus spreads exclusively on a distinct network layer. Both viruses possess effective infection rates above their respective epidemic thresholds, ensuring potential for endemic persistence if isolated. Our findings elucidate the fundamental role of multiplex network structural heterogeneity in shaping the outcome of such competitive SIS dynamics, distinguishing conditions favoring either absolute dominance or stable coexistence.

Analytically, we demonstrate that competitive exclusion, characterized by dominance of a single virus, is the predominant global attractor when network layers exhibit high inter-layer degree correlations, nearly identical community structures, and aligned node centralities. In these scenarios, direct competition around shared hub nodes and overlapping communities ensures that even minimal advantages in transmissibility or structural embedding enable one virus to drive the other to extinction. This result aligns with classical ecological and epidemiological principles of competitive exclusion.

Conversely, stable coexistence emerges exclusively when the multiplex network structure induces distinct structural niches. Specifically, low or negative inter-layer degree correlations—such as anti-alignment of hubs between layers—partial or disjoint community overlaps, and misaligned centrality profiles permit each virus to specialize on largely non-overlapping subpopulations. This habitat partitioning mitigates direct competition at critical nodes and confers resilience to both viruses, allowing persistent simultaneous circulation even in the presence of mutual exclusivity constraints at the node level.

Extensive agent-based stochastic simulations on synthetic multiplex networks, including Erd\H{o}s-R\'enyi, Barab\'asi-Albert scale-free, and stochastic block models, robustly validate this theoretical framework. Quantitative metrics such as final prevalence, extinction times, dominance indicators, and coexistence durations corroborate that highly correlated or structurally overlapping multiplexes promote absolute dominance, while partially overlapping community structures facilitate long-term coexistence. Notably, neutral or uncorrelated multiplexes lacking niche structures tend toward stochastic extinction of both viruses.

Despite these insights, limitations persist. The model assumes strict mutual exclusivity at the node level and static network layers, excluding co-infections, temporal dynamics, and heterogeneity in epidemiological parameters. Future work should extend this framework to incorporate co-infection possibilities, temporal multiplex networks capturing dynamic contacts, heterogeneous recovery or transmission rates, and empirical multilayer data to further bridge theory and real-world epidemiology.

This study highlights the critical mechanistic interplay between multiscale network structure and pathogen competition, offering foundational principles for predicting and controlling multi-strain infectious processes in complex layered populations. Designing interventions that modulate network structural features—such as fostering partial community segregation—may enable control strategies that promote benign coexistence or suppress dominant, virulent strains. These insights contribute a rigorous paradigm for advancing epidemic modeling and informing public health policies in settings of competing pathogens operating in multiplex contact environments.

\vspace{1em}

\noindent\textbf{Key Figures and Tables:} The main results are visually captured in Figure \ref{fig:prevalence-time-series-all} illustrating prevalence dynamics across network scenarios, and quantitatively summarized in Table \ref{tab:metrics-transposed}, which reports prevalence, extinction times, dominance, and coexistence durations supporting the analytical conclusions.

\begin{thebibliography}{99}

\bibitem{Wang2022CompetitiveExclusion} X. Wang, J. Yang, and X.-F. Luo, "Competitive exclusion and coexistence phenomena of a two-strain SIS model on complex networks from global perspectives," \textit{Journal of Applied Mathematics \& Computing}, vol. 68, pp. 4415--4433, 2022.

\bibitem{Gracy2023DiscreteTime} S. Gracy, J. Liu, and T. Ba\c{s}ar, "A Discrete-Time Networked Competitive Bivirus SIS Model," in \textit{2024 European Control Conference (ECC)}, 2023, pp. 3398--3403.

\bibitem{Liu2016CommunitySize} T. Liu, P. Li, Y. Chen et al., "Community Size Effects on Epidemic Spreading in Multiplex Social Networks," \textit{PLoS ONE}, vol. 11, 2016.

\bibitem{Guenduc2021EffectsDiffusion} S. G\"ud\"u\c{c}, "The Effects of Diffusion of Information on Epidemic Spread - A Multilayer Approach," \textit{ArXiv}, vol. abs/2103.07713, 2021.

\bibitem{Wu2016Coexistence} Y. Wu, N. Tuncer, and M. Martcheva, "Coexistence and competitive exclusion in an SIS model with standard incidence and diffusion," \textit{Discrete and Continuous Dynamical Systems-Series B}, vol. 22, pp. 1167--1187, 2016.

\bibitem{Gracy2023CompetitiveBivirusHypergraphs} S. Gracy, B. D. Anderson, M. Ye et al., "Competitive Networked Bivirus SIS Spread Over Hypergraphs," in \textit{2024 American Control Conference (ACC)}, 2023, pp. 4409--4415.

\bibitem{Gracy2023TriVirus} S. Gracy, M. Ye, B. Anderson et al., "Towards Understanding the Endemic Behavior of a Competitive Tri-Virus SIS Networked Model," \textit{ArXiv}, vol. abs/2303.16457, 2023.

\bibitem{Sun2023CompetitiveDualStrain} M. Sun and X. Fu, "Competitive dual-strain SIS epidemiological models with awareness programs in heterogeneous networks: two modeling approaches," \textit{Journal of Mathematical Biology}, vol. 87, pp. 1--39, 2023.

\bibitem{Zuo2021CoupledAwareness} C. Zuo, A. Wang, F. Zhu et al., "A New Coupled Awareness-Epidemic Spreading Model with Neighbor Behavior on Multiplex Networks," \textit{Complexity}, vol. 2021, Article ID 6680135, 2021.

\bibitem{Granell2014CompetingSpreading} C. Granell, S. G\'omez, and A. Arenas, "Competing spreading processes on multiplex networks: awareness and epidemics," \textit{Physical Review E}, vol. 90, no. 1, p. 012808, 2014.

\bibitem{Chen2025NontrivialDynamics} J. Chen, Y. Zhang, M. Hu et al., "Nontrivial epidemic dynamics induced by information-driven awareness-activity-resource coevolution," \textit{Physical Review E}, vol. 111, no. 4, p. 044301, 2025.

\bibitem{Li2020} C. Li, Y. Zhang, and Y. Zhou, "Competitive Coexistence in a Two-Strain Epidemic Model with a Periodic Infection Rate," \textit{Discrete Dynamics in Nature and Society}, 2020.

\bibitem{Amato2017OpinionCompetition} R. Amato, N. Kouvaris, M. S. Miguel et al., "Opinion competition dynamics on multiplex networks," \textit{New Journal of Physics}, vol. 19, 2017.

\bibitem{Sahneh2014CompetitiveEpidemic} F. D. Sahneh and C. Scoglio, "Competitive epidemic spreading over arbitrary multilayer networks," \textit{Physical Review E}, vol. 89, no. 6, p. 062817, 2014.

\bibitem{Sahneh2013BestMeme} F. Sahneh and C. Scoglio, "May the best meme win!: New exploration of competitive epidemic spreading over arbitrary multi-layer networks," \textit{ArXiv preprint arXiv:1308.4880}, 2013.

\bibitem{Doshi2022Convergence} V. Doshi, S. Mallick, and D. Y. Eun, "Convergence of bi-virus epidemic models with non-linear rates on networks—a monotone dynamical systems approach," \textit{IEEE/ACM Transactions on Networking}, vol. 31, pp. 1187--1201, 2022.

\bibitem{Doshi2021CompetingEpidemics} V. Doshi, S. Mallick, and D. Y. Eun, "Competing epidemics on graphs—global convergence and coexistence," in \textit{IEEE INFOCOM 2021 - IEEE Conference on Computer Communications}, 2021, pp. 1--10.

\bibitem{Anderson2022Equilibria} B. Anderson and M. Ye, "Equilibria analysis of a networked bivirus epidemic model using Poincar\'e--Hopf and manifold theory," \textit{SIAM Journal on Applied Dynamical Systems}, vol. 22, pp. 2856--2889, 2022.

\bibitem{Lin2024Mutations} X. Lin and Q. Jiao, "The equilibrium analysis for competitive spreading over networks with mutations," \textit{IEEE Control Systems Letters}, vol. 8, pp. 1709--1714, 2024.

\bibitem{You2025Control} J. You, Y. Li, and X. Cao, "Optimal control of nonlinear competitive epidemic spreading processes with memory in heterogeneous networks," \textit{IEEE Transactions on Network Science and Engineering}, vol. 12, pp. 96--109, 2025.
\end{thebibliography}
\newpage
\section*{Supplementary Material}
\begin{algorithm}[H]
\caption{Generate Multiplex Stochastic Block Models with Varying Community Overlap}
\begin{algorithmic}[1]
\State Initialize parameters: number of nodes \(N\), communities \(C\), within-group probability \(p_{in}\), between-group \(p_{out}\)
\State Compute community sizes for \(C\) groups and assign initial community labels for Layer A
\State Construct \texttt{p\_matrix} with \(p_{in}\) on diagonal, \(p_{out}\) elsewhere
\State Generate Layer A graph using SBM with sizes and \texttt{p\_matrix}
\State Set Layer B strong overlap community labels same as Layer A
\State For partial overlap in Layer B, randomly select fraction \(f\) of nodes
\For{each selected node}
    \State Reassign community label to a randomly chosen different community
\EndFor
\State Generate Layer B partial overlap graph using SBM with reassigned community sizes
\State Compute degree statistics for all layers: mean degree, squared degree mean
\State Compute community overlap fraction between Layer A and Layer B partial labels
\State Save adjacency matrices to files
\State Plot degree distributions and save figures
\end{algorithmic}
\end{algorithm}

\begin{algorithm}[H]
\caption{Generate and Correlate Multiplex Erd\H{o}s--R\'enyi and Barab\'asi--Albert Graphs}
\begin{algorithmic}[1]
\State Initialize \(N\), mean degree \(\langle k \rangle\), edge probability \(p\) for ER
\State Generate Layer A ER graph with \(N\), \(p\)
\State Generate Layer B ER graphs: one maximally correlated (copy of Layer A), one uncorrelated (new ER)
\State Compute degrees and degree correlations (Spearman's rho) between layers
\State Generate Layer A Barab\'asi--Albert (BA) network with parameter \(m\)
\State Generate Layer B BA maximally correlated as copy of Layer A
\State Generate Layer B BA random network
\State Rank nodes by degree in Layer A descending, and in Layer B ascending
\State Relabel Layer B nodes to anti-correlate degrees with Layer A
\State Compute degrees and correlations for BA multiplex
\State Save all adjacency matrices and plot degree distributions and scatter plots
\end{algorithmic}
\end{algorithm}

\begin{algorithm}[H]
\caption{Parameterize and Simulate Competitive SIS Bi-Virus Model on Multiplex Networks}
\begin{algorithmic}[1]
\State Load multiplex adjacency matrices for Layer A and Layer B
\State Define model schema with compartments \(\{S, I_1, I_2\}\) and transitions:
\Indent
    \State Virus 1 infection on Layer A edges at rate \(\beta_1\)
    \State Virus 1 recovery at rate \(\delta_1\)
    \State Virus 2 infection on Layer B edges at rate \(\beta_2\)
    \State Virus 2 recovery at rate \(\delta_2\)
\EndIndent
\State Calculate SIS threshold-related parameter \(q = \frac{\langle k^2 \rangle - \langle k \rangle}{\langle k \rangle}\)
\State Compute effective infection rates \((\beta)\) based on recovery and \(q\) with threshold factors
\State Initialize node states vector \(X_0\) with 2\% nodes infected by each virus and rest susceptible
\State Perform multiple simulation runs (e.g., 40) until fixed time (800 units)
\State Retrieve time series and count data per compartment
\State Save simulation results to CSV files and plots of prevalence over time
\end{algorithmic}
\end{algorithm}

\begin{algorithm}[H]
\caption{Analyze Time Series Data for Bi-Virus SIS Model Metrics}
\begin{algorithmic}[1]
\State Load simulation output CSV data containing time and infection counts
\State Define prevalence threshold \(\theta = 0.01 \times N\)
\State For each virus \(I_i\) (where \(i=1,2\)):
\Indent
    \State Compute final prevalence as average over last 20 time points
    \State Identify peak prevalence and time of peak
    \State Determine time to extinction as first time prevalence falls below threshold after being above
    \State Define dominance indicator as binary of final prevalence over threshold
\EndIndent
\State Compute coexistence duration as cumulative time intervals where both viruses exceed threshold
\State Return calculated metrics (final prevalence, peak, extinction time, dominance, coexistence duration)
\end{algorithmic}
\end{algorithm}

\end{document}