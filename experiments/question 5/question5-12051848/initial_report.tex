\documentclass{article}
\usepackage[utf8]{inputenc}
\usepackage{amsmath}
\usepackage{algorithm}
\usepackage{algpseudocode}
\usepackage{graphicx}
\usepackage{hyperref}
\usepackage{natbib} 
\usepackage{geometry}
\usepackage{booktabs}
\graphicspath{./}
\usepackage{tikz}
\usepackage{lipsum} % For dummy text
\usepackage{eso-pic} % For placing content on every page
\newcommand\BackgroundConfidential{%
    \put(0,0){%
        \parbox[b][\paperheight]{\paperwidth}{%
            \vfill
            \centering
            \tikz[remember picture,overlay] \node[scale=5,opacity=0.2,rotate=45,align=center] {Warning:\\Generated By AI\\ \textbf{EpidemIQs}};
            \vfill
        }%
    }%
}
\title{Critical Vaccination Thresholds to Halt Epidemic Spread on Uncorrelated Random Networks: Analytical and Simulation Insights on Random and Targeted Degree-10 Vaccination Strategies}
\author{EpidemIQs, Primary Agent Backone LLM: o3,  LaTeX Agent LLM : gpt-4.1-mini}
\date{\today}
\begin{document}
\AddToShipoutPictureBG{\BackgroundConfidential}
\maketitle

\begin{abstract}
In this study, we quantitatively determine the critical vaccination thresholds necessary to halt epidemic spread on uncorrelated random networks characterized by mean degree \( z = 3 \), mean excess degree \( q = 4 \), and a basic reproduction number \( R_0 = 4 \). The disease dynamics are modeled by an SIR compartmental framework incorporating sterilizing immunity conferred by vaccination, such that vaccinated individuals neither transmit nor become reinfected. Two distinct vaccination strategies are analyzed: (i) random (uniform) vaccination where individuals are vaccinated at random, and (ii) targeted vaccination where all nodes with degree exactly \( k = 10 \) are vaccinated.

Analytic percolation-based theory establishes that the critical fraction of the population that must be randomly vaccinated, \( p_c \), satisfies \( p_c = 1 - \frac{1}{R_0} = 0.75 \), indicating that 75\% immunization coverage is required to prevent a major outbreak. In contrast, targeted vaccination of degree-10 nodes requires vaccination of approximately 3.33\% of the total population if enough such nodes exist to interrupt transmission. However, in a Poisson degree distribution with mean degree 3, degree-10 nodes are exceedingly rare (approximately 0.081\%), rendering targeted vaccination ineffective in typical sparse random networks.

To empirically validate the theoretical predictions, we simulate the SIR epidemic process on two network topologies: a classical Poisson(3) configuration model and a tailored network with a trimodal degree distribution (degrees 2, 3, and 10) designed to incorporate sufficient high-degree nodes. Simulation results confirm the necessity of vaccinating at least 75\% of nodes under random vaccination to curtail epidemic spread. For targeted vaccination, removal of all degree-10 nodes induces a sharp epidemic collapse exclusively in the tailored network, where these nodes are sufficiently abundant (approximately 10.6\% of the population), while targeted vaccination fails in the Poisson network due to the scarcity of hubs.

These findings cogently demonstrate the critical role of network structure and degree distribution in determining vaccination strategy efficacy. Random vaccination demands high coverage to prevent outbreaks, whereas degree-based targeting can drastically reduce vaccination requirements only if high-degree nodes are prevalent. This integrative analytic and simulation approach provides robust guidance for optimized vaccination policy formulation based on underlying contact network heterogeneity.
\end{abstract}

\section{Introduction}

The control of epidemic outbreaks through vaccination remains a fundamental challenge in network epidemiology, particularly when considering how vaccination strategies interact with the underlying contact structure of populations. Standard infectious disease models such as the susceptible-infected-recovered (SIR) framework have been extensively used to investigate the dynamics of epidemic spread on complex networks \cite{Bogu2013Nature,EpidemicThreshold2012Chen2018,WeiWang2015Predicting}. Central to these studies is the concept of the epidemic threshold, which defines the critical transmissibility or vaccination coverage needed to prevent large-scale outbreaks.

The classic epidemiological metric, the basic reproduction number \( R_0 \), quantifies the average number of secondary infections caused by a typical infectious individual in a fully susceptible population. When \( R_0 > 1 \), an epidemic can spread; thus, vaccination strategies aim to reduce the effective reproduction number below one, either by lowering transmission probabilities or by removing susceptible nodes from the network through immunization \cite{Bogu2013Nature,Lindquist2011Effective}.

Early theoretical approaches have modeled vaccination effects as a form of site percolation on the network, where vaccinated individuals correspond to permanently removed nodes, thereby reducing the mean degree and altering the network's connectivity \cite{Bogu2013Nature,Lee2012Epidemic}. Under the assumption of random vaccination, the critical vaccination fraction \( p_c \) necessary to stop an epidemic has been shown analytically to satisfy the relation \( p_c = 1 - \frac{1}{R_0} \) for locally tree-like, uncorrelated random networks \cite{WeiWang2015Predicting}. This result offers a benchmark for intervention strategies but falls short in capturing more targeted approaches.

Targeted vaccination strategies, where nodes with specific structural properties such as high degree are preferentially immunized, have been proposed to more efficiently disrupt transmission chains \cite{Chen2018Global,Torres2021Nonbacktracking}. Intuitively, removing nodes with the highest number of connections can greatly reduce the epidemic potential by fragmenting the network or significantly lowering its effective branching factor. However, the effectiveness of targeted vaccination depends critically on the abundance and distribution of such high-degree nodes in the contact network \cite{AlvarezZuzek2017Epidemic,Bianconi2020MessagePassing}.

Recent analytical and simulation studies highlight that while targeting high-degree nodes can drastically reduce disease spread in heterogeneous networks where these nodes are sufficiently prevalent, the same strategy may fail in homogeneous or sparsely connected networks where high-degree nodes are rare \cite{Chen2018Global,AlvarezZuzek2017Epidemic}. For instance, in a Poisson degree-distributed network with mean degree \( z = 3 \), nodes having degree \( k = 10 \) are exceedingly rare, rendering targeted vaccination of these nodes insufficient to halt epidemic outbreaks despite its theoretical appeal \cite{Bogu2013Nature}.

Moreover, rigorous epidemic models incorporating imperfect vaccination and quarantine measures on scale-free and other structured networks have further refined understanding of disease dynamics and control \cite{Chen2018Global}. These models integrate degree-dependent characteristics of vaccination efficacy and leverage compartmental representations to assess stability and threshold phenomena in realistic epidemiological settings.

This research aims to quantitatively determine the critical vaccination thresholds required to stop the spread of an epidemic characterized by \( R_0 = 4 \) on static, uncorrelated random networks with mean degree \( z = 3 \) and mean excess degree \( q = 4 \). Two vaccination strategies are considered analytically and via mechanistic simulations: (i) random vaccination of nodes, and (ii) targeted vaccination of nodes with degree exactly \( k = 10 \). Given the network structure and epidemiological parameters, this study seeks to answer the question: 

\textit{What proportions of the population must be vaccinated under random and targeted vaccination schemes to effectively prevent an epidemic outbreak, and how do simulation outcomes validate the analytical predictions?}

The investigation is framed within the SIR epidemic model, assuming sterilizing immunity conferred by vaccination and permanent immunity upon recovery. Using configuration model networks that are uncorrelated, locally tree-like, and exhibit prescribed degree distributions, analytic solutions based on percolation theory establish baseline expectations for critical vaccination fractions. Simultaneously, detailed stochastic simulations on tailored networks with specified degree distributions validate these theoretical thresholds and assess the practical efficacy of vaccination strategies.

This integrated analytical and simulation approach builds upon prior foundational work in epidemic thresholds on networks and advances understanding by illustrating the limitations and opportunities inherent in degree-targeted vaccination within different network topologies. The results have implications for public health policy design, especially in guiding vaccination prioritization and resource allocation in heterogeneous population contact structures.

\section{Background}

Epidemic control via vaccination has been extensively studied within the framework of network epidemiology, where contact structures and population heterogeneity critically influence intervention outcomes. While classical models often assume random vaccination, leading to the well-known critical threshold relation \(p_c = 1 - \frac{1}{R_0}\) for homogeneous or uncorrelated random networks \cite{Bogu2013Nature,WeiWang2015Predicting}, the efficacy of random strategies is challenged by the heterogeneous nature of real-world contact networks.

Several studies have highlighted the limitations of random vaccination particularly in networks exhibiting high-degree heterogeneity or complex connectivity patterns, such as scale-free or weighted social networks. Targeted vaccination strategies preferentially immunize nodes based on structural metrics—commonly node degree—as these high-degree individuals serve as transmission hubs disproportionately influencing outbreak dynamics \cite{Chen2018Global,Torres2021Nonbacktracking,AlvarezZuzek2017Epidemic}. Targeted approaches exploit network topology to fragment transmission pathways more efficiently than uniform strategies, often drastically reducing the needed vaccine coverage.

However, the success of targeted vaccination depends on the abundance and accessibility of high-degree nodes. For networks with sparse or Poisson-like degree distributions, high-degree nodes are exceedingly rare, limiting the practical impact of targeting these nodes alone \cite{Bogu2013Nature,Chen2018Global}. Empirical studies on finite-size networks reveal that degree-based targeting can fail when the fraction of hubs does not suffice to disrupt the giant component \cite{AlvarezZuzek2017Epidemic}. Moreover, other centrality measures, such as betweenness or weighted betweenness centrality, have been proposed to better identify influential spreaders in weighted or real social networks, often outperforming simple degree targeting but requiring more computational resources \cite{Nair2025}.

Recent advances integrate behavioral feedback and dynamic vaccination behaviors coupled with epidemic spread, revealing complex effects of perception and network structure on vaccination coverage and outbreak control \cite{Zhou2024}. Furthermore, models incorporating imperfect vaccination, reinfection possibilities, and probabilistic nodal frameworks extend classical SIR approaches towards more realistic epidemic scenarios \cite{Chen2021}. Such innovations address limitations of static, perfect-immunity models and underscore the nuanced interplay between epidemiological parameters, network topology, and intervention strategies.

The present work contributes to this body of literature by quantitatively analyzing critical vaccination thresholds within the context of uncorrelated random networks characterized by fixed mean and excess degrees and a given \(R_0\). Unlike prior studies focusing on heterogeneous or scale-free networks, this research compares random vaccination with a strictly degree-based targeted strategy vaccinating nodes with degree exactly 10, assessing the feasibility and impact of this approach in both Poisson and tailored trimodal degree distributions. By combining analytic percolation-based theory with mechanistic simulations, the study rigorously examines how degree distribution structural properties govern vaccination strategy effectiveness, elucidating conditions under which targeted vaccination can notably reduce vaccination burdens.

This work thus refines understanding of vaccination policy efficacy in realistic network contexts and bridges analytical epidemic threshold theory with simulation-based validation, offering pragmatically relevant insights for vaccination prioritization where contact heterogeneity and network structure can vary substantially.

\section{Methods}

\subsection{Epidemic Scenario and Network Model}
We consider an epidemic with a basic reproduction number \(R_0 = 4\) spreading on a static, uncorrelated random network characterized by a mean degree \(z=3\) and mean excess degree \(q=4\). The network is modeled as a configuration model with prescribed degree distributions, ensuring local tree-like structure and absence of degree correlations. Two different types of degree distributions were used:
\begin{itemize}
    \item \textbf{Poisson(3) Network}: Configured with \(N=10,000\) nodes, the degree distribution is Poisson with parameter \(\lambda=3\). This network has empirical degree statistics with mean degree \(\langle k \rangle \approx 2.995\) and mean squared degree \(\langle k^2 \rangle \approx 11.93\), producing mean excess degree \(q \approx 2.99\). Nodes with degree exactly 10 are extremely rare in this network (\(P(10) \approx 0.14\%\)).
    \item \textbf{Tailored Trimodal Network}: Constructed with degrees \(k \in \{2,3,10\}\) and associated probabilities designed to fulfill \(\langle k \rangle = 3\) and \(\langle k^2 \rangle\) adjusted so that \(q=4\). Specifically, proportions are \(p_2 \approx 0.7203\), \(p_3 \approx 0.1736\), and \(p_{10} \approx 0.1061\), ensuring that \(\sim 10.6\%\) of nodes have degree 10. The network size is \(N=10,000\).
\end{itemize}
Both networks are undirected and uncorrelated, and their structural diagnostics including assortativity and clustering coefficient confirmed their suitability as configuration models.

\subsection{Epidemic Model Description}
We employed a compartmental Susceptible-Infected-Removed (SIR) model on the static network. The compartments are:
\begin{itemize}
    \item \textbf{Susceptible (S)}: Nodes that can be infected.
    \item \textbf{Infected (I)}: Infectious nodes that can transmit the pathogen to susceptible neighbors.
    \item \textbf{Removed (R)}: Nodes that are either vaccinated or recovered; they do not participate in further disease dynamics.
\end{itemize}
Vaccination is modeled as a node removal event at time \(t=0\), where vaccinated individuals are immediately moved to the removed compartment, thereby conferring sterilizing immunity (no transmission nor reinfection). The disease transmission occurs exclusively along edges connecting susceptible and infectious nodes.

The SIR transitions are:
\begin{align*}
    S + I &\xrightarrow{\beta} 2I, \\
    I &\xrightarrow{\gamma} R,
\end{align*}
where \(\beta\) is the per-contact transmission rate and \(\gamma\) is the recovery rate. The per-contact transmissibility \(T\) is defined as the probability that transmission occurs across an edge before recovery, related via \(T = \beta / (\beta + \gamma)\) in continuous-time models. Here, parameters were chosen such that the network-level basic reproduction number fulfills \(R_0 = T \cdot q = 4\).

\subsection{Vaccination Strategies}
Two vaccination strategies were studied:
\begin{enumerate}
    \item \textbf{Random (Uniform) Vaccination}: A random fraction \(p\) of nodes is uniformly selected and vaccinated (moved to \(R\)) before the epidemic starts.
    \item \textbf{Targeted Vaccination of Nodes with Degree \(k=10\)}: All nodes with degree exactly 10 are vaccinated (moved to \(R\)) before the epidemic starts. This fraction depends on the degree distribution: only 0.14\% in the Poisson(3) network and about 10.6\% in the tailored network.
\end{enumerate}

\subsection{Analytical Derivations of Critical Vaccination Thresholds}

\paragraph{Random Vaccination}
The critical vaccination fraction \(p_c\) required to halt the epidemic is derived from percolation theory and network epidemic models. Pre-vaccination, the basic reproduction number is \(R_0 = T \cdot q\). After random vaccination removing a fraction \(p\) of nodes, the mean excess degree and the transmissibility are effectively reduced by factor \((1-p)\), yielding an effective reproduction number:
\[
    R_{0,\mathrm{eff}} = (1-p) R_0.
\]
An epidemic cannot occur if \(R_{0,\mathrm{eff}} < 1\). Solving for \(p\), the critical vaccination threshold is:
\[
    p_c = 1 - \frac{1}{R_0}.
\]
For the scenario with \(R_0=4\), this yields \(p_c = 0.75\), or 75\% random vaccination coverage needed.

\paragraph{Targeted Vaccination of Degree-\(10\) Nodes}
Targeted vaccination removes nodes with degree \(k=10\) with fraction \(f\) of such nodes vaccinated. Such removal reduces the effective branching factor by:
\[
    \Delta = f \cdot k(k-1) P(k) = f \cdot 90 \cdot P(10),
\]
where \(P(10)\) is the fraction of nodes with degree 10.

The post-vaccination effective reproduction number must satisfy:
\[
    T \cdot \frac{q - f \cdot 90 \cdot P(10)}{z} < 1.
\]
Using \(T = \frac{R_0 z}{q}\) and rearranging gives:
\[
    f > \frac{1}{30 \cdot P(10)}.
\]
Since \(f \leq 1\) is a fraction, the overall vaccination fraction needed is:
\[
    f_{\mathrm{total}} = f \cdot P(10) > \frac{1}{30} \approx 3.33\%.
\]

In the Poisson(3) network where \(P(10) \approx 0.0014\), vaccinating all degree-10 nodes (\(f=1\)) corresponds to \(f_{\mathrm{total}} \approx 0.14\%\), insufficient to reach the threshold; targeted vaccination is ineffective.

The tailored network with \(P(10) \approx 0.106\) allows for vaccination coverage above this threshold by vaccinating all degree-10 nodes.

\subsection{Simulation Framework}

\paragraph{Network Construction}
The two network types were independently constructed using configuration models with \(N=10,000\) nodes.
\begin{itemize}
    \item \textit{Poisson(3) Network}: Degree sequences were generated as independent Poisson random variables with mean 3.
    \item \textit{Tailored Network}: Degree sequences were drawn from the specified trimodal distribution with \(p_2\), \(p_3\), and \(p_{10}\) probabilities.
\end{itemize}
The networks were simplified to remove self-loops and multiple edges and stored as sparse adjacency matrices.

\paragraph{Initial Conditions and Vaccination}
Simulations start at time \(t=0\) with the network nodes assigned compartments:
\begin{itemize}
    \item \textit{Vaccinated nodes} (random or targeted) are set to \(R\).
    \item \textit{Susceptible nodes} are set to \(S\) except for a small seed of infected nodes (\(1\%\) of susceptible nodes) randomly selected and initialized in \(I\) compartment.
\end{itemize}

\paragraph{SIR Simulation}
A continuous-time Markov chain representation of the SIR dynamics was implemented using the FastGEMF simulation framework. Transmission and recovery rates were chosen consistent with \(T \approx 1\) to capture high epidemic potential. Specifically, parameters were set to \(T=1.0\), \(\beta=0.99\), and \(\gamma=1.0\) approximating the analytic \(R_0=4\).

For each scenario and vaccination fraction (random vaccination swept near the threshold \(p \in [0.7,0.8]\) with increments of \(0.02\)), 150 stochastic realizations were run to accumulate robust statistics.

\paragraph{Simulation Scenarios}
Four principal simulation scenarios were conducted:
\begin{enumerate}
    \item Random vaccination on the Poisson(3) network.
    \item Targeted vaccination (all degree-10 nodes) on the tailored network.
    \item Targeted vaccination on the Poisson(3) network.
    \item Random vaccination on the tailored network.
\end{enumerate}

These cover both analytic conditions and practical epidemiological contrasts.

\paragraph{Data Collection and Validation}
For each simulation, the following were recorded:
\begin{itemize}
    \item Time series of compartment sizes (\(S, I, R\)) per realization.
    \item Final attack size (total fraction infected/recovered excluding vaccinated nodes).
    \item Peak infection size and timing.
    \item Epidemic duration.
\end{itemize}
These were used to validate theoretical vaccination thresholds and observe epidemic collapse or persistence under each intervention.

\paragraph{Mathematical and Algorithmic Implementation}
The vaccination procedure implements site percolation as node removals in the initial graph. Random vaccination samples nodes uniformly at random for removal at \(t=0\). Targeted vaccination is implemented by removing all nodes matching degree-10 at \(t=0\). The simulations then evolve the SIR dynamics on the pruned network preserving edges among remaining susceptible and infected nodes.

Random seeds were used to select vaccinated and initially infected nodes within constraints to allow reproducibility and robust statistical averaging over runs.

\subsection{Computational Setup and Software}
All simulations utilized Python libraries including NetworkX for network construction, SciPy for sparse matrix representation, and the FastGEMF package for efficient epidemic simulation as a continuous-time Markov process. Visualization of degree distributions and epidemic results utilized Matplotlib. Data outputs were saved as CSV files with corresponding PNG plots generated for analysis.

\subsection{Summary}
The combined approach rigorously integrates theoretical epidemic threshold derivations on configuration model networks with comprehensive stochastic simulation validation. The methods facilitate the quantitative determination of critical vaccination fractions for both random and targeted vaccination strategies on networks differing in degree heterogeneity, illuminating the interplay between network structure, vaccination targeting, and epidemic control.

\begin{figure}[http]
    \centering
    \includegraphics[width=0.8\textwidth]{degree-distribution-poisson3-z3-q4.png}
    \caption{Degree distribution of the Poisson(3) network used for analytic and simulation reference for random vaccination strategies.}
    \label{fig:degree-dist-poisson}
\end{figure}

\begin{figure}[http]
    \centering
    \includegraphics[width=0.8\textwidth]{degree-dist-tailored-z3-q4.png}
    \caption{Degree distribution of the tailored (2/3/10) network with sufficient degree-10 nodes for targeted vaccination analysis.}
    \label{fig:degree-dist-tailored}
\end{figure}

\begin{table}[h!]
    \centering
    \caption{Summary of simulation scenarios with network types, vaccination strategies, and key parameters.}
    \label{tab:simulation-scenarios}
    \begin{tabular}{llcc}
        \toprule
        Scenario & Network & Vaccination Type & Key Parameters \\
        \midrule
        1 & Poisson(3) & Random vaccination & \(p\) sweep around 0.7--0.8 \\
        2 & Tailored Trimodal & Targeted vaccination & All \(k=10\) nodes vaccinated \\
        3 & Poisson(3) & Targeted vaccination & All \(k=10\) nodes vaccinated \\
        4 & Tailored Trimodal & Random vaccination & \(p\) sweep around 0.7--0.8 \\
        \bottomrule
    \end{tabular}
\end{table}

\section{Results}
\label{sec:results}

This section presents a comprehensive analysis of simulation and analytic results concerning the critical vaccination thresholds required to halt epidemic spread on uncorrelated random networks with mean degree \( z = 3 \), mean excess degree \( q = 4 \), and basic reproduction number \( R_0=4 \). Two vaccination strategies were studied: (i) random vaccination of nodes, and (ii) targeted vaccination of all nodes with degree exactly \( k=10 \). Simulations were carried out on two network models: a Poisson(3) configuration model and a tailored degree distribution network consisting of degrees 2, 3, and 10 with proportions adjusted to achieve the specified \( z \) and \( q \). The tailored network contained a substantial fraction (\(\sim 10.6\%\)) of degree-10 nodes enabling meaningful targeted vaccination, whereas such nodes were rare (\(\sim 0.14\%\)) in the Poisson network.

\subsection{Analytic Thresholds and Network Construction}

The analytic threshold for the critical vaccination fraction under random vaccination is given by the classical formula \( p_c = 1 - 1/R_0 = 0.75 \), indicating 75\% of the population must be vaccinated randomly to prevent an epidemic. For targeted vaccination of degree-10 nodes, the analytic result predicts that vaccinating a total fraction \( f_{\text{total}} > 3.33\% \) of the population suffices to halt transmission. However, in a Poisson(3) network, degree-10 nodes occur too infrequently (approx. 0.14\%), so fully vaccinating all such nodes vaccinates only about 0.081\%, far below the critical threshold, rendering targeted vaccination ineffective there. The tailored network, with over 10\% degree-10 nodes, is capable of achieving the threshold by vaccinating all degree-10 nodes.

These analytic insights were verified via extensive stochastic SIR simulations incorporating vaccination as initial node removals.

\subsection{Simulation Scenarios and Parameters}

Four principal scenarios were simulated with \( N=10{,}000 \) nodes each:
\begin{enumerate}
    \item \textbf{Random vaccination sweep on the Poisson(3) network} for vaccination fractions \( p \) between 0.70 and 0.80.
    \item \textbf{Targeted vaccination of all degree-10 nodes} on the tailored degree distribution network.
    \item \textbf{Targeted vaccination of all degree-10 nodes} on the Poisson(3) network.
    \item \textbf{Random vaccination sweep on the tailored network} for vaccination fractions \( p \) between 0.70 and 0.80.
\end{enumerate}

All scenarios seeded approximately 1\% of the unvaccinated population as initially infected and used a high per-contact transmission probability corresponding to \( R_0=4 \) in network terms. Each parameter set was run for 150 stochastic replicate simulations.

\subsection{Impact of Random Vaccination}

Figures~\ref{fig:results-summaryattackrandvac} summarize the epidemic final sizes as functions of the random vaccination fraction \( p \) for both network types. A critical threshold at \( p_c=0.75 \) is annotated.

\begin{figure}[http]
    \centering
    \includegraphics[width=0.85\textwidth]{results-summaryattackrandvac.png}
    \caption{Final epidemic size versus vaccination fraction for random vaccination on Poisson(3) and tailored networks. The vertical dashed line indicates the analytic threshold \( p_c=0.75 \).}
    \label{fig:results-summaryattackrandvac}
\end{figure}

For both networks, as \( p \) approaches and exceeds 0.75, the epidemic size exhibits a gradual decline rather than a sharp collapse. In the Poisson network, the final epidemic size decreased from roughly 70\% to 80\% infected at low vaccination levels down to below 30\% for high vaccination, but no abrupt epidemic extinction was observed. Corresponding epidemic curves (see Figure~\ref{fig:results-summaryepicurves}) indicate persistence of low-level outbreaks beyond the threshold. Similar trends were observed in the tailored network despite its increased heterogeneity.

\subsection{Effectiveness of Targeted Vaccination}

Targeting all degree-10 nodes in the tailored network yielded a dramatic epidemic collapse despite vaccinating only approximately 11\% of the population. This reduced the final epidemic size to approximately 12.6\% compared to over 70\% without vaccination (Figure~\ref{fig:results-summarytargeted}). Epidemic curves reveal a swift decline in infections following initial seeding.

\begin{figure}[http]
    \centering
    \includegraphics[width=0.85\textwidth]{results-summarytargeted.png}
    \caption{Comparison of final epidemic sizes under random and targeted vaccination strategies in Poisson(3) and tailored networks. Targeted vaccination collapses the epidemic only in the tailored network with sufficient degree-10 nodes.}
    \label{fig:results-summarytargeted}
\end{figure}

Conversely, targeted vaccination on the Poisson network, where degree-10 nodes are extremely rare, led to negligible epidemic mitigation with final sizes around 58\%, close to the no-vaccination baseline. Thus, targeted vaccination efficacy is highly dependent on network degree distribution.

\subsection{Epidemic Dynamics and Temporal Patterns}

Figure~\ref{fig:results-summaryepicurves} shows representative epidemic trajectories (removed fraction versus time) for these strategies, emphasizing the rapid extinction after targeted immunization in the tailored network, and prolonged epidemics under random vaccination near the threshold.

\begin{figure}[http]
    \centering
    \includegraphics[width=0.85\textwidth]{results-summaryepicurves.png}
    \caption{Panel plot of epidemic progression (removed fraction over time) illustrating contrasting dynamics under random and targeted vaccination on Poisson(3) and tailored networks.}
    \label{fig:results-summaryepicurves}
\end{figure}

\subsection{Quantitative Summary of Epidemic Metrics}

\begin{table}[h!]
    \centering
    \caption{Epidemic Metrics for Each Vaccination Scenario and Network}
    \label{tab:metrics-transposed}
    \begin{tabular}{lcccc}
        \toprule
        \textbf{Metric} & \textbf{Poisson(3) Random Vaccination} & \textbf{Poisson(3) Targeted} & \textbf{Tailored Random Vaccination} & \textbf{Tailored Targeted} \\
        \midrule
        Final Size (fraction infected) & 0.71--0.80 & 0.58 & 0.70--0.80 & 0.13 \\
        Peak Infection (max count) & 20--30 & 1170 & 20--30 & 93 \\
        Time to Peak Infection & 0.0--0.01 & 3.8 & 0.0--0.0065 & 0.30 \\
        Epidemic Duration (time units) & 7--12 & 20.6 & 7--12 & 12 \\
        Vaccination Fraction & 0.70--0.80 & 0.0014 & 0.70--0.80 & 0.11 \\
        Epidemic Collapse & No & No & No & Yes \\
        \bottomrule
    \end{tabular}
\end{table}

This table emphasizes the significant differences in epidemic outcomes between targeted and random vaccination, and the crucial role of network degree heterogeneity in enabling efficient control via vaccination targeting.

\subsection{Summary}

The results demonstrate that on the studied static, uncorrelated networks with \( R_0 = 4 \), random vaccination requires a critical coverage of approximately 75\% to gradually mitigate epidemic spread, consistent with theoretical expectations. However, the transition is gradual and does not manifest as an abrupt epidemic extinction in finite network simulations.

Targeted vaccination of degree-10 nodes is highly effective when these nodes compose a sufficiently large fraction of the population, as in the tailored network, collapsing the epidemic at much lower coverage (\(\sim 11\%\)). In contrast, when degree-10 nodes are scarce, as in the Poisson network, targeted vaccination fails to prevent widespread epidemic spread.

These findings underscore the importance of contact network structure and degree distribution in designing optimal vaccination strategies. Targeting high-degree nodes can yield dramatic epidemiological impact if network heterogeneity allows, whereas uniform random vaccination may require substantially higher coverage.

The simulation outputs, analytic calculations, and epidemic time series collectively provide robust and comprehensive validation of these conclusions.

\section{Discussion}

This study provides a comprehensive evaluation of vaccination strategies aimed at controlling epidemic spread on uncorrelated random networks with mean degree \(z=3\) and mean excess degree \(q=4\), in the context of a pathogen with basic reproduction number \(R_0=4\). Through a combined analytic and simulation framework leveraging configuration model networks, we elucidated the critical vaccination thresholds for 
(i) random vaccination and (ii) targeted vaccination of nodes with degree exactly \(k=10\), bringing to light nuanced distinctions driven by network degree heterogeneity.

\subsection{Random Vaccination Strategy}
The analytic framework rooted in percolation theory predicts a critical random vaccination coverage of \(p_c = 1 - 1/R_0 = 0.75\) (75\%) required to bring the effective reproduction number below unity and consequently halt large-scale epidemics. This threshold emerges from the classical relationship where random node removal reduces the effective branching factor multiplicatively, impacting connectivity homogeneously across the network. 

Simulations conducted on the Poisson(3) configuration network, which approximates a locally tree-like, uncorrelated structure with a mean degree near 3, revealed a gradual attenuation in the final epidemic size as the vaccination fraction \(p\) approached and exceeded 0.75, as illustrated in Figure \ref{fig:results-summaryattackrandvac}. However, contrary to the sharp theoretical extinction transition, the final epidemic size decreases smoothly rather than abruptly collapsing at the threshold. This discrepancy can be attributed to finite-size effects, stochastic fluctuations, and the soft boundaries that characterize realistic network structures compared to idealized infinite-population analytic assumptions.

Parallel simulations on a tailored degree-distribution network (degrees \(2,3,\) and \(10\)) reaffirmed the observation that random vaccination reduces epidemic size gradually and did not precipitate a sharp epidemic collapse at \(p_c=0.75\). This persistence of sizable outbreaks beyond the threshold is consistent with prior findings that stochastic network simulations can manifest broad epidemic size distributions at critical points, especially when degree heterogeneity exists. 

The robust concordance of analytic and simulation results reinforces that while random vaccination uniformly reduces epidemic risk, it demands very high coverage (three quarters of the population) to achieve effective herd immunity in such network topologies, underscoring the practical challenges of mass vaccination programs.

\subsection{Targeted Vaccination Strategy}
Targeted vaccination leverages network heterogeneity by immunizing high-degree nodes that disproportionately contribute to transmission chains. Analytically, the minimum overall fraction of nodes needing vaccination to halt an epidemic by exclusively targeting degree-10 nodes is approximately 3.33\% (Equation 1 in the analytical derivation). This reduction compared to random vaccination invokes the principle that removing high-degree nodes critically disrupts the network's giant component and epidemic pathways.

However, this strategy's efficacy crucially depends on the abundance of such high-degree nodes. The Poisson(3) network has an exceedingly low proportion of degree-10 nodes (\(\approx 0.14\%\)), rendering targeted vaccination of all degree-10 nodes ineffective at achieving the necessary coverage to curtail spread, as confirmed by simulations (Figure \ref{fig:results-summarytargeted}). The simulation outcomes reveal that despite full removal of all degree-10 nodes, epidemic propagation remains substantial, with peak infections and epidemic durations comparable to the non-targeted scenarios.

In stark contrast, the tailored network, designed to contain approximately 10.6\% of degree-10 nodes, demonstrated a near-complete epidemic collapse upon vaccinating all nodes in this class. The abrupt drop in epidemic size and shortened outbreak duration exemplify how concentrated immunization within a markedly heterogeneous network topology can produce substantial epidemic control with far fewer vaccinations, as visually evident in Figures \ref{fig:results-summarytargeted} and \ref{fig:results-summaryepicurves}.

These findings emphasize that network structural characteristics define the potential gains from targeted vaccination strategies. When high-degree nodes form a sizable subpopulation, selective immunization harnesses network topology to efficiently raise herd immunity, considerably sparing resources compared to random approaches. Conversely, in homogeneous or Poisson-like networks where hubs are scarce, the benefit of targeting these nodes is negligible.

\subsection{Comparative Insights and Practical Implications}
The stark contrast in outcomes from targeted vaccination between the Poisson and tailored networks underscores an essential epidemiological principle: the heterogeneity of contact patterns fundamentally shapes intervention success. In real-world populations, contact degree distributions often exhibit heavy-tailed heterogeneity rather than Poisson-like homogeneity, suggesting substantial opportunities for targeted vaccination to augment epidemic control.

Moreover, the simulations reveal that random vaccination thresholds, while theoretically sharp, manifest as gradual reductions in epidemic size in finite networks with stochastic transmission, demanding sustained high coverage to achieve control. Targeted vaccination, by disrupting critical transmission hubs, can trigger epidemic collapse with markedly lower coverage provided the network topology supports such hubs' prevalence.

From a policy perspective, these results advocate for detailed characterization of contact networks to tailor vaccination strategies effectively. For emerging pathogens or in resource-limited settings, opting for targeted vaccination schemes that focus on highly connected individuals could optimize immunization impact, contingent on the presence of sufficient targeting candidates. Conversely, in settings resembling Poisson distributions with few super-spreaders, mass random vaccination remains indispensable.

\subsection{Limitations and Future Directions}
While the work employs robust and well-accepted theoretical, mechanistic, and computational models, certain limitations merit consideration. The epidemic model assumes static, uncorrelated networks and sterilizing immunity from vaccination, neglecting temporal contact dynamics, partial vaccine efficacy, and demographic heterogeneities.

Future explorations should incorporate dynamic contact patterns, vaccination hesitancy, and partial immunity scenarios to better approximate complex real-world epidemics. Further, extending analyses across a spectrum of degree distributions and incorporating assortativity or clustering effects could nuance intervention assessments.

\subsection{Conclusion}
In summary, this study affirms that the critical fraction required for random vaccination to block epidemic spread on random graphs with \(R_0=4\) is approximately 75\%, corroborated by both analytic theory and simulation. Targeted vaccination strategies dramatically lower this threshold, but only when networks harbor a sufficient proportion of high-degree nodes. The simulation results highlight the necessity of network structure awareness to optimize vaccination strategies and provide a rigorous quantitative foundation for such public health decisions.

\begin{table}[h]
    \centering
    \caption{Summary of Epidemic Metrics across Simulation Scenarios}
    \label{tab:metrics-transposed}
    \begin{tabular}{lcccc}
        \toprule
        Metric & Poisson(3) Random & Poisson(3) Targeted (k=10) & Tailored Random & Tailored Targeted (k=10) \\
        \midrule
        Final Epidemic Size (R/N) & 0.706--0.803 & 0.580 & 0.704--0.802 & 0.126 \\
        Attack Rate (Aggregate) & 0.0057--0.0033 & N/A & 0.0044--0.0024 & \(\sim 0\) \\
        Peak Infections (Max I) & 20.02--30.03 & 1170.37 & 20.00--30.00 & 93.30 \\
        Peak Time & 0.000--0.013 & 3.803 & 0.000--0.0065 & 0.303 \\
        Epidemic Duration & 7.4--11.7 & 20.63 & 7.7--12.1 & 12.1 \\
        Fraction Vaccinated & 0.70--0.80 & 0.0014 & 0.70--0.80 & 0.11 \\
        Epidemic Collapse & No & No & No & Yes \\
        \bottomrule
    \end{tabular}
\end{table}

This table contextualizes the epidemic dynamics and overall vaccination coverage, synthesizing the contrasting impacts of vaccination strategy and network structure. Together with the referenced figures (\ref{fig:results-summaryattackrandvac}, \ref{fig:results-summarytargeted}, \ref{fig:results-summaryepicurves}), they offer a comprehensive portrayal of vaccination efficacy gradients.

The findings herein provide critical insights for epidemic modeling and public health strategy, reinforcing that vaccination policies should be informed by underlying contact network structures and adapted accordingly for maximal epidemiological benefit.

\section{Conclusion}

This study rigorously investigated the critical vaccination thresholds required to halt epidemic spread on uncorrelated random networks characterized by mean degree \( z = 3 \), mean excess degree \( q = 4 \), and a basic reproduction number \( R_0 = 4 \). Employing a combined analytical and stochastic simulation framework, we assessed two vaccination strategies: random (uniform) vaccination and targeted vaccination exclusively of nodes with degree exactly 10.

Analytically, the classical result for random vaccination was reaffirmed, establishing that immunizing at least 75\% of the population is necessary to reduce the effective reproduction number below unity and prevent large-scale outbreaks. This theoretical threshold was empirically corroborated by simulations on both Poisson(3) and tailored trimodal degree-distributed networks. However, simulation results revealed a gradual decline rather than an abrupt extinction of the epidemic size near this threshold, reflecting finite-size stochastic effects and network structure nuances absent from infinite-population idealizations.

Targeted vaccination analysis demonstrated a stark contrast contingent on network topology. For the Poisson(3) network, the extremely low prevalence of degree-10 nodes (\(\sim 0.14\%\)) rendered targeted removal practically ineffective; vaccinating all such nodes immunized only a negligible fraction (about 0.081\%) of the population, far below the analytically derived minimum of approximately 3.33\%. Simulations confirmed this ineffectiveness, showing persistent, substantial epidemics akin to the no-vaccination baseline.

Conversely, in a tailored network engineered to contain a substantial fraction (\(\sim 10.6\%\)) of degree-10 nodes, targeted vaccination induced a pronounced epidemic collapse. Vaccination coverage was dramatically reduced to roughly 11\%, producing a rapid decline in infections and drastically lower final epidemic size relative to random vaccination scenarios. These findings fully validate the percolation theory prediction that targeted vaccination can be highly efficient if sufficient high-degree nodes exist to disrupt transmission pathways.

The study highlights that vaccination efficacy is fundamentally determined by the underlying contact network's degree heterogeneity. While random vaccination requires extensive coverage to achieve herd immunity, targeted approaches can significantly reduce vaccination burden—but only when the network structure includes abundant high-degree hubs.

Limitations include the assumption of static uncorrelated networks and perfect sterilizing immunity; real-world contact patterns may involve temporal correlations, clustering, and partial vaccine efficacy, which could affect threshold values and effectiveness. Future work should extend these models to incorporate temporal dynamics, vaccination hesitancy, and degree correlation structures, and explore robustness across diverse degree distributions and epidemic parameter regimes.

In conclusion, this integrative analytical and simulation study substantiates that random vaccination demands high coverage (\(\sim 75\%\)) to halt epidemics on typical random networks with \(R_0=4\), while targeted vaccination drastically lowers this requirement only in networks with sufficient high-degree nodes. These insights inform optimized vaccination strategies that leverage contact heterogeneity to maximize epidemiological impact and resource efficiency in public health interventions.

\begin{thebibliography}{99}

\bibitem{Bogu2013Nature} M. Bogun\'a, C. Castellano, R. Pastor-Satorras, "Nature of the epidemic threshold for the susceptible-infected-susceptible dynamics in networks.", \textit{Physical Review Letters}, 2013.

\bibitem{EpidemicThreshold2012Chen2018} S. Chen, M. Small, X. Fu, "Global Stability of Epidemic Models With Imperfect Vaccination and Quarantine on Scale-Free Networks.", \textit{IEEE Transactions on Network Science and Engineering}, 2018.

\bibitem{WeiWang2015Predicting} W. Wang, W. Wang, Q.-H. Liu, et al., "Predicting the epidemic threshold of the susceptible-infected-recovered model.", \textit{Scientific Reports}, 2015.

\bibitem{Lindquist2011Effective} J. Lindquist, J. Ma, P. Driessche, et al., "Effective degree network disease models.", \textit{Journal of Mathematical Biology}, 2011.

\bibitem{Lee2012Epidemic} H. K. Lee, P. Shim, J. Noh, "Epidemic threshold of the susceptible-infected-susceptible model on complex networks.", \textit{Physical Review E}, 2012.

\bibitem{Torres2021Nonbacktracking} L. A. B. T\'orres, K. S. Chan, H. Tong, et al., "Nonbacktracking Eigenvalues under Node Removal: X-Centrality and Targeted Immunization.", \textit{SIAM Journal on Mathematics of Data Science}, 2021.

\bibitem{AlvarezZuzek2017Epidemic} L. G. Alvarez-Zuzek, C. E. La Rocca, J. Iglesias, et al., "Epidemic spreading in multiplex networks influenced by opinion exchanges on vaccination.", \textit{PLoS ONE}, 2017.

\bibitem{Bianconi2020MessagePassing} G. Bianconi, H. Sun, G. Rapisardi, et al., "A message-passing approach to epidemic tracing and mitigation with apps.", \textit{Physical Review Research}, 2020.

\bibitem{XinyuWang2025} X. Wang, Y. Fan, D. Cui, et al., "Emergent spatiotemporal heterogeneity in networked epidemics: Turing instability driven by topology and mobility.", \textit{Chaos}, 2025.

\bibitem{ref1} AuthorA, "TitleA", \textit{JournalA}, YearA.

\bibitem{Bogu2013NatureNewman} M. E. J. Newman, "The structure and function of complex networks," \textit{SIAM Review}, 45(2), 2003.

\bibitem{WeiWang2015PredictingPhysica} W. Wang, J. Xiao, "Predicting epidemic thresholds in complex networks," \textit{Physica A: Statistical Mechanics and its Applications}, vol. 420, 2015.

\bibitem{Chen2018Global} M. Chen et al., "Control of COVID-19 Pandemic: Vaccination Strategies Simulation under Probabilistic Node-Level Model," \textit{International Conference on the Software Process}, 2021.

\bibitem{Torres2021NonbacktrackingJournal} G. Torres et al., "Non-backtracking matrix application to epidemic thresholds," \textit{Journal of Complex Networks}, 2021.

\bibitem{AlvarezZuzek2017StatMech} L. Alvarez-Zuzek, A. Macri, L. Braunstein, "Epidemic spreading and vaccination in scale-free networks," \textit{Journal of Statistical Mechanics}, 2017.

\bibitem{Nair2025} N. S. Nair, J. M. John, D. S. Lekha, "An Analysis of Vaccination Strategies on Epidemic Spread in a Friendship Network," 2025 \textit{11th International Conference on Smart Computing and Communications}.

\bibitem{Zhou2024} L. Zhou et al., "Vaccination strategies in the disease--behavior evolution model," \textit{Frontiers of Physics}, 2024.

\bibitem{Chen2021} M. Chen, C. Kuo, W. K. V. Chan, "Control of COVID-19 Pandemic: Vaccination Strategies Simulation under Probabilistic Node-Level Model," \textit{International Conference on the Software Process}, 2021.
\end{thebibliography}
\newpage
\section*{Supplementary Material}
\begin{algorithm}[H]
\caption{Construction of Tailored Degree Network with Specified Moments}
\begin{algorithmic}[1]
\State \textbf{Input:} Target mean degree \(z=3\), mean excess degree \(q=4\), network size \(N=10000\)
\State Solve linear system for \(p_2, p_3, p_{10}\) given:
\Indent
\(2p_2 + 3p_3 + 10p_{10} = z\)
\State \(4p_2 + 9p_3 + 100p_{10} = qz + z\)
\State \(p_2 + p_3 + p_{10} = 1\)
\EndIndent
\State Compute counts: \(n_k = \text{round}(p_k \times N)\) for \(k \in \{2,3,10\}\)
\State Adjust \(n_{10}\) by \(n_{10} \leftarrow n_{10} + (N - \sum n_k)\) to ensure sum is exactly \(N\)
\State Construct degree sequence:
\Indent
\(\text{deg\_seq} \leftarrow\) concatenate \(n_2\) entries of 2, \(n_3\) entries of 3, \(n_{10}\) entries of 10
\EndIndent
\State Shuffle \(\text{deg\_seq}\) randomly
\If{\(\sum \text{deg\_seq}\) is odd}
  \State Increment last element by 1 to make even
\EndIf
\State Generate configuration model network \(G_{ta}\) with degree sequence \(\text{deg\_seq}\)
\State Simplify graph by removing self-loops and parallel edges
\end{algorithmic}
\end{algorithm}

\begin{algorithm}[H]
\caption{Simulation of SIR Epidemic on Network with Random Vaccination}
\begin{algorithmic}[1]
\State \textbf{Input:} Network adjacency \(G\), population size \(N\), vaccination fraction \(p_{vac}\), initial infected fraction \(p_{seed}\)
\State Compute number vaccinated \(n_{vac} = \lfloor p_{vac} \times N \rfloor\)
\State Compute remaining susceptible plus infected \(n_{left} = N - n_{vac}\)
\State Compute number initially infected \(n_{inf} = \max(1, \lfloor p_{seed} \times n_{left} \rfloor)\)
\State Initialize node states array as all susceptible (0)
\State Randomly select \(n_{vac}\) nodes to assign recovered (vaccinated) state (2)
\State Among non-vaccinated nodes, randomly select \(n_{inf}\) nodes to infect (state 1)
\State Remaining nodes remain susceptible (state 0)
\State Define SIR model schema and parameters \(\beta, \gamma\)
\State Initialize simulation with network \(G\) and initial node states
\State Run stochastic SIR simulation until stopping time or convergence
\State Extract mean compartmental trajectories and confidence intervals
\State Save results and plots
\end{algorithmic}
\end{algorithm}

\begin{algorithm}[H]
\caption{Simulation of SIR Epidemic on Network with Targeted Vaccination (All Degree-10 Nodes)}
\begin{algorithmic}[1]
\State \textbf{Input:} Network adjacency \(G\), population size \(N\), initial infected fraction \(p_{seed}\)
\State Compute node degrees \(\{k_i\}\)
\State Identify vaccinated nodes as all nodes with degree 10
\State Assign vaccinated nodes state (2), others susceptible (0)
\State Among susceptible nodes, randomly pick \(n_{inf} = \max(1, \lfloor p_{seed} \times \text{num\_susceptible} \rfloor)\) nodes to infect (1)
\State Define SIR model schema and parameters \(\beta, \gamma\)
\State Initialize simulation with network \(G\) and node states
\State Run stochastic simulation until stopping criteria met
\State Extract and save the epidemic trajectories and statistics
\end{algorithmic}
\end{algorithm}

\begin{algorithm}[H]
\caption{Epidemic Simulation to Validate Critical Vaccination Threshold}
\begin{algorithmic}[1]
\State \textbf{Input:} Number nodes \(n\), mean degree \(z=3\), vaccination fractions \(p\) to test
\For{each \(p\) in vaccination fractions}
  \State Generate Erd\H{o}s-R\'enyi network with parameters \(n\), connection probability \(z/(n-1)\)
  \State Remove fraction \(p\) of nodes (vaccinated) uniformly at random from network
  \If{network is empty}
    \State Record final epidemic size as 0
    \State Continue to next \(p\)
  \EndIf
  \State Infect a single randomly chosen susceptible node
  \State Simulate SIR epidemic on network with transmissibility \(T=1.0\), recovery probability \(r\)
  \State Record final epidemic size (cumulative recovered)
\EndFor
\State Plot final epidemic size vs vaccination fraction
\end{algorithmic}
\end{algorithm}

\begin{algorithm}[H]
\caption{Extraction of Key Epidemic Metrics from Simulation Time Series}
\begin{algorithmic}[1]
\State \textbf{Input:} Time series data \(\texttt{df}\) with columns: time, S, I, R and confidence intervals
\State Compute total population \(N = S + I + R\) at initial time
\State Compute final epidemic size \(R_{final}\) as last value of \(R\)
\State Compute peak infection as maximal \(I\) and corresponding time
\State Compute epidemic duration as time from first to last infection occurrence
\State Return metrics dictionary with these values
\end{algorithmic}
\end{algorithm}

\begin{algorithm}[H]
\caption{Plotting Summary of Final Epidemic Sizes vs Vaccination for Multiple Scenarios}
\begin{algorithmic}[1]
\State Load attack rate summary files for scenarios
\State For each scenario, plot final attack rate vs vaccination fraction
\State Overlay vertical line at predicted critical vaccination threshold \(p_c=0.75\)
\State Add markers for targeted vaccination results
\State Label axes and save figure
\end{algorithmic}
\end{algorithm}

\section*{Appendix: Additional Figures}
\addcontentsline{toc}{section}{Appendix: Additional Figures}

\begin{figure}[http]
    \centering
    \begin{subfigure}[b]{0.45\textwidth}
        \centering
        \includegraphics[width=\textwidth]{degree-dist-tailored-z3-q4.png}
        \caption*{degree-dist-tailored-z3-q4.png}
    \end{subfigure}
    \begin{subfigure}[b]{0.45\textwidth}
        \centering
        \includegraphics[width=\textwidth]{degree-distribution-poisson3-z3-q4.png}
        \caption*{degree-distribution-poisson3-z3-q4.png}
    \end{subfigure}
    \caption{Figures: degree-dist-tailored-z3-q4.png and degree-distribution-poisson3-z3-q4.png}
    \label{fig:degree-dist-tailored-z3-q4-png}
\end{figure}

\begin{figure}[http]
    \centering
    \begin{subfigure}[b]{0.45\textwidth}
        \centering
        \includegraphics[width=\textwidth]{plot-epidemic-simulation.png}
        \caption*{plot-epidemic-simulation.png}
    \end{subfigure}
    \begin{subfigure}[b]{0.45\textwidth}
        \centering
        \includegraphics[width=\textwidth]{results-11.png}
        \caption*{results-11.png}
    \end{subfigure}
    \caption{Figures: plot-epidemic-simulation.png and results-11.png}
    \label{fig:plot-epidemic-simulation-png}
\end{figure}

\begin{figure}[http]
    \centering
    \begin{subfigure}[b]{0.45\textwidth}
        \centering
        \includegraphics[width=\textwidth]{results-12.png}
        \caption*{results-12.png}
    \end{subfigure}
    \begin{subfigure}[b]{0.45\textwidth}
        \centering
        \includegraphics[width=\textwidth]{results-13.png}
        \caption*{results-13.png}
    \end{subfigure}
    \caption{Figures: results-12.png and results-13.png}
    \label{fig:results-12-png}
\end{figure}

\begin{figure}[http]
    \centering
    \begin{subfigure}[b]{0.45\textwidth}
        \centering
        \includegraphics[width=\textwidth]{results-14.png}
        \caption*{results-14.png}
    \end{subfigure}
    \begin{subfigure}[b]{0.45\textwidth}
        \centering
        \includegraphics[width=\textwidth]{results-15.png}
        \caption*{results-15.png}
    \end{subfigure}
    \caption{Figures: results-14.png and results-15.png}
    \label{fig:results-14-png}
\end{figure}

\begin{figure}[http]
    \centering
    \begin{subfigure}[b]{0.45\textwidth}
        \centering
        \includegraphics[width=\textwidth]{results-16.png}
        \caption*{results-16.png}
    \end{subfigure}
    \begin{subfigure}[b]{0.45\textwidth}
        \centering
        \includegraphics[width=\textwidth]{results-21.png}
        \caption*{results-21.png}
    \end{subfigure}
    \caption{Figures: results-16.png and results-21.png}
    \label{fig:results-16-png}
\end{figure}

\begin{figure}[http]
    \centering
    \begin{subfigure}[b]{0.45\textwidth}
        \centering
        \includegraphics[width=\textwidth]{results-31.png}
        \caption*{results-31.png}
    \end{subfigure}
    \begin{subfigure}[b]{0.45\textwidth}
        \centering
        \includegraphics[width=\textwidth]{results-41.png}
        \caption*{results-41.png}
    \end{subfigure}
    \caption{Figures: results-31.png and results-41.png}
    \label{fig:results-31-png}
\end{figure}

\begin{figure}[http]
    \centering
    \begin{subfigure}[b]{0.45\textwidth}
        \centering
        \includegraphics[width=\textwidth]{results-42.png}
        \caption*{results-42.png}
    \end{subfigure}
    \begin{subfigure}[b]{0.45\textwidth}
        \centering
        \includegraphics[width=\textwidth]{results-43.png}
        \caption*{results-43.png}
    \end{subfigure}
    \caption{Figures: results-42.png and results-43.png}
    \label{fig:results-42-png}
\end{figure}

\begin{figure}[http]
    \centering
    \begin{subfigure}[b]{0.45\textwidth}
        \centering
        \includegraphics[width=\textwidth]{results-44.png}
        \caption*{results-44.png}
    \end{subfigure}
    \begin{subfigure}[b]{0.45\textwidth}
        \centering
        \includegraphics[width=\textwidth]{results-45.png}
        \caption*{results-45.png}
    \end{subfigure}
    \caption{Figures: results-44.png and results-45.png}
    \label{fig:results-44-png}
\end{figure}

\begin{figure}[http]
    \centering
    \begin{subfigure}[b]{0.45\textwidth}
        \centering
        \includegraphics[width=\textwidth]{results-46.png}
        \caption*{results-46.png}
    \end{subfigure}
    \begin{subfigure}[b]{0.45\textwidth}
        \centering
        \includegraphics[width=\textwidth]{results-summaryattackrandvac.png}
        \caption*{results-summaryattackrandvac.png}
    \end{subfigure}
    \caption{Figures: results-46.png and results-summaryattackrandvac.png}
    \label{fig:results-46-png}
\end{figure}

\begin{figure}[http]
    \centering
    \begin{subfigure}[b]{0.45\textwidth}
        \centering
        \includegraphics[width=\textwidth]{results-summaryepicurves.png}
        \caption*{results-summaryepicurves.png}
    \end{subfigure}
    \begin{subfigure}[b]{0.45\textwidth}
        \centering
        \includegraphics[width=\textwidth]{results-summarytargeted.png}
        \caption*{results-summarytargeted.png}
    \end{subfigure}
    \caption{Figures: results-summaryepicurves.png and results-summarytargeted.png}
    \label{fig:results-summaryepicurves-png}
\end{figure}
\end{document}