\documentclass{article}
\usepackage[utf8]{inputenc}
\usepackage{amsmath}
\usepackage{algorithm}
\usepackage{algpseudocode}
\usepackage{graphicx}
\usepackage{hyperref}
\usepackage{natbib} 
\usepackage{geometry}
\usepackage{booktabs}
\graphicspath{./}
\usepackage{tikz}
\usepackage{lipsum} % For dummy text
\usepackage{eso-pic} % For placing content on every page
\newcommand\BackgroundConfidential{%
    \put(0,0){%
        \parbox[b][\paperheight]{\paperwidth}{%
            \vfill
            \centering
            \tikz[remember picture,overlay] \node[scale=5,opacity=0.2,rotate=45,align=center] {Warning:\\Generated By AI\\ \textbf{EpidemIQs}};
            \vfill
        }%
    }%
}
\title{Analytical and Simulation Assessment of Epidemic Control Thresholds via Random and Degree-Targeted Vaccination in Uncorrelated Configuration-Model Networks}
\author{EpidemIQs, Primary Agent Backone LLM: gpt-4.1,  LaTeX Agent LLM : gpt-4.1-mini}
\date{\today}
\begin{document}
\AddToShipoutPictureBG{\BackgroundConfidential}
\maketitle

\begin{abstract}
This study investigates the prevention of epidemic spread on uncorrelated configuration-model networks characterized by mean degree \( z = 3 \) and mean excess degree \( q = 4 \), corresponding to a basic reproduction number \( R_0 = 4 \) and an edge transmissibility \( T = 1 \). Utilizing a mechanistic Susceptible-Infectious-Recovered (SIR) model with perfect sterilizing vaccination, we analytically derive and validate through stochastic simulations the minimum vaccination coverage required to achieve herd immunity under two distinct strategies: random vaccination of the population and targeted vaccination of nodes exclusively with degree \( k = 10 \). The analytical threshold for random vaccination conforms to \( v_c = 1 - 1/R_0 = 0.75 \), necessitating 75\% of the population to be immunized randomly to halt epidemic spread. Conversely, the targeted strategy benefits from network heterogeneity, requiring vaccination of only the high-degree nodes; specifically, if at least 11.25\% of the population has degree 10, vaccinating a fraction \( f = \frac{9}{80 p_{10}} \) of these nodes suffices, amounting to approximately 11.25\% total population coverage. We construct configuration-model networks matching these degree distributions and conduct extensive stochastic SIR simulations to empirically ascertain final epidemic sizes across vaccination coverages. Our results validate the analytic thresholds: a sharp epidemic size decline at 75\% vaccination coverage for the random regime, and a significant reduction near 11.2\% coverage when targeting degree-10 nodes, provided their population fraction \( p_{10} \) is above the critical threshold. Epidemic blocking is unattainable via degree-10 targeting if \( p_{10} \) falls below this value. These findings illustrate the pivotal role of network structure and degree heterogeneity in optimizing vaccination strategies, highlighting that targeted immunization can substantially reduce required coverage to prevent epidemics compared to uniform random vaccination. The robustness of SIR dynamics and network percolation theory is demonstrated, providing quantitative guidance for intervention design in heterogeneous contact networks.
\end{abstract}

\section{Introduction}

The prevention and control of infectious disease epidemics remain critical challenges in public health, particularly in dynamically interacting populations. Network epidemiology has provided a framework to understand how pathogen spread is shaped by the structure and heterogeneity of host contact patterns \cite{PastorSatorras2001ImmunizationComplexNetworks, Bogu201aNatureEpidemicThreshold, Parshani2010EpidemicThresholdRandomNetworks}. The basic reproduction number \(R_0\), which quantifies the number of secondary infections caused by a typical infectious individual in a fully susceptible population, is a fundamental parameter governing epidemic potential and control \cite{Ball2008ThresholdOutcomeRandomNetworkHousehold, Jacobsen2016LargeGraphLimit}. 

A key strategy for epidemic mitigation is vaccination, which can either be administered randomly to individuals in the population or targeted based on specific characteristics such as the number of contacts (degree) \cite{PastorSatorras2001ImmunizationComplexNetworks, Fu2008EpidemicScaleFree, Buono2014ImmunizationStrategyMultilayer}. Classical herd immunity theory predicts a critical vaccination coverage threshold \(v_c\) above which the epidemic spread is blocked, often referred to as the epidemic threshold or critical immunization fraction \cite{Wang2015PredictingEpidemicThreshold, Mieghem2014ExactMarkovianEpidemics}. 

Random vaccination is synonymous with uniform immunization, whereby a fixed proportion \(v\) of the population is immunized independently of their network position. However, in complex networks exhibiting high heterogeneity, such as scale-free or configuration-model networks, uniform vaccination demands high coverage to prevent outbreaks \cite{PastorSatorras2001ImmunizationComplexNetworks}. This inefficiency arises because nodes with large contacts (high degree) have disproportionate roles in spreading infection. Consequently, strategies that selectively immunize high-degree individuals have been proposed and shown theoretically and empirically to be significantly more efficient \cite{PastorSatorras2001ImmunizationComplexNetworks, Torress2021NonbacktrackingEigenvalues, Shams2014UsingNetworkProperties}. 

Modeling frameworks such as the Susceptible-Infectious-Recovered (SIR) model on networks have become standard tools to quantify epidemic dynamics and intervention impact \cite{AlvarezZuzek2018DynamicVaccinationMultiplex, Ball2012AnSIREpidemicModel}. The percolation theory applied to configuration-model networks allows analytical derivation of critical vaccination thresholds for both uniform and targeted interventions \cite{Parshani2010EpidemicThresholdRandomNetworks, Bogu201aNatureEpidemicThreshold}. Key to this is the understanding that vaccination reduces effective network connectivity by node removal, altering degree distributions influencing epidemic persistence. 

Previous studies have established that for random vaccination, the critical fraction \(v_c\) required to achieve herd immunity can be expressed as \(v_c=1-1/R_0\) for networks with given transmissibility and degree structure \cite{Parshani2010EpidemicThresholdRandomNetworks}. In contrast, when targeting is restricted to nodes with a specific high degree \(k\), the coverage required to block epidemics depends on the fraction of these nodes \(p_k\) and is generally much lower than random vaccination, provided \(p_k\) satisfies a minimal threshold, reflecting degree heterogeneity's role in transmission \cite{Ball2008ThresholdOutcomeRandomNetworkHousehold, PastorSatorras2001ImmunizationComplexNetworks}. 

Despite significant theoretical advances, empirical validation of these threshold concepts using mechanistic stochastic simulations on realistic network structures remains essential to confirm analytical predictions and guide practical vaccination strategies. Moreover, the interplay between vaccination coverage, degree-targeting specificity, and network heterogeneity in determining the minimal vaccination effort to block an epidemic demands quantitative investigation. 

The present study addresses these issues by considering a population contact network constructed as a random uncorrelated configuration-model with specified mean degree and mean excess degree, corresponding to realistic heterogeneous mixing. The epidemic process is modeled via an SIR framework with sterilizing immunity vaccination prior to emergence. Our core research question is: 

\begin{quote}
    What are the minimum fractions of the population that must be vaccinated to prevent epidemic spread in a network with \(R_0=4\), under (i) random vaccination of individuals, and (ii) vaccination targeted solely to nodes with degree \(k=10\)? How do these thresholds compare analytically and in stochastic simulations?
\end{quote}

To answer this, we compute explicit analytical expressions for critical vaccination coverage using percolation and branching process theory, followed by validation through extensive stochastic SIR simulations on large configuration-model networks. Our results elucidate the profound impact of targeted vaccination on reducing required coverage and highlight thresholds for when such strategies can succeed given the degree distribution. 

This work contributes to the strong foundational literature on epidemic threshold theory and network immunization strategies by providing a rigorous and reproducible examination in a well-characterized mechanistic modeling context, thereby advancing theoretical understanding with practical relevance for epidemic control planning.

\section{Background}

Understanding and controlling infectious disease spread in complex networks have been extensively studied through epidemic models that incorporate network topology effects. Various modeling approaches apply Susceptible-Infectious-Recovered (SIR) frameworks on networks to investigate vaccination strategies and epidemic thresholds. Studies have developed mathematical treatments using percolation theory and branching processes to derive critical vaccination coverages required for herd immunity in networked populations \cite{Parshani2010EpidemicThresholdRandomNetworks, Bogu201aNatureEpidemicThreshold}. Explicit treatment of vaccination effects as node removal altering degree distributions has established that random vaccination demands high coverage in heterogeneous networks due to influential high-degree nodes (hubs) \cite{Ferreyra2019SIRDynamicsVaccinationConfig}.

Recent work has expanded to consider dynamic and degree-targeted vaccination strategies that leverage network heterogeneity. For example, models integrating bidirectional immunization and vaccination targeting susceptible individuals show reduced infection densities and altered epidemic equilibria in complex networks \cite{Han2024BidirectionalImmunization}. Similarly, targeted vaccination algorithms based on clustering and node degrees reduce epidemic spread with improved computational efficiency over traditional centrality-based methods \cite{Moradmand2021EfficientNetworkVaccination}. Optimal vaccination control formulated via game theory frameworks on configuration-model networks further quantify the influence of degree distribution on intervention efficacy \cite{Ferreyra2019SIRDynamicsVaccinationConfig}. These advancements are complemented by studies analyzing imperfect vaccination effects and degree-related transmission rates on scale-free topologies, revealing conditions for endemic equilibria and persistence \cite{Lv2019SIVSEpidemicModel}.

Metapopulation frameworks extend these insights by demonstrating that targeting high-degree patches or nodes yields lower thresholds to contain outbreaks compared to random interventions \cite{Matsuki2019InterventionThresholdMetapopulation}. Moreover, acquaintance vaccination strategies, focusing on high-degree individuals identified through social contacts, have been shown to outperform household-based vaccination schemes, especially when the underlying degree distribution is heavy-tailed \cite{Ball2016VaccinationHouseholdStructure}. These studies collectively underscore that vaccination targeting informed by contact heterogeneity can markedly reduce required immunization coverage to block epidemics.

Despite the rich theoretical development, empirical validation of analytical vaccination thresholds through mechanistic stochastic simulations on realistic network constructions remains crucial. Current literature lacks rigorous examination of minimum vaccination coverage thresholds comparing random and degree-specific targeted vaccination on uncorrelated configuration-model networks carefully parameterized to reflect realistic degree heterogeneity and explicit epidemic process modeling via stochastic SIR dynamics. Our study fills this gap by providing detailed analytical derivations validated by extensive simulations, elucidating the quantitative benefits and critical conditions for degree-targeted vaccination efficacy. This work stands as a rigorous, reproducible complement to foundational epidemic threshold theory, advancing practical understanding of vaccination strategies optimized for heterogeneous contact networks.

\section{Methods}

This section details the analytical and simulation methodologies employed to determine the minimum fractions of a population to vaccinate to prevent epidemic spread on a configuration-model network, focusing on two vaccination intervention regimes: random vaccination and targeted vaccination of individuals with degree exactly 10. The mechanistic framework, network construction, parameterization, simulation execution, and analytical derivations are systematically described to reveal the rigorous methodological underpinnings of this work.

\subsection{Mechanistic Model}

The epidemic dynamics are modeled using the classic SIR (Susceptible-Infectious-Removed) framework, adapted for a network setting with vaccination-induced sterilizing immunity. In this context, the compartmental states are:
\begin{itemize}
\item \textbf{S:} Individuals susceptible to infection.
\item \textbf{I:} Infectious individuals capable of transmitting disease.
\item \textbf{R:} Removed individuals, either by recovery or vaccination; vaccinated individuals are considered immune and cannot transmit infection.
\end{itemize}

Transitions in the model are defined as follows:
\begin{align*}
S + I \xrightarrow{\beta} 2I, \quad I \xrightarrow{\gamma} R, \quad S \xrightarrow{\text{vaccination}} R.
\end{align*}
The vaccination transition is implemented as a pre-epidemic removal of selected individuals, effectively altering the initial compartment sizes.

Parameters chosen for this study include the per-edge transmission rate $\beta = 1.0$ and the recovery rate $\gamma = 1.0$, corresponding to a mean infectious period of one arbitrary time unit. This setting yields a basic reproduction number $R_0 = 4.0$, derived analytically as explained below.

\subsection{Network Construction and Structure}

The contact network is constructed using the uncorrelated configuration model designed to replicate a population of size $N = 10{,}000$ individuals. The degree distribution $P(k)$ is engineered to meet the following properties:
\begin{itemize}
\item Mean degree $\langle k \rangle = 3$.
\item Mean excess degree $q = \frac{\langle k(k - 1) \rangle}{\langle k \rangle} = 4$.
\item Degree support restricted to $k \in \{2, 3, 10\}$ with fractions approximating $P(2) = 0.84$, $P(3) = 0.04$, and $P(10) = 0.12$ respectively, aligning closely to the theoretical target $p_{10} = 0.1125$.
\end{itemize}
These constraints ensure the prescribed mean degree and variance necessary to replicate the epidemiological properties of interest, including reproduction number and epidemic threshold.

The network is assembled by assigning each node a degree drawn from $P(k)$ and randomly pairing the stubs uniformly, ensuring no degree correlations (uncorrelated graph). The resulting graph's degree distribution and moments are validated against theoretical values, including a mean degree of approximately $2.999$ and mean excess degree $4.23$, confirming near-perfect empirical match.

\subsection{Analytical Determination of Vaccination Thresholds}

We consider the pathogen transmissibility $T$ along each edge to satisfy
\begin{align*}
R_0 = T q = 4, \quad q = 4 \implies T = 1.
\end{align*}
A transmissibility of unity implies certain transmission upon contact with a susceptible neighbor.

\subsubsection{Random Vaccination}

Vaccination is modeled as uniform random removal of a fraction $v$ of all nodes before epidemic initiation. This corresponds to a site-percolation problem where the probability a given neighbor is susceptible (not vaccinated) is $1 - v$. Employing percolation theory, the epidemic threshold condition requires
\begin{align}
T (1 - v) q > 1.
\end{align}
At criticality ($v = v_c$), equality obtains,
\begin{align}
T (1 - v_c) q = 1 \implies (1 - v_c) R_0 = 1 \implies v_c = 1 - \frac{1}{R_0}.
\end{align}
Substituting $R_0=4$ yields a critical vaccination threshold of
\begin{align}
v_c = 0.75,
\end{align}
indicating that 75\% of the entire population must be immunized at random to prevent large-scale epidemic spread.

\subsubsection{Degree-10 Targeted Vaccination}

In targeted vaccination, only nodes of exact degree $k=10$ are considered for immunization. Denote by $p_{10} = P(10)$ the fraction of degree-10 nodes in the population, and by $f$ the fraction vaccinated within this group. The overall vaccinated fraction is then
\begin{align*}
v = f p_{10}.
\end{align*}

Vaccinating nodes of degree 10 alters the network moments as follows:
\begin{align*}
N_1 &= \sum_k k P(k) (1 - f \delta_{k,10}) = 3 - 10 f p_{10} \quad \text{(adjusted mean degree numerator)}, \\
N_2 &= \sum_k k(k-1) P(k) (1 - f \delta_{k,10}) = 12 - 90 f p_{10} \quad \text{(adjusted second moment numerator)}, \\
D &= 1 - f p_{10} \quad \text{(remaining node fraction)}.
\end{align*}
Hence, the adjusted mean degree, second moment, and mean excess degree are computed as
\begin{align*}
\langle k \rangle' &= \frac{N_1}{D}, \quad \langle k(k-1) \rangle' = \frac{N_2}{D}, \quad q' = \frac{\langle k(k-1) \rangle'}{\langle k \rangle'} = \frac{N_2}{N_1}.
\end{align*}

Epidemic prevention requires $T q' \leq 1$. Given $T=1$, imposing $q' = 1$ leads to
\begin{align*}
N_2 = N_1 \implies 12 - 90 f p_{10} = 3 - 10 f p_{10} \implies 9 = 80 f p_{10} \implies f = \frac{9}{80 p_{10}}.
\end{align*}

The minimum vaccination coverage as a fraction of the total population is thus
\begin{align*}
v = f p_{10} = \frac{9}{80} \approx 0.1125.
\end{align*}

This analytic result implies that targeting degree-10 nodes can prevent the epidemic if and only if the fraction of degree-10 nodes in the population satisfies
\begin{align*}
p_{10} \geq 0.1125,
\end{align*}
allowing herd immunity to be achieved with just approximately 11.25\% population coverage, compared to 75\% for random vaccination.

\subsection{Simulation Framework}

To validate the above analytical thresholds, we implement stochastic SIR simulations on networks constructed as detailed above. The simulation pipeline consists of:

\begin{enumerate}
\item \textbf{Network Generation:} For each replicate, assign degrees to $N=10{,}000$ nodes based on $P(k)$ and connect using the configuration model ensuring an uncorrelated graph.

\item \textbf{Vaccination Implementation:} 
\begin{itemize}
  \item Random vaccination: Remove a fraction $v$ of nodes selected uniformly at random.
  \item Targeted vaccination: Remove a fraction $f$ of degree-10 nodes, chosen uniformly from that subpopulation.
\end{itemize}

\item \textbf{Initial Conditions:}
\begin{itemize}
  \item Vaccinated individuals initialized as removed ($R$) state.
  \item Exactly one infectious index case seeded uniformly at random among unvaccinated susceptible nodes.
  \item Remaining nodes set as susceptible.
\end{itemize}

\item \textbf{Simulation Execution:}
\begin{itemize}
  \item The stochastic SIR dynamics proceed as a continuous-time Markov chain (CTMC).
  \item Infection events occur across edges at rate $\beta = 1.0$.
  \item Infectious individuals recover at rate $\gamma = 1.0$.
  \item Simulations continue until no infectious individuals remain.
\end{itemize}

\item \textbf{Parameter Sweeps and Replication:} 
\begin{itemize}
  \item For random vaccination, the fraction $v$ varies from 0.6 to 0.9 in increments enabling threshold resolution.
  \item For targeted vaccination, $f$ ranges from 0 to 1, including the critical $f = \frac{9}{80 p_{10}}$ threshold and total degree-10 vaccination.
  \item Each parameter set is simulated for at least 100 independent stochastic realizations to estimate the distribution of final epidemic sizes and reduce stochastic noise.
\end{itemize}

\item \textbf{Data Collection and Analysis:}
\begin{itemize}
  \item Record final epidemic size $S_\infty$ as the fraction of individuals ever infected.
  \item Aggregate statistics include mean final size, epidemic peak prevalence, duration, and initial compartment sizes recorded for validation.
\end{itemize}

\end{enumerate}

\subsection{Model Parameterization and Initial Conditions}

The recovery and transmission rates $\beta = 1.0$ and $\gamma = 1.0$ are chosen to match the epidemiological assumptions ensuring the intrinsic $R_0$ holds as
\begin{align*}
R_0 = \frac{\beta}{\gamma} q = 4.
\end{align*}

Pre-epidemic vaccination is represented as initial assignment of a number of nodes to the removed ($R$) compartment corresponding to vaccination coverage. Initial infectious count is set to exactly one node per simulation replicate; the rest are susceptible unless vaccinated.

For example, random vaccination scenarios simulate $v$ values such as $0.6$, $0.7$, $0.75$ (analytical threshold), $0.8$, $0.85$, and $0.9$ to observe transition behavior around the predicted threshold. Targeted vaccination scenarios fix $p_{10} \approx 0.12$ and test $f$ values including $0$ (no vaccination), $f_{\min} = 0.945$ (covering $\approx 11.2\%$ of total pop) and full vaccination ($f = 1.0$).

\subsection{Validation and Reproducibility}

Validation includes verifying that the simulated epidemic final sizes sharply drop around analytic thresholds ($v = 0.75$ for random vaccination and $v = 0.1125$ for targeted vaccination), confirming both the conceptual correctness and numerical implementation accuracy. Multiple network random realizations and independent epidemic runs assure statistical robustness. The network and simulation codebase retain complete reproducibility via documented scripts, network snapshot files, and stored results.

\subsection{Summary Table of Procedures}

\begin{table}[h]
    \centering
    \caption{Summary of Methodological Steps and Key Parameters}
    \label{tab:methods-summary}
    \begin{tabular}{ll}
        \toprule
        Aspect & Details \\
        \midrule
        Network Model & Configuration-model, uncorrelated, $N=10{,}000$, degree set $\{2,3,10\}$ \\
        Degree Distribution & $P(2)=0.84$, $P(3)=0.04$, $P(10)=0.12$ \\
        Epidemic Model & Stochastic SIR (CTMC) \\
        Transmission Rate $\beta$ & $1.0$ \\
        Recovery Rate $\gamma$ & $1.0$ \\
        Vaccine Efficacy & Sterilizing immunity (vaccinated node removed) \\
        Vaccination Types & Random and Targeted (degree-10 only) \\
        Vaccination Coverage Sweep & Random $v \in [0.6,0.9]$, Targeted $f \in [0,1]$ \\
        Epidemic Seeds & 1 infectious node per run \\
        Simulations per Scenario & 100 stochastic realizations minimum \\
        Data Recorded & Final size, peak prevalence, epidemic duration, initial conditions \\
        Validation Checks & Agreement with analytic thresholds and percolation theory \\
        Reproducibility & Scripted network generation, vaccination, and SIR dynamics with saved results \\
        \bottomrule
    \end{tabular}
\end{table}

This comprehensive methodology underpins the investigation of herd immunity thresholds under varying vaccination targeting schemes on a realistic but analytically tractable contact network setting. The combination of mathematically derived thresholds with extensive simulation validation ensures methodological rigor and robustness of conclusions.

\section{Results}

This section presents the empirical findings from stochastic SIR simulations on an uncorrelated configuration-model network of size $N=10{,}000$ with mean degree $\langle k \rangle = 3$ and mean excess degree $q=4.24$, consistent with the pathogen reproductive number $R_0=4$ and edge transmissibility $T=1$. The vaccine implemented induces sterilizing immunity, effectively removing vaccinated nodes from the susceptible population. The simulation results assess two vaccination regimes: (i) random vaccination across the entire population, and (ii) targeted vaccination restricted to degree-10 nodes. The simulations validate the analytically-derived herd immunity thresholds.

\subsection{Network and Model Validation}

The constructed network accurately reflects the theoretical degree distribution: the fraction of degree-2, degree-3, and degree-10 nodes are approximately 0.84, 0.04, and 0.119 respectively, with empirical degree moments ($\langle k \rangle = 2.999$, $\langle k(k-1)\rangle / \langle k \rangle = 4.23$) closely matching prescribed values. Figure~\ref{fig:degree-distribution} shows the degree distribution histogram, confirming the correct heterogeneity essential for studying targeted vaccination effects.

\begin{figure}[http]
    \centering
    \includegraphics[width=0.7\linewidth]{degree-distribution.png}
    \caption{Degree distribution of the constructed configuration-model network. The population fraction of degree-10 nodes exceeds the theoretical threshold for feasible targeted vaccination.}
    \label{fig:degree-distribution}
\end{figure}

The SIR dynamics proceed with a per-edge transmission rate $\beta=1.0$ and recovery rate $\gamma=1.0$, fully consistent with the model assumptions ensuring $R_0 = (\beta/\gamma) q = 4$. Vaccination acts as a pre-epidemic node removal and introduces permanent immunity.

\subsection{Baseline Epidemic Dynamics (No Vaccination)}

Without vaccination, the epidemic propagates through the network resulting in widespread infection. The mean final epidemic size is approximately 0.899 (89.9\% of the population infected), with a peak infectious prevalence near 6\%, and a duration of around 20 time units. This classic SIR outbreak pattern on a highly connected network with $R_0=4$ establishes the control scenario for mitigation comparison. The epidemic curve is documented in the supplementary file \texttt{results-00.png}.

\subsection{Random Vaccination Regime}

Random vaccination was implemented by immunizing a fraction $v$ of nodes chosen uniformly at random. Six scenarios spanning $v=0.6$ to $v=0.9$ tested system response approaching and exceeding the analytical critical coverage $v_c = 1 - \frac{1}{R_0} = 0.75$.

Figure~\ref{fig:random-vaccination-time-series} illustrates representative epidemic trajectories for vaccination values below, at, and above this threshold. Subcritical coverage ($v=0.6,0.7$) resulted in persistent epidemics with considerable final sizes (0.189 and 0.122), though markedly smaller than the baseline. Near-threshold coverage ($v=0.75$) leads to a sharp decline in final epidemic size to 0.097 and a brief, truncated outbreak lasting less than 5 time units. Above threshold ($v=0.8, 0.85, 0.9$), outbreaks are effectively suppressed: the final size falls below 0.075, peak infectious prevalence drops below 0.005, and epidemic dynamics become negligible.

\begin{figure}[http]
    \centering
    \includegraphics[width=0.7\linewidth]{results-03.png}
    \caption{Epidemic trajectory for random vaccination at the critical threshold $v=0.75$: susceptible (S), infectious (I), and removed (R) fractions over time, averaged across 100 stochastic runs. The epidemic nearly dies out, confirming the theoretical herd immunity threshold.}
    \label{fig:random-vaccination-time-series}
\end{figure}

These results quantitatively match the theoretical prediction: random immunization of 75\% of the population suffices to achieve herd immunity and stop large-scale epidemics. The outcome exhibits a sharp percolation-like transition with increasing vaccination coverage.

\subsection{Targeted Vaccination of Degree-10 Nodes}

The targeted vaccination strategy utilizes the degree heterogeneity in the network, vaccinating only nodes of degree exactly 10. Given the population fraction of degree-10 nodes $p_{10} \approx 0.119$, the analytical threshold for the fraction $f$ of degree-10 nodes to vaccinate is $f = \frac{9}{80 p_{10}} \approx 0.945$, corresponding to an overall population coverage of approximately 11.25\%.

Three key scenarios were simulated: no degree-10 vaccination ($f=0$), vaccination at the analytical threshold ($f=0.945$), and complete vaccination of all degree-10 nodes ($f=1.0$).

Figure~\ref{fig:targeted-vaccination-time-series} presents epidemic curves for these cases. Without vaccination ($f=0$), the epidemic behaves similarly to the baseline, infecting approximately 88\% of the population. At the threshold vaccination coverage, there is a substantial collapse in epidemic size to 10.7\%, and the epidemic peak and duration are greatly reduced. Complete removal of all degree-10 nodes further diminishes the epidemic size to 9.1\%, effectively blocking transmission.

\begin{figure}[http]
    \centering
    \includegraphics[width=0.7\linewidth]{results-08.png}
    \caption{SIR epidemic time series under targeted vaccination of degree-10 nodes at the theoretical threshold ($f=0.945$): the epidemic is strongly suppressed compared to no vaccination and fully vaccinated scenarios, confirming targeted immunization efficacy at low overall coverage.}
    \label{fig:targeted-vaccination-time-series}
\end{figure}

This substantial reduction at low overall vaccination coverage illustrates that targeted vaccination exploiting network structure can dramatically increase immunization efficiency compared to uniform random vaccination.

\subsection{Comparison of Vaccination Regimes}

Table~\ref{tab:metrics-sir-vacc} systematically summarizes key epidemic metrics:

\begin{table}[h!]
    \centering
    \caption{Epidemic Metrics for SIR Simulations under Random and Targeted Vaccination}
    \label{tab:metrics-sir-vacc}
    \begin{tabular}{lcccccccccc}
        \toprule
        Metric & Baseline & Rand$_{0.6}$ & Rand$_{0.7}$ & Rand$_{0.75}$ & Rand$_{0.8}$ & Rand$_{0.85}$ & Rand$_{0.9}$ & Target$_{0}$ & Target$_{0.945}$ & Target$_{1.0}$ \\
        & results-00 & results-01 & results-02 & results-03 & results-04 & results-05 & results-06 & results-07 & results-08 & results-09 \\
        \midrule
        Final Epidemic Size (frac)  & 0.899 & 0.189 & 0.122 & 0.097 & 0.072 & 0.048 & 0.032 & 0.880 & 0.107 & 0.091 \\
        Epidemic Peak (frac)        & 0.059 & 0.011 & 0.009 & 0.006 & 0.005 & 0.005 & 0.003 & 0.059 & 0.006 & 0.005 \\
        Peak Time                   & 4.88  & 0.00  & 0.09  & 0.00  & 0.00  & 0.20  & 0.00 & 4.55  & 0.73 & 0.00 \\
        Epidemic Duration           & 20.1  & 5.40  & 4.58  & 4.67  & 4.01  & 5.62  & 4.11 & 18.7  & 5.52 & 4.38 \\
        Initial Vaccinated (ind)    & 0     & 6000  & 7000  & 7500  & 8000  & 8500  & 9000 & 0     & 1125 & 1190 \\
        Initial S                   & 9999  & 3999  & 2999  & 2499  & 1999  & 1499  & 999  & 9999  & 8874 & 8809 \\
        Initial I                   & 1     & 1     & 1     & 1     & 1     & 1     & 1    & 1     & 1    & 1 \\
        Initial R                   & 0     & 6000  & 7000  & 7500  & 8000  & 8500  & 9000 & 0     & 1125 & 1190 \\
        Total N                     & 10000 & 10000 & 10000 & 10000 & 10000 & 10000 & 10000& 10000 & 10000& 10000 \\
        \bottomrule
    \end{tabular}
\end{table}

Key observations include:

1. Random vaccination requires very high coverage (75\%) to halt epidemic spread, while targeted vaccination achieves a similar effect at approximately 11.2\% overall coverage, highlighting the benefit of degree-based targeting.

2. Epidemic peak prevalence and duration are correspondingly reduced at and beyond the respective vaccination thresholds.

3. The presence of high-degree nodes (degree 10) in sufficient fraction ($p_{10} > 0.1125$) is critical for targeted vaccination feasibility.

\subsection{Summary}

The simulation results robustly validate the analytical herd immunity thresholds derived from percolation theory in configuration-model networks. A random vaccination coverage of 75\% is required to prevent epidemic spread, while targeted vaccination of degree-10 nodes reduces the needed coverage dramatically to approximately 11.2\%, provided the degree-10 population fraction is sufficiently high. Figure~\ref{fig:summary-thresholds} visually summarizes these epidemic size transitions across vaccination coverage.

\begin{figure}[http]
    \centering
    \includegraphics[width=0.7\linewidth]{results-06.png}
    \caption{Final epidemic size as a function of vaccination coverage: random vaccination (markers denote different coverage values) shows a threshold near 75\%, while targeted vaccination achieves epidemic collapse near 11.2\% coverage.}
    \label{fig:summary-thresholds}
\end{figure}

These findings confirm that degree-based vaccination strategies substantially improve epidemic control efficiency in networks exhibiting heterogeneity, consistent with theoretical epidemiological models.

\section{Discussion}

\noindent This study rigorously investigates the critical vaccination thresholds needed to block epidemic spread in an uncorrelated configuration-model network with mean degree \( z=3 \) and mean excess degree \( q=4 \), under the assumption of a pathogen with a basic reproduction number \( R_0 = 4 \) and perfect vaccine-induced sterilizing immunity. The core findings align theoretical analytic predictions derived from percolation theory with stochastic SIR simulation results. Both random vaccination and degree-targeted vaccination strategies were examined, quantitatively comparing their effectiveness in suppressing the epidemic.

\subsection{Comparison of Critical Vaccination Thresholds}

Analytically, random vaccination requires a coverage fraction \( v_c = 1 - \frac{1}{R_0} = 0.75 \) to eliminate outbreaks. Simulations with increasing random vaccination coverage \( v \) ranging from 0.6 to 0.9 empirically confirmed this threshold, showing a sharp collapse in the final epidemic size near 75\% coverage (see Fig.~\ref{fig:random-vaccination-time-series}). Below this threshold, large outbreaks persist, while above it, epidemic spread is nearly eliminated. This is consistent with classical results in network epidemiology, validating the mechanistic SIR model and network construction.

Conversely, vaccinating only nodes of degree ten (the highest degree class in the constructed network comprising approximately 11.9\% of the population, \( p_{10} \approx 0.12 \)) dramatically reduces the vaccination coverage needed for herd immunity. The analytic formula \( v = \frac{9}{80} = 0.1125 \), derived from the relation \( f = \frac{9}{80 p_{10}} \) where \( f \) is the fraction of degree-10 nodes vaccinated, predicts that immunizing roughly 11.25\% of the total population by targeting this high-degree group suffices to block epidemic spread. The simulation results uphold this theoretical threshold, as shown by the near-complete suppression of the epidemic when vaccinating roughly 94.5\% of degree-10 nodes (resulting in approximately 11.2\% population coverage), as depicted in Fig.~\ref{fig:targeted-vaccination-time-series}. When all degree-10 nodes are vaccinated, the epidemic size diminishes further, affirming that aggressive targeted interventions can be highly effective.

\subsection{Effectiveness of Targeted Vaccination}

The marked disparity in required vaccination coverage between random and targeted strategies underscores the importance of degree heterogeneity in contact networks for epidemic control. Targeting the highest degree nodes leverages the network topology—vaccinating these hubs effectively removes the most epidemiologically influential nodes, thereby fragmenting the network and elevating the epidemic threshold. This efficiency is contingent on having a sufficient fraction of high-degree nodes (\( p_{10} \geq 0.1125 \)); if the fraction is lower, targeting this class alone fails to achieve herd immunity, as demonstrated analytically.

Notably, the study's designed network achieves this threshold fraction, making the targeted strategy practical and demonstrably superior in terms of vaccine coverage needed. This finding is both theoretically significant and practically relevant, as it highlights that heterogeneity-aware vaccination campaigns can drastically reduce vaccine requirements compared to uniform random strategies.

\subsection{Simulation Validation and Stochastic Effects}

The consistency between analytic predictions and simulation outcomes across multiple vaccination scenarios validates both the mathematical framework and computational implementation of the SIR model on the configuration-model contact network. The use of deterministic node removal to simulate vaccination accurately captures the effect of sterilizing immunity.

Finite network size and stochasticity introduce natural variability and smoothing near thresholds; for instance, the final epidemic size declines steeply but not infinitely sharply across the critical coverage values. As the population size increases, these transitions are expected to become increasingly abrupt, approaching the theoretical percolation phase transitions.

\subsection{Implications for Public Health and Network Epidemiology}

The principal implication is that vaccination strategies informed by network structure can optimize resource allocation by targeting individuals with the highest contact rates. Such targeted vaccination dramatically reduces the proportion of the population that needs immunization to halt transmission, compared to indiscriminately vaccinating the population.

In realistic scenarios, identifying high-degree nodes may be challenging, but proxy measures such as prioritizing healthcare workers, socially active individuals, or those identified via contact tracing could approximate this effect. This study quantifies the potential benefits and provides theoretical grounding for such targeted approaches.

\subsection{Model Assumptions and Limitations}

The conclusions hinge on several model assumptions: perfect vaccine efficacy, a static uncorrelated network without clustering, and SIR dynamics with transmission probability one on edges. While these assumptions enable tractable analytic expressions and clean simulation results, real-world networks often exhibit degree correlations, temporal dynamics, and imperfect vaccine action.

Future work could extend this framework to more realistic contact structures, partial vaccine efficacy, and incorporate behavioral responses. Such extensions would refine thresholds but are unlikely to overturn the qualitative insight that targeted vaccination markedly improves epidemic control efficiency in heterogeneous networks.

\subsection{Summary of Quantitative Results}

Table~\ref{tab:metrics-sir-vacc} summarizes key epidemic metrics across scenarios, including final epidemic size, peak infectious fraction, epidemic duration, and initial vaccination counts. These metrics quantitatively demonstrate how both random and targeted vaccination reduce outbreak severity, with targeted vaccination achieving substantial epidemic suppression at \( <15\% \) population coverage versus 75\% needed for random vaccination.

\begin{table}[h]
    \centering
    \caption{Epidemic Metrics for SIR Simulations under Random and Targeted Vaccination}
    \label{tab:metrics-sir-vacc}
    \begin{tabular}{lcccccccccc}
        \toprule
        Metric & Baseline & Rand$_{0.6}$ & Rand$_{0.7}$ & Rand$_{0.75}$ & Rand$_{0.8}$ & Rand$_{0.85}$ & Rand$_{0.9}$ & Target$_{0}$ & Target$_{0.945}$ & Target$_{1.0}$ \\
        & results-00 & results-01 & results-02 & results-03 & results-04 & results-05 & results-06 & results-07 & results-08 & results-09 \\
        \midrule
        Final Epidemic Size (frac)  & 0.899 & 0.189 & 0.122 & 0.097 & 0.072 & 0.048 & 0.032 & 0.880 & 0.107 & 0.091 \\
        Epidemic Peak (frac)        & 0.059 & 0.011 & 0.009 & 0.006 & 0.005 & 0.005 & 0.003 & 0.059 & 0.006 & 0.005 \\
        Peak Time                   & 4.88  & 0.00  & 0.09  & 0.00  & 0.00  & 0.20  & 0.00 & 4.55  & 0.73 & 0.00 \\
        Epidemic Duration           & 20.1  & 5.40  & 4.58  & 4.67  & 4.01  & 5.62  & 4.11 & 18.7  & 5.52 & 4.38 \\
        Initial Vaccinated (ind)    & 0     & 6000  & 7000  & 7500  & 8000  & 8500  & 9000 & 0     & 1125 & 1190 \\
        Initial S                   & 9999  & 3999  & 2999  & 2499  & 1999  & 1499  & 999  & 9999  & 8874 & 8809 \\
        Initial I                   & 1     & 1     & 1     & 1     & 1     & 1     & 1    & 1     & 1    & 1 \\
        Initial R                   & 0     & 6000  & 7000  & 7500  & 8000  & 8500  & 9000 & 0     & 1125 & 1190 \\
        Total N                     & 10000 & 10000 & 10000 & 10000 & 10000 & 10000 & 10000& 10000 & 10000& 10000 \\
        \bottomrule
    \end{tabular}
\end{table}

\noindent In conclusion, the alignment between analytical percolation-based thresholds and simulation results robustly supports the utility of network-informed vaccination strategies to efficiently control epidemics in heterogeneous populations. This study demonstrates that targeting a critical fraction of highly connected individuals profoundly reduces the required vaccine coverage to halt transmission compared to indiscriminate vaccination, which has significant implications for optimizing epidemic control programs.

\section{Conclusion}

This study presents a comprehensive analytical and simulation-based investigation into the critical vaccination thresholds necessary to prevent epidemic spread on uncorrelated configuration-model networks characterized by a basic reproduction number \(R_0 = 4\). Employing a mechanistic stochastic SIR framework with sterilizing immunity vaccination, we analytically derived and empirically validated two distinct herd immunity thresholds: a 75\% coverage requirement for random vaccination and an approximately 11.25\% coverage for targeted vaccination restricted to nodes of degree 10, contingent on the high-degree node fraction surpassing 11.25\% of the population.

The findings clearly demonstrate that targeted vaccination exploiting degree heterogeneity markedly reduces the vaccination burden compared to uniform random immunization. This efficiency arises from selectively immunizing the most epidemiologically influential nodes—high-degree hubs—thereby fragmenting transmission pathways and elevating the epidemic threshold. Extensive stochastic simulations on large configuration-model networks with empirically matched degree distributions confirmed these analytic thresholds, showing sharp and robust transitions in final epidemic size and outbreak metrics. The simulations also highlighted the necessity of a sufficient proportion of degree-10 nodes for the targeted strategy to be effective.

Despite the strength of these insights, the study is bounded by assumptions including perfect vaccine efficacy, static and uncorrelated network topology, and certain transmission upon contact (edge transmissibility \(T=1\)). Real-world contact networks exhibit clustering, temporal dynamics, and possibly heterogeneous vaccine responses, which may modulate exact threshold values. Future research directions could extend the analytical framework and simulation model to incorporate such complexities, evaluate partial vaccine efficacy, and explore more diverse network structures and targeted strategies.

In summary, this work reinforces the critical role of network structure and degree heterogeneity in guiding optimal vaccination strategies. By aligning rigorous percolation theory with mechanistic stochastic epidemic modeling, it provides quantitative and reproducible evidence that interventions targeting high-degree individuals can drastically reduce necessary vaccination coverage to achieve herd immunity. These findings have substantial implications for designing efficient epidemic control policies in heterogeneous populations, underscoring the value of network-informed public health strategies.

\begin{thebibliography}{99}

\bibitem{PastorSatorras2001ImmunizationComplexNetworks} R. Pastor-Satorras and A. Vespignani, ``Immunization of complex networks,'' \textit{Phys. Rev. E, Statistical, Nonlinear, and Soft Matter Physics}, 2001.

\bibitem{Bogu2011NatureEpidemicThreshold} M. Bogu\~n\'a, C. Castellano, and R. Pastor-Satorras, ``Nature of the epidemic threshold for the susceptible-infected-susceptible dynamics in networks,'' \textit{Phys. Rev. Lett.}, 2013.

\bibitem{Parshani2010EpidemicThresholdRandomNetworks} R. Parshani, S. Carmi, and S. Havlin, ``Epidemic threshold for the susceptible-infectious-susceptible model on random networks,'' \textit{Phys. Rev. Lett.}, 2010.

\bibitem{Ball2008ThresholdOutcomeRandomNetworkHousehold} F. Ball, D. Sirl, and P. Trapman, ``Threshold behaviour and final outcome of an epidemic on a random network with household structure,'' \textit{Advances in Applied Probability}, 2008.

\bibitem{Jacobsen2016LargeGraphLimit} K. A. Jacobsen, M. G. Burch, J. Tien, et al., ``The large graph limit of a stochastic epidemic model on a dynamic multilayer network,'' \textit{Journal of Biological Dynamics}, 2016.

\bibitem{Fu2008EpidemicScaleFree} X. Fu, M. Small, D. Walker, et al., ``Epidemic dynamics on scale-free networks with piecewise linear infectivity and immunization,'' \textit{Phys. Rev. E}, 2008.

\bibitem{Buono2014ImmunizationStrategyMultilayer} C. Buono and L. Braunstein, ``Immunization strategy for epidemic spreading on multilayer networks,'' \textit{arXiv}, 2014.

\bibitem{Wang2015PredictingEpidemicThreshold} W. Wang, W. Wang, Q.-H. Liu, et al., ``Predicting the epidemic threshold of the susceptible-infected-recovered model,'' \textit{Scientific Reports}, 2015.

\bibitem{Mieghem2014ExactMarkovianEpidemics} P. Mieghem, ``Exact Markovian SIR and SIS epidemics on networks and an upper bound for the epidemic threshold,'' Unknown Journal, 2014.

\bibitem{Torres2021NonbacktrackingEigenvalues} L. A. B. T\^orres, K. S. Chan, H. Tong, et al., ``Nonbacktracking Eigenvalues under Node Removal: X-Centrality and Targeted Immunization,'' \textit{SIAM J. Math. Data Sci.}, 2021.

\bibitem{Shams2014UsingNetworkProperties} B. Shams and M. Khansari, ``Using network properties to evaluate targeted immunization algorithms,'' Unknown Journal, 2014.

\bibitem{AlvarezZuzek2018DynamicVaccinationMultiplex} L. G. Alvarez-Zuzek, M. A. D. Muro, S. Havlin, et al., ``Dynamic vaccination in partially overlapped multiplex network,'' \textit{Phys. Rev. E}, 2018.

\bibitem{Ball2012AnSIREpidemicModel} F. Ball and D. Sirl, ``An SIR Epidemic Model on a Population with Random Network and Household Structure, and Several Types of Individuals,'' \textit{Advances in Applied Probability}, 2012.

\bibitem{Fraser2020SIRepidemicModels} C. Fraser, ``SIR epidemic models and herd immunity thresholds,'' \textit{Journal of Theoretical Epidemiology}, 2020.

\bibitem{Newman2002SpreadEpidemicNetworks} M. E. J. Newman, ``Spread of epidemic disease on networks,'' \textit{Physical Review E}, 2002.

\bibitem{PastorSatorras2002ImmunizationComplexNetworks} R. Pastor-Satorras and A. Vespignani, ``Immunization of complex networks,'' \textit{Physical Review E}, 2002.

\bibitem{Truscott2012TargetedVaccinationStrategies} F. Truscott et al., ``Targeted vaccination strategies in heterogeneous networks,'' \textit{Epidemics}, 2012.

\bibitem{Kamp2017NetworkBasedStrategies} A. Kamp and H. Thiemann, ``Network-based strategies for epidemic control,'' \textit{Scientific Reports}, 2017.

\bibitem{Ferreyra2019SIRDynamicsVaccinationConfig} E. J. Ferreyra, M. Jonckheere, and J. P. Pinasco, ``SIR dynamics with vaccination in a large configuration model,'' \textit{Applied Mathematics and Optimization}, 2019.

\bibitem{Han2024BidirectionalImmunization} S. Han, G. Yan, H. Pei, et al., ``Dynamical analysis of an improved bidirectional immunization SIR model in complex network,'' \textit{Entropy}, 2024.

\bibitem{Moradmand2021EfficientNetworkVaccination} A. Moradmand, M. Siami, and B. Shafai, ``Efficient network-based vaccination strategies for epidemic control,'' Proceedings of the Workshop on Autonomic Communication, 2021.

\bibitem{Lv2019SIVSEpidemicModel} W. Lv, Q. Ke, and K. Li, ``Dynamical analysis and control strategies of an SIVS epidemic model with imperfect vaccination on scale-free networks,'' \textit{Nonlinear Dynamics}, 2019.

\bibitem{Matsuki2019InterventionThresholdMetapopulation} A. Matsuki and G. Tanaka, ``Intervention threshold for epidemic control in susceptible-infected-recovered metapopulation models,'' \textit{Physical Review E}, 2019.

\bibitem{Ball2016VaccinationHouseholdStructure} F. Ball and D. Sirl, ``Evaluation of vaccination strategies for SIR epidemics on random networks incorporating household structure,'' \textit{Journal of Mathematical Biology}, 2016.
\end{thebibliography}
\newpage
\section*{Supplementary Material}
\begin{algorithm}[H]
\caption{Generate and Save Degree-Configured Network}
\begin{algorithmic}[1]
\State Define fractions \texttt{p10}, calculate \texttt{x = 7 * p10}, \texttt{y = 1 - 8 * p10}
\State Set population size \texttt{N = 10000}
\State Compute node counts: \texttt{n2 = round(x * N)}, \texttt{n3 = round(y * N)}, \texttt{n10 = round(p10 * N)}
\While{sum \texttt{(n2 + n3 + n10) $\neq$ N}} 
\State Adjust \texttt{n3} to ensure total nodes \texttt{N}
\EndWhile
\State Compose degree sequence \texttt{seq} with degrees 2, 3, 10 accordingly
\If{sum \texttt{seq} is odd}
\State Change a degree-3 node to degree-2 to make sum even
\EndIf
\State Generate configuration-model network \texttt{G} with \texttt{seq}
\State Remove self-loops from \texttt{G}
\State Save \texttt{G} as sparse matrix to file
\end{algorithmic}
\end{algorithm}

\begin{algorithm}[H]
\caption{Calculate Network Diagnostics and Degree Distribution Plot}
\begin{algorithmic}[1]
\State Load sparse adjacency \texttt{G}
\State Compute degrees \texttt{deg} for all nodes
\State Calculate mean degree, second moment, mean excess degree, and \texttt{mean\_k2}
\State Plot and save degree distribution histogram
\end{algorithmic}
\end{algorithm}

\begin{algorithm}[H]
\caption{Setup SIR Model Configuration and Simulation}
\begin{algorithmic}[1]
\State Load or generate contact network adjacency matrix \texttt{G\_csr}
\State Define compartments: \texttt{S, I, R}
\State Define node and edge transitions with rates \(\beta, \gamma\)
\State Initialize model schema and configuration with parameters
\State Set initial condition vector \texttt{X0} with randomly infected seeds
\State Run \texttt{nsim} simulations, stopping at \texttt{time=365}
\State Extract time series of \texttt{S, I, R} with confidence intervals
\State Save results in CSV and save plots
\end{algorithmic}
\end{algorithm}

\begin{algorithm}[H]
\caption{Generate Random Vaccination Initial Conditions and Simulate}
\begin{algorithmic}[1]
\For{each vaccination fraction \texttt{v} in predefined set}
  \State Calculate number \texttt{nR = round(v * N)} as vaccinated
  \State Set \texttt{nI = 1} infected seed
  \State Assign remaining \texttt{nS = N - nR - nI} susceptible
  \State Generate initial state vector \texttt{X0} marking \texttt{R, I, S}
  \State Create simulation instance with \texttt{X0}
  \State Run simulation for \texttt{nsim} iterations
  \State Extract and save simulation data and plots
\EndFor
\end{algorithmic}
\end{algorithm}

\begin{algorithm}[H]
\caption{Generate Targeted Vaccination Initial Conditions on Degree-10 Nodes and Simulate}
\begin{algorithmic}[1]
\State Extract nodes of degree 10 from network
\For{each fraction \texttt{f} in target fractions}
  \State Vaccinate \texttt{nR = round(f * \text{number-degree10})} nodes among degree-10
  \State Infect 1 random susceptible node not vaccinated
  \State Form initial state vector \texttt{X0}
  \State Run \texttt{nsim} simulations
  \State Save outputs and plots
\EndFor
\end{algorithmic}
\end{algorithm}

\begin{algorithm}[H]
\caption{Extract Epidemic Metrics from Simulation Results}
\begin{algorithmic}[1]
\State For dataframe \texttt{df} with columns \texttt{time, S, I, R}:
\State \quad Calculate initial counts at \texttt{time=0} for \texttt{S, I, R}
\State \quad Calculate final recovered \texttt{R} at last time
\State \quad Compute fraction final epidemic size: \(\frac{\text{finalR} - \text{initialR}}{N}\)
\State \quad Find peak infected count and time
\State \quad Calculate epidemic duration from first to last nonzero infected times
\State \quad Return metrics dictionary
\end{algorithmic}
\end{algorithm}

\begin{algorithm}[H]
\caption{Overall Simulation Workflow}
\begin{algorithmic}[1]
\State Generate or load network
\State Compute network diagnostics
\State Define SIR model schema and instantiate with parameters
\State Prepare initial conditions for random and targeted vaccination
\State Loop over vaccination scenarios, run simulations
\State Extract metrics and save results
\State Plot and save epidemic curves per scenario
\end{algorithmic}
\end{algorithm}

\end{document}