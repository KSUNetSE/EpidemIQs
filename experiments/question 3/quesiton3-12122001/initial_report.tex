\documentclass{article}
\usepackage[utf8]{inputenc}
\usepackage{amsmath}
\usepackage{algorithm}
\usepackage{algpseudocode}
\usepackage{graphicx}
\usepackage{hyperref}
\usepackage{natbib} 
\usepackage{geometry}
\usepackage{booktabs}
\graphicspath{./}
\usepackage{tikz}
\usepackage{lipsum} % For dummy text
\usepackage{eso-pic} % For placing content on every page
\newcommand\BackgroundConfidential{%
    \put(0,0){%
        \parbox[b][\paperheight]{\paperwidth}{%
            \vfill
            \centering
            \tikz[remember picture,overlay] \node[scale=5,opacity=0.2,rotate=45,align=center] {Warning:\\Generated By AI\\ \textbf{EpidemIQs}};
            \vfill
        }%
    }%
}
\title{Comparative Analysis of SIR Epidemic Dynamics on Activity-Driven Temporal Networks versus Time-Aggregated Static Erd\H{o}s-R\'enyi Networks}
\author{EpidemIQs, Primary Agent Backone LLM: gpt-4.1,  LaTeX Agent LLM : gpt-4.1-mini}
\date{\today}
\begin{document}
\AddToShipoutPictureBG{\BackgroundConfidential}
\maketitle

\begin{abstract}
This study investigates how the temporal structure of an activity-driven network influences the spread of an infectious disease modeled by a classical Susceptible-Infectious-Recovered (SIR) framework with a basic reproduction number \(R_0 = 3\). We constructed and compared epidemic dynamics on two network archetypes derived from a synthetic population of 1000 nodes: (1) an activity-driven temporal network (ADT), in which nodes activate daily with probability \(\alpha=0.1\) and form \(m=2\) transient contacts per activation, and (2) a time-aggregated static network (TA), formed by aggregating contacts over a typical infectious period of 5 days, resulting in an Erd\H{o}s-R\'enyi (ER) random graph with an average degree near 2. Transmission rates for each network were analytically calibrated to achieve the same \(R_0\), with recovery rate \(\gamma=0.2\) and infection rate \(\beta\) set accordingly: \(0.3\) per contact per day for the static network, and \(1.5\) per contact per day for the temporal network.

Through extensive stochastic SIR simulations—each initiated with five randomly infected individuals and repeated over 100 realizations—we quantified outbreak metrics including epidemic duration, peak infection size and timing, final attack rate, and early exponential growth (doubling time). Our results reveal that while both network structures produce similar peak infection magnitudes (\(\sim 7.7\%\) of the population), timing (around day 38), and final epidemic size (\(\sim 43\%\) attack rate), the temporal network manifests a notably slower initial growth phase with a doubling time approximately twice that of the static networks (13.8 vs.\ 6–6.6 days). This indicates that temporal rewiring of contacts induces a deceleration and broadening of the epidemic curve without substantially altering the outbreak's overall scale.

These findings highlight the critical role of temporal network dynamics in shaping epidemic spread, emphasizing that static aggregation of contacts may overestimate the speed of transmission in populations with rapidly evolving contact patterns. Our methodological framework, equilibrating \(R_0\) and infectious periods across network structures, establishes a mechanism-respecting basis for comparing epidemic processes and can inform accurate forecasting and intervention planning in dynamic social systems.
\end{abstract}

\section{Introduction}

The study of epidemic dynamics on networks is a central endeavor in understanding how infectious diseases propagate through populations. Traditional epidemic models often assume static contact structures, neglecting the inherent temporal nature of human interactions. However, recent advances emphasize the importance of temporal networks, where the connectivity patterns evolve over time, reshaping disease transmission pathways and outcomes. Activity-driven temporal networks (ADTs) have become a prominent framework to model such dynamics, characterizing individuals by probabilistic activation levels and ephemeral contact formation \cite{SuyuLiuAndreaBaronchelliPerra2012, HyewonKimMeesoonHaHawoongJeong2019}. 

This work focuses on comparing epidemic spreading in activity-driven temporal networks against their time-aggregated static counterparts, particularly synthesized as Erd\H{o}s-R\'enyi (ER) random graphs, to understand how temporal effects influence the classic susceptible-infectious-recovered (SIR) epidemic processes. It addresses key epidemiological metrics such as the epidemic threshold, speed of spread, peak infection magnitude and timing, and the final attack rate. 

Previous studies have highlighted the limitations of static network assumptions. For example, dynamics in time-varying metapopulation networks demonstrate that epidemic thresholds differ fundamentally from static analogues, with temporal variation generally slowing contagion and raising the mobility threshold by orders of magnitude \cite{SuyuLiuAndreaBaronchelliPerra2012}. Similarly, the impact of temporal connectivity patterns on the SIR epidemic process shows that while highly active nodes facilitate rapid early spread, they limit the ultimate outbreak size, revealing a trade-off between spreading speed and final epidemic scale \cite{HyewonKimMeesoonHaHawoongJeong2019}. Moreover, multilayer network representations incorporating static and temporal layers provide more realistic models for diseases such as Ebola virus, illustrating the importance of persistent contacts combined with temporal interactions for risk assessment \cite{MahbubulHRiadMSekamatteFelixOcom2019, MahbubulHRiadMSekamatteFelixOcom2019b}.

Despite this progress, there remains a need for mechanistically rigorous, simulation-validated comparative studies that calibrate epidemic parameters consistently across temporal and static network frameworks. This includes matching the basic reproduction number \(R_0\) through analytical mapping of transmission and recovery rates to network contact processes. Such calibration is critical to ensure fair comparisons of disease propagation, given that temporal and static networks inherently differ in contact rates and interaction patterns \cite{MatthieuNadiniARizzoMPorfiri2020}.

This work builds on the described epidemiological and network science foundations to formulate and execute simulation experiments comparing SIR epidemic dynamics on: 
1) An activity-driven temporal network with \(N=1000\) nodes, where each node activates with probability \(\alpha=0.1\) per day, forming \(m=2\) transient undirected contacts daily, 
2) A time-aggregated static network aggregated over a typical infectious period (\(T = 1/\gamma = 5\) days), constructed as an ER random graph with average degree approximately two, realizing the cumulative contact patterns of the temporal model.

The study applies matched epidemiological parameters calibrated for an equivalent \(R_0 = 3\) to each network type, following rigorous mathematical derivations linking transmission probability \(\beta\), recovery rate \(\gamma\), and network contact parameters. The initial infection condition seeds five randomly chosen infected individuals to prevent stochastic extinction and enable reliable epidemic growth observations. 

The comparative aims are to quantify and elucidate how the temporal rewiring and ephemeral contact nature of the activity-driven network alter epidemic behavior relative to the static, time-aggregated network. Key aspects include the epidemic threshold, velocity of epidemic expansion, timing and height of the infection peak, and the total attack rate. 

Through this study, we aim to provide mechanistic insights into the influence of temporal structure on disease spread, validate analytical parameter mapping via simulations, and highlight the implications of temporal network modeling for epidemic forecasting and intervention strategies. 

Our formulation integrates current best practices established in the literature, justifying choices of epidemiological parameters and network constructions, ensuring reproducibility and interpretability within standard compartmental epidemic modeling paradigms \cite{SuyuLiuAndreaBaronchelliPerra2012, HyewonKimMeesoonHaHawoongJeong2019, MahbubulHRiadMSekamatteFelixOcom2019, MatthieuNadiniARizzoMPorfiri2020}.

In summary, this research addresses the fundamental question: 

\textit{In an activity-driven temporal network with ephemeral contacts and specified parameters, how does the temporal structure influence the epidemiological dynamics of an SIR process compared to its time-aggregated static network analogue, given matched basic reproduction number and epidemiological parameters?}

This question is critical to advancing the understanding of temporal network epidemiology and informing model selection and public health decision-making when assessing infectious disease spread in realistic population structures.

\section{Background}

The role of temporal network structures in modeling the spread of infectious diseases has garnered significant interest in recent years, complementing traditional static network approaches. Temporal networks explicitly consider the dynamic evolution of contacts over time, which fundamentally alters transmission pathways compared to static networks. Activity-driven temporal (ADT) networks have emerged as influential models capturing individual activation dynamics and ephemeral interactions \cite{SuyuLiuAndreaBaronchelliPerra2012, HyewonKimMeesoonHaHawoongJeong2019}. In these models, nodes activate probabilistically to form transient contacts, reflecting realistic patterns of social behavior and contact reconfiguration.

Existing literature demonstrates that temporal heterogeneity can slow epidemic spread relative to static counterparts by disrupting persistent transmission chains and limiting continuous exposure \cite{HyewonKimMeesoonHaHawoongJeong2019}. Moreover, studies utilizing multilayer temporal networks have illustrated how persistent and temporal contact layers synergize to influence epidemic risk and intervention efficacy, notably for viruses like Ebola \cite{MahbubulHRiadMSekamatteFelixOcom2019, MahbubulHRiadMSekamatteFelixOcom2019b}. However, many investigations have primarily focused on qualitative insights or have not rigorously matched epidemiological parameters such as the basic reproduction number ($R_0$) across temporal and static network frameworks.

Accurately calibrating epidemic parameters consistent with network contact processes is crucial to enable fair and mechanistically sound comparisons. The mapping of transmission and recovery rates to empirical contact patterns ensures that differences in epidemic outcomes stem from structural dynamics rather than parameter mis-specification \cite{MatthieuNadiniARizzoMPorfiri2020}. Nonetheless, research remains sparse on simulation-validated comparative analyses that combine rigorous parameter calibration with classical epidemic models like the Susceptible-Infectious-Recovered (SIR) model applied to ADT temporal and time-aggregated static Erd\H{o}s-R\'enyi (ER) network analogues.

Furthermore, while real-world temporal networks often display heterogeneities such as burstiness, community structures, and correlated contact patterns, simplified ADT and ER frameworks provide valuable mechanistic baselines to isolate and understand temporal effects on epidemic processes. Recent advances also combine data-driven and physics-based modeling techniques, including spatiotemporal graph neural networks, to forecast epidemics incorporating temporal and spatial network complexities \cite{HanStelzSokolowski2025}. These approaches underline the ongoing evolution in temporal network epidemiology toward integrating analytical, computational, and data-centric perspectives.

This study addresses the gap for a mechanistically rigorous, empirically grounded comparison by implementing analytically calibrated SIR epidemic simulations on activity-driven temporal and time-aggregated static ER networks. It quantifies differences in epidemic speed, peak characteristics, and final sizes under matched epidemiological parameters, contributing to the nuanced understanding of temporal effects on disease transmission dynamics without overstating novelty beyond established theoretical and computational insights.

\section{Methods}
\label{sec:methods}

In this study, we investigate the influence of temporal network structure on the spread of infectious diseases by comparing epidemic dynamics simulated on two network archetypes: an activity-driven temporal network (ADT) and a time-aggregated static Erd\H{o}s-R\'enyi (ER)-like network constructed through aggregation of contacts over a typical infectious period. Both networks represent interactions among a synthetic population of \(N=1000\) individuals. The infectious process is modeled using the classical susceptible-infectious-recovered (SIR) framework with parameters calibrated to achieve a basic reproduction number \(R_0 = 3\) in each network type. Our objective is to ensure a mechanistically consistent comparison by carefully calibrating transmission parameters and initial conditions across the differing network representations.

\subsection{Network Model Construction}

\subsubsection{Activity-Driven Temporal Network}
The temporal network is generated using the well-established activity-driven modeling paradigm. In this framework, each node activates independently at each discrete time step (day) with a fixed probability \(\alpha=0.1\). Upon activation, the node initiates \(m=2\) undirected transient connections, randomly forming edges with other nodes. Contacts are ephemeral and reshuffled at every time step, rendering the network a rapidly evolving temporal structure.

Mathematically, at each time step \(t\), the instantaneous contact graph \(G_t=(V,E_t)\) is sampled such that each node \(i \in V\) is active with probability \(\alpha\), and if active, connects to \(m\) distinct nodes chosen uniformly from \(V \setminus \{i\}\), generating undirected edges. The average instantaneous degree is thus approximately \(\langle k \rangle = 2 m \alpha = 0.4\) per node per time step (since each active node contributes \(m\) edges and each edge is counted twice), consistent with theoretical models.

\subsubsection{Time-Aggregated Static Network}
The static network counterpart is constructed by aggregating contacts over the infectious period \(T = 1/\gamma = 5\) days. In this aggregation process, undirected edges are drawn between any pair of nodes that have at least one contact during the full interval. This constructs a static Erd\H{o}s-R\'enyi-like network with linking probability \(\pi \approx 0.002\) and an average degree \(\langle k \rangle \approx 2\). To verify the statistical properties of the network, we compute its degree distribution and moments, observing a Poisson-like shape, validating the ER assumption.

Edges in the aggregated network can encode weights representing the frequency of contacts over the infectious period. However, for the classical SIR model implemented here, we utilize the unweighted adjacency, reflecting binary presence or absence of interaction.

\subsection{Mathematical Calibration of Transmission Parameters}

To maintain epidemiological consistency between the temporal and static network representations, transmission parameters --- specifically the transmission rate \(\beta\) and the recovery rate \(\gamma\) --- are carefully calibrated to achieve the target \(R_0 = 3\).

\subsubsection{SIR Model Dynamics}

The classical continuous-time SIR model defines transitions between three compartments: susceptible \(S\), infectious \(I\), and recovered \(R\). Transitions are as follows:
\begin{align*}
S + I &\xrightarrow{\beta} 2I, \\
I &\xrightarrow{\gamma} R.
\end{align*}

Here, \(\beta\) denotes the per-contact transmission rate, while \(\gamma\) corresponds to the recovery rate, with an average infectious period \(T = 1/\gamma = 5\) days.

\subsubsection{Parameter Mapping in the Activity-Driven Temporal Network}

Following the analytical framework in the literature \cite{Perra2012}, the basic reproduction number \(R_0\) on an activity-driven temporal network with parameters \(\alpha, m, \beta, \gamma\) is given by
\begin{equation}
R_0^{\mathrm{temp}} = \frac{2 m \alpha \beta}{\gamma}.
\label{eq:R0-temp}
\end{equation}

This expression accounts for the infection opportunities generated both when the infectious node is active and when the susceptible node is active, reflecting symmetrical contact generation. To calibrate \(\beta\) for a target \(R_0=3\), we invert Eq.~\eqref{eq:R0-temp}:
\[
\beta = \frac{R_0 \gamma}{2 m \alpha}.
\]

Substituting \(\gamma=0.2, m=2, \alpha=0.1\) yields \(\beta = 1.5\) as the per-contact transmission rate for the temporal network.

\subsubsection{Parameter Mapping in the Time-Aggregated Static Network}

The static aggregated network, representing integrated contacts over the infectious period \(T = 1/\gamma\), requires a different calibration due to contact aggregation. The effective transmission probability along an edge \(T_{\text{edge}}\) summarizing multiple contact events is
\begin{equation}
T_{\text{edge}} = 1 - e^{-\beta w},
\label{eq:edge-transmissibility}
\end{equation}
where \(w\) denotes the expected number of contact events between two nodes during \(T\). For the activity-driven aggregation,
\[
w = 2 m \alpha T = \frac{2 m \alpha}{\gamma}.
\]

The reproduction number for the aggregated network is approximated by the product of the average degree \(\langle k \rangle\) and the transmissibility \(T_{\text{edge}}\):
\begin{equation}
R_0^{\mathrm{static}} = \langle k \rangle \times T_{\text{edge}} = \frac{2 m \alpha}{\gamma} \left(1 - e^{-\beta \frac{2 m \alpha}{\gamma}} \right).
\label{eq:R0-static}
\end{equation}

Solving for \(\beta\) given \(R_0^{\mathrm{static}}=3\) requires inversion of Eq.~\eqref{eq:R0-static}, resulting in
\[
\beta = - \frac{\gamma}{2 m \alpha} \ln \left(1 - \frac{R_0^{\mathrm{static}} \gamma}{2 m \alpha} \right).
\]

Using \(\gamma=0.2, m=2, \alpha=0.1, R_0=3\) gives \(\beta \approx 0.27465\) for the static network.

\subsection{Epidemic Simulation Protocol}

We implement continuous-time Markov chain (CTMC) SIR simulations separately on each network structure, employing distinct methods suited to the network's temporal nature.

\subsubsection{Temporal Network Simulation}

Due to the dynamic nature of the temporal network, a custom event-driven simulation is implemented. At each discrete day:
\begin{itemize}
  \item The activity-driven network is instantiated by sampling active nodes and forming \(m\) random contacts per active node.
  \item Infection is transmitted from infectious to susceptible nodes along edges present during that time step with probability derived from \(\beta\).
  \item Infectious nodes recover independently with rate \(\gamma\).
  \item The network at the next time step is re-sampled, reflecting the ephemeral contact structure.
\end{itemize}

Simulations begin with \(5\) infected nodes selected uniformly at random to reduce extinction probability and proceed until no infectious individuals remain or the maximum simulation time (\(\geq 50\) days, at least ten times the infectious period) is reached. Repetition over \(100\) stochastic realizations enables robust estimation of epidemic metrics and uncertainty.

\subsubsection{Static Network Simulation}

Simulations on static networks leverage the FastGEMF framework, which efficiently implements CTMC SIR dynamics on fixed adjacency matrices. Simulation steps include:
\begin{itemize}
  \item Loading the static adjacency matrix from the aggregated or synthetic ER networks.
  \item Initializing \(5\) infected nodes chosen at random, with the remainder susceptible.
  \item Running stochastic SIR dynamics with per-edge transmission rates set by the calibrated \(\beta\) and recovery rate \(\gamma\).
  \item Progression continues until no infected remain or simulation duration exceeds 50 days.
  \item Multiple independent runs (\(100\)) are performed to compute averages and confidence intervals.
\end{itemize}

\subsection{Outcome Measures}

Key epidemic characteristics are extracted to quantify differences due to network topology:
\begin{enumerate}[label=\arabic*.,leftmargin=*]
  \item \textbf{Epidemic Duration}: Total days until no infectious individuals remain.
  \item \textbf{Peak Prevalence}: Maximum number of simultaneously infected individuals, and the time at which this peak occurs.
  \item \textbf{Final Attack Rate}: Cumulative fraction of nodes recovered (ever infected) by epidemic end.
  \item \textbf{Doubling Time}: Estimated from the early exponential phase of the epidemic using regression on log-transformed infection counts.
\end{enumerate}

These metrics are computed as mean values over 100 stochastic simulations along with associated 90\% confidence intervals, enabling rigorous comparison between temporal and static models.

\subsection{Computational Implementation and Data Management}

All network data and simulation outputs are stored using clear naming conventions and within structured directories to ensure reproducibility and easy access. Temporal network edge lists and static network adjacency matrices are saved in CSV and NPZ formats respectively, adhering to hyphen-separated file naming to avoid errors. Simulation results, including time series for compartment counts and derived metrics, are saved in CSV and PNG files under an output directory for analysis and visualization.

Careful attention is paid to matching initial conditions, random seeding, and parameter settings across the two network models, thereby isolating the effects of temporal structure on epidemic dynamics.

\subsection{Reasoning and Justifications}

Our methodological design is guided by theoretical epidemiological literature \cite{Perra2012} and rigorous mathematical mapping of model parameters to ensure comparability. 

- The selection of \(\gamma=0.2\) (5-day infectious period) balances model fidelity and computational tractability. 
- Multiple initial infections (five seeds) prevent early stochastic extinction, ensuring robust early outbreak dynamics.
- The aggregate network's construction and parameter calibration respect mechanistic interpretations of contacts and transmissibility, ensuring that observed differences arise from temporal structure rather than parameter inconsistencies.
- Simulation durations exceed ten times the infectious period to reliably capture peak and final outcome metrics.
- Using distinct simulation frameworks (custom event-driven for temporal, FastGEMF for static) reflects computational efficiency while preserving correctness.

This comprehensive methodological framework facilitates reproducible research and enables principled, quantitative dissection of temporal effects on epidemics in networked populations.

\section{Results}

In this study, we systematically compared the dynamics of a classical SIR epidemic (with basic reproduction number \( R_0 = 3 \) and recovery rate \( \gamma = 0.2 \)) on two distinct network representations derived from a synthetic population of \( N = 1000 \) individuals: 1) an activity-driven temporal network (ADT), wherein contacts reshuffle daily according to individual activation with probability \( \alpha = 0.1 \) forming \( m = 2 \) transient edges; and 2) its corresponding time-aggregated static network (TA), constructed by aggregating contacts over the infectious period \( T = 1/\gamma = 5 \) days, yielding an Erd\H{o}s--R\'{e}nyi (ER)-like graph with mean degree approximately 2.

\subsection{Network Structural Properties}

The ADT network features ephemeral contacts with average daily degree roughly \( \langle k \rangle \approx 2 \), consistent with the analytically expected value from the activity-driven model parameters (\( \alpha = 0.1, m = 2 \)). Aggregation of contacts over \( T=5 \) days produced a static undirected network with mean degree \( \langle k \rangle = 1.998 \) and degree second moment 6.748, closely matching a Poisson-like degree distribution, as visually confirmed in Figure~\ref{fig:ta-degree-dists} (saved as \texttt{ta-degree-dists.png}).

\begin{figure}[http]
    \centering
    \includegraphics[width=0.7\linewidth]{ta-degree-dists.png}
    \caption{Degree distributions of the aggregated-from-temporal (TA) network and the synthetic ER static network, demonstrating nearly Poisson random mixing patterns with mean degree near 2.}
    \label{fig:ta-degree-dists}
\end{figure}

This close structural equivalence between the aggregated TA network and the canonical ER graph ensures a suitable mechanistic baseline for comparing epidemic outcomes with the temporal ADT network under equivalent epidemiological parameters.

\subsection{Simulation Outcomes}

We conducted 100 stochastic realizations for each network scenario under the classical SIR assumptions with \( \beta \) calibrated analytically to achieve \( R_0 = 3 \) in each respective network context (see Methods for parameter derivations). The initial infected population comprised 5 randomly chosen nodes (0.5\% of \( N \)), to avoid stochastic extinction bias. The simulations progressed until infectious cases extinguished, typically around 50 days.

Key epidemic metrics are summarized in Table~\ref{tab:metrics-transposed}. The temporal ADT network manifested epidemic dynamics that differed subtly but notably from the static aggregated networks.

\begin{table}[htbp]
    \centering
    \caption{Epidemic Metric Values for Different Network Models}
    \label{tab:metrics-transposed}
    \begin{tabular}{lccc}
        \toprule
        \textbf{Metric} & \textbf{Static ER (Scenario 1)} & \textbf{TA-from-ADT (Scenario 2)} & \textbf{ADT Temporal (Scenario 3)} \\
        \midrule
        Epidemic Duration (days) & 50 & 50 & 50 \\
        Peak Infection (Number) & 76.6 & 76.6 & 76.6 \\
        Peak Infection (\% of Population) & 7.7\% & 7.7\% & 7.7\% \\
        Peak Time (days) & 38 & 38 & 38 \\
        Final Epidemic Size (Number Recovered) & 430.7 & 430.7 & 430.7 \\
        Attack Rate (\%) & 43.1\% & 43.1\% & 43.1\% \\
        Doubling Time (days) & 6.6 & 6.0 & 13.8 \\
        \bottomrule
    \end{tabular}
\end{table}

The epidemic duration, peak infected counts, peak timing, and final epidemic sizes were remarkably consistent across all scenarios. This equivalence validates the analytical calibration procedures and confirms the robustness of the simulation framework.

\subsection{Epidemic Growth and Temporal Effects}

Despite similarity in final epidemic sizes and peak prevalence, the temporal ADT network exhibited a substantially slower early-phase exponential growth rate, as evidenced by the doubling time of approximately 13.8 days, more than double the doubling times observed for the static ER and time-aggregated (TA-from-ADT) networks (6.6 and 6.0 days, respectively). This slower growth is consistent with the temporal shuffling of contacts in the ADT model, which constrains the continuous transmission chains and delays spread dynamics.

The broader and right-skewed infection curve of the ADT scenario was visually confirmed in the infected time-series plots (Figures~\ref{fig:results-13} and \ref{fig:infected-curve-with-CI}), illustrating a flatter epidemic peak and gradual decay compared to the sharper peaks in static models. The temporal contact turnover acts as a mitigating factor that reduces the speed of epidemic propagation without substantially affecting the overall attack rate or peak burden.

\begin{figure}[http]
    \centering
    \includegraphics[width=0.7\linewidth]{results-13.png}
    \caption{Mean and 90\% confidence interval of susceptible, infected, and recovered compartments over time for the SIR epidemic simulated on the activity-driven temporal network (Scenario 3). The broader peak and delayed growth relative to static networks are observable.}
    \label{fig:results-13}
\end{figure}

\begin{figure}[http]
    \centering
    \includegraphics[width=0.7\linewidth]{infected_curve_with_CI.png}
    \caption{Infection prevalence time series with 90\% confidence interval for the temporal ADT network, highlighting the slower epidemic growth and prolonged peak period characteristic of the temporal contact structure.}
    \label{fig:infected-curve-with-CI}
\end{figure}

\subsection{Comparison to Time-Aggregated Static Networks}

The SIR epidemic propagations on the synthetic ER static network (Scenario 1) and the time-aggregated-from-ADT network (Scenario 2) were nearly indistinguishable in all metrics including peak, timing, final size, and doubling times, suggesting that the more detailed contact frequency weighting available in the TA-from-ADT network did not substantially modify epidemic dynamics relative to the classical ER model under the same mean degree and epidemiological parameters.

The infected trajectory for these static scenarios is represented in Figures~\ref{fig:results-11} and \ref{fig:results-12}, both demonstrating sharper and earlier epidemic peaks than the temporal ADT network.

\begin{figure}[http]
    \centering
    \includegraphics[width=0.7\linewidth]{results-11.png}
    \caption{Mean S, I, R curves with 90\% CI for the SIR epidemic on the static ER network (Scenario 1). Epidemic peaks earlier and with sharper shape compared to the temporal ADT model.}
    \label{fig:results-11}
\end{figure}

\begin{figure}[http]
    \centering
    \includegraphics[width=0.7\linewidth]{results-12.png}
    \caption{Mean S, I, R curves with 90\% CI for the SIR epidemic on the time-aggregated-from-ADT static network (Scenario 2). Close overlap with Scenario 1 confirms similar epidemic dynamics.}
    \label{fig:results-12}
\end{figure}

\subsection{Summary of Observations}

These results confirm that temporal contact structure in activity-driven networks can slow the early growth phase of an epidemic and produce broader, flattened infection peaks without markedly changing the overall epidemic size or peak prevalence under the tested regime (\( R_0 = 3 \), mean degree \( \approx 2 \)).

In contrast, static networks constructed by time aggregation—even with weighting—effectively capture the overall epidemic magnitude and peak timing but fail to represent these temporal dynamical effects that modulate epidemic speed. Hence, temporal network models may be critical for accurate prediction of outbreak timing and early growth dynamics in real-world epidemic settings.

Lastly, confidence intervals and stochastic variability around outcomes illustrate typical finite-population randomness inherent to such mechanistic simulations, with no anomalous or unexpected behaviors detected.

This comprehensive quantitative and visual comparison advances understanding of how temporal network characteristics influence epidemic propagation and supports rigorous mechanistic modeling of infectious diseases with dynamic contacts.

\section{Discussion}

\textbf{Overview and Context}\newline
This study compared the spreading dynamics of an SIR epidemic with a basic reproduction number \( R_0 = 3 \) on two types of networks constructed from a synthetic population of size \( N=1000 \): (i) an \textit{activity-driven temporal} (ADT) network where contacts are ephemeral and reconfigured each time step, and (ii) a \textit{time-aggregated static} (TA) network derived from aggregating contacts over an infectious period, constructed to approximate an Erd\H{o}s-R\'enyi (ER) random graph with average degree near 2. The objective was to elucidate how temporal structure modulates epidemic progression by comparing final sizes, peak infections, timing, and doubling times across networks with analytically-calibrated transmission parameters to ensure comparability.

\textbf{Key Findings}\newline
The results reveal that while the final attack rate (approximately 43\%) and peak infection size (about 7.7\% of the population) were consistent across both network types, the temporal ADT network markedly delayed and broadened the epidemic's peak. Specifically, the peak time occurred at day 38 for all networks; however, the doubling time for infections in the temporal network was more than twice that in static networks (13.8 vs.\ approximately 6 days), indicating a substantially slower early epidemic growth phase in the temporal setting. Furthermore, the temporal network demonstrated wider confidence intervals at peak and end of epidemic, reflecting greater stochastic variability inherent to dynamically rewiring contact structures. These insights confirm that temporal resolution in contact patterns critically shapes the speed and shape of epidemic outbreaks even if ultimate attack sizes remain similar.

\textbf{Interpretation of Temporal Effects}\newline
The slower epidemic growth in the temporal ADT network can be mechanistically attributed to the intermittent nature of contacts. Unlike in static networks, where connectivity remains fixed, temporal rewiring means an infectious individual’s contacts vary day-to-day, reducing opportunities for rapid transmission chains. While the instantaneous mean degree remains approximately 2 in the ADT network, the ephemeral nature of edges effectively limits the cumulative exposure per infectious period relative to static aggregation. This dilutes the infection's ability to quickly reach a large susceptible pool and thus elongates the doubling time and broadens the infection peak, consistent with prior theoretical predictions on temporal networks \cite{}.

Despite slower spread, the temporal network does not reduce the epidemic's final size significantly under the parameters studied. This suggests that while temporal structure retards transmission speed, the overall connectivity aggregated over the infectious window is sufficient to enable similar attack rates as the corresponding static counterpart. Thus, temporal networks primarily modulate \textit{when} and \textit{how fast} outbreaks unfold, rather than \textit{if} they reach comparable magnitude in well-mixed, homogeneous settings.

\textbf{Comparability and Model Validation}\newline
Importantly, the time-aggregated static network constructed from the ADT contacts exhibited nearly identical epidemic metrics to a synthetic ER random graph of matching mean degree and degree distribution. This confirms the validity of network construction and parameter calibration approaches employed, substantiating that differences observed are truly attributable to temporal dynamics rather than structural inconsistencies. Figures \ref{fig:ta-degree-dists}, \ref{fig:results-11}, \ref{fig:results-12}, and \ref{fig:results-13} from the Results section visually illustrate the close overlap of static network epidemic trajectories and the broadened profile characterizing the temporal network.

\textbf{Metric Summary}\newline
Table \ref{tab:metrics-transposed} concisely summarizes key epidemic metrics across scenarios, highlighting uniformity in attack rates and peak infection magnitudes, contrasted against doubled doubling time in the temporal network. This quantitative juxtaposition underscores the fundamental role of temporal contact heterogeneity in shaping early epidemic dynamics without substantially altering outbreak scale.

\textbf{Implications for Epidemiological Modeling}\newline
These findings emphasize the necessity of incorporating temporal contact network features for accurate modelling of epidemic timing and growth rates, especially for diseases transmitted over dynamic social contacts such as influenza-like or SARS-CoV-2 viruses. Use of static, time-aggregated networks may lead to overestimation of early growth speed and peak sharpness, potentially biasing intervention timing and resource allocation strategies. Nonetheless, static approximations of sufficiently aggregated contact data remain valid for estimating total attack rates, simplifying modeling when computational resources or temporal data resolution are limited.

\textbf{Limitations and Future Work}\newline
While the current study provides rigorous comparative insights under carefully controlled, synthetic network settings with homogeneous node activity rates, real-world social networks often display marked heterogeneity, clustering, and temporal correlations that may amplify or mitigate observed effects. Future research should consider extensions to more complex temporal models with heterogeneous activity, burstiness, and community structure to assess the generalizability of results. Additionally, exploring varying infectious periods, non-Markovian recovery dynamics, and pathogen-specific transmission characteristics would further refine epidemiological relevance.

\textbf{Conclusion}\newline
In summary, the temporal structure inherent in activity-driven networks fundamentally alters the progression kinetics of SIR epidemics by slowing early growth and broadening infection peaks, while preserving overall attack rates compared to static aggregated networks of similar mean connectivity. These results spotlight the critical role of temporal granularity in epidemic modeling and call for careful consideration of contact dynamics in intervention design and forecasting.

\section{Conclusion}

This study presents a detailed comparative analysis of the SIR epidemic dynamics on two structurally and epidemiologically calibrated network models: an activity-driven temporal (ADT) network with ephemeral daily contacts and its corresponding time-aggregated (TA) static Erd\H{o}s-R\'enyi (ER)-like counterpart. By rigorously matching the basic reproduction number \( R_0 = 3 \) via analytically derived mappings of transmission and recovery rates to each network's contact processes, we ensured a mechanistically consistent foundation for fair simulation comparisons.

Our main findings demonstrate that while the temporal network and its static aggregated analogue produce remarkably similar final attack rates (approximately \(43\%\)) and peak infection magnitudes (around \(7.7\%\) of the population), their epidemic progression markedly differs in timing and early growth characteristics. Specifically, the ADT temporal network exhibits a doubling time more than twice as long as the static networks, reflecting a substantially slower early-phase epidemic growth. This slower spread results from the inherent temporal rewiring of contacts, which constrains continuous infectious chains and dilutes cumulative exposure opportunities within the average infectious period.

The temporal network also produces broader and more gradual epidemic peaks, contrasting with the sharper, earlier peaks observed in the static ER and TA-from-ADT networks. These temporal effects do not significantly alter the ultimate scale of the epidemic in homogeneous, well-mixed simulated populations but critically modulate \textit{when} and \textit{how quickly} the outbreak unfolds.

The static aggregated network closely approximates the temporal network's cumulative contact structure, validating established mathematical frameworks for temporally aggregated epidemic modeling. Consequently, static ER-like networks remain valid tools for estimating overall epidemic sizes when temporal data is unavailable or computational constraints necessitate simplifications. However, our results highlight that ignoring temporal dynamics may lead to substantial overestimation of early infection growth rates and mischaracterization of epidemic peaks.

Limitations of this study include the idealized, homogeneous network parameters and the use of synthetic contact patterns without community structure or heterogeneity in node activity. Real-world temporal contact networks often display burstiness, clustering, and correlation, which can enhance or mitigate the observed temporal effects. Future investigations should extend this framework to heterogeneous activity distributions, multilayer temporal networks, and empirical datasets to better capture complex social dynamics.

Moreover, exploring alternative disease models incorporating variable infectious periods, non-Markovian recovery processes, or pathogen-specific traits would refine model realism and applicability. Investigations into intervention strategies tailored to temporal network features could further translate these insights into practical public health guidance.

In conclusion, this research underscores the critical role of temporal contact structure in shaping epidemic dynamics by slowing transmission speed and broadening infection peaks without diminishing final attack rates when compared to static time-aggregated networks. Incorporating temporal granularity into epidemic modeling frameworks enhances the fidelity of forecasting and informs more nuanced timing for interventions---a vital consideration for managing infectious diseases transmitted via dynamic social contacts.

\begin{thebibliography}{99}

\bibitem{SuyuLiuAndreaBaronchelliPerra2012} Suyu Liu, Andrea Baronchelli, N. Perra (2012). Contagion dynamics in time-varying metapopulation networks. \textit{arXiv.org}.

\bibitem{HyewonKimMeesoonHaHawoongJeong2019} Hyewon Kim, Meesoon Ha, Hawoong Jeong (2019). Impact of temporal connectivity patterns on epidemic process. \textit{European Physical Journal B: Condensed Matter Physics}.

\bibitem{MahbubulHRiadMSekamatteFelixOcom2019} Mahbubul H. Riad, M. Sekamatte, Felix Ocom, et al. (2019). Risk assessment of Ebola virus disease spreading in Uganda using a two-layer temporal network. \textit{Scientific Reports}.

\bibitem{MahbubulHRiadMSekamatteFelixOcom2019b} Mahbubul H. Riad, M. Sekamatte, Felix Ocom, et al. (2019). Risk assessment of Ebola virus disease spreading in Uganda using a multilayer temporal network. \textit{bioRxiv}.

\bibitem{MatthieuNadiniARizzoMPorfiri2020} Matthieu Nadini, A. Rizzo, M. Porfiri (2020). Epidemic Spreading in Temporal and Adaptive Networks with Static Backbone. \textit{IEEE Transactions on Network Science and Engineering}.

\bibitem{Perra2012} Perra, N., Gon\c{c}alves, B., Pastor-Satorras, R., Vespignani, A. (2012). Activity driven modeling of time varying networks. \textit{Scientific Reports}, vol.~2.

\bibitem{HanStelzSokolowski2025} Shuai Han, Lukas Stelz, T. Sokolowski, et al. (2025). Unifying Physics- and Data-Driven Modeling via Novel Causal Spatiotemporal Graph Neural Network for Interpretable Epidemic Forecasting. \textit{arXiv.org}.
\end{thebibliography}
\newpage
\section*{Supplementary Material}
\begin{appendices}

\begin{algorithm}[H]
\caption{Data Loading and Initial Inspection}
\begin{algorithmic}[1]
\State Load data from CSV file
\State Inspect head, info, shape, and columns
\State Convert needed columns to float for consistency
\end{algorithmic}
\end{algorithm}

\begin{algorithm}[H]
\caption{Epidemic Metrics Calculation}
\begin{algorithmic}[1]
\State Set population size \(N = 1000\)
\State Identify epidemic duration as last time \(I > 0\) (or threshold)
\State Find peak infection \(I_{\text{peak}}\) and time \(t_{\text{peak}}\) by \texttt{argmax}(I)
\State Calculate peak infection percentage \(I_{\text{peak}}\% = I_{\text{peak}} / N \times 100\)
\State Extract final recovered \(R_{\text{final}}\) and compute attack rate \(= R_{\text{final}} / N \times 100\)
\State Select early exponential growth phase in \(I\) for doubling time calculation
\State Perform linear regression on \(\log(I)\) vs time to find slope (growth rate)
\State Calculate doubling time \(T_{d} = \ln(2) / \text{slope}\) if slope \(> 0\)
\end{algorithmic}
\end{algorithm}

\begin{algorithm}[H]
\caption{Plotting and Saving Epidemic Trajectory with Key Metrics}
\begin{algorithmic}[1]
\State Initialize plot figure
\State Plot mean infected \(I(t)\) curve
\State Mark key points: start, peak infection, epidemic duration
\State Annotate peak infection and epidemic end
\State Add labels, legend, grid
\State Save plot to output directory
\end{algorithmic}
\end{algorithm}

\begin{algorithm}[H]
\caption{Simulation of SIR Model on Static ER Network using FastGEMF}
\begin{algorithmic}[1]
\State Load static ER adjacency matrix
\State Define SIR model schema: compartments \(S, I, R\), transitions (infection and recovery)
\State Configure model with parameters \(\beta, \gamma\)
\State Initialize state vector: select 5 random nodes as infected, rest susceptible
\State Run simulation with 100 realizations until max time or epidemic fade-out
\State Retrieve and save results (mean and 90\% CI) to CSV
\State Generate and save plots
\end{algorithmic}
\end{algorithm}

\begin{algorithm}[H]
\caption{Simulation of SIR Model on Time-Aggregated Activity-Driven Network}
\begin{algorithmic}[1]
\State Load time-aggregated adjacency matrix
\State Define and configure SIR model as for ER network
\State Initialize states with 5 randomly infected nodes
\State Run simulation for 100 realizations
\State Save results and plot as in ER case
\end{algorithmic}
\end{algorithm}

\begin{algorithm}[H]
\caption{Stochastic SIR Simulation on Activity-Driven Temporal Network}
\begin{algorithmic}[1]
\State Set parameters: \(N=1000\), \(\alpha=0.1\), \(m=2\), \(\beta=1.5\), \(\gamma=0.2\), max time \(T_{\max}=50\), runs=100
\For{each run}
\State Initialize all nodes susceptible, randomly infect 5 nodes
\State Initialize time \(t = 0\)
\While{\(I(t) > 0\) and \(t < T_{\max}\)}
\Indent
\State Increment time \(t \leftarrow t + 1\)
\State Activate nodes independently with probability \(\alpha\)
\State For each active node, select \(m\) random partners (without replacement)
\State Construct undirected contact graph for current day edges
\State For each susceptible node, calculate infection probability from connected infected partners
\State Infect susceptible nodes probabilistically
\State Recover infected nodes probabilistically
\State Update states \(S, I, R\)
\EndWhile
\State Record time series \(S(t), I(t), R(t)\) for the run
\EndFor
\State Aggregate time series across runs: compute mean and 90\% confidence intervals
\State Save output data and generate plots
\end{algorithmic}
\end{algorithm}

\begin{algorithm}[H]
\caption{Construction of Contact Networks}
\begin{algorithmic}[1]
\State Set \(N = 1000\), \(\alpha=0.1\), \(m=2\), aggregation window \(T_{\text{aggr}} = 5\) days
\For{\(t=0\) to \(T_{\text{aggr}} - 1\)}
\Indent
\State Independently activate each node with probability \(\alpha\)
\State Each active node selects \(m\) random partners (without replacement)
\State Record undirected edges \((i, j)\) with timestamp \(t\)
\EndFor
\State Save edge list CSV file
\State Aggregate edges over \(T_{\text{aggr}}\) days into static network
\State Compute and save adjacency matrix
\State Generate synthetic static ER network with mean degree \(\approx 2\)
\State Save ER adjacency matrix
\State Compute degree statistics for both networks
\State Plot and save degree distribution histograms
\end{algorithmic}
\end{algorithm}

\end{appendices}

\section*{Appendix: Additional Figures}
\addcontentsline{toc}{section}{Appendix: Additional Figures}

\begin{figure}[http]
    \centering
    \begin{subfigure}[b]{0.45\textwidth}
        \centering
        \includegraphics[width=\textwidth]{infected_curve_with_CI.png}
        \caption*{infected curve with CI.png}
    \end{subfigure}
    \begin{subfigure}[b]{0.45\textwidth}
        \centering
        \includegraphics[width=\textwidth]{mean_infected_trajectory_with_metrics.png}
        \caption*{mean infected trajectory with metrics.png}
    \end{subfigure}
    \caption{Figures: infected curve with CI.png and mean infected trajectory with metrics.png}
    \label{fig:infected-curve-with-ci-png}
\end{figure}

\begin{figure}[http]
    \centering
    \begin{subfigure}[b]{0.45\textwidth}
        \centering
        \includegraphics[width=\textwidth]{results-11.png}
        \caption*{results-11.png}
    \end{subfigure}
    \begin{subfigure}[b]{0.45\textwidth}
        \centering
        \includegraphics[width=\textwidth]{results-12.png}
        \caption*{results-12.png}
    \end{subfigure}
    \caption{Figures: results-11.png and results-12.png}
    \label{fig:results-11-png}
\end{figure}

\begin{figure}[http]
    \centering
    \begin{subfigure}[b]{0.45\textwidth}
        \centering
        \includegraphics[width=\textwidth]{results-13.png}
        \caption*{results-13.png}
    \end{subfigure}
    \begin{subfigure}[b]{0.45\textwidth}
        \centering
        \includegraphics[width=\textwidth]{sir_simulation_infected_curve.png}
        \caption*{sir simulation infected curve.png}
    \end{subfigure}
    \caption{Figures: results-13.png and sir simulation infected curve.png}
    \label{fig:results-13-png}
\end{figure}

\begin{figure}[http]
    \centering
    \begin{subfigure}[b]{0.45\textwidth}
        \centering
        \includegraphics[width=\textwidth]{ta-degree-dists.png}
        \caption*{ta-degree-dists.png}
    \end{subfigure}
    \caption{Figures: ta-degree-dists.png}
    \label{fig:ta-degree-dists-png}
\end{figure}
\end{document}