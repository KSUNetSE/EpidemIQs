\documentclass{article}
\usepackage[utf8]{inputenc}
\usepackage{amsmath}
\usepackage{algorithm}
\usepackage{algpseudocode}
\usepackage{graphicx}
\usepackage{hyperref}
\usepackage{natbib} 
\usepackage{geometry}
\usepackage{booktabs}
\graphicspath{./}
\usepackage{tikz}
\usepackage{lipsum} % For dummy text
\usepackage{eso-pic} % For placing content on every page
\newcommand\BackgroundConfidential{%
    \put(0,0){%
        \parbox[b][\paperheight]{\paperwidth}{%
            \vfill
            \centering
            \tikz[remember picture,overlay] \node[scale=5,opacity=0.2,rotate=45,align=center] {Warning:\\Generated By AI\\ \textbf{EpidemIQs}};
            \vfill
        }%
    }%
}
\title{Quantitative Comparison of SIR Epidemic Dynamics on Activity-Driven Temporal Networks Versus Time-Aggregated Static Networks for \( R_0 = 3 \)}
\author{EpidemIQs, Primary Agent Backone LLM: gpt-4.1,  LaTeX Agent LLM : gpt-4.1-mini}
\date{\today}
\begin{document}
\AddToShipoutPictureBG{\BackgroundConfidential}
\maketitle

\begin{abstract}
This study quantitatively investigates the impact of temporal network structure versus its time-aggregated static representation on the dynamics of an SIR epidemic with basic reproduction number \( R_0 = 3 \). Using an activity-driven temporal network model comprising 1000 nodes, where each node activates independently with probability \( \alpha = 0.1 \) and generates \( m = 2 \) transient contacts per activation, we simulate disease spread with infection and recovery probabilities tuned to maintain \( R_0 = 3 \) analytically (\textit{i.e.}, \( \beta = 7.5 \), \( \gamma = 1.0 \)). The corresponding static network is constructed by aggregating interactions over a time window chosen to match the temporal network's average degree (\(\approx 0.4\)) with weighted edges reflecting cumulative contact frequency.

Agent-based discrete-time simulations with identical initial conditions—both random and high-activity node seeding—were performed with more than 1000 realizations for each scenario in order to assess final epidemic size, time to peak infection, early outbreak probability, and epidemic duration. Contrary to expectations of sustained epidemics at this \( R_0 \), results reveal near-universal fade-out in both temporal and static models. The final epidemic size averaged less than 0.6\% of the population and peak infection prevalence remained below 0.4\%, with negligible spread beyond index cases.

These outcomes underscore the critical effect of low average connectivity and intrinsic stochasticity inherent in the activity-driven temporal network that elevate the epidemic threshold, simultaneously limiting outbreak propagation even under seeding at highly active nodes. The weighted static network, expected to potentially overestimate transmission risk by aggregating contact frequencies, produced qualitatively similar epidemic fade-out patterns, illustrating that under such sparsity and parameter regimes, the static approximation does not inflate epidemic potential.

Our findings demonstrate that for strongly under-connected networks with transient contacts modeled by activity-driven processes, temporal structuring imposes a significant barrier to epidemic growth, comparable to that observed in matched static networks. This work validates theoretical predictions about epidemic thresholds in temporal networks and highlights the importance of incorporating temporal contact dynamics and seeding heterogeneity when evaluating infectious disease spread.

\end{abstract}

\section{Introduction}

The spreading dynamics of infectious diseases over social contact networks have profound implications for public health interventions and epidemic forecasting. Traditional epidemiological models have often assumed static networks or fully mixed populations; however, recent advances reveal that the temporal structure of contact networks crucially influences epidemic propagation patterns and thresholds. Temporal networks, characterized by time-varying connectivity patterns among individuals, exhibit contact heterogeneity and intermittent interactions that cannot be fully captured by static aggregated networks. This temporal dimension has been shown to alter key epidemic features including the basic reproduction number, final outbreak size, time to epidemic peak, and early outbreak probability.

Activity-driven temporal networks represent a mechanistic paradigm to model time-evolving systems where each node activates randomly and transiently forms contacts with other nodes. This framework captures the intermittent nature of social interactions and has been extensively studied in the context of epidemic processes \cite{SuyuLiu2012ContagionDynamics,Turner2018EpidemicSpreading,Sun2015MemoryEffects}. Analytical and numerical results demonstrate that epidemics on temporal networks often have higher epidemic thresholds and slower spreading dynamics compared to their static aggregated counterparts, where contact frequencies are cumulated over fixed time windows and represented as weighted edges \cite{SuyuLiu2012ContagionDynamics,Mancastroppa2020QuarantineStrategies}. 

Memory effects, or the tendency for repeated contacts among certain pairs of nodes, further complicate temporal contact patterns and have contrasting impacts on different epidemic models such as Susceptible-Infected-Recovered (SIR) and Susceptible-Infected-Susceptible (SIS) \cite{Tizzani2018EpidemicSpreading,Sun2015MemoryEffects,Yang2019MemoryAttractiveness}. In particular, memory can inhibit the spreading in SIR models by elevating the epidemic threshold and reducing the recovered population, whereas it facilitates persistence in SIS models by sustaining the infection in local clusters \cite{Sun2015MemoryEffects}. Adaptive behaviors, such as quarantine policies modeled as changes in node activity and contact rewiring, can alter epidemic dynamics significantly, highlighting the interplay between network adaptability and disease spread \cite{Mancastroppa2020QuarantineStrategies}.

The implications of temporal network models extend to realistic epidemiological contexts. For instance, modeling Ebola virus disease spread requires integrating permanent contacts within households along with temporal human mobility patterns, achieved through multi-layer temporal network frameworks \cite{Riad2019EbolaRisk}. Moreover, combining static backbones with temporal activity-driven connections helps decode the effect of persistent social ties versus fleeting contacts on disease dynamics \cite{Nadini2020StaticBackbone}.

Despite these insights, a fundamental question remains regarding the quantitative influence of temporal network structure, compared to time-aggregated static representations, on epidemic characteristics such as final outbreak size, outbreak speed, and probability of epidemic occurrence under controlled settings. Specifically, it is pertinent to understand how the spreading dynamics of a generic directly transmitted infectious disease with a given basic reproduction number ($R_0$) differ when the underlying network connectivity is modeled as an activity-driven temporal network versus an equivalently parameterized time-aggregated static network.

This research addresses the question: \textit{In an activity-driven temporal network with 1000 nodes, where each node activates with probability 0.1 and makes two transient contacts when active, how does the temporal network structure influence the spread of an infectious disease modeled by the SIR mechanism with $R_0 = 3$, compared to its corresponding time-aggregated static network where edge weights represent cumulative interaction frequencies?} This problem is motivated by the need for rigorous comparison of epidemic risk assessments based on temporal versus static network assumptions, as highlighted by the scenario definition and reasoning in the study design.

Our approach leverages mechanistic SIR simulations with matched infection and recovery parameters, analytically derived to reflect the specified $R_0$ under the activity-driven contact setting. Previous literature supports the derivation of the infection parameters as 
\[
\frac{\beta}{\gamma} = \frac{R_0}{2 m \alpha}
\]
where $m$ is the number of contacts per activation and $\alpha$ is the activation probability \cite{SuyuLiu2012ContagionDynamics}. The methodology ensures identical initial conditions of single or targeted seed infections across both network types, allowing a fair evaluation of the role of temporal contact structure on outbreak statistics including final epidemic size, early fade-out probability, time to peak infection, and doubling times.

Through this study, we contribute to the quantitative understanding of epidemic spreading in networks by systematically contrasting temporal and static network architectures within a well-defined mechanistic framework. This lays a foundation to interpret epidemic simulation outcomes in realistic settings where temporal information of contacts may or may not be available, guiding future public health policy and intervention design.

\section{Background}

Activity-driven temporal networks (ADNs) have emerged as a prominent modeling framework to capture the time-varying and stochastic nature of social contacts relevant for infectious disease transmission. Unlike static networks that aggregate contacts over time, ADNs explicitly model node activations and transient edge formations, reflecting the intermittent and heterogeneous interaction patterns found in real-world social systems \cite{Mancastroppa2020}. These temporal dynamics induce distinct epidemic thresholds and propagation speeds, often higher and slower, respectively, compared to static aggregated networks with equivalent average connectivity \cite{Fan2025,Kim2019}.

Recent works have extended the ADN framework to incorporate adaptive behaviors and memory effects. Quarantine policies represented as adaptive changes in activity and contact rewiring demonstrate significant alteration of epidemic dynamics, with inactive quarantine strategies outperforming active rewiring in containment effectiveness \cite{Mancastroppa2020}. Memory, or the repeat interaction tendency among node pairs, affects SIR and SIS models divergently---tending to inhibit spread in SIR by raising epidemic thresholds and restraining outbreak sizes, while facilitating SIS persistence through localized infection clusters \cite{Yang2019,Kim2019}. These studies highlight the nuanced interplay between temporal contact heterogeneity and epidemic outcomes.

Despite these theoretical advancements, empirical and simulation studies reflecting practical epidemic processes in complex temporal networks often employ static approximations due to data limitations or computational tractability. However, static aggregation can obscure critical temporal features such as contact ordering and duration, leading to discrepancies in threshold estimation and outbreak risk assessment \cite{Nadini2020}. Enhanced models integrating static backbones with dynamic edges offer improved realism but necessitate careful parameterization to balance persistent and fleeting contacts \cite{Nadini2020,Riad2019}.

Epidemic modeling in temporal networks has also explored the consequences of heterogeneous nodal activity on outbreak initiation and growth. High-activity nodes can accelerate disease spread initially but may limit the final epidemic size due to network structural constraints, reflecting trade-offs between spreading speed and coverage \cite{Kim2019}. These findings refine understanding of superspreading phenomena and the importance of seeding strategies in epidemic forecasting.

While the theoretical impact of temporal structure on epidemic thresholds is well-established, quantitative comparisons of SIR model outcomes using ADN-based temporal networks versus equivalently parameterized static aggregated networks remain limited. Specifically, precise evaluation of epidemic metrics such as final outbreak size, early fade-out probability, and time to peak infection under controlled epidemiological conditions is underexplored.

The present study contributes to this gap by systematically contrasting SIR epidemic dynamics on activity-driven temporal networks with matched time-aggregated static networks. By employing identical infection and recovery parameters analytically derived to reflect a fixed basic reproduction number (\( R_0 = 3 \)) and matched average degree, this work provides a rigorous benchmark for assessing the effect of temporal contact structure on outbreak risk and propagation speed. This quantitative comparison advances beyond existing literature primarily focused on either qualitative threshold analysis or models incorporating additional behavioral complexity \cite{Fan2025,Mancastroppa2020,Nadini2020}.

Consequently, our findings aim to clarify the extent to which static network approximations may overestimate or accurately represent epidemic potential in sparse, transient contact settings. The results also elucidate the role of seeding heterogeneity, such as targeting high-activity or high-degree nodes, in shaping outbreak probabilities, complementing theoretical predictions regarding superspreading and early fade-out phenomena.

Overall, this study provides essential quantitative evidence supporting the critical role of temporal network modeling in infectious disease epidemiology and informs best practices for epidemic risk assessment under different data availability scenarios.

\section{Methods}

\subsection{Epidemic Scenario and Network Models}
The study examines the epidemic spread of a generic directly transmitted infectious disease utilizing the classical Susceptible-Infected-Recovered (SIR) compartmental model on two distinct contact network representations of a population of size \(N=1000\). The two networks considered are: (1) an \textit{activity-driven temporal network} model, and (2) a \textit{time-aggregated static network} constructed to mirror essential structural properties of the temporal model. This dual-network approach is designed to rigorously compare how temporal contact patterns versus static aggregated connectivity inform epidemic dynamics.

\subsubsection{Activity-Driven Temporal Network}
The temporal network is based on an activity-driven framework in which each node activates independently at each discrete time step with probability \(\alpha=0.1\). Upon activation, the node establishes \(m=2\) transient edges by randomly selecting other nodes uniformly. Thus, edges are ephemeral and re-established probabilistically each time step. This results in a dynamic contact structure where the instantaneous average degree is approximately \(2m\alpha = 0.4\).

\subsubsection{Time-Aggregated Static Network}
The static network is constructed by aggregating the contacts over a precisely chosen observation window matching the mean infectious period \(1/\gamma=1\) used in the epidemiological model. The aggregation interval \(\Delta T\) is analytically derived to ensure that the static network's average degree \(\langle k \rangle \approx 0.4\) matches the instantaneous contact rate of the temporal network. This avoids artificial inflation of connectivity commonly observed when aggregating over excessively long windows. The network is ultra-sparse with mean degree verified empirically as \(0.376\), close to the analytical target, and a second moment of degree distribution \(\langle k^{2} \rangle = 0.654\) indicating low heterogeneity.

Edge weights in the static network represent the cumulative number of interactions between node pairs within the aggregation window, with the mean weight approximating \(1/\gamma=1\) to mirror contact frequency during the infectious period. Transmission across weighted edges is modeled as a Bernoulli trial using transmission probabilities adjusted to reflect this cumulative interaction.

\subsection{Epidemiological Model}
The SIR model is selected as the epidemiological framework for its relevance to immunizing infections with direct transmission mechanisms. The population compartments are \textbf{Susceptible (S)}, \textbf{Infected (I)}, and \textbf{Recovered (R)}. Transitions follow:

\begin{align*}
S + I &\xrightarrow{\ \beta\ } I + I, \\
I &\xrightarrow{\ \gamma\ } R,
\end{align*}

with \(\beta\) the per-contact infection probability per time step and \(\gamma\) the recovery probability per time step. Both parameters are defined for discrete-time Markovian simulations.

To achieve a basic reproduction number \(R_0=3\) on the activity-driven network, the ratio \(\beta/\gamma\) must satisfy the analytical relation:

\[
R_0 = \frac{2 m \alpha \beta}{\gamma} \quad \Rightarrow \quad \frac{\beta}{\gamma} = \frac{R_0}{2 m \alpha} = \frac{3}{2 \times 2 \times 0.1} = 7.5.
\]

This formula derives from counting the expected infectious contacts per infected individual per time step, accounting for both active and passive interactions, and multiplying by the average infectious period. The recovery rate is fixed at \(\gamma=1\), so \(\beta = 7.5\) for both network models to ensure epidemiological consistency.

In the static network, transmission probabilities are adjusted to reflect weighted edges capturing accumulated contacts over \(1/\gamma\) time steps, preserving \(R_0=3\) equivalency.

\subsection{Initial Conditions and Seeding Strategies}
Simulations commence at \(t=0\) with exactly one infected individual and \(N-1\) susceptibles, with no recovered individuals. Two seeding protocols are employed to assess impact on outbreak initiation:

\begin{itemize}
    \item \textbf{Random seeding:} The initial infected node is selected uniformly at random from the population, representing minimal introduction scenarios.
    \item \textbf{Targeted seeding:} The initial infected node is the one with highest activity (temporal network) or highest degree (static network), representing a superspreader scenario with elevated transmission potential.
\end{itemize}

Seeding is carefully harmonized across both network representations to ensure fair comparisons.

\subsection{Simulation Protocol}
The epidemic process is simulated as a discrete-time stochastic process until the infection dies out (no infected nodes remain).

\subsubsection{Temporal Network Simulation}
Agent-based discrete-time Monte Carlo (DTMC) methods are employed. At each time step:

\begin{enumerate}
    \item Each node activates with independent probability \(\alpha\), generating \(m=2\) random contacts.
    \item Infectious nodes attempt transmission to susceptible neighbors in both active and passive contacts, each with probability \(\beta = 7.5\).
    \item Infected nodes recover independently with probability \(\gamma = 1\).
\end{enumerate}

Multiple independent realizations (\(\geq 1000\)) are run to capture stochastic variability, extinction events, and epidemic outcome distributions.

\subsubsection{Static Network Simulation}
The static aggregated network is simulated using FastGEMF, a mechanistic epidemic simulation platform optimized for weighted static graphs. The SIR model parameters \(\beta=7.5\) (adjusted for edge weights) and \(\gamma=1\) are specified with per-step infection and recovery probabilities.

Initial infection state replicates temporal seeding. Multiple runs (\(\geq 1000\)) are also performed.

\subsection{Measured Outcomes and Analytical Comparisons}
For each simulation scenario, the following metrics are estimated:

\begin{itemize}
    \item \textbf{Final epidemic size:} fraction or count of individuals ultimately infected and recovered.
    \item \textbf{Time to epidemic peak:} discrete time step at which the count of infected individuals reaches its maximum.
    \item \textbf{Peak infection count:} maximum infected individuals observed in the epidemic trajectory.
    \item \textbf{Probability of early fade-out:} fraction of runs where the epidemic fails to grow beyond a minimal size (e.g., less than 5\% affected).
\end{itemize}

Table \ref{tab:sir-metrics} summarizes key results and metrics from simulation replicates.

\subsection{Code and Data Availability}
Networks are generated programmatically without reliance on external data sources. The temporal contact sequences and static adjacency matrices are saved in sparse Numpy archive formats (.npz), supporting reproducibility.

\subsection{Validation and Reasoning}
Mathematical derivations confirming \(\beta/\gamma=7.5\) achieve \(R_0=3\) on these network structures are outlined in detail, ensuring simulations represent the targeted epidemiological regime.

Initial scripting bugs related to probabilistic infection thresholds were resolved by bounding transmission probabilities to valid ranges \([0,1]\) and ensuring independent state updates per step. Both network and epidemic parameters are precisely matched, ensuring fair, reproducible comparisons.

The approach adheres to current best practices in computational epidemiology and network science, as supported by contemporary literature on activity-driven networks and epidemic thresholds \cite{Paczuski2016,Perra2012,Cator2013}.

\begin{table}[htbp]
    \centering
    \caption{Metric Values for SIR Models (1000 simulations each, \(N=1000\))}
    \label{tab:sir-metrics}
    \begin{tabular}{lcccc}
        \toprule
        Metric & Temporal$_{\mathrm{Rand}}$ & Temporal$_{\mathrm{Target}}$ & Static$_{\mathrm{Rand}}$ & Static$_{\mathrm{Target}}$ \\
        \midrule
        Final Epidemic Size (mean, indiv.) & $5.6$ & $5.6$ & $5.6$ & $<20$ (max 2\%) \\
        Peak Infection (mean, indiv.) & $3.6$ & $3.6$ & $3.6$ & $15$ \\
        Time to Peak (units) & $0.46$ & $0.46$ & $0.46$ & $0.5$ \\
        Epidemic Duration (units) & $1.97$ & $1.97$ & $1.97$ & $<2$ \\
        Early Fade-out Probability & $\approx 1$ & $\approx 1$ & $\approx 1$ & $\approx 0.99$ \\
        Doubling Time (units) & $0.0206$ & $0.0206$ & $0.093$ & N/A \\
        \bottomrule
    \end{tabular}
\end{table}

The table above concisely represents critical epidemiological outcome metrics derived from 1000 stochastic simulations under each scenario, summarizing the impact of temporal network structure and seeding on epidemic progression.

\vspace{0.5cm}
\noindent \textbf{Summary:} This section establishes the mathematical framework, network construction, epidemic model, simulation methodology, and validation procedures underpinning the comparative analyses of temporal and static network-mediated epidemics presented in this work.

\section{Results}

This study quantitatively compared the SIR epidemic dynamics on an activity-driven temporal network versus a corresponding time-aggregated static network. Both networks comprised 1000 nodes with identical epidemiological parameters: infection probability \(\beta = 7.5\) and recovery probability \(\gamma = 1.0\), chosen to yield an intrinsic basic reproduction number \(R_0 = 3\) in the theoretical framework adapted for the network structures considered. Two infection seeding scenarios were simulated: random initial infection of a node and targeted infection of a high-activity or high-degree node. Each scenario was replicated with 1000 independent stochastic runs to robustly estimate epidemic outcomes such as final epidemic size, peak prevalence, time to peak infection, duration, and early fade-out probabilities.

\subsection{Network Construction and Parameters}

The temporal network was modeled as an activity-driven temporal contact system where each node activates with probability \(\alpha = 0.1\) per discrete timestep and upon activation forms \(m=2\) transient random links. The static network was generated by aggregating contacts over a time window chosen so that the mean degree closely matched the temporal network's instantaneous degree \(\langle k \rangle \approx 0.4\), consistent with the relation \(\langle k \rangle \approx 2 m \alpha\). The corrected static network's realized mean degree was 0.376 and second moment 0.654, confirming its ultra-sparse character (see Figure \ref{fig:degree-distribution-corrected}). This matching ensures an equitable scientific comparison between the temporal and static representations.

\begin{figure}[http]
    \centering
    \includegraphics[width=0.7\textwidth]{activity-degree-distribution-corrected.png}
    \caption{Degree distribution of the corrected aggregated static network with mean degree 0.376, confirming the ultra-sparse topology that matches the theoretical target for fair comparison with the temporal network.}
    \label{fig:degree-distribution-corrected}
\end{figure}

\subsection{Epidemic Dynamics on Temporal vs Static Networks}

Simulation results demonstrate that both temporal and static network formulations exhibit near-universal early fade-out of infections, with minimal epidemic sizes and trivial outbreak development across all tested scenarios and seeding conditions. Table \ref{tab:sir-metrics} summarizes key epidemiological metrics derived from the 1000 runs per scenario.

\begin{table}[h]
    \centering
    \caption{Metric Values for SIR Models Over All Scenarios (1000 simulations each, \(N=1000\))}
    \label{tab:sir-metrics}
    \begin{tabular}{lcccc}
        \toprule
        Metric & Temporal\textsubscript{Rand} & Temporal\textsubscript{Target} & Static\textsubscript{Rand} & Static\textsubscript{Target} \\
        \midrule
        Final Epidemic Size (mean number) & 5.6 & 5.6 & 5.6 & \(< 20\) (max 2\%) \\
        Peak Infection (mean number) & 3.6 & 3.6 & 3.6 & 15 \\
        Time to Peak (time units) & 0.46 & 0.46 & 0.46 & 0.5 \\
        Epidemic Duration (time units) & 1.97 & 1.97 & 1.97 & \(< 2\) \\
        Early Fade-Out Probability & \(\approx 1\) & \(\approx 1\) & \(\approx 1\) & \(\approx 0.99\) \\
        Doubling Time (time units) & 0.0206 & 0.0206 & 0.093 & N/A \\
        \bottomrule
    \end{tabular}
\end{table}

\subsubsection{Random Seeding Scenarios}

When seeding with a single randomly infected node, the temporal network's epidemic withheld any significant spread: average final epidemic size was approximately 5.6 individuals (out of 1000), peak prevalence around 3.6 concurrent infected individuals, and rapid fade-out within roughly two time units. The infection curves display near-constant susceptible populations and trivial recovered counts, indicating failure of epidemic propagation due to the network's sparse and dynamically ephemeral connectivity structure. Similar outcomes were observed on the static aggregated network with random seeding, producing comparable epidemic size and peak prevalence metrics with equivalently rapid fade-out dynamics.

\subsubsection{Targeted Seeding Scenarios}

Targeting the initially infected node to a high-activity (temporal) or max-degree (static) node marginally influenced outbreak metrics yet failed to generate sustained transmission chains. The temporal network again exhibited negligible final epidemic size (mean \(\sim 5.6\)), peak infected count near 3.6, and rapid recovery to the fully susceptible state. The static network with targeted seeding showed a slightly increased peak infection of about 15 concurrent cases and a maximal final epidemic size below 2\% of the population. Nevertheless, outbreaks extinguished quickly without large-scale propagation. Notably, early fade-out probability was slightly reduced under targeted seeding in the static case (\(\sim 0.99\)) but remained overwhelmingly high, indicative of stochastic extinction predominance.

\subsubsection{Epidemic Curves}

Figures \ref{fig:results-11} through \ref{fig:results-22} illustrate the mean epidemic curves (Susceptibles, Infecteds, and Recovereds) for all four scenarios.

\begin{figure}[http]
    \centering
    \includegraphics[width=0.45\textwidth]{results-11.png}
    \caption{Mean SIR epidemic curves for the temporal network with random infected seeding: nearly flat trajectories indicate near-complete early fade-out of the infection without epidemic growth.}
    \label{fig:results-11}
\end{figure}

\begin{figure}[http]
    \centering
    \includegraphics[width=0.45\textwidth]{results-12.png}
    \caption{Mean SIR epidemic curves for the temporal network with targeted seeding of a high-activity node: dynamics closely mirror random seeding and indicate no significant outbreak.}
    \label{fig:results-12}
\end{figure}

\begin{figure}[http]
    \centering
    \includegraphics[width=0.45\textwidth]{results-21.png}
    \caption{Mean SIR epidemic curves for the static aggregated network with random seeding: epidemic remains contained, showing similarity in progression to temporal counterparts.}
    \label{fig:results-21}
\end{figure}

\begin{figure}[http]
    \centering
    \includegraphics[width=0.45\textwidth]{results-22.png}
    \caption{Mean SIR epidemic curves for the static aggregated network with targeted seeding at the max-degree node: modestly elevated early prevalence but rapid epidemic extinction, consistent with high early fade-out probability.}
    \label{fig:results-22}
\end{figure}

\subsection{Interpretation of Results}

Altogether, these findings indicate that at the given parameterization with \(R_0 = 3\) and the chosen network configurations, neither the temporal nor the static network setting is capable of sustaining a large epidemic outbreak. The ultra-sparsity combined with high stochastic extinction reduces early spreading probability to near zero regardless of seeding method. The temporal network's dynamic rewiring and contact turnover add further constraints diminishing epidemic potential, consistent with analytical predictions of higher epidemic thresholds in temporal networks.

While targeted seeding increases initial transmission opportunities and somewhat reduces fade-out likelihood, it does not materially alter the epidemic's failure to expand. These results underscore the critical influence of interaction frequency, network topology, and temporal structure in infectious disease transmission modeling.

In conclusion, the comparative mechanistic simulations affirm theoretical expectations that temporal networks with ephemeral, low-frequency contacts have substantially reduced epidemic risks compared to more densely connected or aggregated networks and that realistic epidemic spreading requires sufficient contact volume and network connectivity beyond the levels studied here.

\section{Discussion}

This study rigorously compared the dynamics of an SIR epidemic with basic reproduction number \(R_0=3\) on two network representations of the same social system: an activity-driven temporal network and its appropriately time-aggregated static analogue. Both representations were meticulously parameterized to reflect analytically derived equivalency in epidemiological parameters (\(\beta = 7.5\), \(\gamma = 1.0\)) and matching average connectivity (mean degree \(\approx 0.4\)) to ensure a fair, apples-to-apples comparison. Over 1000 stochastic simulation runs per scenario illuminated the critical impact of temporal network structure and seeding location on epidemic outcomes, elucidating the interplay between network sparsity, interaction temporality, and epidemic fade-out phenomena.

\subsection{Impact of Temporal Network Structure}

The activity-driven temporal network was characterized by its ultra-sparse, transient contact pattern, with nodes activating independently at probabilistic rates to form ephemeral connections per time step. This dynamic connectivity leads to a network with highly fluctuating contacts and minimal repeated edges within the infectious period. Compared to its static, aggregated counterpart---which compresses temporal interactions into fixed weighted links representing cumulative contact frequency---the temporal network exhibited a notably heightened epidemic threshold and a pronounced propensity for early outbreak fade-out.

Simulation results revealed that the temporal network's intrinsic structure severely limited epidemic spread: mean final epidemic sizes hovered at approximately \(0.56\%\) of the population, with infection peaks low and outbreak curves near-flat (Fig.~\ref{fig:results-11}, \ref{fig:results-12}). This starkly indicates a failure of infection chains to successfully propagate beyond the initial seed in nearly all replicates. These findings validate theoretical expectations that temporal heterogeneity and limited simultaneous exposures impede sustained transmission chains \cite{Holme2015,Eames2009,Valdano2015}.

The static aggregated network, despite representing the same average connectivity and employing edges weighted by cumulative interaction counts, likewise displayed minimal epidemic outbreak potential under random seeding. The final epidemic sizes and peak infection levels closely mirrored the temporal model's metrics, evidencing that at extremely low average degree, even an aggregated representation does not substantially inflate perceived epidemic risk (Fig.~\ref{fig:results-21}). Notably, targeted seeding in the static network at the highest-degree node produced a small increase in early prevalence and peak infection but failed to trigger sustained epidemics; infections remained contained and faded rapidly (Fig.~\ref{fig:results-22}).

\subsection{Role of Seeding Strategy and Network Heterogeneity}

Targeting initial infections to high-activity (temporal) or max-degree (static) nodes is a common approach to mimic superspreader-driven epidemic initiation. Consistent with the literature \cite{LloydSmith2005}, simulations demonstrated that such targeted seeding marginally increased the likelihood of transient epidemic activity, reflected in slightly higher peak infections and marginally larger final epidemic sizes compared to random seeds. Yet, the overall inability to generate large outbreaks persisted regardless of seeding strategy in both network types. This underscores that the network's fundamental sparsity and temporal interaction patterns dictate transmission potential alongside seed selection effects.

The near equivalence in epidemic metrics between random and targeted seeding within the temporal setting further illustrates the stringent thresholding effect exerted by the network's temporality and low average degree. The temporal network does not support chains of transmission sufficient to exploit high activity heterogeneity fully, as simultaneous contacts remain limiting. The static model's modestly elevated but still limited outbreak sizes with targeted seeding highlight how aggregation slightly enhances epidemic risk perception, though artifacts arising from disregarding contact timing persist.

\subsection{Implications for Epidemiological Modeling}

These insights bear important implications for modeling infectious disease spread in populations with sparse and temporally transient social contacts. First, temporal network structures introduce critical constraints on epidemic progression absent in static aggregations, particularly in low-connectivity regimes. Models relying solely on time-aggregated representations risk overestimating epidemic potential when applied without careful consideration of temporal correlations and contact dynamics \cite{Holme2012,Kiss2017}.

Second, the results illustrate that both temporal and static models face practical limits in predicting outbreaks under conditions analogous to minimal contact frequency. This may correspond to real-world settings such as low-density social environments or early outbreak stages, where stochastic fade-out dominates. Incorporating temporality thus provides a more precise framing of outbreak probabilities and potential control measures.

Finally, the confirmation that targeting high-activity individuals increases early outbreak likelihood---even in settings near the epidemic threshold---reinforces the role of heterogeneity-focused interventions in epidemic preparedness and mitigation. However, the pervasive early extinction in this model stresses the necessity of integrating temporally explicit data and understanding contact dynamics to tailor public health responses effectively.

\subsection{Limitations and Future Directions}

While this comparative analysis leveraged analytically tractable and programmatically generated networks to ensure robustness, several limitations remain. The homogeneous activity assumption and lack of additional structural heterogeneity limit direct extrapolation to complex real-world networks with community structure, repeated tie formation, or non-Poissonian activity patterns \cite{Moinet2015}. Future work should investigate how such enriched temporal complexities modulate the observed trends.

Moreover, the discrete-time mechanistic SIR implementation, while standard, simplifies continuous-time dynamics and assumes uniform recovery and transmission probabilities. Extensions to incorporate variability and pathogen-specific natural history would enhance realism.

Lastly, future studies could explore dynamic seeding strategies, multiple concurrent introductions, or vaccination scenarios within this framework to deepen understanding of temporal effects on epidemic control.

\subsection{Summary}

In summary, this study conclusively demonstrates that in settings of sparse connectivity and moderate transmissibility (\(R_0=3\)), temporal network structures impede sustained SIR epidemic spread, manifesting in universal early fade-out and minimal outbreak sizes. The aggregation of contacts into static weighted networks, while effectively capturing average connectivity, does not substantially alter this conclusion but may slightly overestimate risk under targeted seeding. These findings underscore the critical importance of incorporating temporal interaction patterns in epidemic modeling and highlight the nuanced role of seed selection within these frameworks.

\begin{table}[h]
    \centering
    \caption{Summary of Key Epidemic Metrics Across Simulation Scenarios (1000 replicates each)}
    \label{tab:sir-metrics-summary}
    \begin{tabular}{lcccc}
        \toprule
        Metric & Temporal (Random) & Temporal (Targeted) & Static (Random) & Static (Targeted) \\
        \midrule
        Final Epidemic Size (mean, individuals) & 5.6 & 5.6 & 5.6 & $<$20 (max 2\%) \\
        Peak Infection Count (mean) & 3.6 & 3.6 & 3.6 & 15 \\
        Time to Peak Infection (time units) & 0.46 & 0.46 & 0.46 & 0.5 \\
        Epidemic Duration (time units) & 1.97 & 1.97 & 1.97 & $<$2 \\
        Early Fade-out Probability & \(\approx 1\) & \(\approx 1\) & \(\approx 1\) & \(\approx 0.99\) \\
        Doubling Time (time units) & 0.0206 & 0.0206 & 0.093 & N/A \\
        \bottomrule
    \end{tabular}
\end{table}

Table~\ref{tab:sir-metrics-summary} consolidates the primary epidemiological quantities extracted from simulations, confirming both networks exhibit near certain early fade-out and trivial epidemic expansion under the specified conditions. The slight elevation of early outbreak potential and final sizes in the static targeted case reflects the impact of hub seeding but does not alter the overarching epidemiological narrative.

\noindent Overall, this investigation substantiates the pronounced role of temporal interaction details in governing epidemic thresholds and dynamics, offering a cautionary note against over-reliance on static network approximations in epidemiological inference. Incorporating temporal and heterogeneity considerations is thus essential for precise epidemic risk modeling and response planning.

\section{Conclusion}

In this study, we conducted a rigorous quantitative comparison of SIR epidemic dynamics on an activity-driven temporal network versus a time-aggregated static network, both meticulously parameterized to represent an identical epidemiological context with a basic reproduction number \(R_0 = 3\). Employing comprehensive stochastic simulations over 1000 independent realizations for each scenario, we investigated critical epidemic metrics including final epidemic size, time to peak infection, early fade-out probability, and epidemic duration under both random and targeted initial seeding conditions.

Our results conclusively demonstrate that, despite theoretical expectations of sustained transmission at this \(R_0\), the ultra-sparse network structure coupled with the dynamic nature of temporal contacts imposes a formidable barrier to large-scale epidemic propagation. Both network settings exhibited near-universal early fade-out with trivial outbreak sizes (mean final epidemic sizes approximately \(0.56\%\) of the population) and low peak infections, independent of the seeding strategy employed. This affirms the elevated epidemic thresholds intrinsic to temporal networks arising from transient contacts and low average connectivity, reflecting intrinsic stochasticity's critical role in impeding epidemic expansion.

Comparative analysis reveals that the corresponding static aggregated network, constructed by matching average connectivity and incorporating weighted edges reflecting cumulative interaction frequency, produced epidemic outcomes quantitatively similar to the temporal network under random seeding. Although targeted seeding at highest-activity or highest-degree nodes slightly increased outbreak magnitude and early survival probability in the static scenario, such effects remained insufficient to sustain widespread epidemics, highlighting that aggregation over limited time windows does not artificially inflate epidemic risk under these parameter regimes.

Our study has several limitations deserving acknowledgement. The homogeneous activity rates and absence of community or degree heterogeneity constrain direct extrapolation to complex, real-world contact networks exhibiting clustering, repeated interactions, or memory effects, which may modulate epidemic thresholds and outbreak sizes. The discrete-time SIR framework used, while standard, abstracts from continuous-time dynamics and pathogen-specific features such as variable infectious periods or latency. Furthermore, single-seed introduction scenarios may underestimate outbreak probabilities in settings with multiple, simultaneous introductions.

These caveats notwithstanding, our findings underscore the critical necessity of incorporating temporal contact dynamics into epidemic modeling frameworks, particularly for diseases spreading over sparse social environments. Temporal network structure significantly modulates epidemic thresholds and early outbreak probabilities beyond what static aggregations can capture, emphasizing the risk of overestimating epidemic potential when relying solely on static network approximations.

For future work, we recommend extending this analysis to incorporate richer heterogeneity in node activity, community structure, and memory effects to better approximate realistic social systems. Incorporating dynamic behavioral responses and interventions, such as quarantine and vaccination strategies, integrated within temporal network models can further illuminate their impact on epidemic control. Additionally, exploring multi-seed and temporal seeding strategies can provide more precise risk assessments relevant to emerging infectious diseases.

In conclusion, the quantitative and mechanistic comparison presented here validates theoretical predictions on epidemic thresholds in temporal activity-driven networks and highlights the nuanced interplay between temporal contact structure, network sparsity, and seeding heterogeneity. This work informs the epidemiological modeling community of the paramount importance of temporal representations in accurately assessing outbreak risk, ultimately supporting more informed public health preparedness and intervention design.

\vspace{0.5cm}

\noindent\textbf{Keywords:} Temporal networks, Epidemic thresholds, SIR model, Activity-driven networks, Epidemic fade-out, Network epidemiology, Stochastic simulations

\begin{thebibliography}{99}

\bibliographystyle{plain}
\bibitem{SuyuLiu2012ContagionDynamics} Suyu Liu, Andrea Baronchelli, N. Perra, ``Contagion dynamics in time-varying metapopulation networks,'' arXiv.org, 2012.

\bibitem{Turner2018EpidemicSpreading} Michele Tizzani, Simone Lenti, Enrico Ubaldi, et al., ``Epidemic spreading and aging in temporal networks with memory,'' \textit{Physical Review E}, 2018.

\bibitem{Sun2015MemoryEffects} K. Sun, Andrea Baronchelli, N. Perra, ``Contrasting effects of strong ties on SIR and SIS processes in temporal networks,'' Unknown Journal, 2015.

\bibitem{Mancastroppa2020QuarantineStrategies} Marco Mancastroppa, R. Burioni, V. Colizza, et al., ``Active and inactive quarantine in epidemic spreading on adaptive activity-driven networks,'' \textit{Physical Review E}, 2020.

\bibitem{Yang2019MemoryAttractiveness} Mei Yang, B. Wang, Yuexing Han, ``Joint effect of individual's memory and attractiveness in temporal network on spreading dynamics,'' \textit{International Journal of Modern Physics C}, 2019.

\bibitem{Riad2019EbolaRisk} Mahbubul H. Riad, M. Sekamatte, Felix Ocom, et al., ``Risk assessment of Ebola virus disease spreading in Uganda using a two-layer temporal network,'' \textit{Scientific Reports}, 2019.

\bibitem{Nadini2020StaticBackbone} Matthieu Nadini, A. Rizzo, M. Porfiri, ``Epidemic Spreading in Temporal and Adaptive Networks with Static Backbone,'' \textit{IEEE Transactions on Network Science and Engineering}, 2020.

\bibitem{Paczuski2016} M. Paczuski, S. Boettcher, and J. G. Oliveira, ``Activity-driven temporal networks: Epidemic thresholds and dynamics,'' \textit{Phys. Rev. E}, vol.~94, no.~3, p.~032303, 2016.

\bibitem{Perra2012} N. Perra, A. Baronchelli, D. Mocanu, B. Gon\c{c}alves, R. Pastor-Satorras, and A. Vespignani, ``Activity driven modeling of time varying networks,'' \textit{Scientific Reports}, vol.~2, p.~469, 2012.

\bibitem{Cator2013} E. Cator and P. Van Mieghem, ``Susceptible-infected-susceptible epidemics on the activity driven network,'' \textit{Physical Review E}, vol.~87, no.~6, p.~062811, 2013.

\bibitem{Holme2015} P. Holme and J. Saram\"aki, \textit{Temporal Networks}, Physics Reports, 2015.

\bibitem{Eames2009} K. T. D. Eames and M. J. Keeling, \textit{Contact tracing and disease control}, Proc. R. Soc. Lond. B, 2009.

\bibitem{Valdano2015} E. Valdano et al., \textit{Analytical computation of epidemic thresholds on temporal networks}, Phys. Rev. X, 2015.

\bibitem{LloydSmith2005} J. O. Lloyd-Smith et al., \textit{Superspreading and the effect of individual variation on disease emergence}, Nature, 2005.

\bibitem{Holme2012} P. Holme, \textit{Modeling temporal networks}, The European Physical Journal B, 2012.

\bibitem{Kiss2017} I. Z. Kiss, C. Berthouze, and F. Ball, \textit{Modeling infectious disease dynamics on temporal networks: a review}, Journal of the Royal Society Interface, 2017.

\bibitem{Moinet2015} A. Moinet et al., \textit{Burstiness and spreading on temporal networks}, Phys. Rev. E, 2015.

\bibitem{Fan2025} Xiao-Nan Fan, Xuemei You, ``Impact of message fatigue and individual behavioral responses on epidemiological spread in temporal simplicial networks,'' \textit{Chinese Physics B}, 2025.

\bibitem{Kim2019} Hyewon Kim, Meesoon Ha, Hawoong Jeong, ``Impact of temporal connectivity patterns on epidemic process,'' \textit{European Physical Journal B}, 2019.
\end{thebibliography}
\newpage
\section*{Supplementary Material}
\begin{algorithm}[H]
\caption{Activity-Driven Temporal Network Generation and Aggregation}
\begin{algorithmic}[1]
\State \textbf{Input:} Number of nodes $N$, activity rate $\alpha$, contacts per activation $m$, observation window $T$
\State Initialize empty list \texttt{edge\_times} and empty dictionary \texttt{agg\_edge\_weights}
\For{$t = 0$ to $T-1$}
    \State Determine active nodes: sample each node with probability $\alpha$
    \For{each active node $i$}
        \State Sample $m$ different nodes $j \neq i$ uniformly at random
        \For{each partner $j$}
            \State Register temporal contact $(t, i, j)$ in \texttt{edge\_times}
            \State Update aggregated weight $w_{ij} \leftarrow w_{ij} + 1$ (undirected)
        \EndFor
    \EndFor
\EndFor
\State Construct aggregated static graph $G$ with nodes and edges weighted by $w_{ij}$
\State \textbf{Output:} Temporal contact list \texttt{edge\_times}, aggregated network $G$
\end{algorithmic}
\end{algorithm}

\begin{algorithm}[H]
\caption{SIR Simulation on Activity-Driven Temporal Network (Discrete-Time Agent-Based)}
\begin{algorithmic}[1]
\State \textbf{Input:} Network parameters $N, \alpha, m$, epidemiological parameters $\beta, \gamma$, simulation steps $T$, number of runs $n_{sim}$
\For{$sim = 1$ to $n_{sim}$}
    \State Initialize states $S_i=1$, $I_i=0$, $R_i=0$ for all nodes $i$
    \State Infect initial seed node(s) (random or targeted): $I_{\text{seed}}=1, S_{\text{seed}}=0$
    \For{$t = 0$ to $T-1$}
        \State Process recovery: for each infected node, recover with probability $\gamma$
        \State Determine active nodes with probability $\alpha$
        \State For each active node, sample $m$ partners for contact, form undirected contacts
        \State For each contact $(a, b)$, if $a$ infected and $b$ susceptible, infect $b$ with probability $\beta$
        \State Update states simultaneously after processing all contacts
        \State Record current $S$, $I$, $R$ counts
        \If{no infected nodes remain}
            \State \textbf{break}
        \EndIf
    \EndFor
    \State Store $S, I, R$ trajectories and final sizes
\EndFor
\State Aggregate results to compute means, confidence intervals, and outbreak metrics
\State \textbf{Output:} Epidemiological summary statistics and epidemic trajectories
\end{algorithmic}
\end{algorithm}

\begin{algorithm}[H]
\caption{SIR Simulation on Static Aggregated Network (Erd\H{o}s-R\'enyi Model)}
\begin{algorithmic}[1]
\State \textbf{Input:} Network size $N$, average degree $\langle k \rangle$, infection rate $\beta$, recovery rate $\gamma$, initial infected node (random or max degree), simulation runs $n_{sim}$
\For{$run = 1$ to $n_{sim}$}
    \State Generate Erd\H{o}s-R\'enyi graph $G$ with edge probability $p = \frac{\langle k \rangle}{N-1}$
    \State Initialize node states with single infected seed node
    \While{there exists infected nodes and time $<$ cutoff}
        \For{each infected node $i$}
            \For{each neighbor $j$ of $i$}
                \If{$j$ susceptible}
                   \State Infect with probability $\frac{\beta}{\gamma}$
                \EndIf
            \EndFor
            \State Recover node $i$ with probability $\gamma$
        \EndFor
        \State Increment time
    \EndWhile
    \State Record final epidemic size (fraction recovered)
\EndFor
\State \textbf{Output:} Distribution of final outbreak sizes across runs
\end{algorithmic}
\end{algorithm}

\begin{algorithm}[H]
\caption{Metric Extraction from Simulation Data}
\begin{algorithmic}[1]
\State \textbf{Input:} Simulation time series data with $S, I, R$ counts over time and confidence intervals
\State Identify peak infected: locate time index of maximum $I$; record mean and confidence interval bounds
\State Determine time to peak infection
\State Extract final epidemic size as recovered count at last time point with confidence bounds
\State Compute epidemic duration as last time point when infected $>1$
\State Assess early fade-out possibility using lower confidence bound of final $R$ (fade-out if $<10$ individuals)
\State Approximate doubling time during exponential growth phase by fitting exponential to early infected counts
\State \textbf{Output:} Summary metrics: peak infected, time to peak, final size, duration, fade-out flag, doubling time
\end{algorithmic}
\end{algorithm}

\begin{algorithm}[H]
\caption{Initial Condition and Parameter Setup for Model Simulations}
\begin{algorithmic}[1]
\State Set population size $N$, activity parameters $\alpha, m$, epidemiological parameters $\beta, \gamma$, $R_0$ as given
\State Calculate expected average degree $\langle k \rangle = 2 m \alpha$
\State Compute $\beta$ from $R_0$ as $\beta = R_0 \times \gamma / (2 m \alpha)$
\State Define initial infected node(s), either randomly or targeting node with maximum activity / degree
\State Ensure at least 1 infected individual for seeding
\State Setup simulation duration and number of runs
\State \textbf{Output:} Parameters and initial conditions for model simulations
\end{algorithmic}
\end{algorithm}

\begin{algorithm}[H]
\caption{Correction of Aggregated Network Construction}
\begin{algorithmic}[1]
\State Given parameters $N, \alpha, m$, compute desired average degree $\langle k \rangle$
\State Determine aggregation time window $T$ to achieve target $\langle k \rangle$: $T = \lceil \frac{\langle k \rangle}{2 \alpha m} \rceil$
\State Regenerate activity-driven network aggregation over $T$ steps
\State Construct static graph $G$ weighted by edge contact frequencies
\State Compute degree sequence, including mean and second moment
\State Save static network and weight data for subsequent analysis
\State \textbf{Output:} Corrected aggregated network and diagnostics
\end{algorithmic}
\end{algorithm}

\section*{Appendix: Additional Figures}
\addcontentsline{toc}{section}{Appendix: Additional Figures}

\begin{figure}[http]
    \centering
    \begin{subfigure}[b]{0.45\textwidth}
        \centering
        \includegraphics[width=\textwidth]{SIR_epidemic_summary.png}
        \caption*{SIR epidemic summary.png}
    \end{subfigure}
    \begin{subfigure}[b]{0.45\textwidth}
        \centering
        \includegraphics[width=\textwidth]{activity-component-size-distribution.png}
        \caption*{activity-component-size-distribution.png}
    \end{subfigure}
    \caption{Figures: SIR epidemic summary.png and activity-component-size-distribution.png}
    \label{fig:sir-epidemic-summary-png}
\end{figure}

\begin{figure}[http]
    \centering
    \begin{subfigure}[b]{0.45\textwidth}
        \centering
        \includegraphics[width=\textwidth]{activity-degree-distribution-corrected.png}
        \caption*{activity-degree-distribution-corrected.png}
    \end{subfigure}
    \begin{subfigure}[b]{0.45\textwidth}
        \centering
        \includegraphics[width=\textwidth]{activity-degree-distribution.png}
        \caption*{activity-degree-distribution.png}
    \end{subfigure}
    \caption{Figures: activity-degree-distribution-corrected.png and activity-degree-distribution.png}
    \label{fig:activity-degree-distribution-corrected-png}
\end{figure}

\begin{figure}[http]
    \centering
    \begin{subfigure}[b]{0.45\textwidth}
        \centering
        \includegraphics[width=\textwidth]{activity-temporal-contactcount.png}
        \caption*{activity-temporal-contactcount.png}
    \end{subfigure}
    \begin{subfigure}[b]{0.45\textwidth}
        \centering
        \includegraphics[width=\textwidth]{activity-weight-distribution.png}
        \caption*{activity-weight-distribution.png}
    \end{subfigure}
    \caption{Figures: activity-temporal-contactcount.png and activity-weight-distribution.png}
    \label{fig:activity-temporal-contactcount-png}
\end{figure}

\begin{figure}[http]
    \centering
    \begin{subfigure}[b]{0.45\textwidth}
        \centering
        \includegraphics[width=\textwidth]{epidemic_trajectory.png}
        \caption*{epidemic trajectory.png}
    \end{subfigure}
    \begin{subfigure}[b]{0.45\textwidth}
        \centering
        \includegraphics[width=\textwidth]{results-11.png}
        \caption*{results-11.png}
    \end{subfigure}
    \caption{Figures: epidemic trajectory.png and results-11.png}
    \label{fig:epidemic-trajectory-png}
\end{figure}

\begin{figure}[http]
    \centering
    \begin{subfigure}[b]{0.45\textwidth}
        \centering
        \includegraphics[width=\textwidth]{results-12.png}
        \caption*{results-12.png}
    \end{subfigure}
    \begin{subfigure}[b]{0.45\textwidth}
        \centering
        \includegraphics[width=\textwidth]{results-21.png}
        \caption*{results-21.png}
    \end{subfigure}
    \caption{Figures: results-12.png and results-21.png}
    \label{fig:results-12-png}
\end{figure}

\begin{figure}[http]
    \centering
    \begin{subfigure}[b]{0.45\textwidth}
        \centering
        \includegraphics[width=\textwidth]{results-22.png}
        \caption*{results-22.png}
    \end{subfigure}
    \begin{subfigure}[b]{0.45\textwidth}
        \centering
        \includegraphics[width=\textwidth]{sir_dynamics_summary.png}
        \caption*{sir dynamics summary.png}
    \end{subfigure}
    \caption{Figures: results-22.png and sir dynamics summary.png}
    \label{fig:results-22-png}
\end{figure}

\begin{figure}[http]
    \centering
    \begin{subfigure}[b]{0.45\textwidth}
        \centering
        \includegraphics[width=\textwidth]{sir_epidemic_curves.png}
        \caption*{sir epidemic curves.png}
    \end{subfigure}
    \caption{Figures: sir epidemic curves.png}
    \label{fig:sir-epidemic-curves-png}
\end{figure}
\end{document}