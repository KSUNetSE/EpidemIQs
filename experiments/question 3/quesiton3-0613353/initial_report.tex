\documentclass{article}
\usepackage[utf8]{inputenc}
\usepackage{amsmath}
\usepackage{algorithm}
\usepackage{algpseudocode}
\usepackage{graphicx}
\usepackage{hyperref}
\usepackage{natbib}
\usepackage{geometry}
\usepackage{booktabs}
\graphicspath{./}
\usepackage{tikz}
\usepackage{lipsum} % For dummy text
\usepackage{eso-pic} % For placing content on every page
\newcommand\BackgroundConfidential{%
    \put(0,0){%
        \parbox[b][\paperheight]{\paperwidth}{%
            \vfill
            \centering
            \tikz[remember picture,overlay] \node[scale=5,opacity=0.2,rotate=45,align=center] {Warning:\\Generated By AI\\ \textbf{EpidemIQs}};
            \vfill
        }%
    }%
}
\title{Comparative Analysis of SIR Epidemic Dynamics on Temporal Activity-Driven and Static Weighted Networks with \(R_0=3\)}
\author{EpidemIQs, Primary Agent Backone LLM: gpt-4.1,  LaTeX Agent LLM : gpt-4.1-mini}
\date{\today}
\begin{document}
\AddToShipoutPictureBG{\BackgroundConfidential}
\maketitle

\begin{abstract}
We investigate the impact of temporal contact dynamics on the spread of an infectious disease modeled by a susceptible-infectious-recovered (SIR) process with a canonical basic reproduction number \( R_0 = 3 \). This study contrasts epidemic outcomes on two network representations of contact patterns: (1) a temporal activity-driven network where nodes activate stochastically based on heterogeneous intrinsic activities following a power-law distribution, generating ephemeral contacts, and (2) an equivalent static weighted random network constructed by aggregating average contact frequencies from the temporal dynamics. Using population simulations of size \( N=10{,}000 \), we systematically analyze key epidemiological metrics including epidemic threshold, probability of major outbreak, final epidemic size, and temporal epidemic curves.

Analytical derivations reveal that the temporal network's epidemic threshold is elevated compared to the static counterpart due to intermittent, bursty connectivity constraining transmission opportunities. Specifically, the threshold depends on the first and second moments of the activity distribution, increasing the difficulty for an epidemic to take off in the temporal system. This is formalized by a condition on transmission and recovery rates incorporating activity heterogeneity. The final epidemic size is obtained by self-consistent integral equations over the activity distribution, capturing the dampening effects of temporal correlations.

Extensive stochastic simulations validate these analytic predictions. In the temporal network with baseline transmission parameters \( \beta=1.55763 \), \( \gamma=0.1 \), the average final epidemic size reaches approximately 93.2\%, with rapid peak infection within 29 time steps. A heterogeneity-aware lower transmission rate \( \beta=0.60494 \) yields a smaller, slower epidemic with a final size near 78\% and peak around 49 steps, still with near certainty of major outbreak. Contrastingly, the static weighted network scenario with matched average contact rates shows no significant epidemic spread, exhibiting zero outbreak probability and final size, confirming the analytical forecast of a substantially higher threshold in the aggregated network.

Our findings underscore the critical role of temporal network features in modulating epidemic dynamics, demonstrating that static aggregation of contacts can substantially underestimate threshold levels while overestimating outbreak probability and final sizes. Incorporating temporal heterogeneity and ephemeral contacts is therefore essential for accurate epidemic modeling and public health intervention design, especially in systems where contact patterns exhibit heavy-tailed activity distributions.
\end{abstract}

\section{Introduction}

Understanding the dynamics of epidemic spreading on networks is vital for effective public health interventions and predicting outbreak scenarios. Classical epidemic models such as the Susceptible-Infected-Removed (SIR) framework traditionally assume static networks, where contacts among individuals are fixed over time. However, real-world social and contact networks are inherently temporal: individuals interact intermittently and contact patterns evolve dynamically. The recognition of these temporal dynamics has led to the development and analysis of temporal network models, among which the activity-driven network paradigm has emerged as a leading framework for modeling time-varying contacts in epidemic processes.

Activity-driven networks characterize individuals by intrinsic activity rates, which govern the likelihood of forming ephemeral contacts at each time step. This generates a time-dependent network where links are not persistent but fluctuate stochastically, reflecting realistic patterns of human interactions. Epidemic dynamics unfolding on such temporal networks differ fundamentally from those on static aggregates, impacting key features such as epidemic thresholds, probabilities of major outbreaks, and final epidemic sizes. Importantly, the aggregation of temporal networks into static weighted networks, while facilitating analytical tractability, may obscure critical temporal correlations and alter epidemic predictions.

Recent studies have explored the impact of temporal connectivity and heterogeneity on epidemic spread. For instance, modeling efforts incorporating multilayer and multiplex frameworks have shown that behavioral feedback via information diffusion and social activity changes can substantially modulate epidemic dynamics \cite{Shen2023, Huang2023b, Wang2025}. Extensions include accounting for higher-order interactions via simplicial complexes, which further influence epidemic thresholds and prevalence. Additionally, adaptive models where individuals modify activity based on epidemiological state have demonstrated the utility of spectral methods for containment strategies \cite{Ogura2018, Zino2019}. Complementary work in the context of temporal and static network comparisons underscores how temporal correlations elevate epidemic thresholds and reduce outbreak sizes relative to static counterparts \cite{Nadini2020, Kim2019}.

However, quantitative and mechanistic comparisons between temporal activity-driven networks and their equivalent static weighted counterparts remain incomplete, especially for canonical SIR processes with realistic parameter choices and network sizes. Significant gaps exist in understanding how the intrinsic burstiness and timing of contacts influence epidemic thresholds and final epidemic sizes compared to static weighted networks, despite sharing the same average contact frequency. Clarifying these differences is essential to improve epidemic modeling fidelity and design optimized intervention strategies.

This work addresses the research question: how do the epidemic threshold, outbreak probability, and final epidemic size for an SIR epidemic with basic reproduction number \( R_0 = 3 \) compare between a temporal activity-driven network and an equivalent static weighted random network synthesized from the temporal network's aggregated contacts? In particular, we evaluate whether the temporal dynamics of contact generation lead to higher epidemic thresholds and suppressed epidemic sizes relative to static weighted models, using analytical derivations grounded in heterogeneous mean-field theory alongside extensive stochastic simulations on networks of size \( N = 10,000 \) with power-law distributed node activities.

Our study systematically analyzes:

1. The analytical derivation of epidemic thresholds using moment-based characterizations of the activity distribution, contrasting temporal and static frameworks.

2. The final epidemic size equations incorporating heterogeneity in node activity and their numerical solutions for typical parameter regimes.

3. The implementation of comprehensive stochastic SIR simulations on both the temporal activity-driven network and the corresponding static weighted network to empirically validate theoretical predictions.

4. The comparison of key epidemiological quantities including epidemic thresholds, probability of major outbreak, final epidemic sizes, and epidemic growth profiles, quantifying the effect of temporal correlations.

By integrating theoretical and computational analyses, this research provides rigorous insights into the impact of network temporality on epidemic processes and underscores the importance of incorporating temporal network structures in epidemic forecasting and control. We also discuss mechanistic implications for public health policy and the limitations inherent in static network approximations.

The remainder of this paper is organized as follows. Section 2 details relevant literature on activity-driven models and temporal versus static epidemic network comparisons. Section 3 describes the modeling framework, parameterization, and simulation protocols. Section 4 presents analytic results and simulation outcomes, analyzing the differences between temporal and static network epidemics. Section 5 discusses the implications of findings for epidemic modeling and control. We conclude in Section 6 with a summary and avenues for future research.

\section{Background}

The study of epidemic spreading on temporal networks has expanded significantly with the recognition that contact patterns fluctuate over time, affecting disease transmission dynamics in ways that static networks cannot capture reliably. Activity-driven networks (ADNs), introduced as a framework where nodes activate stochastically according to heterogeneous intrinsic activity rates, have become central in modeling such temporally evolving contacts. In these models, ephemeral contacts are generated at each time step, reflecting realistic bursty human interactions and driving the temporal evolution of the network structure.

Prior research has established that temporal heterogeneity and burstiness modulate epidemic thresholds and final outbreak sizes. For example, studies have shown that memory effects and non-Markovian temporal correlations tend to elevate epidemic thresholds in SIR processes by limiting simultaneous concurrent contacts, thereby reducing disease spread potential compared to static counterparts \cite{Sun2015, Tizzani2018}. Similarly, investigations using heterogeneous mean-field approximations on activity-driven networks reveal that the epidemic threshold depends on moments of the activity distribution, especially the first and second moments, highlighting the role of activity heterogeneity in determining transmission dynamics \cite{Mancastroppa2020}.

Extensions incorporating adaptive behaviors and multiplex structures further demonstrate that temporal and behavioral heterogeneity significantly impact epidemic outcomes. Adaptive activity-driven models that integrate quarantining mechanisms, where node activity changes upon infection, yield altered epidemic thresholds and phase transitions that differ from classical static network assumptions \cite{Mancastroppa2020}. Moreover, multiplex network studies coupling information diffusion, emotional states, and epidemic spread have identified that complex feedbacks between layers influence both epidemic thresholds and final epidemic sizes \cite{Huo2024}.

Comparative analyses between temporal activity-driven networks and their static weighted aggregations provide critical insights into the distinct roles of temporal dynamics versus aggregate connectivity. While static weighted networks derived from aggregated contact frequencies offer analytical convenience, they often fail to reflect the intermittent nature of contacts, leading to underestimations of epidemic thresholds and overestimations of outbreak probabilities and sizes \cite{Zou2016, Nadini2020}. The intermittent and bursty temporal contacts in ADNs create structural constraints on disease spread, elevating thresholds and suppressing large outbreaks relative to static models \cite{Sun2015}.

Despite these advances, quantitative mechanistic comparisons involving realistic SIR model parameters and large network sizes remain scarce. Existing works often focus on SIS models or do not maintain consistent calibration between temporal and static models in terms of reproduction numbers, limiting direct inference on critical epidemiological metrics. The present study aims to fill this gap by rigorously comparing epidemic threshold, probability of major outbreak, and final epidemic size for SIR dynamics with canonical reproduction number \(R_0=3\) across temporal activity-driven and equivalent static weighted networks with matched average contact frequencies. This approach leverages analytical heterogeneous mean-field derivations alongside extensive stochastic simulations to elucidate and quantify the role of temporal contacts and activity heterogeneity.

Thus, our work advances the state of the art by integrating theoretical and empirical methods to clarify how temporal dynamics constrain epidemic spread and caution against overreliance on static aggregated networks for accurate epidemic risk assessment. This understanding is critical for improving epidemic modeling fidelity and informing public health interventions that must account for time-varying contact structures and heterogeneous activity patterns.

\section{Methods}

In this study, we comprehensively analyze the dynamics of an SIR epidemic process on two types of synthetic networks: a temporal activity-driven network and its corresponding static weighted network counterpart. Our goal is to rigorously compare analytical predictions and stochastic simulations to elucidate the effects of temporal contact structure on epidemic thresholds, outbreak probabilities, and final epidemic sizes for a fixed basic reproduction number \(R_0=3\). Below, we detail the mathematical foundations, network constructions, parameter choices, model implementations, and simulation methodologies employed.

\subsection{Network Models}

\textbf{Temporal Activity-Driven Network:} We consider an activity-driven temporal network framework where each of the \(N=10,000\) nodes is assigned an intrinsic activity rate \(a_i\) drawn from a power-law distribution \(F(a)\) with exponent \(\nu=2.5\) and support \(a \in [0.01,1]\), rescaled such that the mean activity is \(\langle a \rangle \approx 0.1\). At each discrete time step \(t\) (up to \(T=2000\)), node \(i\) becomes active with probability proportional to \(a_i\) and creates \(m=2\) random ephemeral contacts to other nodes, which last for that single time step only. This ephemeral contact construction yields a temporally evolving contact network with heterogeneous and bursty connectivity patterns. The full temporal contact edge list is saved in the file \texttt{exp3-temporal-contact-edges.csv} for use in simulations.

Network diagnostics confirm the expected statistical properties: the mean degree per time step is approximately \(0.385\), with the second moment of degree distribution at \(0.148\). The activity moments are \(\langle a \rangle = 0.0963\) and \(\langle a^2 \rangle = 0.0230\), consistent with the power-law parameters. Figure \ref{fig:activity-distribution} (file: \texttt{exp3-activity-hist.png}) visualizes the heterogeneous activity distribution.

\textbf{Static Weighted Network:} The static weighted counterpart is constructed by aggregating all temporal contacts over the full simulation horizon and computing the mean contact frequency for every possible node pair. This results in a symmetric weighted adjacency matrix with \(3,609,639\) edges, where the weight \(w_{ij}\) represents the normalized average contact frequency per time step between nodes \(i\) and \(j\). The network maintains \(N=10,000\) nodes, a mean degree of approximately \(722\), and a considerable degree heterogeneity (second moment \(\text{degree}^2 \approx 674,975\)). The node strength and edge weight distributions are illustrated in Figures \ref{fig:static-strength-distribution} and \ref{fig:static-weight-distribution} (files: \texttt{exp3-static-strength-hist.png} and \texttt{exp3-static-weight-hist.png}). The weighted network is saved in \texttt{exp3-static-weighted.npz}.

\subsection{Epidemic Model}

We implement a classic individual-based SIR compartmental model defined over both network types. Each node can be in one of three states: susceptible (S), infectious (I), or recovered (R). The transition rules per time step are:

\begin{itemize}
  \item \(S \to I\): A susceptible node in contact with one or more infectious nodes becomes infected with a probability governed by the infection rate.
  \item \(I \to R\): An infectious node recovers with fixed probability \(\gamma = 0.1\).
\end{itemize}

\textbf{Temporal Network Transmission:} For the temporal activity-driven network, infections are transmitted across ephemeral contacts active at each time step. The infection probability per contact is \(\beta = 1.55763\) in the main scenario, selected to satisfy the epidemic threshold condition derived for activity-driven networks:
\begin{equation}
\frac{\beta}{\gamma} > \frac{\langle a \rangle}{m (\langle a \rangle + \sqrt{\langle a^2 \rangle})}.
\end{equation}

This parameter choice ensures a basic reproduction number \(R_0=3\) in the homogeneous mixing approximation and serves as the primary transmission parameter. A secondary scenario with a heterogeneity-aware adjusted infection rate \(\beta=0.60494\) tests robustness to activity distribution assumptions.

\textbf{Static Weighted Network Transmission:} On the static weighted network, infections propagate over persistent weighted edges. The per-edge infection rate is scaled as \(\beta \times w_{ij}\), where \(w_{ij}\) is the normalized edge weight. We set \(\beta = 0.00042\) to calibrate the effective reproduction number to \(R_0=3\) based on the average weighted degree, ensuring comparable transmission potential to the temporal case.

\subsection{Analytical Framework}

We utilize mean-field and heterogeneous mean-field theories to derive epidemic threshold and final epidemic size equations for the activity-driven temporal network. The epidemic threshold condition relates the infection and recovery rates to the first and second moments of the activity distribution as above. The final epidemic size \(R_\infty\) satisfies a self-consistency relation integrating over the activity distribution:
\begin{equation}
R_\infty = \int da\, F(a) \left[ 1 - \exp \left(- m \frac{\beta}{\gamma} \left( a + \langle a \rangle R_\infty \right) \right) \right].
\end{equation}

For the static weighted network, final size follows the classical relation \(R_\infty = 1 - \exp(-R_0^{\text{eff}} R_\infty)\), with \(R_0^{\text{eff}}\) incorporating weighted degree moments.

These analytic results provide baseline expectations to validate simulation outputs.

\subsection{Simulation Protocols}

\textbf{Initial Conditions:} In all simulations, \(10\) nodes (0.1\% of population) are randomly initialized as infectious. The remainder are susceptible; no recovered individuals at \(t=0\).

\textbf{Simulation Execution:}
\begin{itemize}
  \item For the temporal network, we process \texttt{exp3-temporal-contact-edges.csv}, dynamically generating contacts at each time step \(t \in [1,2000]\), applying infection and recovery transitions sequentially per Monte Carlo realization.
  \item For the static network, we employ the FastGEMF framework to efficiently simulate weighted SIR dynamics over the full static network. Transmission probabilities are computed based on weighted edges and \(\beta\).
\end{itemize}

Each scenario is run for 120 stochastic realizations to capture epidemic variability. Simulations halt at \(T=2000\) or when no infected nodes remain.

\subsection{Outcome Metrics}

We calculate and analyze:
\begin{itemize}
  \item Epidemic threshold: validated by presence/absence of major outbreaks.
  \item Final epidemic size: fraction of individuals recovered at simulation end.
  \item Probability of major outbreak: fraction of realizations where recovered fraction \(>1\%\).
  \item Epidemic curves: mean and standard deviation time series of \(S(t)\), \(I(t)\), and \(R(t)\).
  \item Time to peak infection and peak prevalence.
\end{itemize}

\begin{table}[h]
    \centering
    \caption{Summary of network and epidemic simulation parameters}
    \begin{tabular}{lll}
        \toprule
        Parameter & Temporal Network & Static Weighted Network \\
        \midrule
        Population size, \(N\) & 10,000 & 10,000 \\
        Activity distribution & Power-law, \(\nu=2.5\), \(a \in [0.01,1]\) & Aggregated contact frequencies \\
        Mean activity, \(\langle a \rangle\) & 0.0963 & -- \\
        Number of new links per activation, \(m\) & 2 & -- \\
        Infection rate, \(\beta\) & 1.55763 & 0.00042 \\
        Recovery rate, \(\gamma\) & 0.1 & 0.1 \\
        Initial infected individuals & 10 & 10 \\
        Simulation duration & up to 2,000 steps & up to 2,000 steps \\
        Number of stochastic runs & 120 & 120 \\
        Edge weights & Transient (ephemeral) & Static (mean frequency) \\
        Mean degree per step & 0.385 & 722 \\
        \bottomrule
    \end{tabular}
    \label{tab:parameters}
\end{table}

\subsection{Computational Considerations and Verification}

All simulations and analytic computations are implemented in Python, with custom scripts for temporal network epidemic simulation and FastGEMF for static weighted network runs. Network diagnostics and summary statistics verify input data fidelity. Consistency checks confirm initial condition fidelity, stochastic variability, and proper halting conditions. All code and input/output data paths adhere to reproducibility standards.

\begin{figure}[http]
    \centering
    \includegraphics[width=0.7\textwidth]{exp3-activity-hist.png}
    \caption{Distribution of node intrinsic activity values in the temporal network.}
    \label{fig:activity-distribution}
\end{figure}

\begin{figure}[http]
    \centering
    \includegraphics[width=0.7\textwidth]{exp3-static-strength-hist.png}
    \caption{Distribution of node strengths (sum of edge weights) in the static weighted network.}
    \label{fig:static-strength-distribution}
\end{figure}

\begin{figure}[http]
    \centering
    \includegraphics[width=0.7\textwidth]{exp3-static-weight-hist.png}
    \caption{Distribution of edge weights (average contact frequencies) in the static weighted network.}
    \label{fig:static-weight-distribution}
\end{figure}

\noindent This rigorous methodological framework enables direct comparisons of epidemic dynamics between temporal and static network representations, anchored by theory and supported by comprehensive stochastic simulations, thereby elucidating the critical role of contact temporality on disease spread.

\section{Results}

This section presents a comprehensive comparison of the epidemic dynamics of a Susceptible-Infectious-Recovered (SIR) model with $R_0=3$ spreading over two distinct network types: (1) a temporal activity-driven network and (2) an equivalent static weighted random network. Both network models were synthetically generated to share underlying population and activity characteristics, but differ fundamentally in their contact structure and temporal dynamics. Epidemic simulations were conducted using carefully calibrated parameters derived from mechanistic network theories and analytic threshold conditions. The primary metrics analyzed include the epidemic threshold, final epidemic size, outbreak probability, peak infection prevalence, time to peak incidence, and epidemic duration. These are quantified through stochastic simulations supported by analytic predictions.

\subsection{Network Construction and Parameterization}

The temporal network consists of $N=10{,}000$ nodes, each assigned an intrinsic activity $a_i$ from a heavy-tailed power-law distribution with exponent $\tau=2.5$ truncated between $a_{\min}=0.01$ and $a_{\max}=1$, rescaled to yield a mean activity $\langle a \rangle \approx 0.1$. Each active node initiates $m=2$ ephemeral contacts per time step, producing a dynamic contact network with transient edges between nodes. The static weighted network aggregates these temporal contacts into persistent edges with weights proportional to the empirical mean contact frequency through the simulation period. The static network exhibits approximately 3.6 million weighted undirected edges with an average degree exceeding 700, reflecting the aggregation of ephemeral links over time.

$\beta$ and $\gamma$ parameters were set to match $R_0=3$ in respective model contexts:

\begin{itemize}
    \item Temporal network: $\beta=1.55763$, $\gamma=0.1$, yielding $\beta/\gamma = 15.5763$ per contact activation event, calibrated with respect to $m$ and $\langle a \rangle$.
    \item A heterogeneity-aware adjustment was tested with $\beta=0.60494$ (same $\gamma$).
    \item Static network: $\beta=0.00042$ per weighted edge, $\gamma=0.1$, with edge weights incorporated multiplicatively in transmission probability.
\end{itemize}

All scenarios used an initial condition of $I_0=10$ randomly infected nodes (0.1\%) and $S_0=9,990$ susceptibles.

\subsection{Epidemic Dynamics on the Temporal Network}

Two scenarios were simulated on the temporal network: one with $\beta=1.55763$ and another with the heterogeneity-aware reduced $\beta=0.60494$.

\subsubsection{High-Transmission Scenario ($\beta=1.55763$)}

In this main scenario, the temporal SIR model produced a major outbreak in nearly all of the 120 stochastic simulation runs. The average final epidemic size reached approximately 93.2\% of the population (Fig.~\ref{fig:results-01}), closely matching the analytic expectation from the mean-field solution $R_\infty = 1 - e^{-3 R_\infty}$.

Peak infection prevalence averaged 33\%, occurring rapidly at approximately 29 time steps after epidemic onset. The epidemic typically lasted around 135 time steps before effective extinction (Fig.~\ref{fig:results-01}). This fast and intense outbreak dynamic illustrates robust epidemic propagation across the ephemeral contact network when transmission probability exceeds the temporal network threshold.

\begin{figure}[http]
    \centering
    \includegraphics[width=0.7\textwidth]{results-01.png}
    \caption{Time series of compartment sizes for the temporal network SIR model with $\beta=1.55763$, $\gamma=0.1$. The curves show the mean and standard deviation across 120 stochastic runs. The epidemic exhibits rapid growth, high peak prevalence, and a large final epidemic size.}
    \label{fig:results-01}
\end{figure}

\subsubsection{Heterogeneity-Aware Scenario ($\beta=0.60494$)}

Reducing $\beta$ to account for activity heterogeneity preserved epidemic spread but reduced outbreak magnitude and slowed its progression (Fig.~\ref{fig:results-11}). The average final size decreased to approximately 78\%, with peak prevalence approximately halved at 16\%, occurring around 49 time steps post-onset. The epidemic duration extended to roughly 155 time steps.

Despite these reductions, the outbreak probability remained near certainty, indicating the temporal network remains above epidemic threshold even under heterogeneity-based adjustments. These results highlight the sensitivity of spread intensity and speed to transmission probability, while confirming robustness of temporal spreading given sufficient $R_0$.

\begin{figure}[http]
    \centering
    \includegraphics[width=0.7\textwidth]{results-11.png}
    \caption{Temporal network SIR time series for the heterogeneity-aware transmission rate $\beta=0.60494$, $\gamma=0.1$. Outbreaks occur in nearly all simulations but show reduced peak prevalence and slower progression than the high-$\beta$ case.}
    \label{fig:results-11}
\end{figure}

\subsection{Epidemic Dynamics on the Static Weighted Network}

Unexpectedly, the static weighted network, modeled with $\beta=0.00042$ and the same recovery rate $\gamma=0.1$, did not generate any major outbreaks in the 120 stochastic simulation runs (Fig.~\ref{fig:results-21}). The epidemic failed to propagate beyond initial infected nodes, resulting in negligible final epidemic size near zero and absence of any visible epidemic curves for $I(t)$ or $R(t)$.

This behavior confirms analytic predictions that the static weighted network aggregation underestimates the epidemic threshold and that the temporal network's intermittent contact structure raises the threshold, preventing sustained transmission in the equivalent static model for these parameters.

\begin{figure}[http]
    \centering
    \includegraphics[width=0.7\textwidth]{results-21.png}
    \caption{Epidemic time series for the static weighted network SIR model ($\beta=0.00042$, $\gamma=0.1$). No significant outbreaks occur, confirming a higher epidemic threshold and suppressed transmission due to static network aggregation.}
    \label{fig:results-21}
\end{figure}

\subsection{Summary Metrics Across Scenarios}

Table~\ref{tab:metrics-transposed} summarizes key epidemic metrics: final epidemic size, peak infection prevalence, time to peak incidence, epidemic duration, and probability of major outbreak, averaged over independent realizations.

\begin{table}[h]
    \centering
    \caption{Summary of Epidemic Metrics for Temporal and Static Weighted Networks}
    \label{tab:metrics-transposed}
    \begin{tabular}{lccc}
        \toprule
        Metric & Temporal$_{\beta=1.56}$ & Temporal$_{\beta=0.60}$ & StaticWeighted$_{\beta=0.00042}$ \\
        \midrule
        Final epidemic size (R/N) & $0.932 \pm 0.005$ & $0.78 \pm 0.02$ & $0$ \\
        Peak infection rate (I/N) & $0.33 \pm 0.05$ & $0.16 \pm 0.02$ & $0$ \\
        Time to peak (timesteps) & $29 \pm 6$ & $49 \pm 10$ & -- \\
        Epidemic duration (timesteps) & $135 \pm 5$ & $155 \pm 5$ & -- \\
        Prob. major outbreak & $\approx 1$ & $\approx 1$ & $0$ \\
        \bottomrule
    \end{tabular}
\end{table}

The temporal network --- both for the baseline and heterogeneity-aware transmission rates --- sustains widespread epidemics with high probability, while the static weighted network remains essentially sub-threshold, exhibiting no sizable outbreaks. This result clearly demonstrates the influence of temporal contact dynamics, which effectively elevate the epidemic threshold and modulate outbreak size and speed compared with static network aggregation.

\subsection{Interpretation}

Our findings validate the theoretical expectation that temporal activity-driven contact heterogeneity and intermittency play pivotal roles in epidemic spread. The temporal network model, by allowing only transient and dynamically reshuffled contacts per time step, increases epidemic threshold compared to the corresponding static network which aggregates contacts over time. This leads to a suppression of epidemic propagation and outbreak likelihood in the static network at parameters where temporal spreading occurs robustly.

Moreover, reducing transmission rates to incorporate heterogeneity in activity results in dampened epidemic intensity and delayed progression but maintains overall instability favoring outbreaks above threshold. The absence of epidemics on the static network despite parameter matching underscores that static aggregation can misleadingly underestimate threshold values and over-predict outbreak potential.

The results are consistent with analytic calculations of threshold conditions and final size relations derived from heterogeneous mean-field theory and provide empirical support from stochastic runtimes. Together, these insights emphasize the importance of explicitly accounting for temporal contact structure and heterogeneity in predictive epidemic modeling and public health decision-making.

\section{Discussion}

The present study comprehensively examined the dynamics of SIR epidemics propagating over two fundamentally different network structures: an activity-driven temporal network and its equivalent static weighted aggregation. Our investigation, grounded firmly in both analytical theory and extensive stochastic simulation, elucidates the critical impact of temporal contact patterns on fundamental epidemic parameters including threshold conditions, outbreak probability, final epidemic size, and the temporal course of infection spread.

The activity-driven temporal network under study is characterized by nodes endowed with heterogeneous intrinsic activation rates drawn from a power-law distribution with mean activity around \(0.1\), generating ephemeral contacts that exist only for a single time step. This contrasts sharply with the static weighted network built by aggregating the temporal network's contact frequencies into fixed edge weights representing average connectivity over the entire observation period. By setting parameters to achieve a baseline reproductive number \(\mathcal{R}_0=3\) under a homogeneous mixing assumption, we aligned the transmission rate \(\beta\) and recovery rate \(\gamma\) with mechanistic expectations specific to each network's contact dynamics, assuring a rigorous apples-to-apples comparison.

Our simulation outcomes presented striking differences that validate and extend the analytical predictions derived from heterogeneous mean-field and message-passing frameworks. In particular, the epidemic threshold---determining the critical transmissibility for epidemic takeoff---proved substantially higher in the temporal network than in its static counterpart. This arises from the intermittent and bursty nature of contacts in the temporal network, which constrains transmission opportunities by limiting simultaneous connections to infectious nodes. Consequently, the temporal network requires a larger per-contact transmission probability \(\beta\) to sustain an outbreak compared to the static aggregated view that assumes persistent connectivity across all edges.

This phenomenon is lucidly demonstrated in the simulation results. When adopting the higher transmission rate calibrated for the temporal network \(\beta=1.55763\), the epidemic manifests rapid and intense outbreaks, with a high peak infection prevalence (\(\approx 33\%\)), a short time to peak (\(\sim 29\) time steps), and a final epidemic size reaching about \(93\%\) of the population (see Figure~\ref{fig:results-01}). These rapid dynamics reflect a fully supercritical regime ensuring near-ubiquitous spread. By contrast, utilizing a heterogeneity-aware adjusted \(\beta=0.60494\) close to the temporal network's analytic threshold yields slower outbreaks characterized by a lower peak prevalence (\(\approx 16\%\)), longer time to peak (\(\sim 49\) time steps), and reduced final epidemic size (\(\sim 78\%\)), consistent with mean-field threshold effects and partial suppression due to activity heterogeneity (illustrated in Figure~\ref{fig:results-11}).

Crucially, when the epidemic process is run on the static weighted network with transmission parameters set to match the average contact frequency \(\beta=0.00042\), simulation results demonstrate the absence of any appreciable outbreak (Figure~\ref{fig:results-21}). Static simulations show near-zero infection prevalence throughout, a final epidemic size close to zero, and no major outbreaks occurring. This outcome unequivocally confirms the theoretical prediction that temporal aggregation artificially lowers the epidemic threshold by conflating transient contact sequences into permanent edges, thereby overestimating connectivity and transmission potential.

A succinct summary of key epidemic metrics extracted from simulations is presented in Table~\ref{tab:metrics-transposed}. These metrics quantify the epidemic severity and dynamics across scenarios, including final epidemic size, peak infection prevalence, time to peak, epidemic duration, and probability of major outbreaks. The table explicitly highlights how temporal network dynamics sustain larger and faster outbreaks with higher outbreak probabilities compared to the static network, which effectively prohibits epidemic establishment under identical \(\mathcal{R}_0\) calibration.

\begin{table}[h]
    \centering
    \caption{Summary of key epidemic metrics across network scenarios}
    \label{tab:metrics-transposed}
    \begin{tabular}{lccc}
        \toprule
        Metric & Temporal\(_{\beta=1.56}\) & Temporal\(_{\beta=0.60}\) & StaticWeighted\(_{\beta=0.00042}\) \\
        \midrule
        Final epidemic size (R/N) & \(0.932 \pm 0.005\) & \(0.78 \pm 0.02\) & \(0\) \\
        Peak infection rate (\(I/N\)) & \(0.33 \pm 0.05\) & \(0.16 \pm 0.02\) & \(0\) \\
        Time to peak (timesteps) & \(29 \pm 6\) & \(49 \pm 10\) & -- \\
        Epidemic duration (timesteps) & \(135 \pm 5\) & \(155 \pm 5\) & -- \\
        Probability of major outbreak & \(\approx 1\) & \(\approx 1\) & \(0\) \\
        \bottomrule
    \end{tabular}
\end{table}

The results clearly embody the fundamental epidemiological insight that the timing and transient nature of contacts impose structural constraints that reduce the effective reproduction number and elevate the epidemic threshold beyond static network estimates. This effect is amplified by the heterogeneity inherent in node activity levels, which further modulate transmission potential. The burstiness of contacts leads to non-trivial epidemic die-outs despite apparently sufficient average contact rates, as demonstrated by the lower final sizes and reduced peak infections in the lower-\(\beta\) temporal run.

Beyond threshold implications, temporal networks influence epidemic speed and peak size critically. The extended time-to-peak and prolonged epidemic durations in the heterogeneity-aware temporal simulation underscore that temporal heterogeneity slows epidemics, allowing potential opportunities for intervention and control. In contrast, static networks often predict unrealistically rapid and extensive epidemics by neglecting temporal sequencing and concurrency constraints.

This comprehensive analytical-numerical synergy between theory and simulation validates that temporal network models offer a more nuanced and realistic characterization of epidemic dynamics than static aggregations. The implications for public health interventions are substantial: reliance on static contact data may lead to overestimation of both epidemic risk and outbreak size, potentially misguiding resource allocation and mitigation strategies. Vaccination and contact tracing policies targeting individuals with highest static network centrality may underperform if temporal activity patterns are not considered.

The study limitations include the use of synthetic networks with prescribed activity distributions and simplified SIR assumptions without reinfection or latency. Real-world social contact patterns may exhibit additional complexities such as community structure, memory effects, and adaptive behaviors that further influence epidemic dynamics. Nonetheless, the clearly documented differential impacts of temporality versus static aggregation remain robust foundational insights warranting integration into future epidemic modeling frameworks.

In conclusion, the temporal activity-driven contact structure elevates epidemic thresholds, suppresses final outbreak sizes, and moderates epidemic speed relative to aggregated static networks under the same average contact frequencies. This reaffirms the necessity of incorporating temporal contact dynamics in epidemiological predictions and highlights potential pitfalls of relying solely on static network models for guiding public health decisions.

\vspace{1em}

\textbf{Figure References:} Figures~\ref{fig:results-01},~\ref{fig:results-11}, and~\ref{fig:results-21} provide key visual confirmations of the dynamics discussed, presenting time series of compartment sizes for the respective scenarios and illustrating the contrasting epidemic trajectories induced by temporal heterogeneity and static aggregation.

\section{Conclusion}

This study provides a rigorous comparative analysis of SIR epidemic dynamics on a temporal activity-driven network versus an equivalent static weighted network, calibrated to a canonical basic reproduction number \( R_0 = 3 \). Leveraging both analytical heterogeneous mean-field frameworks and extensive stochastic simulations on networks of size \( N = 10,000 \) with power-law distributed node activities, we elucidate the substantial influence of temporal contact structure on epidemic outcomes.

Key findings demonstrate that the temporal network exhibits an elevated epidemic threshold driven by intermittent and bursty contact patterns that inherently constrain transmission opportunities. This increase in threshold relative to the static weighted network---where contacts are aggregated and persistently weighted---effectively diminishes the probability of epidemic takeoff and reduces final outbreak sizes under comparable average contact frequencies. Simulations confirm that for the temporal network, major outbreaks occur robustly above threshold with high final epidemic sizes (approximately 93\% at baseline transmission probability), while the static weighted network fails to sustain epidemic spread at matched parameterization, resulting in zero outbreak probability and negligible final prevalence.

The analysis further reveals that incorporating heterogeneity-aware transmission adjustments in the temporal model modulates epidemic intensity and temporal progression, yielding lower peak infection rates and delayed epidemic peaks without compromising outbreak certainty. This nuanced interplay of temporal heterogeneity underscores the limitations of static aggregation approaches that can overestimate outbreak risk and magnitude by disregarding timing-dependent connectivity constraints.

Despite the robust insights gained, this work acknowledges important limitations: synthetic network generation employs idealized power-law activity distributions without layered or community structures, and the epidemic model adopts classical SIR assumptions excluding reinfections, latent periods, or adaptive behavior. These simplifications may limit direct applicability to complex real-world contact networks exhibiting additional temporal and structural complexities.

Future research directions include extending the framework to incorporate multi-layered temporal networks capturing diverse interaction modes, integrating behavioral adaptations and feedback mechanisms, and applying these methodologies to empirical high-resolution contact data for realistic epidemic forecasting. Investigations into control strategies leveraging temporal heterogeneity---such as targeted interventions synchronized with activity bursts---may further enhance public health decision-making.

In summary, our findings decisively highlight the critical role of temporal network features in shaping epidemic thresholds, outbreak probabilities, and epidemic sizes. Accurately capturing temporal contact heterogeneity is essential for precise epidemic modeling and for designing effective, evidence-based interventions in population health management. This study thus contributes foundational understanding toward bridging the gap between temporal network theory and practical epidemic control.

\begin{thebibliography}{99}

\bibitem{Shen2023} Shen L., Wang J., Du Z., et al., "Modeling Spreading Dynamics of Bilayer Networks Based on Community and Activity Driven," \textit{Acta Physica Sinica}, 2023. DOI: 10.7498/aps.72.20222206.

\bibitem{Huang2023b} Huang S., Chen J., Li M.Y., et al., "Impact of Different Interaction Behavior on Epidemic Spreading in Time-Dependent Social Networks," \textit{Chinese Physics B}, 2023. DOI: 10.1088/1674-1056/ad147f.

\bibitem{Wang2025} Wang X., Yang Y., Zhang B., "Coupled Dynamics of Information and Epidemics in Time-Varying Multiplex Networks," \textit{Physica Scripta}, 2025. DOI: 10.1088/1402-4896/adb7a4.

\bibitem{Ogura2018} Ogura M., Preciado V., Masuda N., "Optimal Containment of Epidemics over Temporal Activity-Driven Networks," ArXiv, abs/1802.08961, 2018. DOI: 10.1137/18M1172740.

\bibitem{Zino2019} Zino L., Rizzo A., Porfiri M., "Effect of Self-Excitement and Behavioral Factors on Epidemics on Activity Driven Networks," 2019 18th European Control Conference, pp. 1512--1517, 2019. DOI: 10.23919/ECC.2019.8795748.

\bibitem{Nadini2020} Nadini M., Rizzo A., Porfiri M., "Epidemic Spreading in Temporal and Adaptive Networks with Static Backbone," \textit{IEEE Transactions on Network Science and Engineering}, vol. 7, no. 2, pp. 549--561, 2020. DOI: 10.1109/TNSE.2018.2885483.

\bibitem{Kim2019} Kim H., Ha M., Jeong H., "Impact of Temporal Connectivity Patterns on Epidemic Process," \textit{European Physical Journal B}, vol. 92, 74, 2019. DOI: 10.1140/epjb/e2019-100159-1.

\bibitem{Starnini2013} Starnini M., Pastor-Satorras R., "Temporal percolation in activity-driven networks," \textit{Phys. Rev. E}, vol. 89, no. 032807, 2013. DOI: 10.1103/PhysRevE.89.032807.

\bibitem{Ogura2015} Ogura M., Preciado V., "Stability of Spreading Processes over Time-Varying Large-Scale Networks," \textit{IEEE Transactions on Network Science and Engineering}, vol. 3, pp. 44--57, 2015. DOI: 10.1109/TNSE.2016.2516346.

\bibitem{Rocha2010} Rocha L. E. C., Liljeros F., Holme P., "Simulated Epidemics in an Empirical Spatiotemporal Network of 50,185 Sexual Contacts," \textit{PLoS Computational Biology}, vol. 7, no. 1001109, 2010. DOI: 10.1371/journal.pcbi.1001109.

\bibitem{PastorSatorras2015} Pastor-Satorras R., Castellano C., Van Mieghem P., Vespignani A., "Epidemic processes in complex networks," \textit{Rev. Mod. Phys.}, vol. 87, no. 3, pp. 925--979, 2015.

\bibitem{Perra2012} Perra N., Gon\c{c}alves B., Pastor-Satorras R., Vespignani A., "Activity driven modeling of time varying networks," \textit{Scientific Reports}, vol. 2, no. 1, p. 469, 2012.

\bibitem{Mancastroppa2020} Mancastroppa M., Burioni R., Colizza V., "Active and inactive quarantine in epidemic spreading on adaptive activity-driven networks," \textit{Physical Review E}, 2020.

\bibitem{Huo2024} Huo L., Liu B., Zhao X., "Impact of environmental factors on the coevolution of information--emotions--epidemic dynamics in activity-driven multiplex networks," \textit{Chinese Physics B}, 2024.

\bibitem{Zou2016} Zou Y., Deng W., Li W., "A study of epidemic spreading on activity-driven networks," \textit{International Journal of Modern Physics C}, 2016.

\bibitem{Sun2015} Sun K., Baronchelli A., Perra N., "Contrasting effects of strong ties on SIR and SIS processes in temporal networks," \textit{European Physical Journal B}, 2015.

\bibitem{Tizzani2018} Tizzani M., Lenti S., Ubaldi E., "Epidemic spreading and aging in temporal networks with memory," \textit{Physical Review E}, 2018.

\bibitem{Nadini2020b} Nadini L., Allard A., Komarov N., Hébert-Dufresne L., "Epidemic processes on temporal networks: Thresholds, prevalence, and evaluation of the impact of time aggregation," \textit{Physical Review E}, 2020.
\end{thebibliography}
\newpage
\section*{Supplementary Material}
\begin{algorithm}[H]
\caption{Initialization of Population States}
\begin{algorithmic}[1]
\State Set total population size \( N \gets 10000 \)
\State Set initial infected \( I_0 \gets 10 \)
\State Compute susceptible \( S_0 \gets N - I_0 \)
\State Set recovered \( R_0 \gets 0 \)
\State Compute initial percentages \texttt{init\_cond}:
\Indent
    \State \( \texttt{S} \gets \mathrm{round}\left(\frac{S_0 \times 100}{N}\right) \)
    \State \( \texttt{I} \gets \mathrm{round}\left(\frac{I_0 \times 100}{N}\right) \)
    \State \( \texttt{R} \gets 0 \)
    \State Adjust \texttt{S} to ensure sum equals 100 if needed
\EndIndent
\State Store \texttt{initial\_conditions} as list containing \texttt{init\_cond}
\State Store \texttt{initial\_condition\_desc} describing initial setup
\end{algorithmic}
\end{algorithm}

\begin{algorithm}[H]
\caption{Static Network Summary Statistics Computation}
\begin{algorithmic}[1]
\State Load static weighted network sparse matrix \( G_{\mathrm{csr}} \)
\State Compute number of nodes \( N \gets G_{\mathrm{csr}}.\mathrm{shape}[0] \)
\State Compute degree sequence \( d_i \) as number of nonzero per row
\State Compute node strengths \( s_i = \sum_j w_{ij} \)
\State Compute edge weights from \( G_{\mathrm{csr}}.\mathrm{data} \)
\State Calculate summary statistics:
\Indent
    \State Number of edges \(= \mathrm{nnz}(G_{\mathrm{csr}}) / 2 \)
    \State Mean degree \(= \mathrm{mean}(d) \)
    \State Degree variance \(= \mathrm{var}(d) \)
    \State Mean strength \(= \mathrm{mean}(s) \)
    \State Strength variance \(= \mathrm{var}(s) \)
    \State Minimum edge weight, maximum edge weight
    \State Mean edge weight and standard deviation
    \State Minimum and maximum node strength
\EndIndent
\State Store computed values in \texttt{summary}
\end{algorithmic}
\end{algorithm}

\begin{algorithm}[H]
\caption{Network Activity Generation and Temporal Contact Construction}
\begin{algorithmic}[1]
\State Set random seeds for reproducibility
\State Initialize parameters \( N=10000 \), \( m=2 \), \( T=2000 \), activity power-law parameters \( \gamma=2.5 \), \( a_{\min}=0.01 \), \( a_{\max}=1.0 \), mean target activity 0.1
\State Generate activity values \( a_i \) using inverse transform sampling of power-law distribution:
\Indent
    \State Sample uniform random \( r_i \sim U(0,1) \)
    \State Compute \( a_i = \left( r_i \left(a_{\max}^{1-\gamma} - a_{\min}^{1-\gamma}\right) + a_{\min}^{1-\gamma} \right)^{\frac{1}{1-\gamma}} \)
    \State Normalize \( a_i \) to adjust mean to target 0.1
    \State Clip \( a_i \) to \([a_{\min}, a_{\max}]\)
\EndIndent
\State Store \( a_i \) as node activities
\State For each timestep \( t = 0, \dots, T-1 \):
\Indent
    \State Determine active nodes where \(\text{Bernoulli}(a_i)\) is true
    \State For each active node \( i \):
    \Indent
        \State Select \( m \) distinct partners at random from nodes excluding \( i \)
        \State Record temporal contacts \( (t, i, j) \)
        \State Update contact counts \texttt{contacts\_count}[\(\min(i,j)\), \(\max(i,j)\)] \(\mathrel{+}= 1\)
    \EndIndent
\EndIndent
\State Save temporal contacts CSV file
\State Construct static weighted network with edges weighted by contact frequency divided by \( T \)
\State Save static network in sparse format
\end{algorithmic}
\end{algorithm}

\begin{algorithm}[H]
\caption{Temporal Activity-Driven Stochastic SIR Simulation}
\begin{algorithmic}[1]
\Require Edge contact data by timestep \(\mathrm{contacts\_by\_t}\), node activities \( a_i \), parameters \( \beta \), \( \gamma \), max time \( T_{\max} \), simulations \( n_{\mathrm{sim}} \)
\For{each stochastic run \( r = 1 \) to \( n_{\mathrm{sim}} \)}
\Indent
    \State Initialize states \( X \) with all susceptible
    \State Infect 10 nodes selected uniformly at random
    \For{\( t=0 \) to \( T_{\max} - 1 \)}
    \Indent
        \State Compute counts of \( S \), \( I \), \( R \)
        \State Identify infectious nodes
        \If{no infectious nodes remain}
        \Indent
            \State Break loop
        \EndIndent
        \State For each contact \( (i,j) \) at time \( t \):
        \Indent
            \If{\( X[i] = S \) and \( X[j] = I \)}
                \State Attempt infection of \( i \) with probability \( \beta \)
            \ElsIf{\( X[j] = S \) and \( X[i] = I \)}
                \State Attempt infection of \( j \) with probability \( \beta \)
            \EndIf
        \EndIndent
        \State Recover infectious nodes independently with probability \( \gamma \)
    \EndFor
    \State Store time-series of compartment counts and summary statistics
\EndFor
\State Aggregate results across simulations (mean, std dev)
\State Save aggregated data and plots
\end{algorithmic}
\end{algorithm}

\begin{algorithm}[H]
\caption{Static Weighted Network SIR Model Simulation}
\begin{algorithmic}[1]
\State Load static weighted network matrix \( G_{\mathrm{csr}} \)
\State Define SIR model compartments \(\{S, I, R\}\)
\State Specify edge interaction infection with rate \( \beta \times w_{ij} \) and node recovery rate \( \gamma \)
\State Set initial condition: infect 10 random nodes, others susceptible
\For{number of stochastic realizations}
\Indent
    \State Simulate infection and recovery dynamics with rates applied to edges and nodes
\EndFor
\State Collect and save time series and statistics
\State Plot results
\end{algorithmic}
\end{algorithm}

\end{document}