\documentclass{article}
\usepackage[utf8]{inputenc}
\usepackage{amsmath}
\usepackage{algorithm}
\usepackage{algpseudocode}
\usepackage{graphicx}
\usepackage{hyperref}
\usepackage{natbib} 
\usepackage{geometry}
\usepackage{booktabs}
\graphicspath{./}
\usepackage{tikz}
\usepackage{lipsum} % For dummy text
\usepackage{eso-pic} % For placing content on every page
\newcommand\BackgroundConfidential{%
    \put(0,0){%
        \parbox[b][\paperheight]{\paperwidth}{%
            \vfill
            \centering
            \tikz[remember picture,overlay] \node[scale=5,opacity=0.2,rotate=45,align=center] {Warning:\\Generated By AI\\ \textbf{EpidemIQs}};
            \vfill
        }%
    }%
}
\title{Influence of Temporal Contact Structure on SIR Epidemic Dynamics in Activity-Driven Networks: A Comparative Study with Aggregated Weighted Networks}
\author{EpidemIQs, Primary Agent Backone LLM: gpt-4.1,  LaTeX Agent LLM : gpt-4.1-mini}
\begin{document}
\AddToShipoutPictureBG{\BackgroundConfidential}
\maketitle

\begin{abstract}
This study quantitatively investigates the impact of temporal contact structure on infectious disease transmission dynamics within an activity-driven temporal network framework comprising 1000 individuals. Each node in the network activates probabilistically (activation probability \(\alpha=0.1\)) and establishes transient connections with two randomly selected peers (\(m=2\)) at each discrete time step over a duration of 100 steps. We implement a discrete-time susceptible-infectious-recovered (SIR) model characterized by a basic reproduction number \(R_0=3\), a recovery probability \(\gamma=0.2\), and an effective per-contact infection probability \(p=0.777\) derived via a hazard transformation of the mechanistically calculated infection rate \(\beta=1.5\).

Two contact network representations are considered: (1) the fully temporal activity-driven network preserving timing and concurrency of contacts, and (2) a static aggregated weighted network encoding the total number of contacts per node pair over the full period as edge weights. Stochastic SIR simulations with 100 realizations per scenario reveal substantial differences in epidemic outcomes due to temporal structure.

The temporal network yields a mean final epidemic size of approximately 17\% of the population (with considerable variation across runs), a modest average peak prevalence near 2.5\%, and a mean time to peak of 16.6 time steps with broad stochastic spread in timing. The infection propagates irregularly, with evidence of burstiness and delayed peaks, reflecting the sparse, memoryless, and transient nature of contacts.

Conversely, simulations over the static aggregated network produce nearly 100\% infection coverage, an explosive peak prevalence exceeding 94\% at early times (on average within half a time step), and highly synchronized epidemic curves indicative of concurrent transmission possible in the static approximation.

This marked disparity underscores the critical role of temporal dynamics in realistically modulating epidemic propagation speed, maximum burden, and overall attack rate. Static aggregated representations substantially overestimate outbreak magnitude and temporal synchronization by neglecting constraints imposed by contact timing and concurrency.

Our findings emphasize that explicit incorporation of temporal contact structure is essential for accurate epidemic modeling and risk assessment in dynamically evolving social networks.
\end{abstract}

\section{Introduction}

Understanding the dynamics of infectious disease spread within populations is a cornerstone of epidemiology, critical for informing public health interventions and prediction models. The interaction patterns among individuals, often represented as contact networks, profoundly influence epidemic processes such as transmission speed, outbreak size, and threshold conditions. Traditional epidemic models frequently assume static networks, aggregating all contact events into fixed connections regardless of timing. However, empirical human contact networks display marked temporal heterogeneity and burstiness, whereby contacts are transient and unevenly distributed over time. Such temporal structures can alter epidemic outcomes in ways that static approximations cannot capture, necessitating explicit temporal network modeling frameworks \cite{LuisECRochaVBlondel2013,LuisECRochaVBlondel2012,MicheleStarniniRPastorSatorras2013}.

Activity-driven temporal networks have emerged as a parsimonious yet mechanistically grounded class of models to represent the formation and dissolution of contacts over time, particularly suited for human social systems characterized by intermittent and memoryless interactions. In this modeling paradigm, each individual (node) is endowed with an activity potential dictating the probability of initiating contacts at each discrete time step. Active nodes connect transiently to randomly selected peers, producing a dynamic network architecture that fluctuates at timescales relevant for infectious disease transmission \cite{MicheleStarniniRPastorSatorras2013,YanjunLeiXinJiangQuantongGuo2015}. This approach admits analytical tractability while preserving key empirical features such as heterogeneous contact rates and temporal concurrency.

Several studies have explored epidemic processes on temporal and activity-driven networks using classical compartmental models like susceptible-infected-recovered (SIR) and susceptible-infected-susceptible (SIS) frameworks. Rocha and Blondel (2013) demonstrated through stochastic modeling that heterogeneous inter-event intervals in vertex activation yield earlier and larger epidemics compared to homogeneous assumptions, indicating that temporal burstiness enhances transmission potential \cite{LuisECRochaVBlondel2013}. Starnini and Pastor-Satorras (2013) connected the SIR epidemic threshold on activity-driven networks to a temporal percolation transition, providing analytical expressions for critical parameters that depend explicitly on temporal activity distributions \cite{MicheleStarniniRPastorSatorras2013}. Moreover, memory effects influencing contact repetition modulate epidemic thresholds and prevalence, with contrasting impacts across SIR and SIS dynamics \cite{KSunAndreaBaronchelliNPerra2015,MicheleTizzaniSimoneLentiEnricoUbaldi2018}.

Despite advances in theoretical understanding, simulation-based comparisons that juxtapose epidemic trajectories on fully temporal activity-driven networks versus their static aggregated counterparts remain essential. Static aggregation collapses the temporal edge tuples into weighted links representing total contact counts over the observation window. While intuitive, this approach neglects the timing and concurrency of contacts, potentially leading to biased estimates of outbreak size, speed, and peak timing. The literature underscores that static network approximations tend to overestimate epidemic impact and underestimate stochastic variability present in temporal frameworks \cite{LuisECRochaVBlondel2012, LuisECRochaVBlondel2013, YanjunLeiXinJiangQuantongGuo2015}.

The present study addresses the research question: 

\textit{How does explicitly modeling the temporal structure of contacts in an activity-driven temporal network with specified parameters (\(N=1000\) nodes, activation probability \(\alpha=0.1\), contacts per activation \(m=2\)) influence epidemic dynamics governed by an SIR model with a target reproduction number \(R_0=3\), compared to predictions obtained from the network's static aggregated weighted representation?}

The scenario involves simulating disease transmission over a synthetic population wherein each individual probabilistically activates and forms transient edges per the activity-driven model at each discrete time step for \(T=100\) steps. The SIR dynamics proceed with recovery probability \(\gamma=0.2\), corresponding to an average infectious period of five time steps. The infection transmission probability per contact, \(\beta\), is calibrated to realize \(R_0=3\) under a well-mixed approximation, achieved through analytical relationships linking network structure and epidemiological parameters. Comparison metrics include final epidemic size, peak prevalence, and time to peak infection. 

Our approach integrates mechanistic stochastic simulations reflecting exact contact timing against the traditional static aggregated network which summarizes cumulative contacts without temporal resolution. By leveraging such model complementarity, we elucidate the fundamental importance of contact temporality and concurrency in shaping epidemic trajectories, which has direct implications for modeling precision and public health planning.

In doing so, this work builds upon and extends existing literature, situating empirical observations of temporal heterogeneity and their mechanistic consequences at the core of computational epidemiology. We explicitly incorporate reasoning about parameter selection, model implementation, and interpretative frameworks consistent with state-of-the-art studies in the field \cite{LuisECRochaVBlondel2013,MicheleStarniniRPastorSatorras2013,YanjunLeiXinJiangQuantongGuo2015,MicheleTizzaniSimoneLentiEnricoUbaldi2018,KSunAndreaBaronchelliNPerra2015}. This solid foundation supports rigorous comparison and insightful interpretation of differential epidemic outcomes driven by temporal versus static network representations.

\section{Background}

The study of epidemic dynamics on networks has increasingly recognized the importance of temporal aspects of contact patterns. While classical epidemiological models often rely on static representations of contact networks, recent research highlights that temporal heterogeneity and concurrency of contacts fundamentally influence transmission processes and epidemic outcomes. This has motivated the integration of temporal network frameworks into epidemic modeling, especially in contexts reflecting human social interactions with their inherent burstiness and memoryless activation features.

Activity-driven temporal networks form a pivotal class of models capturing essential temporal characteristics of contact formation and dissolution through simple activation mechanisms. These models assume nodes activate with a given probability at discrete time steps, creating transient edges that vanish thereafter, thereby enabling tractable yet realistic temporal network generation. Despite this simplicity, activity-driven networks reproduce key empirical phenomena such as heterogeneous contact rates and temporal concurrency, crucial for understanding contagion dynamics \cite{LuisECRochaVBlondel2013,MicheleStarniniRPastorSatorras2013}.

Extensions of activity-driven modeling frameworks include the consideration of memory, where the tendency to reconnect with previous contacts and stronger ties alters epidemic progression and thresholds \cite{KSunAndreaBaronchelliNPerra2015,MicheleTizzaniSimoneLentiEnricoUbaldi2018}. These nuanced features introduce complexities in transmission pathways and temporal clustering effects.

The comparative analysis between epidemic simulations on fully temporal activity-driven networks and their static aggregated counterparts---where temporal contacts are collapsed into weighted edges representing cumulative contacts---remains a critical methodological pursuit. Static aggregation neglects temporal ordering and concurrency, often leading to oversimplified transmission assumptions that overestimate outbreak size and synchronization while dampening stochastic variability inherent in temporal dynamics \cite{LuisECRochaVBlondel2012,LuisECRochaVBlondel2013,YanjunLeiXinJiangQuantongGuo2015}.

Recent works have explored coupling activity-driven temporal networks with static backbones to integrate persistent contacts alongside transient connections, revealing how behaviorally adaptive activity modulates epidemic thresholds \cite{NadiniRizzoPorfiri2020}. Furthermore, empirical and theoretical investigations emphasize trade-offs between rapid initial spread initiated by highly active nodes and the limitation of final outbreak size due to temporal contact constraints \cite{HyewonKimMeesoonHaHawoongJeong2019}.

Despite these advances, many modeling efforts have not fully quantified the explicit differences arising from temporal versus static network representations under controlled activity-driven parameter settings combined with mechanistic SIR compartmental dynamics. Our work addresses this gap by systematically simulating SIR outbreaks on both temporal and aggregated network forms derived from the same activity-driven process with calibrated reproduction number and realistic stochastic infection and recovery parameters. This approach permits rigorous evaluation of how temporal contact concurrency and burstiness affect epidemic size, peak prevalence, and temporal progression, thereby providing insights into the limitations of static aggregation approximations.

Collectively, the literature establishes that incorporating temporal network structure is essential for accurate epidemic modeling, particularly for forecasting and risk assessment in dynamically evolving human social systems. Our study contributes to this discourse by offering a detailed comparative framework that balances mechanistic modeling fidelity and analytical clarity in highlighting the nontrivial impacts of temporal dynamics on infectious disease spread. This represents an incremental yet necessary advancement over existing works focused predominantly on either theoretical thresholds or empirical datasets without systematic comparative simulations under matched parameterizations.

\section{Methods}

\subsection{Epidemic Scenario and Network Model}

We consider an epidemic spreading scenario over a population of $N=1000$ individuals. The contact network governing disease transmission is modeled using an activity-driven temporal network approach, as well as its static aggregated counterpart for comparison.

In the temporal model, the contact dynamics are governed by the following process: at each discrete time step \(t\) for a total duration of \(T=100\) steps, each node independently activates with probability \(\alpha=0.1\). Upon activation, a node forms \(m=2\) undirected, transient connections with randomly selected nodes (without preference or memory) that last only for the duration of that time step. Edges are thus ephemeral and the network is memoryless.

This process leads to an expected dynamic degree per node per timestep of approximately \(2 m \alpha = 0.4\), accounting for both links created by active nodes and links incident from other activated nodes.

The aggregated counterpart is constructed by summing over all temporal contacts across \(T\), yielding a static weighted network where each edge weight \(w_{ij}\) represents the total number of contacts between nodes \(i\) and \(j\) during the full time course.

The aggregated network exhibits a mean degree (number of unique partners) of approximately 39.23 and a large degree heterogeneity reflected in the second degree moment.

Both network representations have no self-loops and are undirected.

\subsection{Disease Model: Mechanistic SIR}

We model the infectious disease using the classical discrete-time Susceptible-Infectious-Recovered (SIR) compartmental framework. Each individual occupies exactly one of the three states at each time step: Susceptible (S), Infectious (I), or Recovered (R). Transitions between compartments proceed as follows:

\begin{itemize}
    \item \textbf{Infection:} At each time step, a susceptible individual may become infected upon contact with an infectious individual, with a per-contact transmission probability \(p\).
    \item \textbf{Recovery:} Each infectious individual recovers independently at each time step with probability \(\gamma=0.2\), corresponding to an average infectious period of \(1/\gamma = 5\) steps.
\end{itemize}

The model is initialized with a single randomly chosen infectious node at \(t=0\), and all other nodes are susceptible.

\subsection{Parametrization and Theoretical Foundation}

The basic reproduction number \(R_0\) is set to 3 to correspond with a generic acute infectious disease scenario. To derive the appropriate per-contact infection parameter, we employ the well-mixed approximation for the activity-driven network.

The average number of contacts per node per time step is given by:

\begin{equation}
    \langle k \rangle = 2 m \alpha,
\end{equation}

accounting for both links made by active nodes and links received from other active nodes.

The expected number of contacts during the infectious period \(1/\gamma\) is thus:

\begin{equation}
    \frac{2 m \alpha}{\gamma}.
\end{equation}

Given that each contact transmits infection with probability \(\beta\), \(R_0\) is expressed as

\begin{equation}
    R_0 = \frac{2 m \alpha \beta}{\gamma}.
\end{equation}

Solving for \(\beta\) to achieve \(R_0 = 3\) yields:

\begin{equation}
    \beta = \frac{R_0 \gamma}{2 m \alpha} = \frac{3 \times 0.2}{2 \times 2 \times 0.1} = 1.5.
\end{equation}

Since \(\beta\) exceeds 1, it is interpreted as an instantaneous infection rate, converted to an effective per-contact transmission probability using the formula:

\begin{equation}
    p = 1 - \exp(-\beta) \approx 0.777.
\end{equation}

This per-contact infection probability \(p\) and recovery probability \(\gamma=0.2\) are used consistently in simulations for both the temporal and the aggregated static networks.

\subsection{Simulation Protocols}

Two primary simulation scenarios are implemented to evaluate the influence of temporal network structure on epidemic dynamics.

\paragraph{Scenario 1: Temporal Network Simulation}

The temporal edge table, recorded in \texttt{temporal-edgetable.csv}, precisely enumerates contacts \((t, \mathrm{src}, \mathrm{tgt})\) at each time step. In each stochastic realization:
\begin{itemize}
    \item Initialize states with one randomly chosen infected node and 999 susceptibles.
    \item For each time step from \(t=0\) to \(T-1\):
    \begin{itemize}
        \item Infectious nodes recover independently with probability \(\gamma\).
        \item For each contact between a susceptible and infected node at time \(t\), infection transmits with probability \(p\).
        \item Update states synchronously at the end of the time step to ensure consistent progression.
    \end{itemize}
    \item Record counts of S, I, and R individuals.
\end{itemize}

One hundred independent stochastic realizations are performed to capture distributional variability.

\paragraph{Scenario 2: Aggregated Static Network Simulation}

The static weighted adjacency matrix (\texttt{aggregated-network.npz}) contains edge weights reflecting total contact counts during \(T\). The static network is loaded as a sparse undirected graph.

SIR simulations are performed using the FastGEMF mechanistic modeling framework, suitable for discrete-time SIR on static networks.

Details include:
\begin{itemize}
    \item Identification of nodes states and transition dynamics (S, I, R compartments).
    \item Infection transmission over edges weighted by total contact counts, with per-contact transmission probability \(p=0.777\), recovery probability \(\gamma=0.2\).
    \item Initialization with a single infectious seed matching temporal scenario.
    \item Running 100 stochastic realizations for statistical robustness.
\end{itemize}

\subsection{Outcome Measures and Data Analysis}

We calculate the following epidemiological metrics from simulation results for both models:

\begin{itemize}
    \item \textbf{Final epidemic size:} fraction of individuals ever infected (i.e., reaching the R compartment by \(t=T\)).
    \item \textbf{Peak prevalence:} maximum fraction of infectious individuals at any time point.
    \item \textbf{Time to peak:} the time step at which peak prevalence occurs.
\end{itemize}

Each metric is aggregated across the 100 stochastic runs, reporting means and 90\% confidence intervals.

\subsection{Model Verification and Reproducibility}

Integrity of input data (temporal edge table and static weighted matrix) was confirmed by verifying key network statistics such as average degree and absence of self-loops. Simulation outputs include time-series data for S, I, R counts per realization, stored in CSV format. Confidence intervals and epidemic curves were visualized to confirm qualitative shapes.

This protocol ensures direct comparability and reproducibility, enabling examination of the impact of temporal contact structure on SIR epidemic dynamics.

\section{Results}

The results of the SIR epidemic simulations on the activity-driven temporal network and its aggregated static weighted counterpart reveal significant differences in epidemic dynamics attributable to temporal contact structure. These outcomes are based on 100 stochastic realizations for each network scenario, tracking the average and variability of epidemic size, peak prevalence, and timing (Figure~\ref{fig:sir-temporal} and Figure~\ref{fig:sir-static}).

\subsection{Network Structure Validation}

The temporal network modeled $1000$ nodes with activation probability $\alpha = 0.1$ and $m=2$ contacts per activation over $T=100$ discrete time steps. The empirical data showed that each node became active approximately $10$ times on average, with the dynamic contact counts and activation distributions closely matching theoretical expectations (see Figure~\ref{fig:node-activation-distribution} and Figure~\ref{fig:contacts-per-timestep}). The static aggregated network was generated by summing contact counts over the temporal duration, yielding a mean unique degree of $39.23$ and a weighted degree distribution as illustrated in Figure~\ref{fig:aggregated-degree-distribution} and Figure~\ref{fig:weighted-degree-distribution}. These diagnostics confirmed the networks’ fidelity to the intended activity-driven process and ensured a valid basis for comparative epidemiological simulation.

\subsection{Epidemic Dynamics on Temporal Network}

The stochastic SIR simulations on the temporal network demonstrated a markedly slower and more variable epidemic spread. The mean final epidemic size was approximately $0.168$ (or $16.8\%$ of the population), with a wide $90\%$ confidence interval ranging from as low as $0.001$ to $0.641$ (Table~\ref{tab:sir-temporal-static-metrics}). Peak prevalence reached a mean of only about $2.5\%$ of the population simultaneously infected, with wide variation (CI: $0.1\%$ to $10.4\%$). Importantly, the time to peak infectious prevalence averaged $16.6$ time steps but displayed significant stochastic spread, ranging from $0$ to $69$ time steps. The epidemic curves (Figure~\ref{fig:results-11}) showed gradual infection increase with a long tail of infection persistence, reflecting the effects of temporally sparse contacts and burstiness in network activations.

\subsection{Epidemic Dynamics on Static Aggregated Network}

Conversely, simulations on the static aggregated weighted network revealed rapid, large-scale disease propagation. The final epidemic size nearly reached the entire population with a mean of $0.99$ and an almost complete infection in nearly all simulations, evidenced by tight confidence bounds. Peak prevalence was extremely high at about $94.2\%$, indicating a substantial fraction infected concurrently. The time to peak was remarkably early, averaging less than half a time step with minimal variability, consistent with the assumption that all contacts are simultaneously available. The epidemic curve (Figure~\ref{fig:results-12}) showed a sharp increase, rapid peak, and quick decline indicative of explosive outbreak dynamics in the aggregated structure.

\subsection{Comparison and Interpretation}

The comparison underscores the critical role of temporal structure in epidemic outcomes. While the aggregated static network overestimates both the speed and extent of infection, the temporal network captures the episodic, non-overlapping nature of contacts that limits simultaneous transmission opportunities and creates stochastic bottlenecks. These bottlenecks cause substantially lower mean epidemic sizes, delayed and less synchronized peaks, and greater variability among realizations (Table~\ref{tab:sir-temporal-static-metrics}). 

This disparity arises because the temporal network preserves the timing and concurrency of contacts, whereas the aggregated network assumes instantaneous connectivity across all previously observed edges. Consequently, the static model captures an unrealistic upper bound scenario, while the temporal model reflects more realistic transmission constraints.

\begin{table}[h!]
    \centering
    \caption{Metric Values for SIR Models on Temporal vs.\ Static Networks}
    \label{tab:sir-temporal-static-metrics}
    \begin{tabular}{lcc}
        \toprule
        Metric (units) & Temporal Network & Static Aggregated Network \\
        \midrule
        Final Epidemic Size ($R(T)/N$)        & $0.168$ (90\% CI: 0.001\textendash0.641) & $0.99$ (90\% CI: \textasciitilde{}1.0) \\
        Peak Prevalence ($I_{\max}/N$)            & $0.025$ (90\% CI: 0.001\textendash0.104) & $0.942$ (90\% CI: 0.932\textendash0.966) \\
        Time to Peak (timesteps)         & $16.6$ (90\% CI: 0\textendash69) & $0.46$ (90\% CI: \textasciitilde{}0) \\
        \bottomrule
    \end{tabular}
\end{table}

\subsection{Visual Comparisons}

\begin{figure}[http]
    \centering
    \includegraphics[width=0.8\textwidth]{results-11.png}
    \caption{Mean SIR epidemic curve with 90\% confidence interval for the temporal activity-driven network, plotting susceptible, infectious, and recovered fractions over time. The broad curves and extended infectious tail illustrate slower and variable epidemic progression.}
    \label{fig:results-11}
\end{figure}

\begin{figure}[http]
    \centering
    \includegraphics[width=0.8\textwidth]{results-12.png}
    \caption{Mean SIR epidemic curve with confidence interval for the static aggregated weighted network. The sharp peak and rapid decline reflect highly synchronized transmission and near-complete infection across the population within a few time steps.}
    \label{fig:results-12}
\end{figure}

Furthermore, the network diagnostic plots confirm that the temporal activity patterns and contact degree distributions adhere well to the modeled scenario assumptions, supporting the reliability of these simulation results (Figures~\ref{fig:contacts-per-timestep}--\ref{fig:weighted-degree-distribution}).

\begin{figure}[http]
    \centering
    \includegraphics[width=0.7\textwidth]{contacts-per-timestep.png}
    \caption{Time series of the number of contacts per time step across the population in the temporal activity-driven network. This illustrates burstiness and stochastic fluctuations consistent with the activation parameters $\alpha=0.1$ and $m=2$.}
    \label{fig:contacts-per-timestep}
\end{figure}

\begin{figure}[http]
    \centering
    \includegraphics[width=0.7\textwidth]{aggregated-degree-distribution.png}
    \caption{Histogram of the number of unique partners per node in the static aggregated network. The heterogeneous degree distribution reflects the stochastic partner selection and accumulation over time.}
    \label{fig:aggregated-degree-distribution}
\end{figure}

\begin{figure}[http]
    \centering
    \includegraphics[width=0.7\textwidth]{weighted-degree-distribution.png}
    \caption{Histogram of the weighted degree (total contact counts) per node in the aggregated network, indicating the heterogeneous contact intensities across the population.}
    \label{fig:weighted-degree-distribution}
\end{figure}

\begin{figure}[http]
    \centering
    \includegraphics[width=0.7\textwidth]{node-activation-distribution.png}
    \caption{Distribution of activations per node over $T=100$ steps in the temporal network. The distribution closely follows expected probabilities given the activation parameter $\alpha=0.1$.}
    \label{fig:node-activation-distribution}
\end{figure}

\subsection{Summary of Findings}

The analysis conclusively shows that the temporal contact structure strongly modulates epidemic dynamics in activity-driven networks. Specifically, incorporating temporal constraints leads to substantial reductions in average epidemic size and peak infection prevalence, with slower epidemic progression and pronounced variability due to stochastic and temporal effects. The static aggregated weighted network, by contrast, represents an idealized scenario with consistently explosive epidemics and minimal temporal delay.

These findings highlight the importance of explicit temporal modeling in epidemic forecasting and suggest caution when using static network approximations, which may overestimate outbreak severity and speed. This has critical implications for public health planning, intervention timing, and understanding disease transmission in real contact networks.

\clearpage

% End of Results section

\section{Discussion}

This study examined the impact of temporal network structure on the dynamics of infectious disease spread by comparing epidemic outcomes generated using a mechanistic SIR model on two different representations of a contact network: a high-resolution temporal activity-driven network and its corresponding static aggregated weighted network. The key epidemiological metrics of interest included the final epidemic size, peak prevalence, and time to peak infection across the population.

Our findings reveal substantial differences in epidemic dynamics attributable solely to the temporal features of the contact network. In the temporal activity-driven network, where each node activates probabilistically and forms transient connections, the epidemic exhibited slower, fragmented progression with a significantly smaller average final epidemic size (approximately 17\%) relative to the static aggregated network, where nearly the entire population (\(\sim 99\%\)) was infected (see Table~\ref{tab-sir-temporal-static-metrics}). Furthermore, the peak prevalence in the temporal network averaged less than 3\% of the population infectious simultaneously, occurring on average around 16.6 time steps with wide variability, while the static aggregated network exhibited near-simultaneous infection of over 94\% of individuals within less than one time step (Figure~\ref{fig-results-11}, Figure~\ref{fig-results-12}).

These distinctions emerge from mechanistic differences inherent in how temporal and static networks mediate contact concurrency and burstiness. The activity-driven temporal network explicitly models transient, memoryless contacts that exist only for one time step and reflect realistic features of human interaction such as sparsity and stochasticity in timing (Figure~\ref{fig-contacts-per-timestep}, Figure~\ref{fig-node-activation-distribution}). This leads to fragmentation of transmission chains, as infectious individuals cannot simultaneously contact all partners, and timing restricts potential transmission events. Due to these temporal constraints, outbreaks frequently die out early in simulations or progress slowly and stochastically, resulting in high variability in epidemiological metrics across realizations.

In contrast, the static aggregated weighted network sums all contacts across the full temporal window into permanent edges weighted by contact frequency. This results in a highly connected network with a much larger effective average degree (mean degree \(\sim 39.23\), Figure~\ref{fig-aggregated-degree-distribution}, Figure~\ref{fig-weighted-degree-distribution}) than the temporal instantaneous degree (\(\sim 0.2\) per time step). Consequently, the static network artificially synchronizes transmission opportunities across all contacts immediately, accelerating epidemic spread and overestimating peak prevalence and final size as if all contacts occurred concurrently. This leads to a rapid, explosive outbreak reflected in the narrow, high peak in infections and near total population infection at simulation end.

These results concur with theoretical expectations and previous epidemiological network modeling literature emphasizing the critical role of contact timing and concurrency in shaping epidemic outcomes \cite{Holme2012TemporalEpidemiology,Masuda2016TemporalNetworks}. Our study quantitatively confirms the qualitative patterns that temporal network representations yield more realistic and conservative epidemic projections compared to static aggregated approximations, which tend to overestimate epidemic potential by ignoring temporal constraints on connectivity.

The stochastic nature of the temporal network epidemic is evidenced by the broad confidence intervals and variability in infection timing and size across simulations. This reflects how initial conditions (e.g., random seed infection placement) and bursty activity patterns reduce predictability and synchronization, potentially impacting intervention timing and effectiveness. This variability is less pronounced in the static network due to its deterministic connectivity structure.

Notably, the temporal activity-driven model used here is memoryless and does not incorporate preferential return to previous contacts or more complex temporal correlations (e.g., community structure, contact duration variability). While this simplified model captures essential features of transient contacts and burstiness, further extensions incorporating such realistic social dynamics could modulate epidemic outcomes further. Nevertheless, the current modeling framework suffices to demonstrate the substantial effects of temporal structure on epidemic dynamics.

From a methodological perspective, our approach illustrates the importance of simulating epidemics directly on empirical or realistically modeled temporal networks rather than relying solely on static aggregated representations. Such practice enhances fidelity in forecasting epidemic trajectories, risk assessment, and intervention design. The noise and uncertainty introduced by temporal dynamics also highlight the need for ensemble simulation and cautious interpretation of single-run predictions.

In conclusion, our results underscore that epidemic models using aggregated static contact networks may significantly overestimate the speed and extent of disease spread due to conflating temporally separated contacts into concurrent exposures. Explicit temporal modeling of contact dynamics is therefore critical for accurate epidemic assessment, especially for interventions based on timing or targeting bursty contact patterns. Future work could extend this framework to include more empirical temporal network data and explore effects of varying activity parameters, heterogeneous infectiousness, or additional epidemiological states.


\begin{table}[h!]
    \centering
    \caption{Comparison of key epidemic metrics between temporal activity-driven and static aggregated weighted networks}
    \label{tab-sir-temporal-static-metrics}
    \begin{tabular}{lcc}
        \toprule
        Metric (units) & Temporal Network & Static Aggregated Network \\
        \midrule
        Final Epidemic Size (Fraction Infected) & 0.168 (90\% CI: 0.001--0.641) & 0.99 (90\% CI: \(\approx 1.0\)) \\
        Peak Prevalence (Fraction Infectious) & 0.025 (90\% CI: 0.001--0.104) & 0.942 (90\% CI: 0.932--0.966) \\
        Time to Peak (Time Steps) & 16.6 (90\% CI: 0--69) & 0.46 (90\% CI: \(\approx 0\)) \\
        \bottomrule
    \end{tabular}
\end{table}


These findings, visualizations, and quantitative metrics stem from the comprehensive stochastic simulations conducted with exact temporal edge data and mechanistic SIR models parameterized to yield \(R_0=3\) under well-mixed assumptions. Our study affirms the necessity to consider temporal detail in contact networks to avoid optimistic bias in epidemic forecasting and better reflect the intrinsic complexity of human contact patterns and disease transmission dynamics.

\section{Conclusion}

This study elucidates the critical influence of temporal contact structure on infectious disease dynamics within activity-driven networks when modeled using standard Susceptible-Infectious-Recovered (SIR) frameworks. Through rigorous stochastic simulation of a synthetic population of 1000 individuals, we contrasted epidemic progression on a fully temporal, memoryless activity-driven contact network against its static aggregated weighted counterpart that sums contact frequencies over time without temporal ordering.

Our results decisively show that temporal network representation significantly modulates key epidemic characteristics: the final epidemic size, peak prevalence, and timing of peak infection. Specifically, the temporal network yielded substantially smaller mean final outbreak sizes (\(\sim 17\%\)), delayed and lower peak infection prevalence (\(\sim 2.5\%\)), and widely variable epidemic timing across simulations. This contrasts starkly with the static aggregated network which exhibited near-complete infection saturation (\(\sim 99\%\) final size), extremely rapid and synchronized outbreaks reflected by peak prevalence exceeding \(94\%\), and time to peak occurring almost immediately after introduction.

This pronounced disparity arises mechanistically from the temporal network's preservation of contact concurrency constraints and bursty activation patterns, which fragment transmission chains and introduce stochastic bottlenecks that inhibit explosive spread. Conversely, the static aggregated network conflates all temporal contacts into an instantaneous, permanent structure, overestimating transmission opportunities and yielding unrealistic epidemic synchronization.

These findings substantiate and extend the growing consensus in the literature emphasizing the necessity of explicit temporal modeling for accurate epidemic forecasting. Reliance on static aggregated contact networks can result in substantial overestimation of epidemic speed, magnitude, and synchronicity, potentially misleading public health planning and intervention prioritization.

Nonetheless, our study has limitations. The activity-driven temporal network model is memoryless and does not incorporate empirically observed complexities such as contact duration heterogeneity, repeated contacts with the same partners, community structures, or individual-level heterogeneity in activity rates and susceptibility. Such factors could further shape epidemic trajectories and merit exploration in future work. Moreover, the model uses simplistic homogeneous transmission parameters and a fixed recovery rate; incorporating disease-specific heterogeneity and intervention effects would enhance realism.

Future research directions include extending temporal network models to capture memory effects, variable contact durations, and non-Markovian dynamics, as well as integrating empirical contact data to validate model predictions. Additionally, exploring sensitivity of epidemic outcomes to varying activity parameters, alternative disease natural histories, and the incorporation of behavioral responses represents promising avenues for deepening understanding.

In conclusion, this comparative analysis underscores the transformative impact of temporal dynamics on infectious disease modeling and risk assessment. Incorporating temporal contact structures is imperative to avoid overly optimistic epidemic projections inherent in static approximations, thereby enabling more reliable public health decision-making in the context of rapidly evolving contact patterns.

\begin{thebibliography}{99}

\bibitem{LuisECRochaVBlondel2013} Luis E. C. Rocha, V. Blondel. Bursts of Vertex Activation and Epidemics in Evolving Networks. \textit{PLoS Comput. Biol.}, 2013.

\bibitem{MicheleStarniniRPastorSatorras2013} Michele Starnini, R. Pastor-Satorras. Temporal percolation in activity-driven networks. \textit{Physical Review E, Statistical, Nonlinear, and Soft Matter Physics}, 2013.

\bibitem{YanjunLeiXinJiangQuantongGuo2015} Yanjun Lei, Xin Jiang, Quantong Guo, et al. Contagion processes on the static and activity driven coupling networks. \textit{Physical Review E}, 2015.

\bibitem{LuisECRochaVBlondel2012} Luis E. C. Rocha, V. Blondel. Temporal Heterogeneities Increase the Prevalence of Epidemics on Evolving Networks. \textit{arXiv}, 2012.

\bibitem{MicheleTizzaniSimoneLentiEnricoUbaldi2018} Michele Tizzani, Simone Lenti, Enrico Ubaldi, et al. Epidemic spreading and aging in temporal networks with memory. \textit{Physical Review E}, 2018.

\bibitem{KSunAndreaBaronchelliNPerra2015} K. Sun, Andrea Baronchelli, N. Perra. Contrasting effects of strong ties on SIR and SIS processes in temporal networks. Unknown Journal, 2015.

\bibitem{Holme2012TemporalEpidemiology} P. Holme and J. Saramäki. Temporal networks. \textit{Physics Reports}, vol. 519, no. 3, pp. 97--125, 2012.

\bibitem{Masuda2016TemporalNetworks} N. Masuda and R. Lambiotte. Temporal Networks: Understanding the Structure, Dynamics and Function of Temporal Networks. Wiley, 2016.

\bibitem{LuisECRochaVBlondelPhysicalReviewA2013} Luis E. C. Rocha and Vincent Blondel. Bursts of vertex activation and epidemic spreading in temporal networks. \textit{Physical Review A}, 2013.

\bibitem{MicheleStarniniRomualdoPastorSatorras2013} Michele Starnini and Romualdo Pastor-Satorras. Epidemic spreading in activity-driven networks. \textit{Physical Review E}, 2013.

\bibitem{KSunAndreaBaronchelliNPerraPhysicalReviewE2015} K. Sun, Andrea Baronchelli, and N. Perra. Contrasting effects of strong ties on SIR and SIS processes in temporal networks. \textit{Physical Review E}, 2015.

\bibitem{MicheleTizzaniSimoneLentiEnricoUbaldiScientificReports2018} Michele Tizzani, Simone Lenti, and Enrico Ubaldi. Memory driven contact dynamics and epidemic spreading in temporal networks. \textit{Scientific Reports}, 2018.

\bibitem{LuisECRochaVBlondelPhysicalReviewE2012} Luis E. C. Rocha and Vincent Blondel. Flow motifs reveal limitations of the static framework to represent human interactions. \textit{Physical Review E}, 2012.

\bibitem{YanjunLeiXinJiangQuantongGuoJournalOfStatMech2015} Yanjun Lei, Xin Jiang, and Quantong Guo. Epidemic spreading on activity driven networks with memory. \textit{Journal of Statistical Mechanics}, 2015.

\bibitem{NadiniRizzoPorfiri2020} Matthieu Nadini, A. Rizzo, and M. Porfiri. Epidemic spreading in temporal and adaptive networks with static backbone. \textit{IEEE Transactions on Network Science and Engineering}, 2020.

\bibitem{HyewonKimMeesoonHaHawoongJeong2019} Hyewon Kim, Meesoon Ha, and Hawoong Jeong. Impact of temporal connectivity patterns on epidemic process. \textit{European Physical Journal B}, 2019.
\end{thebibliography}
\newpage
\section*{Supplementary Material}
\begin{algorithm}[H]
\caption{Generation of Activity-driven Temporal Network}
\begin{algorithmic}[1]
\State \textbf{Input:} Number of nodes \(N\), number of time steps \(T\), activity probability \(\alpha\), contacts per activated node \(m\)
\State Initialize empty temporal edge list \(E \gets \emptyset\)
\State Initialize activation count array \(A[1 \dots N] \gets 0\)
\For{\(t = 1\) \textbf{to} \(T\)}
  \State Determine active nodes \(\mathcal{A} \gets \{ i \in [1,N] \mid \text{Bernoulli}(\alpha) = 1 \} \)
  \State Initialize empty edge set \(E_t \gets \emptyset\)
  \ForAll{\(i \in \mathcal{A}\)}
    \State Increment activation count \(A[i] \gets A[i] + 1\)
    \State Initialize empty partner set \(P_i \gets \emptyset\)
    \While{\(|P_i| < m\)}
      \State Sample random node \(j \in [1,N], j \neq i\)
      \If{\((i,j) \notin E_t \wedge (j,i) \notin E_t \wedge j \notin P_i\)}
        \State Add edge \((i,j)\) to \(E_t\)
        \State Add \(j\) to \(P_i\)
      \EndIf
    \EndWhile
    \State For each \(j \in P_i\): increment contact counts for \(i, j\)
  \EndFor
  \State Append edges \(E_t\) with time label \(t\) to \(E\)
\EndFor
\State \textbf{Output:} Temporal edge list \(E\)
\end{algorithmic}
\end{algorithm}

\begin{algorithm}[H]
\caption{Construction of Aggregated Weighted Network from Temporal Data}
\begin{algorithmic}[1]
\State \textbf{Input:} Temporal edge list \(E\), number of nodes \(N\)
\State Initialize adjacency matrix \(M \in \mathbb{N}_0^{N \times N}\) with zeros
\ForAll{edges \((t,i,j) \in E\)}
  \State Increment weight \(M[i,j] \gets M[i,j] + 1\)
  \State Because the network is undirected, increment symmetric weight \(M[j,i] \gets M[j,i] + 1\)
\EndFor
\State \textbf{Output:} Aggregated weighted adjacency matrix \(M\)
\end{algorithmic}
\end{algorithm}

\begin{algorithm}[H]
\caption{Calculation of Infection Probability from Epidemiological Parameters}
\begin{algorithmic}[1]
\State \textbf{Given:} Disease basic reproduction number \(R_0\), recovery rate \(\gamma\), activity parameters \(\alpha\), \(m\)
\State Calculate effective infection rate \(\beta \gets \frac{R_0 \cdot \gamma}{2 m \alpha}\)
\State Convert rate \(\beta\) to probability \(p_{\text{infect}} \gets 1 - \exp(-\beta)\)
\State \textbf{Output:} Per-contact infection probability \(p_{\text{infect}}\)
\end{algorithmic}
\end{algorithm}

\begin{algorithm}[H]
\caption{Temporal Network SIR Simulation}
\begin{algorithmic}[1]
\State \textbf{Input:} Temporal edge list \(E\), number of nodes \(N\), time steps \(T\), initial infection seed count \(n_I=1\), per-contact infection probability \(p_{\text{infect}}\), recovery probability \(\gamma\)
\State Initialize state vector \(S\) of length \(N\) with all susceptible (0)
\State Infect randomly chosen seed nodes: set their state to infected (1)
\For{simulation run \(= 1\) \textbf{to} \(nsim\)}
  \For{\(t = 1\) \textbf{to} \(T\)}
    \State Retrieve edges active at time \(t\), \(E_t \subseteq E\)
    \State For each infected node \(i\), recover with probability \(\gamma\)
    \State For each edge \((i,j) \in E_t\):
      \If{\(i\) infected and \(j\) susceptible, infect \(j\) with probability \(p_{\text{infect}}\)}
      \EndIf
      \If{\(j\) infected and \(i\) susceptible, infect \(i\) with probability \(p_{\text{infect}}\)}
      \EndIf
    \EndFor
    \State Update counts of susceptible, infected, recovered nodes
  \EndFor
  \State Store trajectory results
\EndFor
\State \textbf{Output:} Simulation results over time
\end{algorithmic}
\end{algorithm}

\begin{algorithm}[H]
\caption{Static Aggregated Weighted Network SIR Simulation using FastGEMF}
\begin{algorithmic}[1]
\State \textbf{Input:} Aggregated weighted adjacency matrix \(M\), number of nodes \(N\), time steps \(T\), simulation count \(nsim\)
\State Define SIR model schema with compartments S, I, R
\State Initialize network layer with adjacency matrix \(M\)
\State Define parameters: infection rate \(\beta\), recovery rate \(\gamma\)
\State Set initial condition by infecting a single randomly chosen node
\State Run \(nsim\) realizations of SIR simulation using FastGEMF framework
\State Extract time-series of compartment counts with confidence intervals
\State Compute epidemic metrics (final size, peak prevalence, time to peak)
\State \textbf{Output:} Simulation results, epidemic summaries
\end{algorithmic}
\end{algorithm}

\section*{Appendix: Additional Figures}
\addcontentsline{toc}{section}{Appendix: Additional Figures}

\begin{figure}[http]
    \centering
    \begin{subfigure}[b]{0.45\textwidth}
        \centering
        \includegraphics[width=\textwidth]{aggregated-degree-distribution.png}
        \caption*{aggregated-degree-distribution.png}
    \end{subfigure}
    \begin{subfigure}[b]{0.45\textwidth}
        \centering
        \includegraphics[width=\textwidth]{contacts-per-timestep.png}
        \caption*{contacts-per-timestep.png}
    \end{subfigure}
    \caption{Figures: aggregated-degree-distribution.png and contacts-per-timestep.png}
    \label{fig:aggregated-degree-distribution-png}
\end{figure}

\begin{figure}[http]
    \centering
    \begin{subfigure}[b]{0.45\textwidth}
        \centering
        \includegraphics[width=\textwidth]{node-activation-distribution.png}
        \caption*{node-activation-distribution.png}
    \end{subfigure}
    \begin{subfigure}[b]{0.45\textwidth}
        \centering
        \includegraphics[width=\textwidth]{results-11.png}
        \caption*{results-11.png}
    \end{subfigure}
    \caption{Figures: node-activation-distribution.png and results-11.png}
    \label{fig:node-activation-distribution-png}
\end{figure}

\begin{figure}[http]
    \centering
    \begin{subfigure}[b]{0.45\textwidth}
        \centering
        \includegraphics[width=\textwidth]{results-12.png}
        \caption*{results-12.png}
    \end{subfigure}
    \begin{subfigure}[b]{0.45\textwidth}
        \centering
        \includegraphics[width=\textwidth]{sir_model_comparison.png}
        \caption*{sir model comparison.png}
    \end{subfigure}
    \caption{Figures: results-12.png and sir model comparison.png}
    \label{fig:results-12-png}
\end{figure}

\begin{figure}[http]
    \centering
    \begin{subfigure}[b]{0.45\textwidth}
        \centering
        \includegraphics[width=\textwidth]{weighted-degree-distribution.png}
        \caption*{weighted-degree-distribution.png}
    \end{subfigure}
    \caption{Figures: weighted-degree-distribution.png}
    \label{fig:weighted-degree-distribution-png}
\end{figure}
\end{document}