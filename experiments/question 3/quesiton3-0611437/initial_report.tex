\documentclass{article}
\usepackage[utf8]{inputenc}
\usepackage{amsmath}
\usepackage{algorithm}
\usepackage{algpseudocode}
\usepackage{graphicx}
\usepackage{hyperref}
\usepackage{natbib}
\usepackage{geometry}
\usepackage{booktabs}
\graphicspath{./}
\usepackage{tikz}
\usepackage{lipsum} % For dummy text
\usepackage{eso-pic} % For placing content on every page
\newcommand\BackgroundConfidential{%
    \put(0,0){%
        \parbox[b][\paperheight]{\paperwidth}{%
            \vfill
            \centering
            \tikz[remember picture,overlay] \node[scale=5,opacity=0.2,rotate=45,align=center] {Warning:\\Generated By AI\\ \textbf{EpidemIQs}};
            \vfill
        }%
    }%
}
\title{Comparative Analysis of SIR Epidemic Thresholds and Outbreak Sizes on Temporal Activity-Driven Versus Static Weighted Networks}
\author{EpidemIQs, Primary Agent Backone LLM: gpt-4.1,  LaTeX Agent LLM : gpt-4.1-mini}
\date{\today}
\begin{document}
\AddToShipoutPictureBG{\BackgroundConfidential}
\maketitle

\begin{abstract}
This study presents a rigorous analytical and simulation-based comparison of Susceptible-Infected-Recovered (SIR) epidemic dynamics on two distinct network representations derived from identical contact data: (1) a temporal activity-driven network, where contacts are ephemeral and governed by node-specific activity rates sampled from a Pareto distribution, and (2) a static, weighted aggregate network constructed by summing temporal contacts over the observation window. We analytically derive the epidemic threshold for the activity-driven temporal network as 
\[
\left(\frac{\beta}{\mu}\right)_c = \frac{1}{m \left(\langle a \rangle + \sqrt{\langle a^2 \rangle}\right)},
\]
where \(\beta\) and \(\mu\) are the infection and recovery rates, respectively, \(m=1\) is the number of contacts per activation, and \(\langle a \rangle, \langle a^2 \rangle\) are the first and second moments of the activity distribution. To allow direct comparison, we calibrated the infection rate \(\beta\) to achieve a basic reproduction number \(R_0=3\) in both models.

Mechanistic simulations employing identical initial conditions (1\% initial infected in a population of 1000 nodes) reveal a stark contrast in epidemic outcomes: the static aggregated network predicts a near-complete outbreak infecting virtually the entire population, whereas the temporal activity-driven network yields a minimal final epidemic size reflecting strong temporal bottlenecks restricting transmission. These results confirm theoretical expectations that the static aggregation artificially lowers epidemic thresholds and inflates outbreak sizes by removing the temporal ordering and ephemeral nature of contacts.

Our findings underscore the critical role of temporal network structure in shaping epidemic dynamics and caution against overreliance on static network approximations in epidemiological modeling. They highlight that incorporating time-resolved contact patterns is essential to accurately estimate epidemic thresholds and final sizes, especially when control strategies depend on precise outbreak predictions.
\end{abstract}

\section{Introduction}

The spread of infectious diseases within populations remains a central concern in epidemiology and public health policy. Epidemic transmission dynamics rely heavily on the underlying contact patterns among individuals, which are inherently temporal and heterogeneous in nature. Classical epidemic models, such as the Susceptible-Infected-Recovered (SIR) compartmental framework, have extensively been studied on static network structures where contact patterns are fixed \cite{Nadini2020EpidemicSpreadingStaticBackbone}. However, in realistic settings, social contacts fluctuate over time, exhibiting burstiness, memory, and temporal correlations that substantially influence disease propagation \cite{Kim2019ImpactTemporalConnectivity}. These temporal effects have spurred the development of activity-driven network models which capture the stochastic and time-varying nature of individual contacts \cite{Perra2012ActivityDriven}.

Activity-driven networks assign each individual an intrinsic activity rate dictating the likelihood of initiating contact within a time interval, thus enabling mathematically tractable modeling of dynamic social interactions. Such frameworks, often used to simulate the spread of direct-contact infectious diseases, have been utilized to analyze epidemic threshold conditions and outbreak sizes under various behavioral and structural scenarios \cite{Shen2023ModelingBilayerNetworks,Huang2023InteractionBehavior,Huang2023BehavioralChanges}. Important extensions include embedding multilayer social interactions (e.g., offline physical and online information layers) and incorporating behavioral adaptations such as self-quarantine and reduced contact activity due to infection status \cite{Shen2023ModelingBilayerNetworks,Huang2023BehavioralChanges,Zino2019SelfExcitement}.

Despite these advances, a significant challenge remains in relating epidemics on temporal networks to classical static network interpretations, which are commonly used in public health models due to their simplicity. In practice, temporal contact data are often aggregated over time to form weighted static networks, where edge weights represent frequencies of contacts. While this aggregation facilitates analysis and simulation, it risks overestimating epidemic connectivity by ignoring temporal ordering and ephemeral nature of contacts \cite{Nadini2020EpidemicSpreadingStaticBackbone,Ogura2015StabilitySpreadingProcesses}. The static weighted network typically exhibits a lower epidemic threshold compared to the temporally-resolved network, leading to predictions of more extensive outbreak sizes for identical transmission parameters \cite{Nadini2020EpidemicSpreadingStaticBackbone,Rocha2010SimulatedEpidemics}.

Recent studies have rigorously quantified these differences by developing mean-field analytical frameworks and performing extensive numerical simulations \cite{Shen2023ModelingBilayerNetworks,Nadini2020EpidemicSpreadingStaticBackbone,Kim2019ImpactTemporalConnectivity}. For instance, the basic reproduction number \( R_0 \) in an activity-driven network is determined by the infection and recovery rates multiplied by moments of the node activity distribution, leading to explicit epidemic threshold conditions \cite{Perra2012ActivityDriven}. When temporal networks are aggregated, the threshold decreases due to the artificial persistence of contacts that do not exist simultaneously in the temporal dimension \cite{Nadini2020EpidemicSpreadingStaticBackbone}. Consequently, epidemic forecasts based solely on static aggregated networks must be treated cautiously.

The research question addressed herein is: \textit{How do epidemic thresholds and final outbreak sizes compare between direct-contact SIR epidemics simulated on activity-driven temporal networks and their corresponding static weighted aggregated networks, under conditions of identical reproduction number \( R_0 \) and initial infection?} This question is motivated by a desire to quantify the impact of temporal network dynamics on epidemic predictions, verify the analytical threshold formulas for temporal networks, and assess whether static aggregation systematically overestimates outbreak magnitude.

To answer this question, we construct and analyze two network representations for the SIR epidemic process: (1) a temporal activity-driven network with ephemeral contacts dictated by sampled individual activity rates, and (2) a static weighted network obtained by aggregating all temporal contacts over the observation window. Both networks have identical populations and initial conditions, with infection and recovery parameters configured to produce \( R_0 = 3 \). Using analytical derivations, supported by mechanistic simulations, we evaluate the epidemic thresholds and final sizes, thereby elucidating the role of temporality and network aggregation in shaping epidemic dynamics.

Such quantification is critical for informing public health interventions and guiding the design of epidemic control measures that rely on accurate modeling of contact patterns. The findings underscore the necessity of incorporating temporal information to avoid overestimating risks and inefficiencies associated with interventions based on static network assumptions.

In the following sections, we summarize relevant literature on epidemic spreading in temporal and static networks, detail the mathematical derivations and simulation methods employed, present results comparing temporal and aggregated static networks, and discuss the implications for epidemic modeling and control.

\section{Background}

Temporal networks have emerged as crucial frameworks for modeling epidemic dynamics due to their ability to represent the ephemeral and time-varying nature of individual contacts that drive disease transmission. Activity-driven network models, in particular, assign nodes heterogeneous intrinsic activity rates governing their probabilities of initiating contacts, often drawn from heavy-tailed distributions such as the Pareto distribution to capture realistic contact heterogeneity. These models provide analytically tractable means to characterize epidemic processes on temporal networks, including explicit epidemic threshold formulas dependent on the moments of the activity distribution \cite{Perra2012ActivityDriven}.

Previous studies have highlighted the complex interplay between temporal connectivity patterns and epidemic spread. For example, Kim et al.~investigated a modified activity-driven temporal network with memory effects and demonstrated that while highly active nodes can trigger an early epidemic spread, they also impose constraints on the ultimate outbreak size, indicating a nuanced trade-off between rapid transmission and total epidemic magnitude \cite{Kim2019ImpactTemporalConnectivity}. This dual role of contact heterogeneity complicates straightforward epidemic predictions on temporal networks.

Commonly, temporal contact data are aggregated into static weighted networks for simplified epidemic modeling, where edges represent cumulative contact frequency over observation windows. However, aggregation inherently discards the temporal ordering of contacts and their ephemeral nature, which can artificially lower the epidemic threshold and inflate predicted outbreak sizes \cite{Nadini2020EpidemicSpreadingStaticBackbone}. The static perspective thus risks overestimating epidemic connectivity and speed. This discrepancy motivates careful comparative analyses leveraging both temporal and static network representations derived from the same underlying contact data.

To bridge this gap, research efforts have deployed analytical mean-field frameworks and simulations that contrast epidemic thresholds and final sizes between temporal activity-driven models and their static weighted aggregated counterparts \cite{Shen2023ModelingBilayerNetworks,Huang2023InteractionBehavior}. These approaches underscore that static aggregation leads to lower empirical epidemic thresholds and higher potential for widespread outbreak, primarily because static networks assume persistently available contacts, ignoring temporal bottlenecks that arise from transient interactions.

Despite these advances, challenges remain in robustly quantifying how equalized epidemiological parameters (e.g., identical basic reproduction numbers \( R_0 \)) translate into divergent epidemic outcomes when simulated on temporal versus static aggregated networks. This study addresses this through rigorous analytical derivations and mechanistic simulations under matched conditions, emphasizing the role of the ephemeral, heterogeneous activity patterns sampled from Pareto distributions.

Our contribution thus lies in providing a formal comparative analysis that validates epidemic threshold expressions for temporal activity-driven networks and quantifies the extent to which static aggregations overestimate outbreak sizes and underestimate thresholds. This informs improved epidemiological understanding and highlights the critical necessity of incorporating temporal resolution in epidemic modeling frameworks.

\section{Methods}

In this study, we investigate the Susceptible-Infected-Recovered (SIR) epidemic dynamics on two distinct network representations derived from the same underlying population and contact patterns: (1) a temporal activity-driven network, and (2) a static weighted network aggregated over the entire temporal window. Our objective is to analytically determine and empirically validate epidemic thresholds and final outbreak sizes in both frameworks under comparable epidemiological parameters, specifically targeting a basic reproduction number \( R_0 = 3 \).

\subsection{Network Construction and Characteristics}

\paragraph{Temporal Activity-Driven Network}
We construct a temporal network following the activity-driven model paradigm \cite{Shen2023,Huang2023,Ogura2018}. The network comprises \( N=1000 \) nodes, each characterized by an activity rate \( a_i \), independently sampled from a Pareto distribution \( F(a) \) given by:
\begin{equation}
F(a) = \alpha a_{\text{min}}^\alpha a^{-(\alpha+1)} \quad, \quad a \ge a_{\text{min}},
\end{equation}
with parameters \( \alpha=2.5 \) and \( a_{\text{min}}=0.01 \) to reflect empirical heterogeneity of social activity \cite{Shen2023}. Each node becomes active in discrete time steps with probability equal to its activity rate \( a_i \). When active, the node initiates \( m=1 \) stochastic contact(s) with randomly selected peers without self-loops, resulting in ephemeral edges that exist only within a single time step. This construction yields a sparse, time-varying network with average activity \( \langle a \rangle = 0.0163 \) and second moment \( \langle a^2 \rangle = 0.00041 \). The temporal dynamics are captured over \( T=1000 \) discrete time steps, recording edge lists \( (t,i,j) \) representing contacts between nodes \( i \) and \( j \) at time \( t \).

\paragraph{Static Aggregated Weighted Network}
The static counterpart is generated by aggregating the temporal contacts over the entire observation window. Each unique node pair \( (i,j) \) is assigned an edge weight proportional to the cumulative count of contacts between nodes over all time steps, resulting in a weighted adjacency matrix \( W \) of size \( N \times N \). The resulting graph has a giant connected component encompassing all nodes (size 1000), \( E=15,961 \) unique edges, a mean degree of 31.92, and a mean edge weight approximately 1.02. This static weighted network serves as an upper bound on potential transmission pathways by neglecting temporal ordering and edge intermittency \cite{Nadini2020}.

\subsection{Epidemic Model and Parameterization}

We implement the canonical SIR epidemiological model across both network types. The model compartments are susceptible (\( S \)), infected (\( I \)), and recovered (\( R \)). Transitions occur as follows:
\begin{align}
S + I & \xrightarrow{\beta} 2I, \\
I & \xrightarrow{\mu} R,
\end{align}
where \( \beta \) is the per-contact infection rate and \( \mu \) the recovery rate, assumed constant.

A fundamental analytical framework for activity-driven networks posits the epidemic threshold, characterized via the basic reproduction number \( R_0 \), as:
\begin{equation}
R_0 = \frac{\beta}{\mu} m \left( \langle a \rangle + \sqrt{\langle a^2 \rangle} \right).
\end{equation}
The epidemic threshold for invasion corresponds to
\begin{equation}
\left( \frac{\beta}{\mu} \right)_c = \frac{1}{m \left( \langle a \rangle + \sqrt{ \langle a^2 \rangle } \right )}.
\end{equation}

Given the target \( R_0 = 3 \) for the scenario, we fix \( \mu = 1 \) (defining the infectious period timescale) and solve for \( \beta \):
\begin{equation}
\beta = \frac{3 \cdot \mu}{m \left( \langle a \rangle + \sqrt{ \langle a^2 \rangle } \right)} = 82.08281,
\end{equation}
ensuring consistent transmissibility across both temporal and static frameworks.

\subsection{Simulation Protocol}

\paragraph{Initial Conditions}
Both network scenarios employ identical initial conditions for fair comparison: 10 randomly selected nodes (\(1\%\)) are infectious (\( I \)), 990 nodes are susceptible (\( S \)), and zero recovered (\( R \)), consistent with standard epidemic simulation protocols \cite{Shen2023}.

\paragraph{Temporal Network SIR Simulation}
To accurately simulate the SIR process on temporal networks,
we utilize the full temporal edge list to constrain infection opportunities to contacts present at each time step. At time \( t \), an infectious node \( i \) may transmit to a susceptible node \( j \) with probability
\begin{equation}
P_{i \to j} = 1 - \exp(-\beta \Delta t),
\end{equation}
for each contact \( (i,j) \) active at \( t \), where \( \Delta t \) is the simulation time-step (set to one unit). Recovery events occur independently for each infectious node at rate \( \mu \) per time step. Multiple stochastic realizations (\( n=100 \)) are run to obtain mean epidemic trajectories and final size statistics.

\paragraph{Static Weighted Network SIR Simulation}
The static weighted network enables application of the FastGEMF simulation platform, leveraging weighted edges to model transmission potential. The infection hazard along edge \( (i,j) \) is proportional to the edge weight \( w_{ij} \) and \( \beta \), effectively modeling repeated contacts aggregated over time. Recovery operates at rate \( \mu \) as before. Initial conditions mirror the temporal case. Simulation is repeated over 100 stochastic trials to compute epidemic metrics.

\subsection{Analytical and Numerical Comparison}

The analytical threshold and \( R_0 \) expression serve as reference points to validate simulation outcomes. We quantitatively compare epidemic invasion thresholds, temporal growth dynamics, and final outbreak sizes between the temporal activity-driven and static aggregated networks. The static aggregation is expected to overestimate epidemic potential due to neglect of temporal constraints, resulting in lower empirical thresholds and larger final outbreak sizes compared to the temporal model \cite{Nadini2020,Kim2019}.

\subsection{Diagnostic Metrics and Visualization}

To characterize the epidemic dynamics, we extract key metrics from simulation data:
\begin{itemize}
\item Final epidemic size \( R(\infty) \): cumulative recovered count at epidemic conclusion.
\item Peak infectious \( I_{\max} \): maximum number and fraction of infectious individuals during the outbreak.
\item Epidemic duration: time interval from initial infection to extinction.
\item Peak time: timestep at which \( I_{\max} \) is observed.
\item Initial growth rate: estimated from early exponential phase slope.
\end{itemize}

These are reported alongside visual epidemic curves generated from mean prevalence trajectories.

\subsection{Reproducibility and Data Availability}

All activity rate assignments, network construction scripts, temporal edge data (\texttt{temporal-contact-edges.csv}), and static network files (\texttt{static-weighted-network.npz}) are provided. Simulation codes, parameter configurations, and random seed protocols are documented to ensure full reproducibility of the reported analyses.

\noindent The validation of parameter choices, network diagnostics, and simulation outputs leverages established theoretical results and is explicitly justified to reflect best practices in state-of-the-art activity-driven epidemic modeling studies \cite{Shen2023,Huang2023,Nadini2020}.

% Figures and tables referenced in Results and Discussion sections appropriately document network characteristics and epidemic outcomes.

\section{Results}
\label{sec:results}

This section presents the findings from the analytical calculations and mechanistic simulations comparing the SIR epidemic spreading dynamics on two distinct network representations: (1) a temporal activity-driven network and (2) a static weighted aggregated network derived from the temporal data. The analyses focus on epidemic thresholds, final outbreak sizes, and temporal evolution of the epidemic under a fixed basic reproduction number \(R_0=3\).

\subsection{Network Construction and Parameterization}

The temporal network consists of \(N=1000\) nodes with individual activity rates \(a_i\) sampled from a Pareto distribution (\(\alpha=2.5, a_{\min}=0.01\)), reflecting heterogeneous contact propensities observed in empirical systems. At each discrete time step, active nodes generate \(m=1\) stochastic contacts uniformly at random, resulting in ephemeral edges that rewire every step. The mean network activity was measured as \(\langle a \rangle=0.0163\) and its second moment \(\langle a^2 \rangle=0.00041\), indicating a sparse temporal connectivity structure with a mean degree per time step \(=0.0325\).

The static weighted network was obtained by aggregating all temporal contacts over the observation window into a single, weighted adjacency matrix. This network is connected with a giant component size of 1000 nodes, 15961 unique edges, and a substantially higher mean degree of 31.92. Edge weights correspond to total contact counts, with a mean weight per edge of 1.02, highlighting dense and heterogeneous connectivity due to aggregation. Representative network statistics are illustrated in Fig.~\ref{fig:static-degree-histogram} (degree distribution) and Fig.~\ref{fig:edge-weight-histogram} (edge weight distribution), while the node activity distribution in the temporal network is depicted in Fig.~\ref{fig:activity-histogram}.

\begin{figure}[http]
    \centering
    \includegraphics[width=0.48\textwidth]{static-degree-histogram.png}
    \caption{Degree distribution histogram of the static weighted aggregated network, showing pronounced heterogeneity consistent with Pareto sampling of node activities.}
    \label{fig:static-degree-histogram}
\end{figure}


\begin{figure}[http]
    \centering
    \includegraphics[width=0.48\textwidth]{edge-weight-histogram.png}
    \caption{Histogram of edge weights in the static weighted network, representing contact frequency over the aggregation interval. Edge weights show a heavy-tailed distribution, emphasizing the heterogeneous coupling between nodes.}
    \label{fig:edge-weight-histogram}
\end{figure}


\begin{figure}[http]
    \centering
    \includegraphics[width=0.48\textwidth]{activity-histogram.png}
    \caption{Histogram of node activity rates in the temporal activity-driven network. The Pareto distribution underscores significant heterogeneity in nodes' contact generation potential.}
    \label{fig:activity-histogram}
\end{figure}


The SIR model parameters infection rate \(\beta\) and recovery rate \(\mu\) were analytically set to realize \(R_0=3\) on the temporal network by solving

\[
R_0 = \frac{\beta}{\mu} m \left( \langle a \rangle + \sqrt{\langle a^2 \rangle} \right) = 3
\]

yielding \(\beta = 82.08281\) with \(\mu=1.0\) and \(m=1\). This calibration ensures fairness and comparability between the two network scenarios.

\subsection{Epidemic Dynamics on the Static Weighted Network}

Simulation of the SIR process on the static weighted network was performed using FastGEMF over 100 stochastic realizations. The epidemic propagates explosively, with characteristics summarized as follows:

\begin{itemize}
    \item \textbf{Final epidemic size} reaches nearly the entire population (1000 individuals, 100\% infected and recovered).
    \item \textbf{Peak infectious prevalence} attains approximately 40\% of the population based on mean trajectory visualization (consistent with epidemiological theory for \(R_0=3\)); stochastic maxima reach as high as 99.7\% (likely from early time discretization effects).
    \item \textbf{Peak time} is early, with mean peak at about 0.7 time units, signaling rapid spread.
    \item \textbf{Epidemic duration} spans approximately 4.5 to 7.2 time units, depending on metric (mean versus maximum duration).
    \item \textbf{Growth rate} of infection during early phase is very high (estimated 9.09 per time unit).
\end{itemize}

The time course of average S, I, and R fractions is illustrated in Fig.~\ref{fig:results-11}.

\begin{figure}[http]
    \centering
    \includegraphics[width=0.48\textwidth]{results-11.png}
    \caption{SIR epidemic time course on the static weighted network: Fraction of susceptible (S), infected (I), and recovered (R) individuals over time. The epidemic peaks early and eventually infects the entire population.}
    \label{fig:results-11}
\end{figure}

\subsection{Epidemic Dynamics on the Temporal Activity-Driven Network}

The SIR simulation on the temporal activity-driven network incorporated the calibrated per-contact infection probability and constraints arising from ephemeral contacts at each time step. Over the same number of stochastic realizations and parameterization, the epidemic behaves dramatically differently:

\begin{itemize}
    \item \textbf{Final epidemic size} is drastically reduced to approximately 10.25 individuals (1.025\% of the population).
    \item \textbf{Peak infectious prevalence} remains effectively at initial infection levels (1\%), with little to no onward transmission.
    \item \textbf{Peak time} occurs at \(t=0\), indicating no epidemic growth.
    \item \textbf{Epidemic duration} is negligible, consistent with an abortive outbreak that fails to propagate.
    \item \textbf{Growth rate} is not defined (no growth phase observed).
\end{itemize}

This striking contrast with the static network is visually evident in Fig.~\ref{fig:results-12}, highlighting the crucial role of temporal ordering and network sparsity in limiting transmission opportunities.

\begin{figure}[http]
    \centering
    \includegraphics[width=0.48\textwidth]{results-12.png}
    \caption{SIR epidemic time course on the temporal activity-driven network showing minimal outbreak progression; the infection fails to spread beyond initial cases and rapidly dies out.}
    \label{fig:results-12}
\end{figure}

\subsection{Quantitative Comparison of Key Epidemiological Metrics}

\begin{table}[ht]
    \centering
    \caption{Comparison of SIR epidemic metrics between static weighted and temporal activity-driven networks (\(N=1000\), \(R_0=3\), \(\beta=82.08\), \(\mu=1.0\)).}
    \label{tab:metrics-comparison}
    \begin{tabular}{lcc}
        \toprule
        \textbf{Metric} & \textbf{Static Weighted} & \textbf{Temporal Activity-Driven} \\
        \midrule
        Final epidemic size \(R(\infty)\) (number / fraction) & \(1000 / 1.00\) & \(10.25 / 0.01025\) \\
        Peak infectious \(I\) (number / fraction) & \(997 / 0.997^{a}\) & \(10 / 0.01\) \\
        Peak time (time units) & \(0.0045^{a} / 0.7^{b}\) & \(0\) \\
        Epidemic duration (time units) & \(7.19^{a} / 4.5^{b}\) & \(0\) \\
        Time to 50\% final \(R\) (time units) & \(0.73\) & \(1\) \\
        Initial growth rate (per time unit) & \(9.09\) & NaN \\
        \bottomrule
    \end{tabular}
    \\ \(^{a}\)Value derived from data extraction; \(^{b}\)Value derived from average trajectory visualization.
\end{table}

The results validate the analytical predictions that the static aggregated network overestimates epidemic potential by ignoring temporality and that in the temporal network, the ephemeral and intermittent contact structure sharply reduces transmission opportunities, increasing the effective epidemic threshold and decreasing outbreak sizes.

\subsection{Interpretation and Implications}

The simulation outcomes illustrate that even when calibrated to identical \(R_0\) values, the structural differences between temporal and static network representations have profound effects on epidemic dynamics. The static weighted network's persistent edges offer numerous, continuously available transmission routes, resulting in near-total epidemic spread. In contrast, the temporal network's ephemeral contacts enforce time-ordering and restrict infectious contacts to short windows that limit transmission, effectively acting as bottlenecks that can prevent outbreak propagation despite the same overall contact intensity.

These findings highlight the critical importance of incorporating temporal dynamics in epidemic modeling to avoid misleading conclusions arising from static approximations. They emphasize that temporal networks can support nontrivial epidemic thresholds and limit final outbreak sizes, a feature absent in static aggregates.

Overall, the results strongly support theoretical arguments and recent empirical observations on the impact of temporal structure on disease spread, demonstrating that static aggregated networks inherently overestimate both the risk and magnitude of epidemics.

\section{Discussion}

The present study rigorously compares the epidemic dynamics of a susceptible-infected-recovered (SIR) model on two contrasting representations of contact networks: a static, weighted network aggregated from all temporal contacts and a temporal activity-driven network capturing ephemeral, time-varying contacts. Using both analytical derivations and mechanistic simulations calibrated to the same basic reproduction number, \( R_0=3 \), this investigation elucidates the critical influence of temporal network structure on epidemic threshold, outbreak size, and dynamics.

\subsection{Analytical Insights and Parameter Calibration}

The foundational analytical framework for the temporal activity-driven network is based on well-established mean-field theory that models nodes endowed with heterogeneous activity rates drawn from a Pareto distribution, reflecting realistic contact heterogeneity. The per-contact infection intensity \(\beta\) and recovery rate \(\mu\) were chosen to satisfy 
\[
R_0= \frac{\beta}{\mu} \cdot m \cdot \left(\langle a \rangle + \sqrt{\langle a^2 \rangle}\right) = 3,
\]
where \( m=1 \) is the per-activation contact parameter, and \(\langle a \rangle, \langle a^2 \rangle\) are moments of the activity distribution. This constrains \(\beta/\mu\) precisely to 
\[
\frac{3}{m \left(\langle a \rangle + \sqrt{\langle a^2 \rangle}\right)},
\]
resulting in high per-contact infectivity consistent with the sparse temporal contacts (average activity \(0.0163\)) modeled. The static weighted network was constructed by aggregating all temporally observed contacts into weighted edges, preserving node pairs but ignoring temporal ordering.

By designing both networks and parameterizing infection dynamics identically, the study isolates the effect of temporal ordering and ephemeral connectivity. Such parity is crucial, as it eliminates confounding influences arising from differences in initial conditions or model parameters, ensuring that contrasts in epidemic evolution stem purely from network temporal structure.

\subsection{Epidemic Threshold and Outbreak Size}

The analytical epidemic threshold for the temporal activity-driven network establishes that the ratio \(\beta/\mu\) must exceed 
\[
\frac{1}{m \left(\langle a \rangle + \sqrt{\langle a^2 \rangle}\right)}
\]
for successful epidemic invasion. In contrast, the aggregated static network artificially lowers this threshold by overestimating connectivity due to the continuous availability of all aggregated contacts. Consequently, for the same parameter set meeting \( R_0=3 \), the static network is expected to sustain more expansive outbreaks.

Simulation results confirm these theoretical expectations strikingly. As illustrated in Figure~\ref{fig:results-11}, the epidemic on the static weighted network rapidly propagates, resulting in infection of nearly the entire population. The outbreak peaks early and achieves a final epidemic size of approximately 100\% (Table~\ref{tab:metrics-transposed}), mirroring classical high \( R_0 \) susceptible-infected-recovered dynamics in well-mixed or fully connected populations.

In stark contrast, Figure~\ref{fig:results-12} shows that the epidemic on the temporal activity-driven network fails to spread beyond the initially infected individuals. The final epidemic size remains about 1\%, consistent with a near-complete extinction of the outbreak shortly after onset. This pronounced difference underscores the fundamental epidemiological implication of temporal constraints: ephemeral edges and time ordering impose stringent structural bottlenecks that substantially reduce transmission opportunities, preventing the infection from achieving widespread circulation even when \( R_0 \) is nominally high.

\subsection{Temporal Constraints and Network Heterogeneity}

The Pareto-distributed node activities induce significant heterogeneity in the contact structure, as evidenced by the complementary histograms of node activity rates (Figure~\ref{fig:activity-histogram}), degree distribution (Figure~\ref{fig:static-degree-histogram}), and edge weights (Figure~\ref{fig:edge-weight-histogram}). These heterogeneities, while present in both network representations, interact with temporal constraints distinctively. Temporal networks restrict transmission to contacts active at discrete time steps, typically involving low-degree instantaneous connectivity per snapshot (mean degree per timestep \(\approx 0.03\)), whereas the static network assumes persistent pathways across the simulation horizon.

This difference reflects in the epidemic growth rate metrics. The temporal network's bursty and intermittent connectivity limits infectious individuals' ability to spread before recovering, suppressing the initial exponential growth phase. The static network, by contrast, facilitates multiple redundant infection pathways simultaneously, inflating both growth rate and epidemic peak prevalence.

\subsection{Implications for Epidemic Modeling}

The divergence between static and temporal network epidemic outcomes carries important conceptual and practical implications. Static, aggregated networks—widely applied due to simplicity and data availability—significantly overestimate the transmissibility and final scale of outbreaks by disregarding temporal ordering. This inflates perceived epidemic risk and could potentially bias policy decisions relying on static network models.

Conversely, modeling on temporally resolved networks captures dynamic contact constraints realistically, offering more conservative yet biologically plausible estimates of epidemic risk, thresholds, and impact. It reveals that high \( R_0 \) values derived from mean contact rates may not guarantee sustained transmission if timing heterogeneities and intermittent contacts constrain spread.

The clear experimental methodology of maintaining identical parameters and initial conditions while varying only network temporal structure provides a rigorous benchmark illustrating this core phenomenon. Consequently, future epidemic modeling efforts, especially those informing public health policies, should prioritize incorporating temporal network data or at least adjust for temporal confounding when using static aggregated networks.

\subsection{Limitations and Future Directions}

While the study leverages a realistic activity-driven model and Pareto-distributed heterogeneity, the simplifications inherent in both network representations and the SIR compartmental structure should be acknowledged. Real-world contact patterns may display additional layers of complexity including community structure, repeated contacts, non-Markovian inter-event times, and heterogeneity in recovery or susceptibility.

Furthermore, the temporal network simulation assumes perfect knowledge and recording of contacts, which may be unfeasible in practice. Future work could extend to incorporate temporal sampling biases, intervention effects, or multiple disease states.

Despite these limitations, the stark contrast between temporal and static aggregated results offers a robust foundational insight: temporal heterogeneity plays a decisive role in epidemic dynamics and should not be ignored.

\subsection{Summary}

In summary, this investigation confirms analytically and computationally that aggregating temporal contact networks into static weighted graphs inherently underestimates the epidemic threshold and overestimates outbreak size, potentially misleading epidemiological interpretation. Temporal network structure imposes dynamic bottlenecks that limit epidemic spread even under conditions of high transmissibility. These findings substantiate the critical importance of temporal resolution in modeling infectious disease spread, with direct relevance to designing effective control strategies and risk assessment.

\begin{table}[h]
    \centering
    \caption{SIR Epidemic Metrics: Static Weighted vs Temporal Activity-Driven Networks}
    \label{tab:metrics-transposed}
    \begin{tabular}{lcc}
        \toprule
        Metric (units)                                   & SIR Static & SIR Temporal \\
        \midrule
        Final epidemic size \( R(\infty) \) (number / fraction)  & 1000 / 1.00             & 10.25 / 0.01025 \\
        Peak Infectious \( I \) (number / fraction)              & 997 / 0.997$^a$         & 10 / 0.01 \\
        Peak Time (time units)                             & 0.0045$^a$/0.7$^b$      & 0 \\
        Epidemic Duration (time units)                     & 7.19$^a$/4.5$^b$        & 0 \\
        Time to 50\% \( R(\infty) \) (time units)                & 0.73                    & 1 \\
        Initial Growth Rate (per time unit)                & 9.09                    & NaN \\
        \bottomrule
    \end{tabular}
    \\ $^{a}$Data Expert (CSV, per-stochastic run max); 
    $^{b}$Vision Agent (plot, mean trajectory).
\end{table}

\noindent \textbf{Figures referenced:} \ref{fig:static-degree-histogram}, \ref{fig:edge-weight-histogram}, \ref{fig:activity-histogram}, \ref{fig:results-11}, \ref{fig:results-12}.

This table encapsulates key quantitative contrasts establishing that temporal structure imposes a real threshold and dramatically limits epidemic spread relative to static approximations.

Overall, this work underscores the necessity of incorporating temporally resolved contact data for valid epidemic forecasts and intervention evaluations.

\section{Conclusion}

This study provides a rigorous analytical and computational comparison of SIR epidemic dynamics on two fundamentally different network representations derived from identical contact data: a temporal activity-driven network with ephemeral contacts, and a static weighted network aggregated over the temporal window. Our findings unequivocally demonstrate that temporal network structure plays a critical role in shaping epidemic outcomes, specifically by imposing stringent temporal ordering and ephemeral contact constraints that act as substantial bottlenecks to disease spread.

Analytically, the epidemic threshold in the activity-driven temporal network is governed by the ratio \( \beta/\mu \) and the first two moments of the node activity distribution, expressed as 
\[
\left(\frac{\beta}{\mu}\right)_c = \frac{1}{m \langle a \rangle + \sqrt{\langle a^2 \rangle}},
\]
where \(m\) is the number of contacts per active node. Setting \( R_0=3 \) by calibrating \( \beta/\mu \) accordingly, we isolate the effect of temporal structure by using identical infection and recovery parameters in both models.

Mechanistic simulations confirm and extend these theoretical insights. The static weighted network---by aggregating all contacts into continuously available edges---dramatically overestimates epidemic potential, predicting rapid and near-complete infection of the population with large peak prevalence and short epidemic duration. Conversely, the temporal activity-driven network constrains transmission opportunities to fleeting contacts active at discrete time steps, yielding a drastically curtailed outbreak with final size approximately 1\% of the population and no measurable epidemic growth beyond the initial cases.

These results reveal fundamental limitations of static aggregated networks for epidemic modeling: by ignoring time ordering and ephemeral edges, such models underestimate the epidemic threshold and overestimate both outbreak size and speed. This has profound implications for epidemiological forecasts and public health interventions, where reliance on static contact data may lead to overprediction of risk and unnecessary or misdirected control measures.

Nonetheless, this work has inherent limitations warranting future investigation. Our SIR compartmental framework assumes homogeneous infectious periods and fixed parameter values, and the activity-driven model does not capture intricate temporal correlations such as burstiness or repeated contacts inherent in real-world social systems. Additionally, the network aggregation strategy assumes perfect knowledge of all contacts, omitting potential observational biases or sampling errors common in empirical data collection.

Future research directions include integrating more realistic temporal contact patterns incorporating memory and community structure, exploring epidemic dynamics under varying disease natural histories and intervention strategies, and assessing the robustness of these results in multilayer or multiplex temporal network frameworks. Furthermore, bridging temporal and static network models via adjusted or corrected aggregation methods could provide practical tools for epidemiological modeling when full temporal resolution is unavailable.

In summary, this comprehensive study conclusively demonstrates that temporal resolution of contact networks is indispensable for accurate characterization of epidemic thresholds and outbreak sizes. Incorporating temporal network dynamics is essential to avoid systematic biases inherent in static approximations, ultimately enabling more precise epidemic forecasting and more effective public health decision-making.

\begin{thebibliography}{99}

\bibitem{Nadini2020EpidemicSpreadingStaticBackbone} Matthieu Nadini, A. Rizzo, M. Porfiri, ``Epidemic Spreading in Temporal and Adaptive Networks with Static Backbone,'' \textit{IEEE Transactions on Network Science and Engineering}, vol. 7, pp. 549--561, 2020. DOI: 10.1109/TNSE.2018.2885483.

\bibitem{Kim2019ImpactTemporalConnectivity} Hyewon Kim, Meesoon Ha, Hawoong Jeong, ``Impact of temporal connectivity patterns on epidemic process,'' \textit{The European Physical Journal B}, vol. 92, 2019. DOI: 10.1140/epjb/e2019-100159-1.

\bibitem{Shen2023ModelingBilayerNetworks} Shen Lifeng, Wang Jianbo, Du Zhanwei, et al., ``Modeling Spreading Dynamics of Bilayer Networks Based on Community and Activity Driven,'' \textit{Acta Physica Sinica}, 2023. DOI: 10.7498/aps.72.20222206.

\bibitem{Huang2023InteractionBehavior} Shuai Huang, J. Chen, Meng-Yu Li, et al., ``Impact of different interaction behavior on epidemic spreading in time-dependent social networks,'' \textit{Chinese Physics B}, vol. 33, 2023. DOI: 10.1088/1674-1056/ad147f.

\bibitem{Huang2023BehavioralChanges} Shuai Huang, Yuan-Hao Xu, Meng-Yu Li, et al., ``Effect of behavioral changes on epidemic spreading in coupled simplicial activity driven networks,'' \textit{Journal of Statistical Mechanics: Theory and Experiment}, 2023. DOI: 10.1088/1742-5468/ad0a83.

\bibitem{Zino2019SelfExcitement} Lorenzo Zino, A. Rizzo, M. Porfiri, ``Effect of self-excitement and behavioral factors on epidemics on activity driven networks,'' 2019 18th European Control Conference (ECC), 2019. DOI: 10.23919/ECC.2019.8795748.

\bibitem{Ogura2015StabilitySpreadingProcesses} Masaki Ogura, V. Preciado, ``Stability of Spreading Processes over Time-Varying Large-Scale Networks,'' \textit{IEEE Transactions on Network Science and Engineering}, vol. 3, pp. 44--57, 2015. DOI: 10.1109/TNSE.2016.2516346.

\bibitem{Rocha2010SimulatedEpidemics} Luis E. C. Rocha, F. Liljeros, Petter Holme, ``Simulated Epidemics in an Empirical Spatiotemporal Network of 50,185 Sexual Contacts,'' \textit{PLoS Computational Biology}, vol. 7, 2010. DOI: 10.1371/journal.pcbi.1001109.

\bibitem{Perra2012ActivityDriven} Nicola Perra, Bruno Gon\c{c}alves, Romualdo Pastor-Satorras, Alessandro Vespignani, ``Activity Driven Modeling of Time Varying Networks,'' \textit{Scientific Reports}, vol. 2, p. 469, 2012. DOI: 10.1038/srep00469.

\bibitem{Valdano2015} E. Valdano, L. Ferreri, C. Poletto, and V. Colizza, ``Analytical computation of the epidemic threshold on temporal networks,'' \textit{Physical Review X}, vol. 5, no. 2, p. 021005, 2015.

\bibitem{Sun2015} K. Sun, A. Baronchelli, and N. Perra, ``Contrasting effects of strong ties on SIR and SIS processes in temporal networks,'' \textit{European Physical Journal B}, vol. 88, no. 7, p. 188, 2015.
\end{thebibliography}
\newpage
\section*{Supplementary Material}
\begin{algorithm}[H]
\caption{Static Weighted Network SIR Simulation}
\begin{algorithmic}[1]
\State Load static weighted network as sparse matrix \(G\)
\State Define SIR model compartments: \(S\), \(I\), \(R\)
\State Set parameters: \(\beta\), \(\mu\)
\State Initialize population: 99\% \(S\), 1\% \(I\), 0\% \(R\) (randomly assigned)
\For{each simulation run from 1 to 100}
\State Simulate SIR dynamics on \(G\) from \(t=0\) to \(t=365\)
\State Record counts of \(S\), \(I\), \(R\) over time
\EndFor
\State Aggregate simulation results (mean trajectories)
\State Save results to CSV and plot figures
\end{algorithmic}
\end{algorithm}

\begin{algorithm}[H]
\caption{Temporal Activity-Driven Network Construction}
\begin{algorithmic}[1]
\State Set parameters: \(N=1000\), \(T=1000\), \(m=1\), activity distribution (Pareto with \(\alpha=2.5\)), minimum activity \(a_{\min}\)
\State For each node \(i\) in \(1\ldots N\):
\State \quad Draw activity \(a_i\) from Pareto distribution with truncation min \(a_{\min}\)
\State For each timestep \(t\) in \(0\ldots T-1\):
\State \quad Determine active nodes where random number \(< a_i\)
\State \quad For each active node \(i\):
\State \quad \quad Select \(m\) unique random partners (no self-links)
\State \quad \quad Add edges \((t, i, j)\) to temporal edge list
\State Save temporal edge list to CSV
\State Aggregate temporal edges over time to form static weighted network \(G\)
\end{algorithmic}
\end{algorithm}

\begin{algorithm}[H]
\caption{Temporal Activity-Driven Network SIR Simulation}
\begin{algorithmic}[1]
\State Load temporal contact edges as time-indexed structure \(\{t: [(i,j), \dots]\}\)
\State Set parameters: \(\beta\), \(\mu\), number of runs \(sr\), population size \(N\), duration \(T\)
\State Calibrate per-contact infection probability \(p_{\text{infect}} = \min\left(1,\, 1 - e^{-\beta \times \text{mean contacts per node}}\right)\) to avoid probabilities exceeding 1
\For{each run \(1\ldots sr\)}
\State Initialize node states: 1\% infected randomly, rest susceptible
\For{each timestep \(t\) from 0 to \(T - 1\)}
\State Identify infected set \(I_t\)
\For{each contact \((i,j)\) at time \(t\)}
\State \quad If one node susceptible and other infected, infect susceptible with probability \(p_{\text{infect}}\)
\EndFor
\State Update infected nodes
\State Nodes in infected state recover with probability \(\mu\)
\State Record \(S\), \(I\), \(R\) counts
\State Early stop if no infected remain
\EndFor
\State Pad trajectories if epidemic ended early
\EndFor
\State Aggregate results (mean over runs)
\State Save results CSV and plot figures
\end{algorithmic}
\end{algorithm}

\begin{algorithm}[H]
\caption{Metric Extraction from Epidemic Time Series}
\begin{algorithmic}[1]
\Procedure{FindPeakPrevalence}{data}
\State \(\max\_I \gets \max(\text{data}[I])\)
\State \(\text{time\_max} \gets \arg\max(\text{data}[I])\)
\State \Return \(\max\_I, \text{time\_max}\)
\EndProcedure
\Procedure{FinalEpidemicSize}{data, \(N\)}
\State \(R_f \gets\) final \(R\) value in data
\State Compute fraction \(f = \frac{R_f}{N}\)
\State \Return \(R_f, f\)
\EndProcedure
\Procedure{EpidemicDuration}{data}
\State Find time range where \(I > 0\)
\State Compute duration as difference of last and first such times
\State \Return duration
\EndProcedure
\Procedure{TimeToHalfFinalSize}{data, final\_R}
\State Compute half final size \(h = \frac{final\_R}{2}\)
\State Find earliest time \(t_h\) where \(R \ge h\)
\State \Return \(t_h\)
\EndProcedure
\Procedure{FitExponentialGrowth}{data}
\State Extract early infection phase with small \(R\)
\State Fit curve \(I(t) = I_0 e^{rt}\)
\State \Return growth rate \(r\)
\EndProcedure
\end{algorithmic}
\end{algorithm}

\end{document}