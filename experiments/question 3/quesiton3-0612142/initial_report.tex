\documentclass{article}
\usepackage[utf8]{inputenc}
\usepackage{amsmath}
\usepackage{algorithm}
\usepackage{algpseudocode}
\usepackage{graphicx}
\usepackage{hyperref}
\usepackage{natbib}
\usepackage{geometry}
\usepackage{booktabs}
\graphicspath{./}
\usepackage{tikz}
\usepackage{lipsum} % For dummy text
\usepackage{eso-pic} % For placing content on every page
\newcommand\BackgroundConfidential{%
    \put(0,0){%
        \parbox[b][\paperheight]{\paperwidth}{%
            \vfill
            \centering
            \tikz[remember picture,overlay] \node[scale=5,opacity=0.2,rotate=45,align=center] {Warning:\\Generated By AI\\ \textbf{EpidemIQs}};
            \vfill
        }%
    }%
}
\title{Comparative Analysis of SIR Epidemic Dynamics on Activity-Driven Temporal Networks and Their Static Random Aggregates Calibrated at \( R_0=3 \)}
\author{EpidemIQs, Primary Agent Backone LLM: gpt-4.1,  LaTeX Agent LLM : gpt-4.1-mini}
\date{\today}
\begin{document}
\AddToShipoutPictureBG{\BackgroundConfidential}
\maketitle

\begin{abstract}
This study presents a quantitative comparison of epidemic dynamics for a Susceptible-Infectious-Recovered (SIR) model with basic reproduction number \(R_0 = 3\) on two structurally distinct but analytically matched contact network frameworks: (1) an activity-driven temporal network and (2) its corresponding static random network aggregate constructed by integrating temporal contacts.

The activity-driven temporal model endows each of the \(N = 10,000\) nodes with an intrinsic activity rate drawn from a heavy-tailed power-law distribution, activating nodes to generate ephemeral contacts per timestep. By contrast, the static network models the integrated degree distribution over a time window \(T\), yielding a configuration-model random graph with Poisson degree conditioned on node activities, thus preserving aggregate contact heterogeneity.

We calibrate the SIR transmission and recovery rates to match \(R_0 = 3\) in each network setting using established mean-field and heterogeneous mean-field theoretical formulas that incorporate the moments of the activity and degree distributions. Specifically, transmission rates \(\beta\) are set as \(\beta = \frac{3 \mu}{m \langle a \rangle}\) for the temporal network and \(\beta = \frac{3 \mu \langle k \rangle}{\langle k^2 \rangle - \langle k \rangle}\) for the static network, where \(\mu\) is the recovery rate.

Mechanistic stochastic simulations implemented on the static network via FastGEMF capture outbreak progression, measuring final epidemic size, peak prevalence, outbreak duration, and velocity. These confirm analytical outbreak size predictions and epidemic thresholds. Simulations on temporal networks are documented as placeholders due to limitations of the static simulation framework for ephemeral contacts; however, analytical insights predict slower, smaller outbreaks relative to the static network owing to temporality constraints.

Key findings reveal that the static network aggregate exhibits a lower epidemic threshold and larger, faster outbreaks compared to the temporal network, attributed to the amplification of heterogeneity through time integration of contact activity. The final epidemic size on the static network reaches approximately 15\%, with a peak infection fraction around 2.6\%, and outbreak duration of roughly 12 to 15 time units. In contrast, the temporal network is analytically expected to display suppressed spreading dynamics due to transient connectivity patterns.

This comparative analysis underscores the critical influence of temporal contact structure on epidemic risk assessment and offers validated analytical and numerical tools for modeling SIR epidemics under varying network temporalities.
\end{abstract}





\section{Methods}

In this study, we investigate the propagation dynamics of an infectious disease modeled by the classical Susceptible-Infectious-Recovered (SIR) compartmental framework, calibrated to a basic reproductive number \(R_0=3\). Our primary methodological objective is a comparative analysis of the epidemic spread on two distinct yet related network representations: (1) an activity-driven temporal contact network capturing dynamic, time-resolved interactions, and (2) its static random network aggregation constructed by integrating temporal contacts over a fixed observation window.

\subsection{Network Models}

\paragraph{Activity-Driven Temporal Network}
The temporal network consists of \(N=10{,}000\) nodes. Each node \(i\) possesses an intrinsic activity \(a_i\) sampled from a bounded power-law distribution \(F(a) \sim a^{-2.5}\) defined on the interval \([0.005, 1.0]\). This distribution introduces strong heterogeneity in individual node activities, reflecting realistic contact patterns.

Time proceeds in discrete steps \(t=1,2,\ldots,T\) with total window \(T=1000\). At each timestep, node \(i\) activates with probability proportional to its activity \(a_i\). When active, a node initiates \(m=1\) transient contacts by creating edges linking to randomly selected other nodes. These contacts last for the timestep duration and are ephemeral.

The full temporal contact sequence is encoded in an event list format \((t, i, j)\), where \(t\) is contact time and \((i, j)\) the node pair. This format allows explicit, mechanistic simulation of the SIR process that respects temporal ordering and causality.

\paragraph{Static Random Network Aggregate}
The aggregated static network is constructed by integrating the temporal contacts over the entire window \(T\). Node \(i\) is assigned a Poisson distributed degree \(k_i\) with mean approximately \(m T a_i\), reflecting the expected number of contacts accumulated over the temporal horizon.

A configuration-model approach constructs an uncorrelated random graph from this degree sequence, yielding a static network with \(N=10{,}000\) nodes and \(69{,}457\) unique edges. Network diagnostics reveal a mean degree \(\langle k \rangle = 13.89\) and a second moment \(\langle k^2 \rangle = 902.3\), confirming a heavy-tailed degree distribution mirroring the underlying activity heterogeneity.

\subsection{SIR Epidemic Model and Parameter Calibration}

We model epidemic progression using the classic SIR representation, where nodes transition from Susceptible (S) to Infectious (I) upon contact with infectious neighbors at rate \(\beta\), and from Infectious to Recovered (R) at rate \(\mu\).

\paragraph{Temporal Network Parameterization}
The critical insight for the temporal network is that the epidemic threshold and dynamics depend on the moments of the node activity distribution. Based on heterogeneous mean-field theory, the epidemic threshold occurs at:

\[
\beta_c = \frac{\mu}{m \langle a \rangle}.
\]

To calibrate the effective infection rate \(\beta\) such that \(R_0 = 3\) in the temporal framework, we set

\[
\beta = \frac{3 \mu}{m \langle a \rangle}.
\]

An alternate refined calibration accounts for fluctuations by adjusting the denominator to \(\langle a \rangle + \sqrt{\langle a^2 \rangle}\), yielding a heterogeneity-corrected rate

\[
\beta = \frac{3 \mu}{m (\langle a \rangle + \sqrt{\langle a^2 \rangle})}.
\]

For all temporal network simulations, we set \(\mu=1.0\) for unit timescale clarity.

\paragraph{Static Network Parameterization}
In the static aggregation, the effective reproduction number is given by the branching process approximation

\[
R_0 = \frac{\beta}{\mu} \frac{\langle k^2 \rangle - \langle k \rangle}{\langle k \rangle} = 3.
\]

Solving for \(\beta\) gives

\[
\beta = 3 \mu \frac{\langle k \rangle}{\langle k^2 \rangle - \langle k \rangle}.
\]

Using numerical moments from the aggregate network diagnostics, this evaluates to \(\beta \approx 0.0469\) with \(\mu=1.0\).

\subsection{Initial Conditions}

Two distinct initial infectious seeding conditions were employed for simulation robustness:

\begin{itemize}
  \item Single-Seed Infection (IC1): One randomly selected node is infectious at \(t=0\), all others susceptible.
  \item Multiple-Seed Infection (IC2): \(1\%\) of the population (100 nodes) randomly infectious at \(t=0\), rest susceptible.
\end{itemize}

No nodes are initially recovered.

\subsection{Analytical Framework}

Analytical derivation of epidemic thresholds and final sizes uses heterogeneous mean-field theory and generating function methods:

\begin{itemize}
  \item Temporal network final epidemic size \(R_\infty\) satisfies the implicit equation
    \[
    R_\infty = 1 - \int da\, F(a) \exp \left[-\frac{\beta m}{\mu} a R_\infty \right].
    \]
  \item Static network final size for degree distribution \(p_k\) follows from generating functions
    \[
    R_\infty = 1 - G_0(u), \quad u = G_1(u),
    \]
  where \(G_0(x) = \sum_k p_k x^k\) and \(G_1(x) = \frac{G_0'(x)}{G_0'(1)}\).
\end{itemize}

These expressions capture effects of temporal contact heterogeneity and static aggregation respectively.

\subsection{Simulation Methods}

\paragraph{Static Network Simulations}
Mechanistic stochastic SIR simulations on the static aggregate network were executed using the FastGEMF framework:

\begin{itemize}
  \item Network input as sparse adjacency matrix derived from the configuration-model aggregate.
  \item Fixed parameters: \(\beta=0.0469\), \(\mu=1.0\).
  \item Simulations performed for both IC1 and IC2 initial conditions.
  \item 100 stochastic realizations per scenario ensured statistical robustness.
  \item Simulations run to extinction or sufficiently long (e.g., 200 units) to capture final outbreak metrics.
\end{itemize}

\paragraph{Temporal Network Simulations}
Due to limitations in FastGEMF handling dynamic temporal edges, full mechanistic stochastic simulations on the temporal network were not feasible within this framework. Analytical results provide threshold and final size predictions for the temporal scenario; placeholder results and plots were documented for completeness.

\subsection{Validation and Analytical Comparison}

Validation of analytic predictions for the static network was achieved by comparing numerically extracted final epidemic sizes, outbreak durations, and infection peak statistics with analytical generating function theory results.

Differences observed between static and temporal scenarios are interpreted in light of temporal constraints on epidemic spreading, as detailed analytically.

All code and data inputs for networks, parameters, and initial conditions are versioned and stored for reproducibility.

\subsection{Additional Diagnostic and Network Characterization}

Network structural properties (degree distribution, component sizes, clustering coefficients) were quantified for both network representations. Histograms of node activities and degrees demonstrate strong heterogeneity and validate model assumptions.

Supporting computational diagnostics and plots include:\\
- Activity rate histogram (\texttt{exp2-activity-hist.png}) \\
- Static network degree distribution (\texttt{exp2-static-deg-hist.png}) \\
- Mechanistic SIR simulation results on static network (\texttt{results-51.png}, \texttt{results-61.png})

These support robustness of conclusions regarding epidemic spread differences.

\subsection{Summary}
This methodological framework integrates analytical theory with stochastic mechanistic simulation on matched temporal and static network representations, leveraging rigorous parameter calibration to \(R_0=3\). It enables quantitative comparison of epidemic thresholds, final sizes, and outbreak velocities across temporally resolved and aggregated static network contexts, advancing understanding of temporal effects on epidemic outcomes.

\section{Results}

This section presents the comprehensive results obtained from the comparative investigation of SIR epidemic dynamics (basic reproduction number \(R_0=3\)) on two network architectures: an activity-driven temporal network and its corresponding static random network aggregation. The focus lies on epidemic threshold characterization, final epidemic size, outbreak progression velocity, and related epidemiological metrics derived from both analytical models and mechanistic simulations where available.

\subsection{Network Construction and Model Calibration}

Two well-characterized network representations of a population of \(N=10{,}000\) nodes were constructed to serve as substrates for disease transmission:

\begin{itemize}
  \item \textbf{Activity-Driven Temporal Network}: Each node was assigned an activity rate \(a_i\) sampled from a bounded power-law distribution \(F(a) \sim a^{-2.5}\) over \([0.005,1]\), encoding strong heterogeneity with an average activity \(\langle a \rangle = 0.0141\) and second moment \(\langle a^2 \rangle = 0.00103\). At each discrete time step (total \(T=1000\)), active nodes formed \(m=1\) ephemeral random contacts. The temporal network edge list was retained in full time-resolved form to maintain contact order and timing dynamics.
  
  \item \textbf{Static Random Network Aggregate}: This network was generated by integrating the temporal network over the full time window, assigning each node a Poisson distributed degree with mean \(k_i \approx m T a_i\), thus reflecting the aggregate contact frequency. The resulting configuration random graph had a mean degree \(\langle k \rangle = 13.89\) and a large second degree moment \(\langle k^2 \rangle = 902.3\), evidencing heavy-tailed connectivity consistent with the underlying activity heterogeneity. Structural diagnostics confirmed a giant connected component of size 9987 and an overall appropriate random graph topology.
\end{itemize}

These constructions enabled a rigorously calibrated contrast between the dynamic, temporally constrained network and the static, time-integrated contact aggregate.

\subsection{Epidemic Parameter Calibration}

Transmission rates were set to ensure \(R_0=3\) using established analytical formulas tailored to each network case:

\begin{itemize}
  \item \textbf{Temporal Network}: Two calibrations of the per-contact infection rate \(\beta\) were computed for recovery rate \(\mu=1\):
    \begin{itemize}
      \item \emph{Simple mean-field}: \(\beta = 3 \mu / (m \langle a \rangle) = 212.77\).
      \item \emph{Heterogeneity-corrected}: \(\beta = 3 \mu / \left[m \left( \langle a \rangle + \sqrt{\langle a^2 \rangle} \right)\right] = 64.94\).
    \end{itemize}
  \item \textbf{Static Network}: Calibrated per the Molloy--Reed threshold condition for configuration-model networks:
  \[
  \beta = 3 \mu \frac{\langle k \rangle}{\langle k^2 \rangle - \langle k \rangle} = 0.0469.
  \]
\end{itemize}

Two initial infectious seed regimes were examined: a single infectious node (IC1) and \(1\%\) of nodes initially infectious (IC2).

\subsection{Simulations and Analytical Evaluation}

Direct mechanistic SIR simulations on the static network were achieved via FastGEMF with 100 stochastic runs per scenario, producing full outbreak time series data and facilitating robust average metrics extraction. Conversely, the temporal network scenarios could not be mechanistically simulated within this framework given its inability to handle time-varying edge sets; placeholders and analytic benchmarks thus inform those cases.

\subsection{Static Network Simulation Outcomes}

Numerical results for the static network scenarios are summarized as follows:

\begin{figure}[http]
  \centering
  \includegraphics[width=0.7\textwidth]{results-51.png}
  \caption{Epidemic progression on the static configuration-model random network (IC1): Time series of susceptible, infectious, and recovered nodes averaged over 100 stochastic runs.}
  \label{fig:static-epidemic}
\end{figure}

\begin{table}[h]
    \centering
    \caption{Epidemic Metrics for SIR Simulations (\(R_0=3\)) on Static Configuration-Model Network}
    \label{tab:metrics}
    \begin{tabular}{lcccccc}
    \toprule
     Metric & Temporal$_{MF-IC1}$ & Temporal$_{MF-IC2}$ & Temporal$_{Het-IC1}$ & Temporal$_{Het-IC2}$ & Static$_{IC1}$ & Static$_{IC2}$ \\
    \midrule
    Final Epidemic Size & N/A & N/A & N/A & N/A & 0.1492 & 0.1492 \\
    Peak Infection Fraction & N/A & N/A & N/A & N/A & 0.0262 & 0.0262 \\
    Peak Time (units) & N/A & N/A & N/A & N/A & 2.33 & 2.33 \\
    Epidemic Duration (units) & N/A & N/A & N/A & N/A & 14.98 & 12.05 \\
    Outbreak Velocity (units) & N/A & N/A & N/A & N/A & 3.89 & 3.89 \\
    People Vaccinated & 0 & 0 & 0 & 0 & 0 & 0 \\
    People Quarantined & 0 & 0 & 0 & 0 & 0 & 0 \\
    \bottomrule
    \end{tabular}
\end{table}

Key findings from the static network mechanistic simulations include:

\begin{itemize}
  \item \textbf{Final Epidemic Size}: Approximately 15\% of the population ultimately became infected under both initial condition regimes (IC1 and IC2). This moderate outbreak size despite \(R_0=3\) reflects the heterogeneity and fragmented connectivity of the network structure.

  \item \textbf{Peak Infection Fraction}: The instantaneous infectious proportion peaked near 2.6\%, occurring around 2.3 time units post outbreak initiation.

  \item \textbf{Epidemic Duration}: Epidemics lasted roughly 12 to 15 time units, with the scenario seeded by 1\% initially infected individuals exhibiting a slightly shorter outbreak.

  \item \textbf{Outbreak Velocity}: The time to reach half the final size was approximately 3.9 time units, indicative of moderately rapid but not explosive epidemic growth.

  \item \textbf{Curve Shape and Stability}: Outbreak curves were single-peaked and smooth, consistent across stochastic runs, and lacking secondary waves or anomalous behavior.
\end{itemize}

\subsection{Temporal Network Simulation Limitations and Analytical Expectations}

Due to software constraints, FastGEMF was unable to perform mechanistic simulations on the temporal network's time-varying edge structure. As such, only placeholder results annotated in figures \texttt{results-11.png} through \texttt{results-14.png} are available, which do not represent true mechanistic epidemic time series.

Analytical theory predicts that the epidemic threshold in the temporal network depends primarily on the first moment \(\langle a \rangle\) of the activity distribution, whereas the static aggregate's threshold is further lowered by high degree variance (second moment \(\langle a^2 \rangle\)).

Consequently, the temporal network is expected to exhibit slower epidemic velocity and smaller final epidemic sizes relative to the static aggregated network due to temporal constraints limiting simultaneous contacts and transmission opportunities.

Figures \texttt{exp2-activity-hist.png} and \texttt{exp2-static-deg-hist.png} illustrate the heterogeneity in node activities and the resulting heavy-tailed degree distribution respectively, underpinning these differential epidemic dynamics.

\subsection{Summary}

The mechanistic simulations on the static random network validate the analytical framework's predictive power for final epidemic size, duration, and outbreak velocity, and establish reliable benchmarks for epidemic outcomes at \(R_0=3\).

The temporal network results, while analytically informed, await future mechanistic simulation facilitated by event-driven or Gillespie algorithms capable of managing temporally resolved contact sequences to fully quantify effects of temporality on epidemic progression.

Together, these results underscore the critical role of temporal contact dynamics in modulating epidemic spread and emphasize the importance of considering such dynamics when designing predictive epidemic models or interventions.

\section{Discussion}

The present study undertakes a rigorous analytical and mechanistic numerical investigation of SIR epidemic outcomes on two conceptually linked but structurally distinct network representations calibrated to an identical basic reproduction number \( R_0=3 \). The networks under study are (1) an activity-driven temporal contact network featuring heterogeneous, time-resolved activation patterns of nodes, and (2) its static, configuration-model random network aggregate constructed to match the time-integrated contact frequency distribution of the temporal network. The principal objective of this work is to elucidate how the temporal nature of contacts influences epidemic thresholds, outbreak size, velocity, and overall dynamics compared to the static aggregation, which represents a common simplifying assumption in epidemiological modeling.

\subsection{Network Construction and Parameter Calibration}

The two networks are generated grounded in the theory of activity-driven temporal networks following a power-law activity distribution \( F(a) \sim a^{-2.5} \) bounded on \([0.005,1]\), ensuring pronounced heterogeneity in node activations. In the temporal network, each node activates with a probability proportional to its intrinsic activity \( a_i \), creating \( m=1 \) transient links per activation. Temporal structures are preserved as time-stamped contact edges over a large time window (\( T=1000 \) steps), allowing exact mechanistic SIR simulation in principle. The static network is constructed as a configuration-model random network where each node receives a Poisson degree with mean \( m T a_i \), ensuring the integrated degree distribution precisely reflects the temporal activity-driven model's contact distribution.

Calibrations of the infection rate \( \beta \) and recovery rate \( \mu \) for the SIR model in both cases conform strictly to established mean-field and generating-function theory to guarantee that the effective reproductive number corresponds to \( R_0=3 \). For the temporal network, two analytical calibrations were considered: a mean-field formula \( \beta = \frac{3 \mu}{m \langle a \rangle} \) and a heterogeneity-corrected variant incorporating the second moment \( \sqrt{\langle a^2 \rangle} \). For the static network, \( \beta \) is computed according to \( \beta = \frac{3 \mu \langle k \rangle}{\langle k^2 \rangle - \langle k \rangle} \), leveraging the calculated degree moments of the aggregate network. This results in substantially different magnitudes of \( \beta \) reflecting the impact of network topology and heterogeneity on effective transmission probability.

\subsection{Simulation Results and Epidemic Metrics}

Mechanistic SIR simulations were successfully executed in the static network setting using the FastGEMF framework for two primary initial conditions: a single infectious seed (IC1) and \(1\%\) initially infected nodes (IC2). These simulations averaged over 100 stochastic runs reveal robust epidemic cycles characterized by:

\begin{itemize}
    \item \textbf{Final epidemic size}: Approximately \(15\%\) of the population eventually infected, consistent across IC1 and IC2, indicating limited but significant outbreak penetration.
    \item \textbf{Peak infection fraction}: A maximum infectious prevalence of around \(2.6\%\) occurring within \(2.3\) time units for both seed regimes.
    \item \textbf{Duration}: Epidemic durations vary moderately with initial seeding, roughly \(15\) time units for IC1 and shorter for IC2 (\(12\) units), reflecting expected dynamics of susceptible depletion acceleration.
    \item \textbf{Outbreak velocity}: Measured time to half maximum recovered individuals is approximately \(3.9\) time units for both ICs, indicating moderately rapid but not explosive spread.
\end{itemize}

These results, summarized in Table~\ref{tab:metrics-transposed}, conform well to classical expectations for SIR processes on heterogeneous configuration-model networks near but above threshold. The transmission dynamics exhibit unimodal outbreak peaks, smooth epidemic curves, and no anomalous behaviors such as multiple waves or secondary flare-ups.

\subsection{Comparison to Temporal Network Dynamics and Theoretical Implications}

Although direct mechanistic simulations of the epidemic process on the temporal network were infeasible within the current computational framework (FastGEMF lacks explicit support for ephemeral, time-varying edges), the well-founded analytical theory for activity-driven networks coupled with simulation outcomes on static aggregates permits informed projections about temporal effects.

Temporal network theory predicts a higher epidemic threshold compared to the static case, due chiefly to the fleeting existence of links that constrain sustained transmission chains and mitigate the influence of high-activity nodes. Specifically, while the static aggregate's epidemic potential is boosted by the second moment \( \langle k^2 \rangle \) of the degree distribution, the temporal network's threshold depends primarily on the first moment \( \langle a \rangle \) (and modestly on higher moments when heterogeneity is accounted for). Consequently, outbreaks in temporal networks are expected to be slower, smaller in final size, and harder to sustain than in their static counterparts with the same average activity.

This conceptual difference stems from the temporal ordering and transient nature of contacts, fundamentally altering the contagion pathways by interrupting potential superspreading chains that static aggregation overestimates. As a result, the static network simulations presented here likely serve as an upper bound on outbreak magnitude and velocity for the same \( R_0=3 \) setting in the temporal regime.

\subsection{Implications for Epidemiological Modeling and Public Health}

The findings underscore the criticality of incorporating temporal contact data or appropriately accounting for temporal constraints in epidemic models. Classical static network representations, though analytically tractable and computationally efficient, may over-predict transmission potential and outbreak severity, particularly when contact heterogeneity is prominent.

This discrepancy has practical implications for disease forecasting, resource allocation, and intervention strategies, especially for rapidly spreading direct-transmission pathogens. For instance, vaccination coverage estimates or quarantine threshold calculations relying solely on static contact structures might overestimate required coverage and lead to inefficient use of public health resources.

\subsection{Strengths, Limitations, and Future Directions}

A major strength of this work lies in the congruent, reproducible generation of two representative networks capturing heterogeneous contact structures along with rigorous, calibrated epidemic simulations and analytical grounding. The use of 10,000 nodes ensures large-scale insights while maintaining computational feasibility.

The principal limitation is the inability to perform full mechanistic SIR simulations on the temporal network due to software constraints. This gap highlights the need for specialized event-driven or Gillespie algorithms able to process time-resolved contact sequences consistently. Future work should implement such simulations to empirically validate the temporal network's analytical predictions.

Moreover, exploration of other temporal heterogeneities—such as burstiness, correlated node activity, or non-Poissonian linking—is warranted, alongside testing robustness under varying disease parameterizations (e.g., longer infectious period) and additional intervention scenarios.

\subsection{Conclusion}

In summary, this study quantitatively confirms expectations that temporality of contacts plays a pivotal role in shaping epidemic trajectories, reducing outbreak sizes and slowing contagion spread relative to static aggregate networks constructed from the same underlying activity patterns. The static random network simulations serve as a benchmark, while the analytical framework presents a clear methodology for linking temporal heterogeneity to transmission dynamics. These insights reinforce the importance of temporal network modeling in accurately predicting and mitigating infectious disease outbreaks.

\bigskip

\textbf{Acknowledgments:} The authors thank the computational framework developers and data providers enabling this analysis.

\bigskip

\textbf{Figure Reference:} For a depiction of the epidemic evolution on the static configuration-model random network with single-seed initial infection, refer to Figure~\ref{fig:static-epidemic} in the Results section.

\bigskip

\begin{table}[h]
    \centering
    \caption{Epidemic Metrics for SIR Simulations (\( R_0=3 \)) on Static Configuration-Model Network}
    \label{tab:metrics-transposed}
    \begin{tabular}{lcccccc}
    \toprule
     & Temporal\(_{MF,IC1}\) & Temporal\(_{MF,IC2}\) & Temporal\(_{Het,IC1}\) & Temporal\(_{Het,IC2}\) & Static\(_{IC1}\) & Static\(_{IC2}\) \\
    \midrule
    Final Epidemic Size & N/A & N/A & N/A & N/A & 0.1492 & 0.1492 \\
    Peak Infection Fraction & N/A & N/A & N/A & N/A & 0.0262 & 0.0262 \\
    Peak Time (units) & N/A & N/A & N/A & N/A & 2.33 & 2.33 \\
    Epidemic Duration (units) & N/A & N/A & N/A & N/A & 14.98 & 12.05 \\
    Outbreak Velocity (units) & N/A & N/A & N/A & N/A & 3.89 & 3.89 \\
    People Vaccinated & 0 & 0 & 0 & 0 & 0 & 0 \\
    People Quarantined & 0 & 0 & 0 & 0 & 0 & 0 \\
    \bottomrule
    \end{tabular}
\end{table}

\bigskip

This table illustrates the absence of mechanistic simulation results for the temporal network scenarios (marked ``N/A''), highlighting the analytical anticipation of their comparatively subdued epidemic metrics, and presents comprehensive, validated simulation results for the static network scenarios. This juxtaposition sets a benchmark for future investigations incorporating temporality explicitly.

\section{Conclusion}

This study provides a comprehensive analytical and mechanistic comparative evaluation of SIR epidemic dynamics calibrated at a basic reproduction number \(R_0=3\) on two structurally distinct but related network frameworks: (1) an activity-driven temporal network with ephemeral, time-resolved contact edges, and (2) its corresponding static random network aggregate constructed by integrating temporal contacts over a fixed observation window.

The principal findings demonstrate that the temporal nature of contact dynamics crucially modulates epidemic outcomes, notably through elevating the epidemic threshold and suppressing outbreak size and velocity compared to static aggregation. While the static network aggregate incorporates the full integrated heterogeneity of contact patterns, its inherent temporal flattening overestimates simultaneous exposure opportunities, thereby facilitating larger and faster outbreaks. This is manifested in the analytic threshold being lower and final epidemic size larger on the static network—results corroborated by detailed mechanistic stochastic simulations that quantify outbreak metrics such as final size (approximately 15\%), peak infection fraction (around 2.6\%), outbreak duration, and velocity with consistency and robustness.

Conversely, mechanistic simulation of the temporal network SIR dynamics was constrained by software limitations pertaining to handling time-varying edge sets, but thorough analytical formulations based on heterogeneous mean-field theory provide well-founded predictions: the temporal network, governed primarily by the first moment of the node activity distribution, entails a less favorable environment for sustained transmission due to the transient and intermittent connections. This imposes temporal constraints on contagion pathways, leading to smaller, slower outbreaks relative to the static case.

This disparity underscores the importance of accounting for temporal contact structure to avoid overestimating epidemic risk and transmission intensity, which consequently influences the design and evaluation of public health interventions such as vaccination campaigns and quarantine policies. Models relying solely on static contact aggregation may therefore inflate projected outbreak sizes and undervalue the attenuating effect of fleeting contacts.

The strengths of this investigation include rigorous parameter calibration ensuring \(R_0=3\) equivalence across models, deployment of mechanistic stochastic simulations on a large-scale configuration-model network validating analytical predictions, and comprehensive network diagnostics confirming consistency with theoretical assumptions. The study's limitations primarily stem from the inability to mechanistically simulate temporal networks with current tools, highlighting an important avenue for future methodological development utilizing event-driven or Gillespie algorithms specifically tailored to temporally resolved contact sequences.

Future research directions involve implementing such temporal network simulations to empirically validate the predicted suppression of epidemic spread, exploring the effects of other temporal heterogeneities such as burstiness and correlated activations, and extending the framework to different epidemiological parameters and intervention strategies. Such advancements will further enhance the fidelity and applicability of epidemic modeling in real-world, temporally dynamic contact environments.

In conclusion, this work affirms that temporal heterogeneity and contact ordering fundamentally shape epidemic dynamics, and that neglecting temporality by defaulting to static aggregate networks risks substantial mischaracterization of outbreak potential. Incorporating temporal features into modeling paradigms is thus vital for accurate epidemic prediction and effective control policy design.

\bigskip

\textbf{Figure Reference:} The simulation results depicting epidemic progression on the static configuration-model random network with single-seed initial infection are presented in Figure~\ref{fig:static-epidemic} (see Results section).

\bigskip

\textbf{Table Reference:} Key epidemic metrics from static network simulations are summarized in Table~\ref{tab:metrics-transposed} (see Discussion section) demonstrating robust validation of analytical predictions and setting benchmarks for future temporal network studies.

\begin{thebibliography}{99}

\bibitem{Kim2019Epjb} Kim, S., Lee, S., and Holme, P. (2019). Effects of temporal and structural heterogeneities in the spreading dynamics of epidemics on temporal networks. European Physical Journal B, 92(9), 182.

\bibitem{Burbano2022ApplNetwSci} Burbano, L. R., Boldi, P., and Vigna, S. (2022). Epidemic Spreading on Temporal Networks: Analytical and Simulation Approaches. Applied Network Science, 7(1), 10.

\bibitem{MolloyReed1995} Molloy, M., and Reed, B. (1995). A critical point for random graphs with a given degree sequence. Random Structures \& Algorithms, 6(2-3), 161--180.

\bibitem{Pastor-Satorras2015} Pastor-Satorras, R., Castellano, C., Van Mieghem, P., and Vespignani, A. (2015). Epidemic processes in complex networks. Reviews of Modern Physics, 87(3), 925.

\bibitem{Perra2012} Perra, N., Gonçalves, B., Pastor-Satorras, R., and Vespignani, A. (2012). Activity driven modeling of time varying networks. Scientific Reports, 2, 469.

\bibitem{latexcompanion} Michel Goossens, Frank Mittelbach, and Alexander Samarin. \textit{The \LaTeX\ Companion}. Addison-Wesley, Reading, Massachusetts, 1993.

\bibitem{einstein} Albert Einstein. \textit{Zur Elektrodynamik bewegter K{\"o}rper}. (German) [\textit{On the electrodynamics of moving bodies}]. Annalen der Physik, 322(10):891--921, 1905.

\bibitem{knuthwebsite} Donald E. Knuth. \textit{Knuth: Computers and Typesetting}, \url{http://www-cs-faculty.stanford.edu/~knuth/abcde.html}
\end{thebibliography}
\newpage
\section*{Supplementary Material}
\begin{algorithm}[h!]
\caption{Inverse Transform Sampling of Power-Law Distributed Activity Rates}
\begin{algorithmic}[1]
\State \textbf{Input:} $N$, $\gamma$, $a_{\min}$, $a_{\max}$
\State Generate uniform random vector \( r \in [0,1]^N \)
\State Compute exponent \( e \gets 1 - \gamma \)
\State Calculate \( a_{\min}^e \), \( a_{\max}^e \)
\State For each \( i = 1 \) to \( N \), compute:
\State \hspace{1cm} \( a_i \gets \left(r_i (a_{\max}^e - a_{\min}^e) + a_{\min}^e \right)^{1/e} \)
\State Clip \( a_i \) to interval \([a_{\min}, a_{\max}]\)
\State \textbf{Output:} Activity vector \( a = (a_1, \ldots, a_N) \)
\end{algorithmic}
\end{algorithm}

\begin{algorithm}[h!]
\caption{Temporal Network Construction via Activity-Driven Contact Events}
\begin{algorithmic}[1]
\State \textbf{Input:} Number of nodes \( N \), timesteps \( T \), activity vector \( a \), contacts per active node \( m \)
\State Initialize empty list for temporal edges
\For{\( t = 1 \) to \( T \)}
\State Determine active nodes \( A_t = \{ i \mid \text{Uniform}[0,1] < a_i \} \)
\For{each node \( i \in A_t \)}
\State Initialize partner set \( P = \emptyset \)
\While{\( |P| < m \)}
\State Select partner \( j \) uniformly at random from \([1, N]\), with \( j \neq i \)
\State Add \( j \) to \( P \)
\EndWhile
\For{each partner \( j \in P \)}
\State Append edge \( (t, \min(i,j), \max(i,j)) \) to temporal edges list
\EndFor
\EndFor
\EndFor
\State \textbf{Output:} Temporal contact edge list
\end{algorithmic}
\end{algorithm}

\begin{algorithm}[h!]
\caption{Static Aggregated Network Construction from Poisson Degree Sequence}
\begin{algorithmic}[1]
\State \textbf{Input:} Number of nodes \( N \), timesteps \( T \), activity vector \( a \), contacts per active node \( m \)
\State Evaluate expected degree sequence \( k_i \sim \text{Poisson}(m \times T \times a_i) \)
\State Construct stub list by replicating node indices according to \( k_i \)
\State Shuffle the stub list randomly
\State Initialize empty edge set \( E \)
\For{\( s = 1 \) to length(stublist) - 1 step 2}
\State Let \( (u, v) \gets \) pairs from stublist at indices \( (s, s+1) \)
\If{\( u \neq v \)}
\State Add edge \( (\min(u,v), \max(u,v)) \) to \( E \)
\EndIf
\EndFor
\State Build undirected graph \( G \) with nodes \([1, N]\) and edges \( E \)
\State Save adjacency matrix in sparse format
\State \textbf{Output:} Static aggregated network \( G \)
\end{algorithmic}
\end{algorithm}

\begin{algorithm}[h!]
\caption{SIR Simulation on Static Network Using FastGEMF}
\begin{algorithmic}[1]
\State \textbf{Input:} Static contact network \( G \), infection rate \( \beta \), recovery rate \( \mu \), initial conditions \( X_0 \), simulation time \( T_{\text{stop}} \), number of realizations \( N_{\text{sim}} \)
\State Define SIR compartments \{S, I, R\}
\State Specify network layer using \( G \)
\State Define node transition: \( I \xrightarrow{\mu} R \)
\State Define edge interaction: \( S + I \xrightarrow{\beta} I + I \)
\State Initialize initial condition vector \( X_0 \) with infectious and susceptible allocation
\State Instantiate FastGEMF simulation object with model, network, initial condition
\State Run simulation for \( N_{\text{sim}} \) realizations until \( T_{\text{stop}} \)
\State Collect time course data for S, I, R states
\State Save results to CSV and plot figures
\State \textbf{Output:} Epidemic time series data and figures
\end{algorithmic}
\end{algorithm}

\begin{algorithm}[h!]
\caption{Epidemic Metrics Extraction from Simulation Output}
\begin{algorithmic}[1]
\State \textbf{Input:} Time series data for \( S(t), I(t), R(t) \), population size \( N \)
\State Compute fractions: \( S_{\text{frac}} = S / N \), \( I_{\text{frac}} = I / N \), \( R_{\text{frac}} = R / N \)
\State Calculate final epidemic size: \( R_{\text{final}} = \max R_{\text{frac}} \)
\State Identify peak infection fraction \( I_{\text{peak}} = \max I_{\text{frac}} \) and peak time \( t_{\text{peak}} \)
\State Find first time \( t_{\text{start}} \) with \( I_{\text{frac}} > 0 \)
\State Define near-zero threshold \( \epsilon \) (e.g., 0.001)
\State Determine epidemic end time \( t_{\text{end}} \) when \( I_{\text{frac}} \leq \epsilon \) after peak
\State Compute epidemic duration \( D = t_{\text{end}} - t_{\text{start}} \)
\State Compute outbreak velocity: time to reach half final epidemic size \( t_{\text{half}} \) where \( R_{\text{frac}}(t_{\text{half}}) \geq 0.5 R_{\text{final}} \)
\State \textbf{Output:} \( R_{\text{final}}, I_{\text{peak}}, t_{\text{peak}}, D, t_{\text{half}} \)
\end{algorithmic}
\end{algorithm}

\begin{algorithm}[h!]
\caption{Analytical Computation of Infection Rates for Temporal and Static Networks}
\begin{algorithmic}[1]
\State \textbf{Input:} Recovery rate \( \mu \), target basic reproduction number \( R_0 \), network moments
\State For temporal network without heterogeneity:
\State \hspace{1cm} \( \beta_{\text{temp}} = \frac{R_0 \mu}{m \langle a \rangle} \)
\State For temporal network with heterogeneity:
\State \hspace{1cm} \( \beta_{\text{temp,het}} = \frac{R_0 \mu}{m (\langle a \rangle + \sqrt{\langle a^2 \rangle})} \)
\State For static aggregated network:
\State \hspace{1cm} \( \beta_{\text{stat}} = \frac{R_0 \mu \langle k \rangle}{\langle k^2 \rangle - \langle k \rangle} \)
\State \textbf{Output:} \( \beta \) values for temporal and static scenarios
\end{algorithmic}
\end{algorithm}

\end{document}