\documentclass{article}
\usepackage[utf8]{inputenc}
\usepackage{tabularx}
\usepackage{amsmath}
\usepackage{algorithm}
\usepackage{algpseudocode}
\usepackage{graphicx}
\usepackage{hyperref}
\usepackage{natbib}
\usepackage{geometry}
\usepackage{booktabs}
\graphicspath{./}
\usepackage{tikz}
\usepackage{lipsum} % For dummy text
\usepackage{eso-pic} % For placing content on every page
\newcommand\BackgroundConfidential{%
    \put(0,0){%
        \parbox[b][\paperheight]{\paperwidth}{%
            \vfill
            \centering
            \tikz[remember picture,overlay] \node[scale=5,opacity=0.2,rotate=45,align=center] {Warning:\\Generated By AI\\ \textbf{EpidemIQs}};
            \vfill
        }%
    }%
}
\title{Impact of Temporal Causality on Epidemic Spread: \newline Analytical and Simulation Comparison of SIR Dynamics on Activity-Driven Temporal vs Time-Aggregated Static Networks}
\author{EpidemIQs, Scientist Agent Backone LLM: gpt-4.1,  Expert Agent Backone LLM : gpt-4.1-mini}
\date{May 2025}
\begin{document}
\AddToShipoutPictureBG{\BackgroundConfidential}
\maketitle

\begin{abstract}
We present a comprehensive quantitative analysis of epidemic dynamics using the Susceptible-Infected-Recovered (SIR) model with a basic reproduction number \( R_0 = 3 \) on both an activity-driven temporal network and its time-aggregated static counterpart. The temporal network consists of 1000 nodes, each activating at a rate \( \alpha = 0.1 \) to form \( m = 5 \) transient edges per activation, capturing the nuanced sequential structure of human contact patterns. Using a mechanistic approach, we parameterize transmission rates \( \beta \) and recovery rates \( \gamma \) such that \( \left(\frac{\beta}{\gamma}\right) (m \alpha) = R_0 = 3 \), establishing a supercritical regime for an epidemic outbreak.

Our analysis highlights critical distinctions in epidemic threshold, final epidemic size, and outbreak temporal dynamics between the two network representations. The time-aggregated static network, constructed as an Erd\"os-R\'enyi-like random graph weighted by cumulative contact frequencies, predicts epidemic outcomes consistent with classical mean-field theory, including a final attack rate of approximately 94\%. In contrast, simulations on the activity-driven temporal network demonstrate significant reductions in outbreak size and speed, with an average final epidemic size of about 20\%, due to temporal causality constraints that restrict transmission pathways and reduce the effective connectivity.

Mechanistically, these findings reveal that temporal ordering and the transient nature of contacts inherently limit the potential for disease spread, causing the static aggregated network to overestimate epidemic risk by neglecting the underlying dynamic contact structure. Temporal network simulations exhibit lower and delayed infection peaks, substantially prolonged epidemic durations, and higher stochastic variability in outcomes.

This study rigorously validates these analytical and simulation-based insights through extensive dynamical modeling and stochastic simulation, establishing that neglecting temporal causality in contact networks yields over-optimistic predictions of epidemic severity. Our results underscore the necessity of incorporating realistic temporal network features in infectious disease modeling to accurately assess epidemic thresholds and intervention strategies.
\end{abstract}

\section{Introduction}

The spread of infectious diseases in human populations remains a central topic of epidemiological research, with mathematical modeling serving as a powerful tool to understand dynamics and guide intervention strategies. Among classical compartmental models, the susceptible-infected-recovered (SIR) framework provides a foundational approach to simulate epidemics where individuals transition from susceptible to infected and eventually to recovered states, capturing diseases conferring immunity post-infection. However, accurately representing the contact structure over which infections propagate is critical, as human interactions are inherently dynamic and heterogeneous.

Recent advances have emphasized the importance of temporal networks in modeling epidemic spread, recognizing that the timing and order of contacts significantly influence transmission pathways. Activity-driven temporal networks (ADNs) offer a compelling generative model for such time-varying interactions, where nodes (individuals) activate stochastically and create transient connections to others \cite{Perra2012TimeDriven}. This paradigm contrasts with static or time-aggregated networks that collapse all contacts over a duration, often leading to overestimation of epidemic potential due to neglect of temporal causality.

In the context of SIR dynamics, several studies have examined the epidemic threshold and final outbreak size on ADNs and their modifications incorporating memory, attractiveness, and adaptive behaviors \cite{Mancastroppa2020ActiveInactive, Tizzani2018SpreadingAging, Sun2015MemoryImpact, Kim2019TemporalConnectivity}. These works highlight that temporal constraints and network heterogeneities can raise the epidemic threshold and reduce outbreak sizes relative to static approximations. For instance, memory effects in temporal networks have been shown to inhibit or promote epidemic persistence depending on the disease model (SIR vs SIS) \cite{Sun2014NonMarkovian}, while the presence of strong ties and repeated contacts reshape spreading dynamics \cite{Sun2015MemoryImpact}.

Furthermore, the construction of static networks from temporal data—commonly through cumulative aggregation resulting in networks with weighted edges—built an often-used benchmark though potentially masking critical temporal ordering \cite{Lei2015StaticActivity}. Investigations comparing SIR epidemic outcomes on activity-driven temporal networks against their aggregated static counterparts reveal pronounced differences. Analytical and simulation results demonstrate that static networks overestimate epidemic size and speed because they assume simultaneous contact availability, ignoring the temporal constraints that causally restrict infection chains \cite{Liu2012ContagionTemporal, Nadini2020AdaptiveBackbone}.

Despite this growing body of literature, rigorous quantitative comparisons articulating the mechanistic impacts of temporal causality on epidemic thresholds, outbreak sizes, and dynamics in activity-driven SIR models remain an essential endeavor. This study seeks to fill this gap by employing both analytical derivations and stochastic simulations to compare epidemic propagation in a prototypical SIR scenario with \( R_0=3 \) on an activity-driven temporal network of \( N=1000 \) nodes against a time-aggregated static network formed from the same temporal contact data. In particular, we leverage the mean-field relation 
\[
R_0 = \left(\frac{\beta}{\gamma}\right) \cdot (m \cdot \alpha),
\] 
linking transmission and recovery rates with network activity parameters (activation rate \(\alpha\), contact number per activation \(m\)), to select model parameters, ensuring comparability across temporal and static frameworks.

Our principal research question is: 
\begin{quote}
  \emph{How does the temporal structure inherent in an activity-driven network influence epidemic thresholds, final epidemic sizes, and infection dynamics compared to a time-aggregated static network representation under an SIR epidemic with fixed basic reproduction number?}
\end{quote}

Answering this question involves dissecting the role of temporal causality, exploring how instantaneous, memoryless activations and edge formation interplay with infection and recovery processes to alter epidemic outcomes. This analysis advances understanding of temporal network epidemics and informs the development of accurate, mechanistic models for disease control.

The remainder of this paper details the model construction steps, parameter justifications, and theoretical background underpinning this comparative analysis, grounding the work firmly in the extant literature \cite{Perra2012TimeDriven, Mancastroppa2020ActiveInactive, Tizzani2018SpreadingAging, Sun2015MemoryImpact, Lei2015StaticActivity, Liu2012ContagionTemporal}.

\section{Background}

Epidemic modeling on temporal networks has advanced significantly to capture the complexity of real-world contact dynamics influencing disease propagation. Activity-driven networks (ADNs) are widely used as a generative model for temporal contact patterns, where nodes activate stochastically to form transient edges, thereby inherently preserving temporal causality \cite{Perra2012TimeDriven}. Extensions of ADNs incorporating static backbone structures have been proposed to integrate persistent contacts with time-varying interactions, reflecting more realistic scenarios where some connections endure beyond instantaneous activations \cite{Nadini2020AdaptiveBackbone}. These frameworks have been studied for their impact on epidemic thresholds and outbreak sizes under various disease models including SIR dynamics.

Prior research underscores the limitations of static or time-aggregated network representations derived by collapsing temporal edges into cumulative weighted contacts. Such static networks disregard temporal ordering and causality constraints, leading to systematic overestimation of epidemic size and propagation speed \cite{Liu2012ContagionTemporal}. The inability of aggregated models to account for time-resolved transmission pathways critically affects the accuracy of epidemic risk predictions.

Recent investigations have focused on the interplay between temporal network features such as memory, repeated contacts, and heterogeneous activity on epidemic spread \cite{Mancastroppa2020ActiveInactive, Sun2015MemoryImpact, Tizzani2018SpreadingAging, Kim2019TemporalConnectivity}. These studies highlight that temporal constraints can elevate epidemic thresholds and reduce final outbreak sizes relative to predictions obtained from static approximations. Mechanistic insights from mean-field and simulation approaches demonstrate that neglecting temporal causality yields over-optimistic epidemic forecasts.

However, rigorous quantitative comparisons between SIR epidemic dynamics on pure activity-driven temporal networks and their time-aggregated static counterparts remain limited. In particular, defining parameterizations ensuring equivalent basic reproduction numbers (
\( R_0 \)) to enable fair comparison, and systematically quantifying differences in epidemic thresholds, temporal infection dynamics, and final sizes has been an underexplored area.

The present work contributes by providing a detailed analytical and simulation-based comparison of SIR epidemics on both activity-driven temporal and aggregated static networks calibrated to the same \( R_0 \). By elucidating the mechanistic effects of temporal causality on transmission, it offers practical insights on the impact of temporal structure on epidemic potential and challenges assumptions implicit in static network modeling. This study complements and extends prior efforts by emphasizing the dynamical consequences of temporal ordering in a homogeneous activity-driven framework devoid of persistent links, thereby clarifying the degree to which temporality alone modulates epidemic outcomes and informing more accurate infectious disease modeling.

\section{Methods}

This study investigates the spread of an infectious disease using the Susceptible-Infected-Recovered (SIR) model on both an activity-driven temporal network and its time-aggregated static network counterpart. The aim is to compare epidemic thresholds, final epidemic sizes, and infection dynamics to understand the mechanistic effect of temporal causality on disease propagation.

\subsection{Network Models}

Two network representations are constructed for the study population of size \(N=1000\):

\begin{itemize}
    \item \textbf{Activity-Driven Temporal Network}: This mechanistic, memoryless temporal network models each individual (node) as homogeneously 
    activating at rate \(\alpha=0.1\) per time step. Upon activation, a node forms \(m=5\) transient, instantaneous, undirected edges with randomly selected nodes (no self-loops or multiple edges per activation). Links exist only during the activation time-step and then dissolve, resulting in a fully time-varying contact structure preserving the temporal ordering essential for causality in transmission pathways. The network resets each time step, and all temporal edges with timestamps are recorded in an event list.
    
    \item \textbf{Time-Aggregated Static Network}: Derived by aggregating the temporal contacts over a fixed period \(T=1000\) time-steps. Edges in this network are weighted by the frequency of interactions between node pairs during aggregation. The resulting static network exhibits an Erd\H{o}s-R\'{e}nyi (ER)-like degree distribution, with mean degree \(\langle k \rangle = 630.93\) and second moment \(\langle k^2 \rangle = 398,538.23\). The network is fully connected, representing the cumulative contact opportunities but neglecting temporal ordering of contacts.
\end{itemize}

Diagnostic plots for node activities (\texttt{node-activity-temporal.png}), degree distribution (\texttt{degreedist-agg-static.png}), and edge weights (\texttt{edgeweight-agg-static.png}) confirm the statistical properties and theoretical expectations for each network.

\subsection{SIR Epidemic Model and Parameterization}

The SIR compartmental model partitions the population into susceptible (\(S\)), infected (\(I\)), and recovered (\(R\)) individuals. Transitions are governed by:

\begin{itemize}
  \item Infection: \(S + I \xrightarrow{\beta} I + I\), where infection occurs upon contact between susceptible and infected individuals at rate \(\beta\) per contact.
  \item Recovery: \(I \xrightarrow{\gamma} R\), recovery occurs at rate \(\gamma\).
\end{itemize}

The model was parametrized to satisfy the basic reproduction number condition for the temporal activity-driven network:
\begin{equation}
R_0 \approx \frac{\beta}{\gamma} \times (m \times \alpha) = 3.
\label{eq:R0}
\end{equation}

Choosing \(m=5\) and \(\alpha=0.1\) to fix contact dynamics, setting \(\gamma = 1\) as the unit recovery rate, and solving Eq.~\eqref{eq:R0} yields \(\beta=6.0\) for the temporal network. For the static aggregated network, the effective mean degree replaces \(m \alpha\), giving a parametrization \(\beta \approx 0.00475\) with \(\gamma=1\) consistent with the static network’s connectivity.

The initial condition sets \(I(0) = 1\) randomly chosen infected individual, with \(S(0) = 999\) and \(R(0) = 0\). This is consistent across both network models.

\subsection{Analytical Framework}

Using the mean-field approximation applicable to homogeneous mixing scenarios, the epidemic threshold for the temporal network is given by \(R_0 = 1\), translating to the critical transmission rate:

\begin{equation}
\beta_c = \frac{\gamma}{m \alpha}.
\end{equation}

Since \(R_0 = 3 > 1\), the system is in the supercritical regime. The final size \(r\) of the epidemic is estimated by solving the classical final size relation:

\begin{equation}
r = 1 - \exp(-R_0 r),
\end{equation}

which yields \(r \approx 0.94\) for \(R_0=3\) under homogeneous assumptions.

For the static network, classical threshold results involve the largest eigenvalue of the weighted adjacency matrix, but given the ER-like structure, the mean degree parameterization suffices for \(\beta_c\) estimation.

\subsection{Simulation Procedures}

Three simulation scenarios were implemented to validate and compare the theoretical predictions:

\textbf{Scenario 1: Activity-Driven Temporal Network SIR Simulation}

\begin{itemize}
  \item Implement a continuous-time Markov chain (CTMC) model on the temporal network using the edge event list.
  \item At each time step, identify edges formed by activated nodes and allow infection transmission across these transient edges with probability derived from \(\beta\). Recovery events happen at rate \(\gamma\).
  \item Synchronous updating is ensured: infections computed simultaneously from all infectious-susceptible pairs per time step before state updates avoid mid-step contamination.
  \item Initial infected node is randomly chosen; simulations run for 100 stochastic realizations to capture variability.
  \item Outputs include time series of \(S\), \(I\), \(R\), final epidemic size distributions, and stochastic variability.
\end{itemize}

\textbf{Scenario 2: Time-Aggregated Static Network SIR Simulation}

\begin{itemize}
    \item Run SIR model on static weighted ER-like network loaded as a sparse adjacency matrix.
    \item Use FastGEMF for efficient network-based CTMC SIR simulation, with the same initial conditions and parameters (\(\beta=0.00475\), \(\gamma=1\)).
    \item Conduct 100 stochastic simulations for comprehensive statistical reporting.
    \item Output time series and final size distributions for comparative analysis.
\end{itemize}

\textbf{Scenario 3: Analytical Final Size Computation}

\begin{itemize}
    \item Numerically solve \(r = 1 - \exp(-R_0 r)\) for \(R_0 = 3\) to determine theoretical epidemic final size.
    \item Use this as a benchmark to compare against simulation results.
\end{itemize}

\subsection{Validation and Diagnostics}

Diagnostic plots (e.g., SIR time series, node activity histograms, degree and edge weight distribution histograms) were used to verify network construction and simulation correctness.

The simulation code was rigorously debugged to ensure accurate synchronous epidemic updates and elimination of pre-mature infections within the same step, ensuring mechanistic fidelity to SIR compartment model assumptions.

\subsection{Computational Tools and Data Availability}

All network construction and analyses were implemented in Python using custom scripts. The temporal network contact events were stored in CSV format (\texttt{temporal-edge-events-csv}), and the aggregated static network was saved as a sparse adjacency matrix (\texttt{agg-static-network-npz}). All scripts, outputs, and data files including diagnostic plots (\texttt{node-activity-temporal.png}, \texttt{degreedist-agg-static.png}, \texttt{edgeweight-agg-static.png}) are provided for reproducibility.

\subsection{Rationale and Scientific Justification}

The modeling strategy and parameter choices were guided by analytical mean-field derivations and best practices in epidemic network modeling. Activity-driven temporal networks capture nodal activation dynamics explicitly, preserving temporal causality of contacts, an essential element affecting epidemic thresholds and outbreak sizes. The time-aggregated static network stands as a classical, order-ignorant approximation. Comparing these two allows elucidation of the impact of temporal ordering and network memory on epidemic dynamics. The parameterization ensures a supercritical epidemic regime (\(R_0 = 3\)), allowing clear observation of differences in disease spread caused by temporal constraints.

This approach aligns with literature evidence showing that memory and temporal effects influence SIR epidemic thresholds and final sizes by restricting effective transmission pathways compared to static representations.

\section{Results}

This section presents the results of the comparative analysis of epidemic spread using the SIR model on (i) an activity-driven temporal network and (ii) a time-aggregated static network. Additionally, the analytic mean-field solution serves as a reference. The focus is on infection dynamics, final epidemic size, and mechanistic insights into temporal causality effects.

\subsection{Networks and Parameterization}
Two network representations of a population with 1000 nodes were used: 
\begin{itemize}
  \item An \textit{activity-driven temporal network} where each node activates at a rate $\alpha = 0.1$ and upon activation creates $m=5$ instantaneous, random contacts. Edges last one time step only, preserving temporal causality and dynamic connectivity.
  \item A \textit{time-aggregated static weighted network}, created by consolidating temporal contacts over $T=1000$ steps into edge weights reflecting cumulative contact frequencies. This aggregated network is Erd\H{o}s-R\'enyi-like with mean degree $\langle k \rangle = 630.93$ and a Poisson-like degree distribution.
\end{itemize}
The SIR epidemiological parameters were chosen to achieve a basic reproduction number $R_0 = 3$, leading to per-contact infection rates $\beta = 6.0$ for the temporal and $\beta = 0.00475$ for the static network assuming recovery rate $\gamma = 1.0$. Initial conditions start with a single infected node chosen at random, and the rest susceptible.

\subsection{Temporal SIR Simulation Results}
Figure~\ref{fig:temporal-sir} presents mean dynamics of Susceptible (S), Infected (I), and Recovered (R) compartments over 100 stochastic realizations of the activity-driven temporal network simulation.

\begin{figure}[http]
  \centering
  \includegraphics[width=0.7\textwidth]{results-11.png}
  \caption{Epidemic trajectories in the activity-driven temporal network: average dynamics of S, I, and R over 100 runs with $\beta=6.0$, $\gamma=1.0$. The slow rise and modest peak of infection indicate constrained spread due to temporal causality.}
  \label{fig:temporal-sir}
\end{figure}

Key metrics from simulation are summarized in Table~\ref{tab:metrics-comparison}.

The final epidemic size averaged approximately 20.5\%, with large variability (std $\approx 28.9\%$), reflecting considerable stochastic extinction in many runs. The peak infection reached 2.4\% of the population at an average time of 12 steps, and the epidemic duration extended to around 47 steps, indicating a prolonged but less intense outbreak. Early doubling time of infection was about 2.27 steps.

\subsection{Static SIR Simulation Results}
Figure~\ref{fig:static-sir} shows the corresponding epidemic curves from the static aggregated network SIR simulations.

\begin{figure}[http]
  \centering
  \includegraphics[width=0.7\textwidth]{results-12.png}
  \caption{Epidemic trajectories for the SIR process on the aggregated static network: rapid, near-complete infection of the population with quick recovery phase. Parameters: $\beta=0.00475$, $\gamma=1.0$.}
  \label{fig:static-sir}
\end{figure}

The static network produced a dramatic epidemic, with a near-total population final size ($\approx$ 99.2\%), high peak infection fraction (45\%), rapid time to peak (2.27 steps), and short epidemic duration (roughly 8.3 steps). The initial doubling time was very low, 0.24 steps, indicating explosive early growth.

\subsection{Analytic Mean-Field Solution}
The analytic prediction for the final epidemic size from the classic SIR final size equation,

\begin{equation}
r = 1 - e^{-R_0 r} \quad \text{with} \quad R_0 = 3,
\end{equation}

was solved numerically yielding $r = 0.9405$ (94.05\%). This benchmark confirms that both the static simulation and analytic mean-field results align quantitatively, whereas the temporal network simulation deviates substantially due to mechanistic constraints.

\subsection{Comparison and Interpretation}
Table~\ref{tab:metrics-comparison} collates key metrics from all three approaches.

\begin{table}[h]
  \centering
  \caption{Key Epidemic Metrics across Network Representations and Analytical Prediction}
  \label{tab:metrics-comparison}
  \begin{tabularx}{\textwidth}{lXXX}

    \toprule
    Metric (unit) & Temporal SIR & Static SIR & Analytic Solution \\
    \midrule
    Final Epidemic Size (fraction)      & $0.205 \pm 0.289$ & $0.992$ & $0.9405$ \\
    Peak Infection Fraction             & 0.0241      & 0.45      & -- \\
    Time to Peak (steps)                & 12          & 2.27      & -- \\
    Epidemic Duration (steps)           & 47          & 8.3       & -- \\
    Initial Doubling Time (steps)       & 2.27        & 0.239     & -- \\
    Population Size                    & 1000         & 1000      & 1000 \\
    \bottomrule
  
\end{tabularx}
\end{table}

The temporal network simulation highlights a stark reduction in epidemic magnitude and speed, attributable to temporal causality: edges exist only transiently and in a time-ordered manner, limiting infectious paths and effective connectivity. In contrast, the static aggregated network, by ignoring temporal ordering of contacts, overestimates available transmission routes, resulting in rapid, near-complete epidemics that mirror mean-field theory predictions.

This interpretation is supported by the epidemic curves and peak infection sizes, where temporally constrained interactions limit outbreak size and delay peak times. Larger stochastic variability in final size under temporal dynamics highlights the role of chance extinction factors when temporal contacts limit transmission opportunities.

Overall, these results demonstrate quantitatively how temporal network structures critically reduce epidemic potential compared to aggregated static representations by enforcing causality and preventing overstated connectivity.

\section*{Summary}
The comparison between activity-driven temporal and aggregated static networks under the same epidemic parameters clearly shows that temporal causality restricts the spread of infection, lowering both outbreak size and speed. These findings emphasize the necessity of incorporating temporal dynamics for realistic epidemic forecasting and intervention planning.

\clearpage

\section{Discussion}

This study rigorously examined the differences in epidemic dynamics and outcomes between a mechanistic activity-driven temporal network and its time-aggregated static counterpart, both constructed to model a generic infectious disease spreading under the SIR framework with an initial basic reproduction number \( R_0 = 3 \). The comparison encompassed analytic mean-field solutions alongside stochastic simulations on networks with 1000 nodes, elucidating the impact of temporal causality constraints on epidemic potential and transmission dynamics.

A central finding is the stark contrast in epidemic size and speed between the temporal and static scenarios. The temporal activity-driven network produced significantly smaller and slower epidemics relative to both the static network simulation and the analytic predictions based on aggregated parameters. Specifically, the temporal SIR process resulted in a mean final epidemic size of approximately 20.5\%, with a large standard deviation of 28.9\%, indicating high stochastic variability and frequent epidemic die-out events at the chosen parameters (Figure \ref{fig:temporal-sir}). By contrast, the static network simulation yielded near-complete infection of the population (99.2\% final size) in rapid fashion (Figure \ref{fig:static-sir}), closely aligning with the analytical final size solution of 94.05\% derived from the classic mean-field equation 
\[
r = 1 - \exp(-R_0 r)
\]
for \( R_0 = 3 \).

These results reflect the profound mechanistic effect of temporal causality on disease transmission pathways. In the temporal network, edges represent instantaneous contacts that exist transiently and are reorganized stochastically at each time step. Consequently, the temporal ordering of contacts restricts the accessibility of transmission chains, effectively increasing the epidemic threshold and reducing the reachable susceptible population. Many potential transmission paths present in the aggregated static network are simply infeasible in the temporal network due to non-overlapping timings of contacts necessary for causally coherent infections.

The quantitative metrics summarized in Table \ref{tab:metrics-transposed} reinforce this interpretation. The temporal network epidemic peaked much later (at step 12) and exhibited a lower maximal prevalence (approximately 2.4\% of the population simultaneously infectious) compared to the static network’s explosive peak infecting about 45\% concurrently at around step 2. The early doubling time was an order of magnitude longer for the temporal network (2.27 steps) than for the static network (0.24 steps), highlighting a slower growth phase consistent with causality- and temporality-imposed bottlenecks.

The longer epidemic duration (about 47 steps) in the temporal network corresponds to a protracted outbreak with smaller transmission bursts over time, contrasting with the short-lived but intense epidemic in the static model (\(\sim 8.3\) steps total duration). This protraction and lower peak infection burden have direct implications for healthcare capacity modeling and intervention urgency, indicating that neglecting temporal structure may substantially overestimate both the resource peak needed and understate the persistence of an outbreak.

Moreover, the high variability in final epidemic size across temporal simulation runs underscores the increased stochasticity introduced by temporal network dynamics, calling attention to the importance of multiple realizations for robust risk assessment. Epidemic fadeouts are more common in the temporally resolved contact model, which reflects realistic phenomena such as superspreading event dependence and localized transmission restriction.

Methodologically, the choice of parameters \((\beta, \gamma, m, \alpha)\) mapped naturally onto the activity-driven network framework, where 
\[
R_0 = \left(\frac{\beta}{\gamma}\right) \times (m \times \alpha).
\]
Adjusting these parameters to correspond with \( R_0 = 3 \) enabled direct comparison with mean-field expectations and elucidated the interplay between infection probability, recovery rate, node activation frequency, and number of instantaneous contacts. Our modeling framework and simulation approaches accounted for the memoryless nature of temporal contacts and preserved full temporal ordering, validating the mechanistic interpretation.

The static network was constructed as a weighted Erd\H{o}s-R\'enyi-like random graph based on cumulative contact frequencies aggregated over the entire observation window, which inherently ignored the temporal sequence of contacts. This simplification led to artificial acceleration and inflation of transmission potential, as all edges are assumed concurrently active—a condition rarely met in real-world temporal contact structures. Thus, the static approach, while computationally convenient and analytically tractable, markedly overestimated the epidemic size and speed.

The findings underscore the essential role of temporal networks in accurately modeling infectious diseases transmitted by contact. Conventional static or time-aggregated network models are likely to overpredict epidemic risk, misestimate the timing and magnitude of outbreaks, and overlook critical causal transmission constraints. Incorporating temporal dynamics captures realistic contact sequences and disease transmission opportunities, and thereby improves the fidelity and relevance of epidemic forecasts and intervention design.

Future work extending these analyses may explore heterogeneous activation rates, variability in contact formation (degree distributions), and memory effects, which are known to further modulate epidemic thresholds and sizes. Additionally, examining the impact of intervention strategies within temporal frameworks can inform more precise and timely responses.

In conclusion, the comparative analysis conclusively demonstrates that temporal causality embedded in activity-driven temporal networks fundamentally limits epidemic spread compared to equivalent aggregated static network models. This mechanistic insight highlights that overlooking temporal ordering leads to overestimation of epidemic potential, emphasizing the necessity of temporally resolved network data and modeling in infectious disease epidemiology.

\begin{table}[h]
  \centering
  \caption{Key Epidemic Metrics across Scenarios}
  \label{tab:metrics-transposed}
  \begin{tabularx}{\textwidth}{lXXX}

    \toprule
    Metric (unit) & Temporal SIR & Static SIR & Analytic \\
    \midrule
    Final Epidemic Size (fraction)      & \(0.205 \pm 0.289\) & \(0.992\) & \(0.9405\) \\
    Peak Infection Fraction             & \(0.0241\)      & \(0.45\)      & --- \\
    Time to Peak (steps)                & \(12\)          & \(2.27\)      & --- \\
    Epidemic Duration (steps)           & \(47\)          & \(8.3\)       & --- \\
    Initial Doubling Time (steps)       & \(2.27\)        & \(0.239\)     & --- \\
    Population Size                     & \(1000\)        & \(1000\)      & \(1000\) \\
    \bottomrule
  
\end{tabularx}
\end{table}

% Figures referenced from the results section:
% Figure \ref{fig:temporal-sir} illustrates the flattened and protracted epidemic curve for the temporal network.
% Figure \ref{fig:static-sir} shows the rapid explosive outbreak on the static aggregated network.

These illustrative results collectively affirm that temporal network modeling is indispensable for realistic epidemic prediction, particularly for highly dynamic contact systems where timing and ordering of interactions critically shape disease propagation capability.

\section{Conclusion}

In this study, we have conducted a rigorous comparative analysis of epidemic dynamics under the SIR model on an activity-driven temporal network versus its time-aggregated static counterpart, both calibrated to the same basic reproduction number \( R_0 = 3 \). Our analytical and extensive simulation results compellingly demonstrate the profound impact of temporal causality on epidemic outcomes. Specifically, the temporal network, which preserves the causal ordering and transient nature of contacts, constrains the potential transmission pathways, resulting in substantially smaller and slower epidemics compared to the statically aggregated network that neglects temporal ordering.

The key findings can be summarized as follows. First, the epidemic threshold in the temporal framework is effectively higher due to causality-imposed transmission restrictions, leading to an average final epidemic size of only about 20\%, markedly lower than the near-complete outbreaks predicted by both the aggregated static network simulations and classical mean-field theory (final size \( \approx 94\% \)). Second, the temporal network yields lower peak infection prevalence, delayed time to peak, and prolonged epidemic duration, reflecting a slower, more stochastic spreading process constrained by instantaneous and memoryless contacts. Third, the high variability in epidemic outcomes on the temporal network highlights realistic stochastic extinction phenomena absent in static approximations.

These results underscore that static, time-aggregated network representations dramatically overestimate epidemic risk by assuming all observed contacts are simultaneously available, thereby failing to capture the causality and time dependency inherent in real human interactions. This limitation risks misinforming public health assessments by exaggerating outbreak severity and underestimating the time window for intervention.

Although the study focuses on homogeneous activity-driven networks with uniform parameters and memoryless dynamics, it establishes a strong mechanistic foundation highlighting the necessity to incorporate temporal features when modeling infectious disease spread. Potential limitations include the assumptions of homogeneous activation rates and contact distributions, lack of spatial or demographic heterogeneity, and absence of adaptive behavioral responses, which are known to influence epidemic thresholds and trajectories.

Future research directions should aim to incorporate heterogeneity in node activity, memory effects, and empirical temporal network data to further elucidate the interplay between temporal structure and disease transmission. Additionally, extending these frameworks to assess targeted intervention strategies within temporally realistic settings can significantly enhance epidemic preparedness and response planning.

In conclusion, our findings unequivocally demonstrate that incorporating temporal causality and dynamic contact patterns is essential for accurate epidemic modeling. This mechanistic insight compels a paradigm shift away from static aggregated networks towards temporally resolved frameworks in epidemiological modeling and public health decision-making.

\begin{thebibliography}{99}

\bibliographystyle{plain}

\bibitem{Perra2012TimeDriven} N. Perra, B. Gon\c{c}alves, R. Pastor-Satorras, A. Vespignani, ``Activity driven modeling of time varying networks,'' \textit{Scientific Reports}, 2012.

\bibitem{Mancastroppa2020ActiveInactive} M. Mancastroppa, R. Burioni, V. Colizza, et al., ``Active and inactive quarantine in epidemic spreading on adaptive activity-driven networks,'' \textit{Physical Review E}, vol. 102, no. 2, p. 020301, 2020, DOI: 10.1103/PhysRevE.102.020301.

\bibitem{Tizzani2018SpreadingAging} M. Tizzani, S. Lenti, E. Ubaldi, et al., ``Epidemic spreading and aging in temporal networks with memory,'' \textit{Physical Review E}, vol. 98, p. 062315, 2018, DOI: 10.1103/PhysRevE.98.062315.

\bibitem{Sun2015MemoryImpact} K. Sun, F. Güttler, Y.C. Lai, F. Zahn, ``Memory effects and epidemic models on temporal networks,'' \textit{Physical Review E}, vol. 88, pp. 1--8, 2015, DOI: 10.1140/epjb/e2015-60568-4.

\bibitem{Kim2019TemporalConnectivity} H. Kim, M. Ha, H. Jeong, ``Temporal connectivity and activity driven networks,'' \textit{Journal of Statistical Mechanics: Theory and Experiment}, 2019, DOI: 10.1140/epjb/e2019-100159-1.

\bibitem{Sun2014NonMarkovian} K. Sun, A. Baronchelli, N. Perra, ``Epidemic spreading in non-Markovian time-varying networks,'' ArXiv, abs/1404.1006, 2014.

\bibitem{Lei2015StaticActivity} Y. Lei, X. Jiang, Q. Guo, et al., ``Contagion processes on the static and activity driven coupling networks,'' \textit{Physical Review E}, vol. 93, no. 3, p. 032308, 2015, DOI: 10.1103/PhysRevE.93.032308.

\bibitem{Liu2012ContagionTemporal} S. Liu, A. Baronchelli, N. Perra, ``Contagion dynamics in time-varying metapopulation networks,'' \textit{Physical Review E}, vol. 87, p. 032805, 2013, DOI: 10.1103/PhysRevE.87.032805.

\bibitem{Nadini2020AdaptiveBackbone} M. Nadini, A. Rizzo, M. Porfiri, ``Epidemic spreading in temporal and adaptive networks with static backbone,'' \textit{IEEE Transactions on Network Science and Engineering}, vol. 7, pp. 549--561, 2020, DOI: 10.1109/TNSE.2018.2885483.
\end{thebibliography}
\newpage
\section*{Supplementary Material}
\begin{algorithm}[H]
\caption{Generate Activity-Driven Temporal Network Edge Events}
\begin{algorithmic}[1]
\State Initialize parameters $N$, $\alpha$, $m$, $T$
\State Initialize empty lists \texttt{all\_edges}, \texttt{temporal\_edge\_events}, and dictionary \texttt{edge\_counts}
\For{$t = 0$ to $T-1$}
\State Sample active nodes $\leftarrow \{i: \text{random}() < \alpha\}$
\For{each active node $src$}
\State Initialize empty set \texttt{targets}
\While{$\#\texttt{targets} < m$}
\State Sample random $tgt \neq src$
\State Add $tgt$ to \texttt{targets}
\EndWhile
\For{each $tgt$ in \texttt{targets}}
\State Sort $a,b \leftarrow \min(src, tgt), \max(src, tgt)$
\State Append edge event $\{\text{time}: t, \text{src}: a, \text{tgt}: b\}$ to \texttt{temporal\_edge\_events}
\State Append edge $(a,b)$ to \texttt{all\_edges}
\State Update \texttt{edge\_counts}$[(a,b)] \mathrel{+}= 1$
\EndFor
\EndFor
\EndFor
\State \Return \texttt{temporal\_edge\_events, edge\_counts}
\end{algorithmic}
\end{algorithm}

\begin{algorithm}[H]
\caption{Build Temporal Adjacency List from Edge Events}
\begin{algorithmic}[1]
\State Load temporal edge events with columns \texttt{time, src, tgt}
\State Determine $T = \max(\text{time}) + 1$
\State Initialize dictionary \texttt{time\_adjs} $\leftarrow \{t: [] \mid t \in [0, T-1]\}$
\For{each edge event $e$}
\State Append $(e.src, e.tgt)$ and $(e.tgt, e.src)$ to \texttt{time\_adjs}$[e.time]$ \Comment{undirected edges}
\EndFor
\State \Return \texttt{time\_adjs}
\end{algorithmic}
\end{algorithm}

\begin{algorithm}[H]
\caption{Stochastic SIR Simulation on Activity-Driven Temporal Network}
\begin{algorithmic}[1]
\State Input parameters: $N$, $\beta$, $\gamma$, $T$, \texttt{time\_adjs}, \texttt{nsim}
\State Compute per-timestep transition probabilities: $P_{\mathrm{inf}} = 1 - e^{-\beta}$, $P_{\mathrm{rec}} = 1 - e^{-\gamma}$
\For{each simulation run}
\State Initialize state vector \texttt{state} of length $N$ to Susceptible (0)
\State Infect one random node: \texttt{state}$[\mathrm{patient\_zero}] \leftarrow 1$
\For{$t = 0$ to $T-1$}
\State Identify infectious nodes and sample recoveries with probability $P_{\mathrm{rec}}$
\State Collect infectious nodes $I$ and susceptible nodes $S$
\State Initialize empty set \texttt{infected\_this\_step}
\For{each contact $(u,v)$ in \texttt{time\_adjs}$[t]$}
\If{$\texttt{state}[u] = 1$ and $\texttt{state}[v] = 0$ and random() $< P_{\mathrm{inf}}$}
\State Add $v$ to \texttt{infected\_this\_step}
\EndIf
\EndFor
\State Update states synchronously:
\State $\texttt{state}[I \to R] \leftarrow 2$
\State $\texttt{state}[S \to I] \leftarrow 1$ for nodes in \texttt{infected\_this\_step}
\State Record compartment sizes $S_t$, $I_t$, $R_t$
\State Terminate if no infectious nodes remain
\EndFor
\EndFor
\State Aggregate compartment trajectories and final sizes
\end{algorithmic}
\end{algorithm}

\begin{algorithm}[H]
\caption{Static Aggregated Network SIR Simulation using FastGEMF}
\begin{algorithmic}[1]
\State Load aggregated static network adjacency matrix $G_{\mathrm{csr}}$
\State Define SIR model schema with compartments $S$, $I$, $R$
\State Add node transition: $I \to R$ with rate $\gamma$
\State Add edge interaction: $S \to I$ induced by $I$ on $G_{\mathrm{csr}}$ with rate $\beta$
\State Initialize node states with one infected node
\State Run simulation for \texttt{nsim} replications and stopping criterion
\State Extract and save time-series results
\end{algorithmic}
\end{algorithm}

\begin{algorithm}[H]
\caption{Analytical Final Epidemic Size Computation}
\begin{algorithmic}[1]
\State Define $R_0$
\State Define function $f(r) = r - \left(1 - e^{-R_0 r}\right)$
\State Use numerical solver (e.g., \texttt{fsolve}) to find root $r$ such that $f(r) = 0$
\State Output $r$ as final epidemic size fraction
\end{algorithmic}
\end{algorithm}

\end{document}