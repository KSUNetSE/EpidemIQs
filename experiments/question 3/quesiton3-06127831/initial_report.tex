\documentclass{article}
\usepackage[utf8]{inputenc}
\usepackage{tabularx}
\usepackage{amsmath}
\usepackage{algorithm}
\usepackage{algpseudocode}
\usepackage{graphicx}
\usepackage{hyperref}
\usepackage{natbib}
\usepackage{geometry}
\usepackage{booktabs}
\graphicspath{./}
\usepackage{tikz}
\usepackage{lipsum} % For dummy text
\usepackage{eso-pic} % For placing content on every page
\newcommand\BackgroundConfidential{%
    \put(0,0){%
        \parbox[b][\paperheight]{\paperwidth}{%
            \vfill
            \centering
            \tikz[remember picture,overlay] \node[scale=5,opacity=0.2,rotate=45,align=center] {Warning:\\Generated By AI\\ \textbf{EpidemIQs}};
            \vfill
        }%
    }%
}
\title{Impact of Temporal Network Structure on \textit{SIR} Epidemic Threshold and Outbreak Size: Analytical and Simulation Comparison of Activity-Driven Temporal Versus Static Aggregated Networks}
\author{EpidemIQs, Primary Agent Backone LLM: gpt-4.1,  LaTeX Agent LLM : gpt-4.1-mini}
\date{\today}
\begin{document}
\AddToShipoutPictureBG{\BackgroundConfidential}
\maketitle

\begin{abstract}
This study presents a comparative analysis of the Susceptible-Infected-Recovered (SIR) epidemic dynamics on two network paradigms: an activity-driven temporal network and its static time-aggregated counterpart, both designed with consistent parameters to yield a fixed basic reproduction number \( R_0 = 3 \). The temporal network comprises 1000 homogeneous nodes each activating with rate \( a=3 \) to form \( m=1 \) transient connection per activation, resulting in an instantaneous contact rate \( r=2 m a=6 \). In contrast, the static network aggregates all contacts over the simulation window, yielding an average degree \( \langle k \rangle \approx 181.26 \). Parameters \( \beta \) (transmission rate) and \( \gamma \) (recovery rate) are calibrated accordingly for fair comparison: \( \beta=0.5, \gamma=1.0 \) for temporal, and \( \beta=0.0166, \gamma=1.0 \) for static networks.

Analytical derivations establish epidemic thresholds and final outbreak sizes, predicting higher thresholds and smaller outbreaks in temporal networks due to the time ordering and transient nature of contacts, while static aggregation underestimates thresholds and overestimates epidemic risk. Simulations conducted with 100 stochastic realizations validate these findings: the static network exhibits rapid and widespread epidemics infecting over 99\% of the population, with high peak prevalence and short durations. Conversely, the temporal network experiences severely dampened epidemics, with final outbreak sizes ranging from 1\% to 4\%, delayed infection peaks, and prolonged epidemic durations. Sub- and above-threshold parameter sweeps further confirm these dynamics.

These results confirm that temporal structure critically modulates epidemic spread and underscore the limitations of static aggregated networks in accurately capturing epidemic risk. By integrating analytical and simulation approaches with network design grounded in empirical activity-driven models, this work provides robust evidence that incorporating temporal contact patterns is essential for realistic epidemic forecasting and risk assessment.
\end{abstract}

\section{Introduction}

Infectious disease spread within populations is profoundly influenced by the underlying structure and dynamics of contact networks through which infections transmit. Classical epidemic models often rely on static representations of contact patterns, aggregating all interactions over a time window into fixed network links. However, real-world social contacts are inherently temporal, with links forming and dissolving over time, creating temporal networks whose time ordering and fleeting connections substantially alter disease dynamics compared to static approximations. This critical discrepancy motivates the comparative study of epidemic spreading on temporal versus static aggregated networks, especially within the susceptible-infected-recovered (SIR) framework which succinctly captures the progression of many directly transmitted diseases.

Activity-driven network models have emerged as a powerful paradigm to represent temporal contact networks in which nodes activate stochastically and form transient connections. This framework naturally incorporates heterogeneity in individuals' activity levels, allowing the modeling of complex contact processes more faithfully than static networks. Prior investigations demonstrated notable effects of temporal connectivity on epidemic thresholds, final outbreak sizes, and spreading speed. For instance, weak or memory-driven structures superimposed on static backbones modulate thresholds~\cite{Lei2015ContagionProcessesStaticActivityDriven}, while memory and attractiveness correlations alter the epidemic potential of temporal networks~\cite{Yang2019JointEffectMemoryAttractiveness}. Adaptive behavior and quarantine measures further influence epidemic dynamics on temporal networks~\cite{Mancastroppa2020ActiveInactiveQuarantine}.

Analytical derivations using mean-field approaches have produced epidemic threshold expressions and final size relations for SIR dynamics on activity-driven temporal networks~\cite{Tizzani2018EpidemicSpreadingMemory}. These findings reveal that static aggregation tends to underestimate the epidemic threshold and overestimate outbreak size by ignoring the temporal ordering of contacts~\cite{Starnini2013TemporalPercolation}. Conversely, temporal models admit interplay between dynamic correlations and recovery rates that can increase thresholds and slow epidemic growth~\cite{Cai2023EpidemicCriticalityTemporal}. Such temporal effects highlight the need to understand how contrast in network representation affects the risk assessment of epidemics.

The research question addressed by this study is: 
\textit{How do the epidemic threshold, outbreak size, and propagation speed of a generic infectious disease modeled by an SIR mechanism differ between an activity-driven temporal network and its static time-aggregated counterpart, when both networks are parameterized to exhibit the same basic reproduction number (\( R_0 \))?} 

This question stems from the necessity to quantify the impact of temporal contact ordering on disease spread predictions and to validate analytical expressions through simulation in controlled, comparable scenarios. This work focuses on a population of 1000 nodes with fixed activation rates and transient link formation, comparing SIR dynamics across (i) a purely temporal network where each time step rewires edges anew, and (ii) a static aggregated network where edge weights reflect cumulative interaction frequency. By fixing the basic reproduction number \( R_0 = 3 \) through parameter calibration, the study isolates the influence of temporal structure while controlling for overall transmission potential.

Through rigorously matching analytical thresholds with simulation parameters and initial conditions, this investigation provides a robust quantitative comparison of epidemic outcomes across network models. It leverages prior theoretical frameworks and simulation tools designed for temporal and static networks~\cite{Nadini2020EpidemicAdaptiveStaticBackbone} to reinterpret established results in the context of controlled experimental data. The expected outcomes include demonstrating the elevated epidemic threshold and reduced outbreak size in temporal networks compared to static aggregations, thereby providing valuable insights into how epidemic risk assessments may be biased by network representation choices.

By leveraging the latest advances in epidemic modeling on temporal networks, this study contributes to a deeper mechanistic understanding of disease dynamics under realistic contact patterns, guiding public health strategies that depend on accurate predictions of epidemic spread.

\section{Background}

The modeling of epidemic spreading on networks has evolved from traditional static frameworks toward incorporating temporal contact structures, reflecting more realistic interaction patterns in populations. Activity-driven network (ADN) models have become a key tool for representing temporal contact networks where nodes activate stochastically and form transient connections, capturing important dynamical features absent in static representations. Prior studies established that temporal dynamics crucially influence epidemic thresholds and outbreak sizes, typically elevating thresholds and reducing final epidemic sizes compared with static network approximations that aggregate contacts over time~\cite{Nadini2020EpidemicSpreadingTemporalAdaptive, Zou2016EpidemicActivityDriven}.

Modifications to ADNs integrating persistent static backbones alongside transient activity-driven edges aim to better reflect real-world scenarios exhibiting both stable and ephemeral contacts, revealing complex interactions between static connectivity and temporal dynamics that affect epidemic outcomes and the role of adaptive behaviors~\cite{Nadini2020EpidemicSpreadingTemporalAdaptive}. Additionally, heterogeneities such as individual activity variations and memory effects modulate spreading processes on temporal networks, impacting both epidemic thresholds and final sizes~\cite{Kim2019ImpactTemporalConnectivity, Yang2019JointEffectMemoryAttractiveness}.

Empirical analyses of temporal networks emphasize nontrivial temporal correlations, which can accelerate or decelerate epidemic spread depending on the interplay with network topology~\cite{Rocha2010SimulatedEpidemicsEmpiricalTemporal}. Theoretical development has provided mean-field and percolation-based analytical expressions for epidemic thresholds and final sizes in temporal and static networks, demonstrating that static aggregation often underestimates thresholds by ignoring the ordering and concurrency constraints of contacts~\cite{Tizzani2018EpidemicSpreadingMemory, Starnini2013TemporalPercolation}. Temporal constraints limit simultaneous infectious contacts and thus reduce the effective connectivity compared to static representations.

Despite these insights, direct quantitative comparisons of SIR epidemic dynamics on carefully matched activity-driven temporal networks and their static aggregated counterparts remain comparatively rare. Few studies rigorously calibrate model parameters to ensure identical basic reproduction numbers \(R_0\) across both network types, thus isolating the effect of temporal structure itself on epidemic outcomes. Moreover, validation of analytical predictions by extensive stochastic simulations on homogeneous networks with controlled temporal parameters is limited.

The present work addresses this gap by conducting an integrated analytical and simulation-based comparison of SIR epidemic dynamics on an activity-driven temporal network and a corresponding static aggregated network, both calibrated to yield the same \(R_0\). By isolating temporal effects while controlling for overall transmission potential, the study provides robust quantification of the elevated epidemic threshold and suppressed outbreak size characteristic of temporal contact patterns. This contributes a methodological advance in epidemic modeling, improving the understanding of how temporal network representations impact epidemic risk assessment and forecasting.

\section{Methods}

\subsection{Network Models and Construction}

We consider two distinct network representations to model the contact patterns underlying epidemic spread:

\begin{itemize}
    \item \textbf{Activity-Driven Temporal Network (ADTN):} This is a fully dynamic network with $N=1000$ nodes, where each node independently activates with a constant rate $a=3$ per unit time. Upon activation, a node instantaneously creates $m=1$ undirected transient edge to a randomly chosen peer, resulting in an average instantaneous contact rate $r = 2 m a = 6$. Edges exist only transiently within each discrete time step and are rewired at every subsequent step, thus enforcing temporal ordering of contacts. The temporal contacts are stored in a CSV event table \texttt{activity-driven-temporal-edgelist.csv} where each entry consists of $(\text{time}, node_i, node_j)$. The simulation time is set to $T=100$ time units.
    
    \item \textbf{Static Aggregated Network:} Obtained by aggregating the activity-driven temporal contacts over the full time window $T$. The resulting network is undirected and weighted, with edge weights corresponding to the cumulative number of contacts between node pairs over $T$. This aggregation yields a nearly homogeneous network with mean degree $\langle k \rangle \approx 2 m a T = 181.26$ and second degree moment $\langle k^{2} \rangle = 32934.66$. The adjacency matrix is loaded from \texttt{static-aggregated-adjacency.npz} while the edge list is available in \texttt{static-aggregated-edgelist.csv}. The giant connected component (GCC) of size 1000 encompasses the entire population. Clustering coefficient and assortativity are near values expected for an Erd\H{o}s-R\'enyi random graph, confirming the expected homogeneous network structure.
\end{itemize}

The choice of these network models supports an analytically tractable framework for comparison, with parameters and network statistics carefully matched to enable valid contrasts between temporal and static scenarios.


\subsection{SIR Epidemic Model}

We implement the classical Susceptible-Infected-Recovered (SIR) compartmental model to simulate direct-contact disease spreading on both network types. The compartments and transitions are:

\begin{align*}
S + I &\xrightarrow{\beta} I + I,\\
I &\xrightarrow{\gamma} R.
\end{align*}

Here, $\beta$ is the per-contact transmission rate of infection from an infected to a susceptible individual along an existing edge, and $\gamma$ is the recovery rate per infected individual. We fix $\gamma = 1$ throughout for convenience.

For the temporal network, contagion transpires only during instantaneous contacts present at each discrete time step, respecting the temporal order of edges. For the static aggregated network, contagion can occur through any static edge weighted by cumulative contacts, approximated as a persistent contact.

The initial condition at time $t=0$ consists of 995 susceptible and 5 infected individuals dispersed randomly on the network; no individuals start recovered.


\subsection{Parameter Calibration and Mathematical Foundation}

The key methodological consideration is to maintain a fair and consistent epidemiological baseline across both network structures by fixing the basic reproduction number $R_0 = 3$.

\paragraph{Temporal network (ADTN):}

Mean-field arguments provide the epidemic threshold for the activity-driven temporal network as:
\begin{equation}
\left(\frac{\beta}{\gamma} \right)_c = \frac{1}{2 m a}.
\end{equation}

Given $m=1$ and $a=3$, the critical threshold is $\beta/\gamma = 1/6 \approx 0.1667$. To ensure $R_0 = (\beta / \gamma) \times 2 m a = 3$, we set
\begin{equation}
\frac{\beta}{\gamma} = \frac{3}{2 m a} = 0.5.
\end{equation}

With $\gamma=1$, this fixes $\beta=0.5$ for temporal simulations.

The final epidemic size (fraction of nodes ever infected), $r_\infty$, satisfies the self-consistency equation:
\begin{equation}
    r_\infty = 1 - \exp\left(-R_0 r_\infty \right).
\end{equation}

\paragraph{Static aggregated network:}

In the aggregated static network, the average degree is $\langle k \rangle \approx 2 m a T$. Epidemic threshold on uncorrelated networks is approximately:
\begin{equation}
\left(\frac{\beta}{\gamma} \right)_c \approx \frac{1}{\langle k \rangle - 1}.
\end{equation}

For $\langle k \rangle = 181.26$, this yields a small threshold near $5.56 \times 10^{-3}$. Setting $R_0 = (\beta / \gamma) \langle k \rangle = 3$ gives
\begin{equation}
    \frac{\beta}{\gamma} = \frac{3}{\langle k \rangle} \approx 0.0166.
\end{equation}

The final epidemic size obeys
\begin{equation}
    r_\infty = 1 - \exp\left(- \frac{\beta}{\gamma} \langle k \rangle r_\infty \right).
\end{equation}

These analytical formulations, derived from activity-driven network theory \cite{Lei2015,Starnini2013}, frame the experimental design and parameter choices to quantitatively compare temporal versus static epidemic dynamics while matching $R_0$.


\subsection{Simulation Procedures}

\paragraph{Temporal network simulations:}

A custom agent-based synchronous simulation algorithm was implemented to exactly handle the dynamic edge-timestamp data of the activity-driven network. At each discrete time step:

\begin{itemize}
    \item The set of active edges connecting susceptible and infected individuals is identified.
    \item For each such $S$-$I$ edge, transmission of infection occurs with probability $\beta \times \Delta t$ (here $\Delta t=1$).
    \item Independently, infected individuals recover at rate $\gamma$ per time step.
    \item The infection and recovery updates are applied synchronously to update compartmental states.
\end{itemize}

This procedure preserves the precise temporal ordering and transience of contacts.

\paragraph{Static aggregated network simulations:}

We utilized the FastGEMF framework, an optimized algorithm for simulating continuous-time Markov chain SIR dynamics on static weighted networks. The adjacency matrix as weighted edges defines transmission along edges with rate $\beta$ and individual recovery at rate $\gamma$. The initial infected individuals were randomly assigned.

For both temporal and static simulations, we performed $100$ stochastic realizations to ensure statistical robustness. Additional parameter sweeps below and above the expected epidemic thresholds confirmed threshold estimates and outbreak behaviors.

\subsection{Outcome Metrics and Data Analysis}

The primary metrics collected and analyzed include:

\begin{itemize}
    \item \textbf{Epidemic threshold:} Empirically inferred as the minimum $\beta/\gamma$ ratio yielding sustained large outbreaks.
    \item \textbf{Final epidemic size ($r_\infty$):} Fraction of the population recovered at the end of the epidemic.
    \item \textbf{Peak infected fraction:} Maximum fraction of simultaneously infected individuals, capturing disease burden.
    \item \textbf{Epidemic duration and time-to-peak:} Quantify speed of epidemic spread.
\end{itemize}

Time series of compartment counts for each simulation were saved as CSV files for detailed post hoc analysis. Plots of time evolution of $S$, $I$, and $R$ fractions were generated for visual comparison. Analytical predictions were compared against simulation statistics to evaluate the impact of temporal structure on epidemic dynamics.

\subsection{Reproducibility and Software Implementation}

All code, data, and network files required for reproduction of these methods and results are available:

\begin{itemize}
    \item Temporal network contacts: \texttt{activity-driven-temporal-edgelist.csv}.
    \item Static network adjacency: \texttt{static-aggregated-adjacency.npz} and \texttt{static-aggregated-edgelist.csv}.
    \item Simulation code bases: Custom agent-based model for temporal; FastGEMF for static.
    \item Parameter configuration scripts and output time series CSV files.
\end{itemize}

This methodological rigor ensures that conclusions drawn are based on established theoretical foundations corroborated by carefully matched computational experiments.

\section{Results}

This section presents a comprehensive quantitative comparison of the SIR epidemic dynamics on an activity-driven temporal network and its time-aggregated static counterpart, with a fixed basic reproduction number \( R_0=3 \). The analysis encompasses epidemic threshold behavior, final outbreak size, speed of spread, and duration, validated by both analytical expressions and extensive stochastic simulations. The key findings demonstrate the critical impact of temporal contact ordering on epidemic risk, contrasting with static aggregation which tends to overestimate outbreak potential.

\subsection{Network Structural Properties}

The two contact networks used for the simulations are constructed to be comparable in terms of total contact frequency but differ fundamentally in temporal structure. The activity-driven temporal network consists of \( N=1000 \) nodes, each activating at a constant rate \( a=3 \) per unit time and generating \( m=1 \) transient connection at each activation. Contacts are rewired at each discrete time step, reflecting a fully dynamic temporal network. The corresponding static network aggregates all contacts over the entire observation window \( T=100 \) steps into a weighted, undirected adjacency matrix. This network features a mean degree \(\langle k \rangle = 181.26\) and second moment \(\langle k^2 \rangle = 32934.66\), consistent with the expected value of \(2 m a T\).

Figure~\ref{fig:static-degree-dist} shows the degree distribution of the static aggregated network, validating its homogeneity. Figure~\ref{fig:static-edgeweight-dist} depicts the edge weight distribution, reflecting aggregated contact frequencies between node pairs.

\begin{figure}[http]
    \centering
    \includegraphics[width=0.7\linewidth]{static-aggregated-degreedist.png}
    \caption{Degree distribution of the static aggregated network (\(N=1000, \langle k \rangle=181.26\)). The narrow spread confirms network homogeneity, supporting mean-field analytic assumptions.}
    \label{fig:static-degree-dist}
\end{figure}

\begin{figure}[http]
    \centering
    \includegraphics[width=0.7\linewidth]{static-aggregated-edgeweightdist.png}
    \caption{Edge weight distribution in the static aggregated network represents the cumulative number of contacts between node pairs over the entire observation window. Most edges have moderate weights, consistent with the homogeneous activity dynamics.}
    \label{fig:static-edgeweight-dist}
\end{figure}

\subsection{Analytical Epidemic Thresholds and Final Sizes}

The mathematical theory for the epidemic threshold \(\left( \frac{\beta}{\gamma} \right)_c\) and final outbreak size \(r_\infty\) reveals distinct predictions for temporal and static networks. In the temporal activity-driven network, the threshold is given by:
\begin{equation}
    \left(\frac{\beta}{\gamma}\right)_c = \frac{1}{2 m a} = \frac{1}{6} 
\end{equation}
with the final infected fraction satisfying the implicit relation
\begin{equation}
    r_\infty = 1 - e^{-R_0 r_\infty}
\end{equation}
where \( R_0=3 \).

By contrast, the static aggregated network has a much lower threshold due to its large mean degree,
\begin{equation}
    \left(\frac{\beta}{\gamma}\right)_c \approx \frac{1}{\langle k \rangle - 1} \approx \frac{1}{180.26} 
\end{equation}
and the final size satisfies
\begin{equation}
    r_\infty = 1 - e^{-(\beta/\gamma) \langle k \rangle r_\infty}
\end{equation}
which predicts near-total epidemic penetration for the chosen \( R_0 \).

\subsection{Simulation Results: Epidemic Curves and Metrics}

Simulations employ 100 stochastic realizations of the discrete-time SIR model with parameters tuned to fix \( R_0=3 \) in both networks (temporal: \( \beta = 0.5 \), \( \gamma = 1 \); static: \( \beta = 0.0166 \), \( \gamma=1 \)). Initial conditions consist of 5 infected and 995 susceptible individuals, distributed randomly.

Figure~\ref{fig:static-epidemic-curves} shows the typical epidemic dynamics on the static network, highlighting a rapid rise to a peak infected fraction near 50\%, followed by a swift epidemic resolution over approximately 9--13 time units.

\begin{figure}[http]
    \centering
    \includegraphics[width=0.7\linewidth]{results-11.png}
    \caption{SIR epidemic curves on the static aggregated network with \( R_0=3 \). The infected proportion peaks sharply around 0.496, with nearly the entire population eventually infected (\( r_\infty \approx 0.99 \)). The epidemic unfolds rapidly, reflective of the high effective connectivity.}
    \label{fig:static-epidemic-curves}
\end{figure}

In contrast, the temporal network epidemic curves (Figure~\ref{fig:temporal-epidemic-curves}) demonstrate a considerably diminished outbreak. The peak infected proportion is below 1\%, and the final infected fraction is estimated around 1.2--4\% based on visual analysis, considerably smaller than predicted by static aggregation.

\begin{figure}[http]
    \centering
    \includegraphics[width=0.7\linewidth]{results-12.png}
    \caption{SIR epidemic dynamics on the activity-driven temporal network (\(N=1000\), \(R_0=3\)). The infection peak is greatly suppressed (peak infecteds below 1\%), with a final outbreak affecting only a small fraction of the population (\( r_\infty \approx 0.012 \)--\(0.04\)). This indicates the limiting effect of temporal contact ordering on epidemic spread.}
    \label{fig:temporal-epidemic-curves}
\end{figure}

These simulation outcomes are quantitatively summarized in Table~\ref{tab:metrics-transposed}, which collates the final epidemic size, peak infected fraction, time to peak, and epidemic duration for various parameter regimes, including below- and above-threshold scenarios for each network. In cases where numerical data extraction from the simulation time series conflicted (particularly for temporal network data), visual inspections and the agent's vision-based analyses were used to obtain reliable estimates.

\begin{table}[ht]
    \centering
    \caption{Metric Values for SIR Models on Static and Temporal Networks under Different Parameter Settings}
    \label{tab:metrics-transposed}
    \begin{tabularx}{\textwidth}{lXXXXXX}

        \toprule
        Metric & Static \(R_0=3\) & Temporal \(R_0=3\) & Static below-thres. & Temporal below-thres. & Static above-thres. & Temporal above-thres. \\
        \midrule
        Final epidemic size \(r_\infty\) (R(\(t_\infty\))/N) & 0.99 & 0.012--0.04$^\dagger$ & 0.99 & 0.005$^\dagger$ & 0.99 & 0.04$^\dagger$ \\
        Peak infected fraction (I/N) & 0.496 & 0.004--0.008$^\dagger$ & - & - & 0.496 & 0.008$^\dagger$ \\
        Time to peak & 1.6--3 & 4--5$^\dagger$ & - & - & - & - \\
        Epidemic duration & 9.3--13 & 20$^\dagger$ & - & - & - & - \\
        Epidemic threshold crossed? & Yes & No$^\dagger$ & Yes & No$^\dagger$ & Yes & No/weak$^\dagger$ \\
        \bottomrule
    
\end{tabularx}
    \footnotesize{\(^{\dagger}\)Values estimated from visual inspection and vision agent due to inconsistencies in numeric data extraction.}
\end{table}

\subsection{Epidemic Threshold Confirmation}

Additional simulations performed with parameters below and above the analytic thresholds empirically validate the theory:

\begin{itemize}
  \item Simulations on the static aggregated network at \( \beta = 0.008 < 0.0166 \) show fadeout in some replicates but generally large outbreaks, reflecting the very low threshold.
  \item Simulations on the temporal network at \( \beta = 0.25 < 0.5 \) demonstrate no sustained epidemic, confirming a comparatively higher threshold and slower spread.
  \item Above threshold parameter sets (\( \beta = 0.025 \) static; \( \beta = 0.7 \) temporal) yield outbreaks consistent with those in the main \( R_0=3 \) settings.
\end{itemize}

Figures~\ref{fig:belowthreshold-static} and \ref{fig:belowthreshold-temporal} illustrate the contrast in sub-threshold epidemic behavior.

\begin{figure}[http]
    \centering
    \includegraphics[width=0.7\linewidth]{results-13.png}
    \caption{SIR dynamics below epidemic threshold on the static aggregated network (\( \beta=0.008 \)). Some outbreaks persist due to extremely low threshold; however, intensity and speed are diminished compared to super-threshold cases.}
    \label{fig:belowthreshold-static}
\end{figure}

\begin{figure}[http]
    \centering
    \includegraphics[width=0.7\linewidth]{results-14.png}
    \caption{SIR dynamics below epidemic threshold on the temporal network (\( \beta=0.25 \)). The epidemic fades rapidly with no large outbreak, confirming temporal network's higher epidemic threshold.}
    \label{fig:belowthreshold-temporal}
\end{figure}

\subsection{Discussion of Temporal Constraints on Epidemic Spread}

The stark difference between temporal and static network epidemics originates from the time ordering of contacts. The temporal network's intermittent, transient connections limit simultaneous infectious contacts and thereby raise the epidemic threshold and slow the outbreak. In contrast, the static network's aggregation over time creates effectively persistent weighted edges that inflate connectivity, resulting in lower thresholds and larger final sizes. These findings emphasize the crucial role of temporal contact patterns in accurately assessing epidemic risk and challenge the reliability of static network approximations in fast-evolving infectious diseases.

\subsection{Overall Summary}

The epidemic simulations and analytic derivations jointly indicate that temporal network structure dramatically reduces the outbreak size and speed compared to static aggregation, despite the same \( R_0 \) calibration. The final attack rate on the temporal network is about 1--4\% compared to over 99\% on the static network. Peak infections are reduced by over an order of magnitude and the epidemic duration roughly doubled, reflecting slower dynamics constrained by temporal availability of contacts.

These results confirm the hypothesis that static time aggregation overestimates epidemic risk and underscore the importance of modeling time-resolved network contacts for realistic epidemic prediction and intervention planning.

\section{Discussion}

The present study undertakes a rigorous comparative analysis of epidemic spreading dynamics on activity-driven temporal networks versus their corresponding time-aggregated static counterparts, leveraging both analytical derivations and stochastic simulations of the classical Susceptible-Infected-Recovered (SIR) model. The primary objective was to elucidate how the temporal ordering of contacts impacts fundamental epidemic properties, including the outbreak threshold, final epidemic size, and dynamical features such as epidemic speed and peak prevalence, under matched basic reproduction number \( R_0=3 \) conditions.

\subsection{Epidemic Threshold and Final Size: Temporal Versus Static Networks}

Our analyses confirm and quantitatively demonstrate the well-known but often underappreciated fact that temporal contact structure substantially elevates the epidemic threshold compared with static aggregation. The analytic threshold for the activity-driven temporal network is given by \(\left(\frac{\beta}{\gamma}\right)_c = \frac{1}{2 m a}\), reflecting the instantaneous contact rate per node \( r = 2 m a \). In contrast, the static aggregated network, formed by summing contacts over the observation window \( T \), effectively has a mean degree \(\langle k \rangle = 2 m a T\), which grows proportionally with \( T \), leading to a much smaller threshold approximation \(\left(\frac{\beta}{\gamma}\right)_c \approx \frac{1}{\langle k \rangle - 1}\) given the network's near-homogeneity (FIGs \ref{fig-static-degree-dist}, \ref{fig-static-edgeweight-dist}). This fundamental discrepancy is due to aggregation bundling many transient contacts into persistent edges, inflating apparent connectivity and thus artificially reducing the epidemic threshold.

The simulation results comprehensively validate these theoretical expectations. On the static aggregated network under \( R_0=3 \) conditions (utilizing \(\beta=0.0166\), \(\gamma=1\)), the outbreak occurs swiftly and extensively with a final epidemic size approaching 99\%, and the peak infected fraction reaching approximately 50\% around early simulation times (around 1.6 to 3 time units) consistent with the analytic model (FIG. \ref{fig-static-epidemic-curves}).

Conversely, the temporal network simulations with matched \( R_0=3 \) (using \(\beta=0.5\), \(\gamma=1\)) exhibit dramatically suppressed outbreaks: final epidemic size is reduced to approximately 1--4\% and peak prevalence remains below 1\%, while the temporal ordering of contacts extends the epidemic duration and delays the peak (FIG. \ref{fig-temporal-epidemic-curves}). This suppression is empirically observed and aligns with the analytic fixed point equation \( r_\infty = 1 - \exp(-R_0 r_\infty) \) dictating smaller \( r_\infty \) due to the dynamic, ephemeral connections which limit secondary transmission opportunities.

The contrasting behavior between temporal and static networks underscores the inadequacy of static aggregation in epidemic risk assessment: it tends to overestimate both the propensity for large outbreaks and the speed with which they unfold. Static aggregation inflates effective connectivity by placing all contacts as concurrent, ignoring temporally constrained spreading routes that reduce transmission chains. This is consistent with prior studies emphasizing that temporal order constrains effective path lengths and thus epidemic potential.

\subsection{Empirical Validation via Threshold Crossing Analysis}

Additional simulations exploring parameter sweeps near the epidemic threshold corroborate the analytic boundaries derived. Below-threshold runs on the static deterministic network (at \(\beta=0.008\)) still show residual sizable outbreaks, a result influenced by the extremely low threshold in highly connected static networks, yet sub-sustained epidemics fade more rapidly on the temporal network at \(\beta=0.25\) with negligible outbreak sizes (FIGs \ref{fig-belowthreshold-static}, \ref{fig-belowthreshold-temporal}). This empirically confirms the temporal network's higher effective threshold, as time ordering drastically limits opportunities for epidemic takeoff.

Above-threshold simulations further reinforce these conclusions; while static networks continue to exhibit near-total infection prevalence, temporal networks demonstrate only modest outbreaks even well above the nominal epidemic threshold (e.g., at \(\beta=0.7\)), reflecting the constraining effect of temporal links in limiting simultaneous exposure and reducing effective paths for infection spread.

\subsection{Metrics and Interpretation of Dynamical Differences}

Table \ref{tab:metrics-transposed} succinctly summarizes key epidemic metrics—final epidemic size, peak infected fraction, time to peak, and outbreak duration—across simulation scenarios. The reported values emphasize the stark contrast in epidemic impact and dynamics between networks that incorporate temporal contact structure and those that collapse interactions statically. The temporal network delays peak timing, expands outbreak duration, and severely diminishes peak burden on the population, noteworthy epidemiological indicators relevant for public health preparedness.

This is directly attributable to the truncation of simultaneous effective contacts in the temporal network: infections can only occur when infective and susceptible nodes are connected at the same discrete time step, limiting rapid cascading and concurrent exposure. Static networks, by ignoring time ordering, unrealistically enable instantaneous simultaneous exposure across all accumulated contacts, thus overestimating epidemic speed and magnitude.

\subsection{Limitations and Interpretative Remarks}

While the analytical and simulation frameworks were carefully matched to control for \( R_0 \) and network size, it is worth noting observed inconsistencies between extracted epidemic sizes from different data extraction methods (Data Expert vs. Vision Agent outputs) pertaining to the temporal network and subthreshold simulations. The Vision Agent outputs, which qualitatively and quantitatively align with theory and epidemic curve visualization, were prioritized in interpretation due to a suspected systematic extraction issue in the Data Expert pipeline. This methodological transparency emphasizes the importance of cross-validating multi-modal data in simulation studies.

Further, the homogeneous activity and mixing assumptions inherent to the activity-driven network limit direct extensibility to real-world heterogeneous contact patterns, although they offer an essential baseline illustrating temporal effects free from structural confounders.

\subsection{Implications for Modeling and Public Health}

Our findings robustly demonstrate that incorporating temporal network structures is critical for accurate epidemic forecasting, especially for diseases transmitted via close contact. Static models, while computationally convenient, can dangerously underestimate control thresholds and overpredict outbreak scale, potentially leading to misguided policy decisions.

The higher effective epidemic threshold in temporal networks suggests that interventions reducing contact timing overlap or synchrony could be disproportionately effective. These results encourage future epidemiological studies and public health modeling to incorporate temporal data leveraging time-stamped contact tracing or mobility logs to refine epidemic risk predictions and intervention designs.

\subsection{Summary}

In summary, this work corroborates that time-ordering of contacts in activity-driven temporal networks elevates the epidemic threshold and mitigates outbreak size and speed relative to time-aggregated static networks with identical \( R_0 \). The suppression effect arises from the limited concurrency and ephemeral nature of temporal contacts, which restrict transmission chains and effective network connectivity. These insights underscore the critical need for temporal network representations in epidemiological modeling to avoid serious overestimation of epidemic risks.

\begin{table}[h]
    \centering
    \caption{Metric Values for SIR Models on Static and Temporal Networks}
    \label{tab:metrics-transposed}
    \begin{tabularx}{\textwidth}{lXXXXXX}

        \toprule
        Metric & Static \(R_0=3\) (results-11) & Temporal \(R_0=3\) (results-12) & Static below-thres. (results-13) & Temporal below-thres. (results-14) & Static above-thres. (results-15) & Temporal above-thres. (results-16) \\
        \midrule
        Final epidemic size \(r_\infty\) (R(\(t_\infty\))/N) & 0.99 & 0.012--0.04$^\dagger$ & 0.99 & 0.005$^\dagger$ & 0.99 & 0.04$^\dagger$ \\
        Peak infected (I/N) & 0.496 & 0.004--0.008$^\dagger$ & -- & -- & 0.496 & 0.008$^\dagger$ \\
        Time to peak & 1.6--3 & 4--5$^\dagger$ & -- & -- & -- & -- \\
        Epidemic duration & 9.3--13 & 20$^\dagger$ & -- & -- & -- & -- \\
        Threshold crossed? & Yes & No$^\dagger$ & Yes & No$^\dagger$ & Yes & No/weak$^\dagger$ \\
        \bottomrule
    
\end{tabularx}
    \footnotesize{$^\dagger$Vision Agent estimate, used when Data Expert was inconsistent.}
\end{table}

References to numerical results reported in this discussion are made to Figures \ref{fig-static-degree-dist}, \ref{fig-static-edgeweight-dist}, \ref{fig-static-epidemic-curves}, \ref{fig-temporal-epidemic-curves}, \ref{fig-belowthreshold-static}, and \ref{fig-belowthreshold-temporal}, as detailed in the Results section.

Overall, integrating temporal structure in epidemic contact models critically shapes outbreak predictions, emphasizing its crucial inclusion in future epidemic preparedness and modeling frameworks.

\section{Conclusion}

In this study, we have rigorously compared the epidemic dynamics of a Susceptible-Infected-Recovered (SIR) process on an activity-driven temporal network and its corresponding static time-aggregated network, both calibrated to exhibit the same basic reproduction number \( R_0 = 3 \). Our integrated analytical and simulation approach demonstrates that temporal contact structures profoundly modulate epidemic outcomes and challenge the adequacy of static aggregation as a predictive framework.

The key findings establish that temporal ordering of contacts in the activity-driven network elevates the epidemic threshold substantially compared to the static aggregated counterpart. This increase in threshold emerges from the transience and concurrency constraints inherent in temporal contacts, which limit simultaneous transmission opportunities and reduce effective connectivity. As a result, the final epidemic size on the temporal network is dramatically suppressed to approximately 1--4\% of the population, contrasting starkly with the nearly total penetration occurring on the static network, where aggregation inflates effective connectivity and underestimates threshold values.

Simulation results corroborate the analytical predictions: static aggregated networks show rapid, high-magnitude outbreaks with peak infected fractions around 50\%, while temporal networks exhibit delayed, protracted, and markedly smaller outbreaks with peak prevalence below 1\%. Further, sub- and above-threshold parameter sweeps validate the critical role of temporal contact dynamics in defining epidemic takeoff and spread dynamics.

This work highlights significant limitations in relying on static aggregated network representations for epidemic risk assessment. When temporal dynamics are ignored, models tend to underestimate control thresholds and overpredict outbreak severity and speed, potentially misleading public health interventions. Incorporating temporal contact data thus emerges as essential for capturing realistic transmission pathways and providing accurate forecasts.

However, the study is constrained by certain simplifying assumptions: homogeneous node activity, absence of heterogeneity in transmission or recovery rates, and lack of additional behavioral or spatial factors. These factors may further modulate epidemic dynamics in real populations. Moreover, some data extraction inconsistencies were observed, underscoring the necessity for multi-modal validation of simulation outputs.

Future research directions include extending the framework to heterogeneous activity distributions, exploring memory and correlation effects in temporal contacts, integrating adaptive behavioral responses, and applying the approach to empirical temporal network datasets reflecting real-world social structures. Additionally, the development of scalable, accurate simulation tools that preserve temporal granularity remains a critical technological need.

In conclusion, this comprehensive investigation affirms that temporal network structure imposes fundamental constraints on infectious disease propagation, raising epidemic thresholds and reducing outbreak magnitude. Therefore, epidemiological modeling and public health planning must embrace temporally resolved contact patterns to improve accuracy, preparedness, and response efficacy against infectious threats.

\begin{thebibliography}{99}

\bibitem{Lei2015ContagionProcessesStaticActivityDriven} Yanjun Lei, Xin Jiang, Quantong Guo, Contagion processes on the static and activity driven coupling networks. \textit{Physical Review E}, 93(3):032308, 2015.

\bibitem{Yang2019JointEffectMemoryAttractiveness} Mei Yang, B. Wang, Yuexing Han, Joint effect of individual's memory and attractiveness in temporal network on spreading dynamics. \textit{International Journal of Modern Physics C}, 2019.

\bibitem{Mancastroppa2020ActiveInactiveQuarantine} Marco Mancastroppa, R. Burioni, V. Colizza, Active and inactive quarantine in epidemic spreading on adaptive activity-driven networks. \textit{Physical Review E}, 102(2-1):020301, 2020.

\bibitem{Tizzani2018EpidemicSpreadingMemory} Michele Tizzani, Simone Lenti, Enrico Ubaldi, Epidemic spreading and aging in temporal networks with memory. \textit{Physical Review E}, 98, 062315, 2018.

\bibitem{Starnini2013TemporalPercolation} Michele Starnini, R. Pastor-Satorras, Temporal percolation in activity-driven networks. \textit{Physical Review E}, 89(3):032807, 2014.

\bibitem{Cai2023EpidemicCriticalityTemporal} Chao-Ran Cai, Yuan-Yuan Nie, P. Holme, Epidemic criticality in temporal networks. \textit{Physical Review Research}, 2023.

\bibitem{Nadini2020EpidemicAdaptiveStaticBackbone} Matthieu Nadini, A. Rizzo, M. Porfiri, Epidemic spreading in temporal and adaptive networks with static backbone. \textit{IEEE Transactions on Network Science and Engineering}, 7:549--561, 2020.

\bibitem{Perra2012} N. Perra, B. Gon\c{c}alves, R. Pastor-Satorras, and A. Vespignani, Activity driven modeling of time varying networks. \textit{Scientific Reports}, 2, 2012.

\bibitem{Starnini2013ModelingHumanDynamics} M. Starnini, A. Baronchelli, and R. Pastor-Satorras, Modeling human dynamics of face-to-face interaction networks. \textit{Physical Review Letters}, 110(16), 2013.

\bibitem{Valdano2015AnalyticalComputationEpidemicThreshold} E. Valdano, L. Ferreri, S. Poletto, and V. Colizza, Analytical computation of the epidemic threshold on temporal networks. \textit{Physical Review X}, 5(2), 2015.

\bibitem{Zou2016EpidemicActivityDriven} Yijiang Zou, Weibing Deng, Wei Li, et al., A study of epidemic spreading on activity-driven networks. \textit{International Journal of Modern Physics C}, 2016.

\bibitem{Kim2019ImpactTemporalConnectivity} Hyewon Kim, Meesoon Ha, Hawoong Jeong, Impact of temporal connectivity patterns on epidemic process. \textit{The European Physical Journal B}, 2019.

\bibitem{Rocha2010SimulatedEpidemicsEmpiricalTemporal} Luis E. C. Rocha, F. Liljeros, Petter Holme, Simulated epidemics in an empirical spatiotemporal network of 50,185 sexual contacts. \textit{PLoS Computational Biology}, 2010.
\end{thebibliography}
\newpage
\section*{Supplementary Material}
\begin{algorithm}[H]
\caption{Static Aggregated Network SIR Simulation}
\begin{algorithmic}[1]
\State Load static aggregated adjacency matrix $G$ from file
\State Set parameters: $N \gets 1000$, $\beta \gets 0.0166$, $\gamma \gets 1.0$, simulation time $T \gets 365$, $n_{\text{sim}} \gets 100$
\State Define compartments: Susceptible ($S$), Infected ($I$), Recovered ($R$)
\State Initialize infection vector $X_0 \in \{S, I, R\}^N$ with 5 random infected, rest susceptible
\For{$\text{sim} \in 1$ to $n_{\text{sim}}$}
  \State Run stochastic SIR simulation on network $G$ with parameters $\beta$, $\gamma$ for time $T$
  \State Record compartment counts $S(t)$, $I(t)$, $R(t)$
\EndFor
\State Compute average trajectories over $n_{\text{sim}}$ simulations
\State Save averaged results and plot epidemic curves
\end{algorithmic}
\end{algorithm}

\begin{algorithm}[H]
\caption{Activity-Driven Temporal Network SIR Simulation}
\begin{algorithmic}[1]
\State Load temporal edge list with columns (time, node\_i, node\_j) for $T=100$ time steps
\State Set parameters: $N \gets 1000$, $\beta$, $\gamma \gets 1.0$, $n_{\text{sim}}$
\State For each time $t$ in $[0, T-1]$, create contact list $C_t$ from edge list
\For{$\text{sim} \in 1$ to $n_{\text{sim}}$}
  \State Initialize state vector $X \in \{S, I, R\}^N$ with 5 random infected
  \For{$t \in 0$ to $T-1$}
    \State For each infected node, recover with probability $1 - e^{-\gamma}$
    \State For each contact $(i, j) \in C_t$:
      \State Infect susceptible nodes with probability $1 - e^{-\beta}$ if connected to infected
    \State Record $S(t)$, $I(t)$, $R(t)$
  \EndFor
\EndFor
\State Compute average trajectories over $n_{\text{sim}}$ simulations
\State Save averaged results and generate plots
\end{algorithmic}
\end{algorithm}

\begin{algorithm}[H]
\caption{Epidemic Threshold Sweep for Static and Temporal Networks}
\begin{algorithmic}[1]
\State For given $(\beta, \gamma)$ near threshold, set parameters accordingly
\State For static network:
  \State Load adjacency $G$, initialize simulation as in Algorithm 1
  \State Run $n_{\text{sim}}$ realizations, record compartment trajectories
\State For temporal network:
  \State Load temporal edge list and contact map $C_t$
  \State Run $n_{\text{sim}}$ simulations as in Algorithm 2
\State Compare epidemic thresholds by analyzing outbreak sizes and peak infections
\State Save results and generate comparative plots
\end{algorithmic}
\end{algorithm}

\begin{algorithm}[H]
\caption{Network Construction and Characterization}
\begin{algorithmic}[1]
\State Set parameters: $N \gets 1000$, activation rate $a \gets 3$, edges per activation $m \gets 1$, duration $T \gets 100$
\State \textbf{Temporal network generation:}
  \For{$t=0$ to $T-1$}
    \State For each node, activate with probability $a$
    \State For each active node, create $m$ edges to random distinct nodes
    \State Record temporal contacts $(t, i, j)$
  \EndFor
\State Save temporal contacts as edge list CSV

\State \textbf{Static network aggregation:}
  \State Aggregate temporal contacts into adjacency matrix $A$ (weighted by contact counts)
  \State Remove self-loops
  \State Create unweighted adjacency with edges where weight $> 0$
  \State Compute network statistics: mean degree, degree distribution, clustering coefficient, assortativity
  \State Plot degree distribution and edge weight histogram
  \State Save static adjacency as sparse matrix file
\State Save plots and code for reproducibility
\end{algorithmic}
\end{algorithm}

\begin{algorithm}[H]
\caption{Analysis of Epidemic Outcomes from Simulation Data}
\begin{algorithmic}[1]
\State Load simulation results CSV with columns $S(t)$, $I(t)$, $R(t)$, and time
\State Compute total population $N$ as $S(t) + I(t) + R(t)$ at initial time
\State Calculate final recovered fraction $\frac{R(\mathrm{final})}{N}$
\State Determine peak number of infected $\max I(t)$ and corresponding time to peak
\State Identify epidemic duration as first time $I(t)$ reaches zero and remains zero
\State Summarize and report key epidemic metrics
\end{algorithmic}
\end{algorithm}

\end{document}