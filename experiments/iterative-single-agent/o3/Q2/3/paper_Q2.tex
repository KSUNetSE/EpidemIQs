% Compilable LaTeX document generated by AI
\documentclass[10pt,conference]{IEEEtran}
\usepackage{amsmath,amsfonts,amssymb}
\usepackage{graphicx}
\usepackage{hyperref}
\usepackage{booktabs}
\begin{document}

\title{Mechanisms for Epidemic Fade--Out: Declining Infectives Versus Susceptible Depletion -- Analytical Insights and Network Simulations}

\author{Anonymous Author}

\maketitle

\begin{abstract}
Understanding why the chain of transmission of an infectious disease ultimately breaks is a cornerstone of epidemic theory and control.  Classical deterministic theory attributes epidemic fade--out either to an intrinsic decline in the number of infectious individuals when the basic reproduction number $\mathcal R_0<1$, or to the exhaustion of susceptible hosts once the fraction of susceptibles drops below $1/\mathcal R_0$ for initially super--critical outbreaks ($\mathcal R_0>1$).  Using a joint analytical and simulation--based study on a static Erd\"os--R\'enyi contact network ($N=1000$, $\langle k\rangle\approx10$), we quantify both mechanisms.  Two parameterisations of an SIR process are considered: (i) a sub--critical scenario ($\mathcal R_0=0.8$) and (ii) a super--critical scenario ($\mathcal R_0=3.0$).  Exact stochastic network simulations with FastGEMF corroborate the deterministic prediction that fade–out in the first case is governed by the monotonic decline of infectives while ample susceptibles remain ($S_{\infty}\approx95\%$), whereas in the second case it is the depletion of susceptibles that reverses infection growth and terminates transmission ($S_{\infty}\approx15\%$).  The results provide quantitative evidence for both mechanisms and highlight their dependence on $\mathcal R_0$ and network structure.
\end{abstract}

\section{Introduction}
Breaking an epidemic chain of transmission can occur for fundamentally different reasons.  In sub--critical settings where the basic reproduction number $\mathcal R_0<1$, each infectious individual, on average, transmits to fewer than one new host; therefore the number of infectives $I(t)$ decays exponentially and the outbreak dies out mainly because the pool of infectives vanishes \cite{AndersonMay1991}.  Conversely, if $\mathcal R_0>1$ the incidence initially grows, but transmission eventually ceases once the susceptible fraction $S(t)/N$ falls below $1/\mathcal R_0$ \cite{KermackMcKendrick1927}.  Although these principles are well established in homogeneous--mixing theory, their manifestation on realistic contact networks warrants reassessment \cite{PastorSatorras2001,DiLauro2021}.

This paper re--examines the fade--out mechanisms analytically and validates them with stochastic simulations of an SIR process on an Erd\"os--R\'enyi (ER) network.  We show quantitatively how the two routes to termination emerge under different $\mathcal R_0$ regimes and provide measurable criteria, derived from network moments, that link the microscopic infection rate $\beta$ to an emergent $\mathcal R_0$.

\section{Methodology}
\subsection{Network Construction}
A static ER graph with $N=1000$ nodes was generated using connection probability $p=0.01$.  Python NetworkX implementation yielded mean degree $\langle k\rangle=9.976$ and second moment $\langle k^{2}\rangle=108.922$.  The adjacency matrix was stored in compressed sparse row (CSR) format (\texttt{network.npz}).

\subsection{Mechanistic Model}
An SIR compartmental model with states $\{S,I,R\}$ was encoded in FastGEMF.  Edges mediate infection at per--contact rate $\beta$; infectives recover at rate $\gamma=0.2\,\text{d}^{-1}$.  Two parameter sets were defined:
\begin{itemize}
 \item \textbf{Sub--critical}: $\mathcal R_0=0.8$, implying $\beta= \mathcal R_0\,\gamma / q$ where $q=(\langle k^{2}\rangle-\langle k\rangle)/\langle k\rangle$.  This gives $\beta=0.0161$.
 \item \textbf{Super--critical}: $\mathcal R_0=3.0$, hence $\beta=0.0605$.
\end{itemize}
Initial conditions place $1\%$ of nodes at random in $I$ and the remainder in $S$.

\subsection{Analytical Framework}
For a deterministic SIR model on a well--mixed population the infection dynamic obeys
\begin{equation}
\frac{\mathrm d I}{\mathrm d t}=\beta \frac{S}{N} I-\gamma I=I\bigl(\beta \tfrac{S}{N}-\gamma\bigr).\label{eq:dIdt}
\end{equation}
The sign of the bracket determines epidemic growth.  Two cases arise:
\begin{enumerate}
 \item[1)] $\mathcal R_0=\beta/\gamma <1$.  Then $\beta S/N-\gamma<0$ for all $t$ because $S\le N$.  Hence $I(t)$ decays monotonically and extinction results from the \emph{decline of infectives}.  A large susceptible residue remains: $S_{\infty}\approx N$.
 \item[2)] $\mathcal R_0>1$.  The term in brackets is initially positive until $S(t)$ falls below $N/\mathcal R_0$.  At that time, $I(t)$ reaches its peak, after which Eq.~\eqref{eq:dIdt} becomes negative.  Fade--out therefore requires \emph{susceptible depletion}.  Final size solves $R_{\infty}/N=1-\exp(-\mathcal R_0 R_{\infty}/N)$ \cite{DiekmannHeesterbeek2000}.
\end{enumerate}
These conclusions are expected to remain qualitatively valid on locally tree--like networks when $\beta$ is rescaled through the excess degree $q$ \cite{PastorSatorras2001}.

\subsection{Simulation Protocol}
FastGEMF version~0.2.0 was used to run one stochastic realisation per parameter set for an equivalent horizon of $160$~days.  Code is available in \texttt{simulation--12.py}.  Temporal compartment counts were exported to \texttt{results--11.csv} (sub--critical) and \texttt{results--12.csv} (super--critical) together with plotted trajectories (Figures~\ref{fig:sub}--\ref{fig:super}).

\section{Results}
\subsection{Trajectory Characteristics}
Figure~\ref{fig:sub} shows that with $\mathcal R_0=0.8$ infections decline immediately; the peak number of infectives is only $12$ individuals, reached on day~2.  The susceptible pool remains virtually intact ($S_{\infty}=953$).

By contrast, Figure~\ref{fig:super} illustrates explosive initial growth for $\mathcal R_0=3.0$, peaking at $244$ infectives on day~13, after which incidence collapses as susceptibles are exhausted.  Only $15\%$ of nodes stay uninfected ($S_{\infty}=147$).

\begin{figure}[t]
  \centering
  \includegraphics[width=0.95\columnwidth]{results-11.png}
  \caption{Sub--critical scenario ($\mathcal R_0=0.8$).  Infection declines monotonically while susceptibles remain abundant.}
  \label{fig:sub}
\end{figure}

\begin{figure}[t]
  \centering
  \includegraphics[width=0.95\columnwidth]{results-12.png}
  \caption{Super--critical scenario ($\mathcal R_0=3.0$).  Fade--out occurs after susceptible depletion reduces the effective reproduction number below unity.}
  \label{fig:super}
\end{figure}

\subsection{Quantitative Metrics}
Table~\ref{tab:metrics} summarises key indicators extracted from the CSV files.
\begin{table}[h]
\centering
\caption{Simulation metrics.}
\label{tab:metrics}
\begin{tabular}{lcccc}
\toprule
Scenario & $I_{\max}$ & $t_{\max}$ [d] & $R_{\infty}$ & $S_{\infty}$\\
\midrule
Sub--critical & 12 & 2.0 & 47 & 953\\
Super--critical & 244 & 13.1 & 853 & 147\\
\bottomrule
\end{tabular}
\end{table}

\section{Discussion}
The deterministic analysis predicts two distinct routes to epidemic extinction depending on the initial reproduction number.  Stochastic network simulations confirm these predictions quantitatively.

In the sub--critical case, the driving mechanism is the inherent reproductive deficit ($\mathcal R_0<1$) causing $I(t)$ to decay irrespective of susceptible availability.  Policy implications include that modest non--pharmaceutical interventions capable of pushing $\mathcal R_0$ marginally below one are sufficient to stop transmission without immunising the majority.

In the super--critical regime, the epidemic can only be halted after substantial immunisation through infection or vaccination has reduced the susceptible fraction below $1/\mathcal R_0$.  The final size closely matches the root of the transcendental equation for $R_{\infty}$, demonstrating that classical final size theory retains power on sparse random networks when parameters are appropriately renormalised.

Limitations include the small network size and absence of clustering; empirical networks with strong heterogeneity may lower the herd immunity threshold via selective removal of high--degree nodes \cite{DiLauro2021}.  Future work will explore these effects.

\section{Conclusion}
Analytical derivations and stochastic network simulations jointly show that epidemic fade--out can occur either because infectives decline when $\mathcal R_0<1$, or because susceptible depletion reduces the effective reproduction number below unity in initially super--critical outbreaks.  Recognising which mechanism dominates in a given context is vital for designing proportionate control measures and vaccination targets.

\section*{Appendix}
Python scripts and data files are available in the accompanying repository.  Key code snippets for simulation and analysis are provided below.
\begin{verbatim}
# simulation-12.py (excerpt)
beta = R0*gamma/q
mc = (fg.ModelConfiguration(SIR)
      .add_parameter(beta=beta, gamma=gamma)
      .get_networks(c=G))
sim = fg.Simulation(mc, initial_condition=initial,
                    stop_condition={'time':160}, nsim=1)
sim.run()
\end{verbatim}

\begin{thebibliography}{99}
\bibitem{KermackMcKendrick1927} W.~O. Kermack and A.~G. McKendrick, ``A contribution to the mathematical theory of epidemics,'' \emph{Proc. Roy. Soc. A}, vol. 115, no. 772, pp. 700--721, 1927.
\bibitem{AndersonMay1991} R.~M. Anderson and R.~M. May, \emph{Infectious Diseases of Humans}. Oxford, U.K.: Oxford Univ. Press, 1991.
\bibitem{DiekmannHeesterbeek2000} O.~Diekmann and J.~A.~P. Heesterbeek, \emph{Mathematical Epidemiology of Infectious Diseases}. John Wiley \& Sons, 2000.
\bibitem{PastorSatorras2001} R.~Pastor--Satorras and A.~Vespignani, ``Epidemic spreading in scale--free networks,'' \emph{Physical Review Letters}, vol.~86, no.~14, pp.~3200--3203, 2001.
\bibitem{DiLauro2021} F.~Di Lauro, L.~Berthouze, M.~Dorey \emph{et~al.}, ``The impact of contact structure and mixing on control measures and disease--induced herd immunity in epidemic models: A mean--field model perspective,'' \emph{Bulletin of Mathematical Biology}, vol.~83, pp.~1--26, 2021.
\end{thebibliography}

\end{document}