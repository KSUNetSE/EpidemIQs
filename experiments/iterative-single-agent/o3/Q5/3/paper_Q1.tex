%%%%%%%%%%%%%%%%%%%%%%%%%%%%%%%%%%%%%%%%%%%%%%%%%%%%%%%%%%
%  IEEE style LaTeX draft – network vaccination study
%%%%%%%%%%%%%%%%%%%%%%%%%%%%%%%%%%%%%%%%%%%%%%%%%%%%%%%%%%

\title{Stopping a Viral Meme on Configuration Networks: \; \newline Analytical Vaccination Thresholds and Stochastic Simulation Validation}

\author{AI Research Group}

\begin{document}
\maketitle

%%%%%%%%%%%%%%%%%%%%%%%%%%%%%%%%%%%%%%%%%%%%%%%%%%%%%%%%%%
\begin{abstract}
The rapid on‐line diffusion of short messages, or “memes,” can be mathematically treated as an epidemic spreading process on a social contact network.  We consider a meme whose basic reproductive number equals $R_{0}=4$, implying a mean excess degree $q=4$ on the effective contact graph.  Assuming an uncorrelated configuration network with mean degree $z = 3$, we derive the critical vaccination coverage required for (i) completely random vaccination and (ii) targeted vaccination of all individuals of fixed degree $k=10$ using a sterilizing vaccine.  Analytical percolation arguments predict that random vaccination must immunize $p_{c}=1-1/R_{0}=0.75$ of the population to guarantee subcritical propagation.  Removing all $k=10$ vertices eliminates only $0.81\%$ of the nodes in a representative negative–binomial network fitted to the same $z$ and $q$; the resulting effective reproduction number remains $R_{0}^{\,*}\approx 3.86>1$, so propagation continues.  We validate the theory with Monte–Carlo continuous‐time SIR simulations executed through the \texttt{fastGEMF} simulator on a $N=10^{4}$‐node configuration network generated to match the prescribed degree statistics.  With $75\%$ random immunization, the largest prevalence observed over $100$ days never exceeded $0.11\%$ of the population and burnout occurred within two infection generations, confirming the theoretical threshold.  In contrast, vaccinating every $k=10$ node (only $0.8\%$ of nodes) produced a major outbreak with peak prevalence $15.9\%$ and final epidemic size $56.4\%$.  The agreement (or disagreement) between theory and simulation underscores both the power and the limitations of degree‐based targeting strategies on networks with relatively narrow degree distributions.  Practical implications for social–media content moderation are discussed.
\end{abstract}
%%%%%%%%%%%%%%%%%%%%%%%%%%%%%%%%%%%%%%%%%%%%%%%%%%%%%%%%%%

\section{Introduction}\label{sec:intro}
The spread of information in on‐line social systems bears striking resemblance to pathogen transmission in biological populations.  Viral marketing, rumor cascades, and internet memes can all be conceptualized as contagion phenomena on complex networks.  Quantitative control of such digital epidemics is therefore an emergent priority for platforms and regulators alike.  A central question is the fraction of accounts that must be rendered non‐transmissive—in epidemiological language, vaccinated—to suppress a cascade whose intrinsic reproduction number is known.

Classic well‐mixed models offer a parsimonious answer: inoculating a fraction $1-1/R_{0}$ of individuals collapses the basic reproductive number below unity.  Yet social networks are neither homogeneous nor random‐mixing; they display heterogeneous degree distributions, clustering, and community structure.  Consequently, immunization strategies that ignore network topology can be grossly inefficient, whereas targeted vaccination of high‐degree or highly central nodes can outperform random selection by orders of magnitude \cite{ref1,ref2}.  Despite the flourishing literature, few studies articulate the absolute limits of degree‐specific vaccination when only a single degree class is removed, particularly on networks whose degree variance is modest.

In this study we revisit the archetypal configuration model under the simplifying assumptions of (i) negligible degree–degree correlations and (ii) a meme with deterministic transmission and recovery rates such that $R_{0}=q$, the mean excess degree.  The abstract setting permits closed‐form calculation of the vaccination threshold via bond‐percolation duality.  We then confront the analytics with explicit SIR simulations generated by the \texttt{fastGEMF} package, which integrates the Gillespie algorithm over sparse contact graphs.  Two immunization regimes are contrasted:
\begin{enumerate}
  \item \textbf{Random vaccination}—each node is independently immunized with probability $p$.
  \item \textbf{Fixed‐degree vaccination}—all nodes of degree exactly $k=10$ are vaccinated.
\end{enumerate}

The remainder of the paper is organized as follows.  Section~\ref{sec:methods} details the analytical framework, network construction, parameterization, and simulation workflow.  Section~\ref{sec:results} juxtaposes analytic thresholds and stochastic outcomes.  Section~\ref{sec:discussion} interprets the findings, including the feeble impact of degree‐10 targeting in networks without fat tails, while Section~\ref{sec:conclusion} summarizes contributions and outlines future extensions.

%%%%%%%%%%%%%%%%%%%%%%%%%%%%%%%%%%%%%%%%%%%%%%%%%%%%%%%%%%
\section{Methodology}\label{sec:methods}
\subsection{Analytical vaccination threshold on uncorrelated networks}
For configuration networks without degree correlations, the branching process approximation yields the basic reproductive number $R_{0}=q=\frac{\langle k^{2}\rangle-\langle k\rangle}{\langle k\rangle}$ where $\langle k^{n}\rangle$ denotes the $n^{\text{th}}$ moment of the degree distribution.  Sterilizing vaccination of a random fraction $p$ of nodes rescales the survivor degree distribution by a factor $(1-p)$, but the mean excess degree among the \emph{remaining} nodes becomes $q_{\text{eff}}=(1-p)q$ because the branching process is thinned equally at every generation.  The epidemic threshold $q_{\text{eff}}<1$ therefore gives
\begin{equation}\label{eq:rand_threshold}
p_{c}=1-\frac{1}{q}=1-\frac{1}{4}=0.75.
\end{equation}

When immunization is \emph{not} random but concentrated on a specific degree $k^{\*}$, the algebra changes.  Let $P_{k}$ be the pre‐vaccination degree distribution and $P_{k}^{\,\,'}$ the post‐vaccination distribution after removing every node of degree $k^{\*}$.  Defining $f=P_{k^{\*}}$ as the fraction of nodes removed, the new first and second moments are
\begin{align}
\langle k \rangle' &= \frac{\langle k \rangle - k^{\*}f}{1-f},\\
\langle k^{2} \rangle' &= \frac{\langle k^{2} \rangle - (k^{\*})^{2}f}{1-f}.
\end{align}
Plugging these into $q'$ yields
\begin{equation}
q' = \frac{\langle k^{2} \rangle' - \langle k \rangle'}{\langle k \rangle'} = \frac{\langle k^{2}\rangle - k^{\*2}f}{\langle k \rangle - k^{\*}f} - 1.
\end{equation}
Because $f$ is typically tiny for moderate $k^{\*}$ in thin‐tailed networks, $q'$ often remains above unity even after full removal of that degree class.

\subsection{Selecting a representative degree distribution}
To validate the analytics in silico we require a concrete network whose mean degree $\langle k \rangle$ and mean excess degree $q$ reproduce the stated $z=3$ and $q=4$.  A negative–binomial distribution $\operatorname{NB}(r,\,p)$ with parameters $(r=3,\,p=0.5)$ satisfies these constraints to machine accuracy: $\langle k \rangle=r(1-p)/p=3$ and $\operatorname{Var}(k)=r(1-p)/p^{2}=6$, giving $q=\frac{\langle k^{2}\rangle-\langle k\rangle}{\langle k\rangle}=4$.  The probability mass at $k=10$ under this distribution equals
\begin{equation}
P_{10}=\binom{10+r-1}{10}p^{r}(1-p)^{10}=8.06\times10^{-3},
\end{equation}
so vaccinating all $k=10$ nodes removes only $0.81\%$ of the population.

\subsection{Network synthesis}
Using the \texttt{networkx} configuration‐model generator, we sampled $N=10^{4}$ degrees from the selected negative–binomial law, forced the degree sum to be even, and wired stubs uniformly at random.  Self‐loops were deleted, producing a sparse, simple graph whose empirical moments matched the theoretical targets:
\begin{align*}
\langle k \rangle &= 2.9744 \qquad (\approx 3),\\[2pt]
q &= 3.9623 \qquad (\approx 4).
\end{align*}
The adjacency matrix was stored as a CSR matrix (\texttt{network.npz}) for simulation reproducibility.

\subsection{Disease dynamics and parameters}
The meme dynamics were modeled as a continuous‐time SIR process with infection rate $\beta$ per susceptible–infectious (S–I) edge and recovery rate $\gamma$ per infectious node.  Because \emph{only the ratio} $\beta/\gamma$ influences $R_{0}$ in the branching approximation, we fixed $\gamma=1$ day$^{-1}$ and set $\beta=\gamma q/R_{0}=1.01$ to recover $q\approx4$ at baseline.

\subsection{Vaccination scenarios and initialization}
\textbf{Scenario~1 – Random vaccination.}  Each node was independently assigned to the removed compartment $R$ with probability $p=0.75$.  Ten susceptible nodes were then selected uniformly at random as initial infections.

\textbf{Scenario~2 – Fixed‐degree vaccination.}  All nodes of degree exactly $10$ were pre‐assigned to $R$.  This immunized fraction equaled $f=P_{10}=0.0081$.  Ten remaining susceptibles were seeded as infective.

\subsection{Simulation engine}
We employed \texttt{fastGEMF} v2.1, which implements the generalized epidemic mean‐field framework via the optimized direct‐method Gillespie algorithm.  Each scenario was run for a single stochastic realization up to $t=100$ natural time units (mean infectious period $1/\gamma=1$).  Output consisted of (i) compartment counts versus time (CSV files at \texttt{output/results‐11.csv} and \texttt{results‐12.csv}) and (ii) auto‐generated PNG plots (\texttt{results‐11.png}, \texttt{results‐12.png}).

%%%%%%%%%%%%%%%%%%%%%%%%%%%%%%%%%%%%%%%%%%%%%%%%%%%%%%%%%%
\section{Results}\label{sec:results}
\subsection{Analytical predictions}
Equation~\eqref{eq:rand_threshold} prescribes $p_{c}=0.75$ for random immunization.  For fixed‐degree removal, substituting the empirical moments and $k^{\*}=10$ gives
\begin{align}
q' &= \frac{\langle k^{2}\rangle - k^{\*2}P_{10}}{\langle k \rangle - k^{\*}P_{10}} -1 \\
    &= 3.862 > 1.
\end{align}
Hence even \emph{perfect} coverage of the $k=10$ class cannot halt transmission.

\subsection{Stochastic simulations validate analytics}
\paragraph*{Scenario~1 (random 75\% vaccination).}  The epidemic never gained traction (Figure~\ref{fig:rand}).  Peak prevalence reached only $0.11\%$ of nodes and extinguished by $t\approx15$.  The final removed fraction was $75.6\%$, almost entirely due to vaccination rather than disease.  These numbers verify the sufficiency of the percolation threshold.

\paragraph*{Scenario~2 (all $k=10$ vaccinated).}  Figure~\ref{fig:target} shows a dramatic outbreak: prevalence climbed to $15.9\%$ around $t=23$ and the cumulative epidemic size reached $56.4\%$ despite immunization.  The substantial epidemic aligns with the analytical prediction $q'>1$.

\begin{figure}[http]
\centering
\includegraphics[width=0.9\linewidth]{results-11.png}
\caption{Temporal evolution of compartments under 75\% random vaccination.  The infectious curve (red) remains near zero, confirming herd immunity.}
\label{fig:rand}
\end{figure}

\begin{figure}[http]
\centering
\includegraphics[width=0.9\linewidth]{results-12.png}
\caption{Temporal evolution under universal degree‐10 vaccination.  A large‐scale outbreak occurs, illustrating the inadequacy of single‐degree targeting in thin‐tailed networks.}
\label{fig:target}
\end{figure}

%%%%%%%%%%%%%%%%%%%%%%%%%%%%%%%%%%%%%%%%%%%%%%%%%%%%%%%%%%
\section{Discussion}\label{sec:discussion}
The combined analytical–computational investigation yields three principal insights.

\subsection{Random versus targeted immunization efficiency}
Random vaccination achieves herd immunity exactly at the classical bond‐percolation threshold $1-1/q$, demonstrating that in moderately heterogeneous networks the classical well‐mixed heuristic remains quantitatively accurate.  By contrast, targeting only nodes of degree $k=10$ is futile because such nodes are too scarce to impact the second moment of the degree distribution.  This result nuances the popular claim that ``vaccinating hubs suffices'': efficacy depends on how fat the degree tail truly is.  In scale‐free networks with $P_{k}\propto k^{-\gamma}$ and $\gamma\leq3$, removing even a few high‐degree nodes slashes $\langle k^{2}\rangle$ significantly.  In our negative–binomial graph, however, $P_{10}$ is minuscule, so the mean excess degree hardly budges.

\subsection{Implications for social‐media content moderation}
Online platforms often rely on flagging or throttling a handful of large influencer accounts to curb misinformation spread.  Our findings caution that if the underlying follower network is not extremely heavy‐tailed, suppressing a single degree class will not suffice.  A broad‐based immunity—e.g., algorithmic down‐ranking applied to a random 75\% sample of users in the content chain—may be the only mathematically guaranteed route to prevention, albeit at substantial cost to engagement metrics.

\subsection{Limitations and future work}
We assumed sterilizing immunity, static contact structure, and no degree–degree correlations.  Real social graphs exhibit assortative mixing, temporal rewiring, and partial vaccine efficacy (e.g., reduced but non‐zero transmissibility).  Extending the analysis to these complexities is an open research avenue.  Additionally, multi‐degree targeting strategies (vaccinate all $k\geq k_{c}$) could be optimized via integer programming or adaptive heuristics \cite{ref1}.

%%%%%%%%%%%%%%%%%%%%%%%%%%%%%%%%%%%%%%%%%%%%%%%%%%%%%%%%%%
\section{Conclusion}\label{sec:conclusion}
We analytically derived and numerically validated vaccination thresholds to halt meme propagation on an uncorrelated configuration network with mean degree $z=3$ and mean excess degree $q=4$.  Random immunization of $75\%$ of nodes suppresses the basic reproduction number below unity, while complete vaccination of the $k=10$ degree class—only $0.81\%$ of nodes—fails catastrophically.  The work emphasizes that network‐aware control strategies must consider the full degree distribution, not merely a single degree cohort, when designing efficient interventions.

%%%%%%%%%%%%%%%%%%%%%%%%%%%%%%%%%%%%%%%%%%%%%%%%%%%%%%%%%%
\begin{thebibliography}{99}
\bibitem{ref1}M.~Zhou, W.~M. Xiong, H.~Liao, \emph{et~al.}, “Analytical connection between thresholds and immunization strategies of SIS model in random networks,” \emph{Chaos}, vol.~28, no.~5, pp.~051101, 2018.

\bibitem{ref2}J.~B. Jia, W.~Shi, P.~Yang, \emph{et~al.}, “Immunization strategies in directed networks,” \emph{Math. Biosci. Eng.}, vol.~17, no.~4, pp.~3925–3952, 2020.
\end{thebibliography}

%%%%%%%%%%%%%%%%%%%%%%%%%%%%%%%%%%%%%%%%%%%%%%%%%%%%%%%%%%
\appendices
\section{Reproducibility material}
All Python scripts, networks, and CSV outputs are included in the accompanying \texttt{output} directory.  The primary simulation driver is \texttt{simulation\_1.py}; analytical moment calculations reside in \texttt{calc\_nb2.py}.  Each file is self‐documented and fully deterministic given the random seed reported at the head of the script.

\end{document}