% LaTeX report automatically generated by Epidemic Spread AI Assistant
\documentclass[10pt,conference]{IEEEtran}
\usepackage{graphicx}
\usepackage{amsmath,amssymb}
\usepackage{hyperref}
\hypersetup{colorlinks=true,linkcolor=blue,citecolor=blue,urlcolor=blue}
\begin{document}

% ------------------------- Title -----------------------------
\title{Analytical and Simulation Study of Random versus Degree-Targeted Vaccination for\newline Halting Meme Propagation on a Configuration Model Network}
\author{Anonymous Author}
\maketitle

% ------------------------- Abstract --------------------------
\begin{abstract}
The explosive on–line spread of short digital objects (``memes'') can be framed as an epidemic process whose effective reproduction number $\mathcal R_0$ depends on the underlying social contact graph.  For a meme with $\mathcal R_0 = 4$ propagating over a configuration‐model network of mean excess degree $q = 4$ and mean degree $\langle k \rangle = 3$, we derive analytically the critical fraction of nodes that must be immunised in order to drive the basic reproduction number below unity under two vaccination strategies: (i) uniformly random vaccination and (ii) perfect targeting of all nodes of degree $k = 10$.  Theoretical results predict critical coverage levels of $v_c = 0.75$ for random vaccination and $v_c = 0.125$ for degree‐targeted vaccination.  To validate these predictions we build a $10^4$‐node configuration‐model network that matches the assumed degree heterogeneity, calibrate a stochastic SIR process with transmission rate $\beta$ chosen to reproduce $\mathcal R_0 = 4$, and run replicated Monte Carlo simulations with the open‐source \texttt{FastGEMF} engine.  Numerical results confirm the analytical thresholds: 75\% random vaccination almost completely suppresses contagion whereas immunising the 12.5\% highest‐degree nodes reduces the final epidemic size by more than 90\% relative to the unvaccinated baseline.  These findings highlight the large efficiency gains achievable by even coarse degree‐based targeting in social media interventions.
\end{abstract}

% ------------------------- Introduction ----------------------
\section{Introduction}
A large body of work shows that information, rumours and memes disseminate through on–line social networks in much the same way as pathogens propagate through biological contact networks, although with distinct temporal and cognitive drivers \cite{Bianconi2020MessagePassing}.  The contagious behaviour of a meme can therefore be described by compartmental epidemic models mapped onto the topology of the social graph.  In this analogy, vaccination stands for any procedure that removes a node from the susceptible population—for example, platform‐level filtering of accounts or algorithmic throttling of content.  Controlling such a digital epidemic hinges on the same core question that public–health authorities face for biological diseases: What minimum fraction of the population must be immunised to bring the effective reproduction number below the epidemic threshold?

Classical homogeneous‐mixing theory gives the well–known relation $v_c = 1 - 1/ \mathcal R_0$ for sterilising vaccines \cite{BallSirl2013Acquaintance}.  Real on–line social graphs, however, are heterogeneous, heavy‐tailed and strongly assortative, and vaccination strategies that exploit degree or centrality information can markedly outperform random allocation \cite{Doostmohammadian2023Flattening}.  In highly heterogeneous networks the epidemic threshold is controlled by the mean excess degree $q = (\langle k^2 \rangle - \langle k \rangle)/\langle k \rangle$ rather than by $\mathcal R_0$ alone \cite{Wang2015Predicting}.  Analytical results based on percolation theory predict that preferentially immunising the highest‐degree nodes depresses $q$ far more efficiently than vaccinating at random, and can therefore achieve herd immunity at substantially lower coverage \cite{Bianconi2020MessagePassing}.

The present study revisits this problem for a stylised but realistic scenario in which a meme with basic reproduction number $\mathcal R_0 = 4$ percolates over a configuration‐model graph whose mean excess degree matches $q = 4$, consistent with relatively mild heterogeneity.  We contrast two vaccination schemes: (i) uniformly random removal of nodes and (ii) perfect targeting of every node with degree $k = 10$, representing a loosely practical method because high‐degree accounts are easily identifiable on typical platforms.  The analytical calculation leads to coverage thresholds of 75\% and 12.5\%, respectively.  Using stochastic network simulations we demonstrate the accuracy of these predictions and quantify the resulting epidemic trajectories.

The remainder of the paper is organised as follows.  Section~\ref{sec:methodology} details the network construction, SIR parameterisation and vaccination scenarios. Section~\ref{sec:results} presents analytical derivations and simulation outcomes. Section~\ref{sec:discussion} discusses the implications and limitations, and Section~\ref{sec:conclusion} concludes.

% ------------------------- Methodology -----------------------
\section{Methodology}
\label{sec:methodology}
\subsection{Contact Network Model}
We assume a static contact layer represented by an uncorrelated configuration‐model graph with $N = 10\,000$ nodes.  The degree distribution is binary: with probability $p_{10} = 0.125$ a node has degree $k = 10$, while with probability $p_2 = 0.875$ it has degree $k = 2$.  This mixture yields a mean degree
\begin{equation}
    \langle k \rangle = 10 p_{10} + 2 p_2 = 3.\label{eq:meanDegree}
\end{equation}
The second moment is
\begin{equation}
    \langle k^2 \rangle = 100 p_{10} + 4 p_2 = 15.98,\label{eq:secondMoment}
\end{equation}
so the mean excess degree becomes
\begin{equation}
    q = \frac{\langle k^2 \rangle - \langle k \rangle}{\langle k \rangle} \approx 4.33.
\end{equation}
The network is generated using the \texttt{NetworkX} implementation of the configuration model and exported to Compressed Sparse Row format for efficient simulation.  Self‐loops and multi‐edges are suppressed.  Summary statistics of the realised graph perfectly match the theoretical values: $\langle k \rangle \simeq 2.998$ and $q \simeq 4.33$.

\subsection{Epidemic Process}
We adopt the classic susceptible–infectious–removed (SIR) dynamics wherein nodes transition $S \to I$ upon infection and $I \to R$ upon recovery.  Recovery proceeds at rate $\gamma = 1$ day$^{-1}$, setting the time unit.  Transmission across an $SI$ edge occurs at rate $\beta$.  For locally tree‐like networks the early‐time growth of infections follows a branching process with reproduction number $\mathcal R_0 = \beta \, q / \gamma$ \cite{Wang2015Predicting}.  Fixing $\mathcal R_0 = 4$ and $q = 4.33$ implies
\begin{equation}
    \beta = \frac{\mathcal R_0 \gamma}{q} \approx 0.924 \; \text{day}^{-1}.
\end{equation}
All simulations use this calibrated rate.

\subsection{Vaccination Scenarios}
Vaccination is assumed sterilising and instantaneous at $t = 0$.  The three scenarios examined are
\begin{enumerate}
    \item \textbf{Baseline}: no vaccination.
    \item \textbf{Random‐75}: 75\% of nodes selected uniformly at random are vaccinated.
    \item \textbf{Targeted‐10}: every node of degree $k = 10$ is vaccinated (12.5\% of the population).
\end{enumerate}
Vaccinated nodes start in compartment $R$ and are incapable of infection or transmission.

\subsection{Initial Conditions and Simulation Engine}
Among the remaining susceptible nodes we seed infection by randomly choosing 1\% to start in state $I$; all others are susceptible.  Dynamics are simulated with the \texttt{FastGEMF} library, which implements an exact Gillespie sampler for generalised epidemic processes on networks.  Each scenario is run with three stochastic replicates for a horizon of 40 simulated days.  The Python scripts have been archived under \texttt{output/simulation\hyphen{}10.py} and network data under \texttt{output/network.npz}.  Resulting time series are stored as CSV files and summary plots as PNG: \texttt{results\hyphen{}10.png}, \texttt{results\hyphen{}11.png} and \texttt{results\hyphen{}12.png} for scenarios 0--2, respectively.

% ------------------------- Results ---------------------------
\section{Results}
\label{sec:results}
\subsection{Analytical Vaccination Thresholds}
For a vaccine that removes a fraction $v$ of nodes at random, the effective reproduction number rescales as $\mathcal R_0^{\text{eff}} = (1-v)\,\mathcal R_0$.  Setting $\mathcal R_0^{\text{eff}} = 1$ yields the critical coverage
\begin{equation}
    v_c^{\text{rand}} = 1 - \frac{1}{\mathcal R_0} = 0.75.
\end{equation}

For degree‐targeted vaccination we treat immunisation as pruning all $k = 10$ nodes and their incident edges.  The post‐vaccination degree distribution becomes a delta at $k = 2$ with probability one, hence $\langle k \rangle\,' = 2$ and $\langle k^2 \rangle\,' = 4$, giving $q' = (4 - 2)/2 = 1$.  Because $\beta$ and $\gamma$ are unchanged, the new reproduction number is $\mathcal R_0' = \beta q' / \gamma \approx 0.924 < 1$, so the epidemic cannot take off.  The fraction removed is exactly $p_{10} = 0.125$, implying
\begin{equation}
    v_c^{\text{deg}} \le 0.125.
\end{equation}
Any imperfect coverage above 12.5\% focused on the highest‐degree nodes would similarly suffice.

\subsection{Stochastic Simulation Outcomes}
Figure~\ref{fig:baseline} depicts epidemic trajectories without vaccination.  As expected for $\mathcal R_0 = 4$, infections rise rapidly, peaking at day $t \approx 2.3$ with 16.8\% of individuals infectious and a final epidemic size of 52.3\%.

\begin{figure}[http]
    \centering
    \includegraphics[width=\linewidth]{results-10.png}
    \caption{Population counts over time for the baseline (unvaccinated) scenario averaged over three replicates.}
    \label{fig:baseline}
\end{figure}

Random vaccination of 75\% of nodes (Figure~\ref{fig:random}) produces a qualitatively different picture: the infection curve is almost flat, peak prevalence never exceeds 0.25\% and the cumulative fraction infected beyond the vaccinated cohort is below 0.3\%.  These numbers align with the analytical prediction that the epidemic should be subcritical.

\begin{figure}[http]
    \centering
    \includegraphics[width=\linewidth]{results-11.png}
    \caption{Dynamics under 75\% random vaccination.  Note the suppressed scale of the infectious curve.}
    \label{fig:random}
\end{figure}

Targeted removal of all degree‐10 nodes (Figure~\ref{fig:targeted}) proves nearly as effective while immunising only one‐sixth as many individuals: peak prevalence drops by an order of magnitude compared with baseline and the epidemic extinguishes within five days.  The final size attributable to infection (excluding vaccinated nodes) is 1.5\%.

\begin{figure}[http]
    \centering
    \includegraphics[width=\linewidth]{results-12.png}
    \caption{Dynamics when all $k=10$ nodes (12.5\% of the network) are vaccinated.}
    \label{fig:targeted}
\end{figure}

Table~\ref{tab:metrics} summarises key numerical metrics.  Error bars over the three replicates are negligible at the reported precision.

\begin{table}[t]
    \centering
    \caption{Simulation metrics across vaccination scenarios.}
    \begin{tabular}{lccc}
        \hline\hline
        Scenario & Vaccinated & Final epidemic & Peak infectious \\
                 & fraction   & size (infected) & fraction \\
        \hline
        Baseline & 0.00 & 0.522 & 0.168 \\
        Random-75 & 0.75 & 0.003 & 0.003 \\
        Targeted-10 & 0.124 & 0.015 & 0.009 \\
        \hline\hline
    \end{tabular}
    \label{tab:metrics}
\end{table}

\subsection{Consistency Between Theory and Simulation}
The simulated final sizes confirm that both vaccination strategies push the system into the subcritical regime, with Random-75 slightly outperforming Targeted-10, as expected from the lower residual $q'$.  Crucially, the relative efficiency gain—herd immunity with roughly one‐sixth the coverage—is borne out quantitatively.

% ------------------------- Discussion ------------------------
\section{Discussion}
The analytical thresholds derived from degree moments accurately forecast the qualitative epidemiological regimes observed in fully stochastic network simulations.  While the targeted scheme requires knowledge of node degree, many social platforms readily expose follower counts or interaction frequencies, making such information practically accessible.  Even coarse targeting of the top‐decile accounts may therefore deliver a large return on investment relative to blanket interventions.

Our configuration‐model assumption deliberately removes clustering and degree–degree correlations, both of which can alter epidemic thresholds \cite{Doostmohammadian2023Flattening}.  In assortative or clustered networks the benefits of degree targeting can be amplified because high‐degree nodes tend to connect disproportionately to each other, forming super‐spreader cores.  Conversely, if high‐degree nodes mainly connect to low‐degree periphery nodes, random vaccination may become competitive.  Future work should examine these nuances using the message‐passing frameworks of \cite{Bianconi2020MessagePassing}.

Another limitation is perfect vaccine efficacy.  For digital interventions such as account bans or automated content filters the assumption is closer to reality than for biological vaccines, yet partial compliance and evasion tactics can reintroduce contagion pathways.  Extending the analysis to leaky or waning immunity is straightforward within the same percolation formalism \cite{BallSirl2013Acquaintance} and remains an important direction.

Finally, we fixed the infection seed fraction at 1\%.  Lower seeding would magnify stochastic fade‐out, especially in the Random-75 scenario, but would not change the threshold logic.  High seed levels, emulating coordinated meme campaigns, make targeted vaccination even more valuable because they remove the most prolific sources.

% ------------------------- Conclusion ------------------------
\section{Conclusion}
We analysed the vaccination coverage required to suppress a meme with reproduction number $\mathcal R_0 = 4$ on a mildly heterogeneous configuration‐model network.  Closed‐form calculations show that uniform random vaccination demands 75\% coverage, whereas immunising all degree‐10 nodes—only 12.5\% of the population—suffices.  Stochastic simulations with \texttt{FastGEMF} corroborate these figures, demonstrating dramatic reductions in peak prevalence and final epidemic size under both strategies, with degree targeting offering a six‐fold efficiency gain.  The study underscores the importance of network heterogeneity in designing resource‐efficient interventions against the viral spread of information.

% ------------------------- References -----------------------
\begin{thebibliography}{99}

\bibitem{Bianconi2020MessagePassing}
G.~Bianconi, H.~Sun, and G.~Rapisardi, ``A message‐passing approach to epidemic tracing and mitigation with apps,'' \emph{Phys. Rev. Research}, vol.~3, p. L012014, 2021.

\bibitem{BallSirl2013Acquaintance}
F.~Ball and D.~Sirl, ``Acquaintance vaccination in an epidemic on a random graph with specified degree distribution,'' \emph{J. Appl. Probab.}, vol.~50, no.~4, pp. 1147--1168, 2013.

\bibitem{Doostmohammadian2023Flattening}
M.~Doostmohammadian and H.~Rabiee, ``Network‐based control of epidemics via flattening the infection curve: High‐clustered vs. low‐clustered social networks,'' \emph{Social Network Analysis and Mining}, vol.~13, no.~1, p.~62, 2023.

\bibitem{Wang2015Predicting}
W.~Wang, W.~Wang, Q.~H. Liu, et~al., ``Predicting the epidemic threshold of the susceptible‐infected‐recovered model,'' \emph{Scientific Reports}, vol.~6, p. 24676, 2016.

\end{thebibliography}

% ------------------------- Appendix -------------------------
\appendices
\section{Python Code Listings}
The repository accompanying this article includes the following scripts saved under the \texttt{output} directory:
\begin{itemize}
    \item \texttt{network\hyphen{}construct\hyphen{}code.py}: generates the configuration‐model graph and exports it as \texttt{network.npz}.
    \item \texttt{simulation\hyphen{}10.py}: implements the SIR dynamics and vaccination scenarios using \texttt{FastGEMF}.
\end{itemize}
Interested readers may reproduce all experiments by running these scripts with Python~3.10 and the dependencies listed in the repository README.

\end{document}