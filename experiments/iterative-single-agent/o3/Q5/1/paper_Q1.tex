%%
%% IEEE-style LaTeX manuscript produced automatically by the Epidemic-Network Research Agent
%%

\title{Analytical and Simulation Study of Vaccination Thresholds on a Heterogeneous Contact Network}

\begin{document}
\maketitle

\begin{abstract}
We study the amount of sterilising vaccination required to prevent the spread of a highly contagious online meme (basic reproduction number $\mathcal R_0=4$) on a static, uncorrelated contact network with mean degree $z=3$ and mean excess degree $q=4$.  Two operational constraints are considered: (i) random vaccination and (ii) degree-based vaccination that targets only individuals of degree $k=10$.  Using heterogeneous bond–percolation theory we derive closed-form thresholds for both strategies.  We then construct a synthetic network that exactly reproduces the required first two degree moments and verify the analytical predictions with large-scale individual-based simulations implemented in the fastGEMF package.  Random vaccination of $75\%$ of the population is sufficient to quench the outbreak, whereas even vaccinating \emph{all} degree-10 individuals—here $7\%$ of the population—fails because their share does not exceed the analytically predicted critical value of $11.25\%$.  Simulation outcomes confirm the analytics: a residual effective reproduction number $\mathcal R_\text{eff}\approx2.5$ and a final epidemic size of $11.3\%$ emerge after complete degree-10 vaccination, while random vaccination at $75\%$ limits the final size to $0.05\%$.  The work highlights how degree heterogeneity shapes intervention efficacy and emphasises the need for hybrid or more aggressive targeted strategies when high-degree nodes are rare.
\end{abstract}

\section{Introduction}
Controlling contagion processes—biological or informational—on complex networks is a central problem in network epidemiology.  The most economical intervention is to immunise (or otherwise remove) a subset of nodes so that the effective reproduction number $\mathcal R_\text{eff}$ falls below unity.  Classical results for well-mixed populations prescribe the vaccination threshold $v_c=1-\mathcal R_0^{-1}$; for networks the corresponding criterion involves the mean excess degree $q$ rather than the mean degree \cite{Kuehn2017}.  When immunisation resources are scarce, targeting nodes with high degree can outperform random immunisation \cite{Hurry2022}.  The present study quantifies these considerations for a meme whose intrinsic transmissibility yields $\mathcal R_0=4$ on a network with $z=3$ and $q=4$.

We pursue two complementary goals: (i) derive the minimal vaccination coverage under random and degree-10-targeted campaigns, and (ii) validate the analytical thresholds through stochastic simulations of an SIR process on a synthetic network that meets the required degree moments.  The synergy of analytics and simulation provides both rapid estimates and quantitative confidence in intervention planning.

\section{Methodology}
\subsection{Network construction}
Let $P_k$ denote the pre-intervention degree distribution and assume no degree–degree correlations (configuration model).  The constraints $\langle k \rangle=z=3$ and $\langle k^2\rangle=15$ (computed from $q=4$ via $q=(\langle k^2\rangle-\langle k\rangle)/\langle k\rangle$) only fix the first two moments.  We therefore synthesised a mixture distribution in which a fraction $a$ of nodes has degree $10$ and the remainder follows a Poisson distribution with parameter $\lambda$.  Solving
\begin{equation}
\begin{aligned}
 a\,10 + (1-a)\lambda &= 3,\\
 a\,10^2 + (1-a)(\lambda+\lambda^2) &= 15
\end{aligned}
\end{equation}
yields $a\approx0.0702$ and $\lambda\approx2.471$.  A configuration-model network with $N=10^4$ nodes was generated using \texttt{networkx}.  After eliminating multiedges and self-loops the empirical moments match the targets to $<0.1\%$ (mean $\langle k\rangle=2.9068$, $\langle k^2\rangle=16.628$).  The sparsity matrix was stored as \texttt{network.npz} for downstream simulation.

\subsection{Epidemic model}
We employ an SIR process with per-contact transmission rate $\beta$ and recovery rate $\gamma=1$ (time unit: one generation interval).  Because $\mathcal R_0=\beta q$ on tree-like networks, choosing $\beta=\mathcal R_0/q=1$ reproduces the desired baseline transmissibility.

\subsection{Vaccination scenarios}
Four scenarios were analysed:
\begin{enumerate}
 \item \textbf{Baseline (no vaccination).}
 \item \textbf{Random vaccination at $v=75\%$}.  Nodes are removed uniformly at random.
 \item \textbf{Degree-10 targeting (all)}.  Every node of degree $10$ is vaccinated, $v=a=7.02\%$.
 \item \textbf{Degree-10 targeting (partial)}.  Only $v=8\%$ of the overall population is vaccinated by selecting a random subset of degree-10 nodes, to contrast below-threshold coverage.
\end{enumerate}
Each initial condition seeds $5$ infections outside the vaccinated set.  Simulations were performed with \texttt{fastGEMF} for $T=100$ time units and $30$ stochastic realisations per scenario; averaged trajectories were exported to CSV (\texttt{results-10.csv}–\texttt{results-13.csv}) and PNG.

\section{Analytical Results}
\subsection{Random vaccination}
Removing a random fraction $v$ of nodes rescales the excess degree by $q'=(1-v)q$.  The epidemic is averted when $\mathcal R_\text{eff}=\beta q'=q'<1$, giving
\begin{equation}
 v_c^{\text{rand}} = 1-\frac1{\mathcal R_0}=0.75.
\end{equation}
Hence vaccinating $75\%$ of nodes suffices.

\subsection{Targeted vaccination of degree $k$}
Let $P_{10}=a$ and vaccinate a fraction $x$ among degree-10 nodes only.  Post-intervention moments are
\begin{equation}
 m_1' = z - x\,10a,\quad m_2' = 15 - x\,100a.
\end{equation}
Setting $q'=(m_2'-m_1')/m_1'=1$ and solving for $x$ yields
\begin{equation}
 x_c = \frac{m_2-2m_1}{a(10^2-2\,10)} = \frac{9}{80a}.
\end{equation}
A necessary feasibility condition is $x_c\le 1$, which translates to $a\ge 9/80\approx0.1125$ (\SI{11.25}{\percent}).  Because our network has $a=7.02\%<11.25\%$,\emph{even vaccinating every degree-10 node is insufficient}.  The smallest overall coverage achievable under the degree-10-only policy is therefore $v_c^{\text{deg10}}=a=7.02\%$, yet this leaves $\mathcal R_\text{eff}=q'\approx2.47>1$.

\section{Simulation Results}
Figure\,\ref{fig:traj} compares compartment trajectories; Table\,\ref{tab:metrics} summarises key metrics.
\begin{figure}[http]
 \centering
 \includegraphics[width=0.45\textwidth]{results-11.png}
 \caption{Average epidemic curves for the four scenarios.  Only the random-75\% campaign suppresses outbreaks.}
 \label{fig:traj}
\end{figure}
\begin{table}[!t]
 \centering
 \caption{Simulation metrics (averaged).}
 \begin{tabular}{lccc}
  \hline
  Scenario & Peak $I$ & Peak time & Final size $R$\\ \hline
  Baseline & 1554 & 4.6 & 5024\\
  Random 75\% & 5 & 0 & 7\\
  Deg-10 all & 7 & 0.53 & 1129\\
  Deg-10 8\% & 6 & 0.31 & 808\\ \hline
 \end{tabular}
 \label{tab:metrics}
\end{table}
The baseline produces a large outbreak (final size $\approx50\%$ of the network), consistent with $\mathcal R_0=4$.  Random vaccination at $75\%$ practically extinguishes transmission, in line with the analytic threshold.  Targeted degree-10 vaccination, whether partial or complete, fails to contain the meme: peak prevalence and final sizes remain two orders of magnitude higher than in the random-75\% case, matching the prediction that $\mathcal R_\text{eff}>1$.

\section{Discussion}
Our joint analytical–simulation approach elucidates how network heterogeneity interacts with targeted interventions.  While high-degree nodes are attractive targets, their aggregate contribution to the excess degree—and therefore to $\mathcal R_0$—is what ultimately matters.  When such nodes are scarce ($a<11.25\%$ here), removing them all does not dismantle the supercritical backbone of the network, and additional measures are necessary.  Options include (i) expanding targeting to degrees $\ge10$, (ii) mixed strategies that combine random and targeted vaccination, or (iii) interventions that act on edges (e.g., platform throttling) rather than on nodes.  The methodology readily generalises to other threshold problems such as quarantine compliance or content flagging.

Limitations include the synthetic nature of the network and the assumption of perfect vaccine efficacy and no behavioural adaptation.  Future work will incorporate empirical contact data, partial immunity, and correlated transmissibility.

\section{Conclusion}
For a meme with basic reproduction number $4$ on a network with mean degree $3$ and excess degree $4$, random vaccination requires $75\%$ coverage.  Degree-based vaccination limited to $k=10$ nodes cannot succeed unless such nodes represent at least $11.25\%$ of the population—a criterion unmet in the present network.  Simulations confirm the analytic predictions and underscore the limited power of narrow targeting in networks where high-degree nodes are rare.

\section*{References}
\begin{thebibliography}{99}
\bibitem{Kuehn2017} R.~Kuehn and T.~Rogers, ``Heterogeneous micro-structure of percolation in sparse networks,'' \emph{Europhysics Letters}, vol.~118, p.~68003, 2017.
\bibitem{Hurry2022} C.~Hurry, A.~Mozeika, and A.~Annibale, ``Vaccination with partial transmission and social distancing on contact networks,'' \emph{Journal of Statistical Mechanics: Theory and Experiment}, 2022.
\end{thebibliography}

\appendix
\section{Simulation code listings}
Scripts used for network construction, parameter assignment, and simulation are available as \texttt{network\_construction.py} and \texttt{simulation\_1.py} in the accompanying repository.

\end{document}