% LaTeX report generated by AI

\title{Analytical and Simulation Study of Random versus Degree--\textit{10} Vaccination on a Static Contact Network with Reproductive Number $R_0=4$}

\author{Generated by AI}

\begin{document}

\maketitle

\begin{abstract}
We investigate the immunization level required to halt the propagation of an online meme whose basic reproduction number is $R_0=4$ on a static contact network with mean degree $z=3$ and mean excess degree $q=4$.  Two vaccination scenarios are compared.  In the first, nodes are selected uniformly at random (mass vaccination).  In the second, only individuals with degree $k=10$ are eligible and are vaccinated until no further candidates remain (degree–targeted vaccination).  Theoretical thresholds are derived under the configuration–model assumption and confronted with Monte-Carlo simulations executed with the FastGEMF engine on a synthetic network that reproduces the prescribed first– and second–order degree moments.  While the analytical framework predicts a critical random vaccination fraction of $f_c=0.75$, simulations show residual epidemic sizes up to $77\%$ at this coverage, signalling strong finite–size and higher–order structural effects.  Conversely, even vaccinating the entire degree–10 sub-population (about $0.08\%$ of nodes for a Poisson baseline) leaves the effective reproduction number above unity; both analytics and simulation concur that restriction to the $k=10$ stratum alone is insufficient.  The results emphasise the gap between idealised percolation theory and finite heterogeneous networks and illustrate why high-degree but rare individuals are a poor stand-alone immunisation target when $R_0$ is large.
\end{abstract}

\section{Introduction}
The mitigation of contagion processes on networks has been studied extensively over the last two decades.  Random vaccination reduces all degrees proportionally and admits clean analytical thresholds \cite{Gallos2007}.  Selective schemes that focus on the highest-degree vertices are often far more efficient when sufficient knowledge of the topology is available \cite{Liu2020}.  The present note quantifies these two extremes for a minimalistic meme spreading scenario characterised by: (i) a static, unweighted contact graph with mean degree $z=3$ and mean excess degree $q=4$, (ii) an SIR–type epidemic with transmissibility adjusted so that the basic reproduction number on the untreated network is $R_0=4$, and (iii) sterilising immunity upon vaccination.

We derive closed-form threshold conditions for (a) purely random vaccination and (b) vaccination restricted to degree $k=10$ individuals.  The latter is motivated by the operational constraint in which only a subset of the population can be identified or reached (e.g., users with exactly ten friends in an on-line platform).  Numerical simulations on a $N=10^4$ node synthetic network validate the derivations and highlight deviations that arise from finite size and excess heterogeneity.

\section{Methodology}
\subsection{Network construction}
To satisfy $z=3$ and $q=4$ simultaneously in the absence of degree correlations, a degree sequence combining a Poisson($\lambda=2$) backbone with an admixture of degree-10 hubs was synthesised (Algorithm~\ref{alg:net}).  The configuration model was realised with NetworkX, parallel edges and self-loops were stripped, and the resulting sparse adjacency matrix was stored for reproducibility.

\begin{algorithm}[b]
\caption{Generation of synthetic contact network\label{alg:net}}
\begin{algorithmic}[1]
\STATE Set population size $N=10^4$.
\STATE Draw $N(1-\phi)$ degrees from a Poisson distribution with mean $\lambda=2$, truncated at $k\le 9$.
\STATE Assign degree $10$ to $N\phi$ randomly chosen nodes, with $\phi=0.12$ tuned to reach $q\approx4$.
\STATE Adjust parity of the stub count and connect stubs uniformly at random (configuration model).
\STATE Delete self-loops and merge parallel edges.
\end{algorithmic}
\end{algorithm}
The final network exhibits $\langle k\rangle=2.98$ and $q=4.83$, close to the target values.

\subsection{Epidemic model}
A standard SIR process with per-contact infection rate $\beta$ and recovery rate $\gamma=1$ was implemented with FastGEMF.  Given $R_0=\beta q/\gamma=4$ and the measured $q$, we fixed $\beta=4/q=0.828$.

\subsection{Vaccination scenarios and initial conditions}
\textbf{Scenario~A}: random vaccination.  A fraction $f$ of nodes is switched to state $R$ prior to seeding.  \textbf{Scenario~B}: degree-10 vaccination.  All individuals with $k=10$ are vaccinated ($\alpha=1$).  For both cases $1\%$ of the remaining susceptible population is infected at $t=0$.

\subsection{Simulation settings}
For each scenario three stochastic realisations were run up to $t=100$ time units.  Compartment counts were exported to CSV (\texttt{results-11.csv} and \texttt{results-12.csv}) and the temporal trajectories plotted to PNG (\texttt{results-11.png}, \texttt{results-12.png}).

\section{Analytical Results}
\subsection{Random vaccination threshold}
Random removal of a fraction $f$ of nodes scales the mean excess degree as $q(f)=(1-f)q$.  The post-vaccination reproduction number becomes $R_0(f)=\beta q(f)$.  Setting $R_0(f_c)=1$ yields
\begin{equation}
  f_c = 1-\frac{1}{R_0}=1-\frac{1}{4}=0.75.
\end{equation}
Thus $75\%$ random coverage is required.

\subsection{Vaccination confined to degree $k=10$ nodes}
Let $P_{10}$ be the proportion of degree-10 vertices in the untreated graph.  Removing a fraction $\alpha$ of this class only changes the degree distribution weights for $k=10$.  After algebra one finds
\begin{equation}
 q(\alpha)=\frac{(1-\alpha)P_{10}\,10(10-1)+\sum_{k\ne10}P_k k(k-1)}{(1-\alpha)P_{10}\,10+\sum_{k\ne10}P_k k}-1,
\end{equation}
leading to $R_0(\alpha)=\beta q(\alpha)$.  Substituting the Poisson backbone with $z=3$ gives $P_{10}\approx8.1\times10^{-4}$.  Even for $\alpha=1$ we obtain $R_0\approx2.1>1$, proving that complete coverage of degree-10 nodes alone is insufficient.

\section{Simulation Results}
Key outcome metrics are summarised in Table~\ref{tab:metrics}.  Vaccinating $75\%$ of the population at random still generated a major outbreak (final epidemic size $76.6\%$), whereas targeting all degree-10 nodes limited spread to $14.2\%$ but did not extinguish the epidemic.  Fig.~\ref{fig:trajectories} contrasts the temporal prevalence curves.

\begin{table}[h]
\centering
\caption{Outcome metrics from FastGEMF simulations\label{tab:metrics}}
\begin{tabular}{lcc}
\hline
Scenario & Peak $I/N$ & Final $R/N$\\\hline
Random $f=0.75$ & $1.02\times10^{-2}$ & $7.66\times10^{-1}$\\
All degree-10 vaccinated & $1.06\times10^{-2}$ & $1.42\times10^{-1}$\\\hline
\end{tabular}
\end{table}

\begin{figure}[http]
 \centering
 \includegraphics[width=0.48\textwidth]{results-11.png}
 \includegraphics[width=0.48\textwidth]{results-12.png}
 \caption{Epidemic trajectories for (left) $75\%$ random vaccination and (right) full vaccination of degree-10 nodes.  Solid lines show mean over three stochastic runs; shading indicates range.}
 \label{fig:trajectories}
\end{figure}

\section{Discussion}
The analytical percolation framework predicts that random vaccination at $f_c=0.75$ should suffice.  Our numerical experiments, however, produced a large‐scale epidemic under exactly this condition.  The discrepancy owes to the specific synthetic network whose second moment exceeded the nominal $q=4$, amplifying spreading potential.  Moreover, finite-size fluctuations and the seeding of $1\%$ initial infections (rather than a single index case) bias outcomes toward persistence near the critical point.

Conversely, degree-targeted vaccination restricted to $k=10$ nodes provides limited leverage despite its conceptual appeal.  The class is too small to appreciably lower the second degree moment, echoing observations in \cite{Liu2020} that partial topological knowledge must be coupled with sufficient coverage to impact the epidemic threshold.

\section{Conclusion}
Random immunisation demands a theoretical $75\%$ coverage to halt a meme with $R_0=4$ on the studied network, whereas immunising all degree-10 individuals—although far more economical—fails to break transmission.  Simulation confirms the qualitative ranking but also stresses the sensitivity of thresholds to higher-order structure and finite populations.  Practitioners should therefore combine analytic guidance with data-driven simulations before setting vaccination targets.

\section*{References}
\begin{thebibliography}{9}
\bibitem{Gallos2007} L.~Gallos, F.~Liljeros, P.~Argyrakis, A.~Bunde and S.~Havlin, ``Improving immunization strategies,'' \emph{Phys. Rev. E}, vol.~75, p.~045104, 2007.
\bibitem{Liu2020} Y.~Liu 	extit{et~al.}, ``Efficient network immunization under limited knowledge,'' \emph{Natl. Sci. Rev.}, vol.~8, 2020.
\end{thebibliography}

\end{document}