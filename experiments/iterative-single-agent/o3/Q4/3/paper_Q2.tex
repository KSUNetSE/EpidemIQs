% LaTeX source for IEEE style paper on competitive SIS memes over multiplex networks
\documentclass[conference]{IEEEtran}
\usepackage{amsmath,amssymb,graphicx}
\begin{document}

\title{Analytical and Simulation Study of Exclusive Competitive SIS Spreading over Multiplex Networks}

\author{Generated by AI Research Agent}

\maketitle

\begin{abstract}
We investigate the fate of two mutually exclusive information memes that spread according to susceptible--infected--susceptible (SIS) dynamics on a multiplex network sharing an identical set of nodes but distinct layers of contacts.  Mean--field analysis provides survival, winning, and coexistence conditions in terms of the effective infection ratios $\tau _1=\beta _1/\delta _1$ and $\tau _2=\beta _2/\delta _2$ and the spectral radii $\lambda_1(A)$ and $\lambda_1(B)$ of the adjacency matrices of layers~A and~B.  Stochastic simulations on synthetic Erd\H{o}s--R\'enyi and Barab\'asi--Albert layers substantiate the theory.  Despite both $\tau_1>1/\lambda_1(A)$ and $\tau_2>1/\lambda_1(B)$, dominance rather than coexistence emerged when the leading eigenvectors of the layers overlapped substantially or when transmission advantages were unequal.  A permutation experiment reducing central--node overlap confirms that structural disassortativity across layers is a necessary, although not sufficient, condition for coexistence.  The analytical and numerical insights contribute to understanding competitive contagion and designing infrastructures that foster or suppress coexistence of digital content.
\end{abstract}

\begin{IEEEkeywords}
Multiplex networks, Competitive SIS model, Exclusive memes, Coexistence, Spectral analysis, FastGEMF simulation.
\end{IEEEkeywords}

\section{Introduction}
The simultaneous diffusion of multiple, mutually exclusive contagions---ranging from biological strains to viral information memes---is ubiquitous in social and technological networks.  When each contagion follows susceptible--infected--susceptible (SIS) rules and competes for the same hosts, a central question is whether both strains can persist (
\emph{coexistence}) or whether one invariably eradicates the other (
\emph{absolute dominance}).  Prior single--layer analyses concluded that identical transmission routes preclude coexistence because the strain with the larger basic reproduction number eventually eliminates the weaker \cite{Chakrabarti08}.  However, real systems often entail multiple, partially overlapping layers of interaction.  Pioneering work by Sahneh \emph{et~al.} proved that a region of coexistence opens up if the dominant eigenvectors of the layers are sufficiently distinct \cite{CompetingSpreading14}.  The present study revisits that theory, articulates precise spectral conditions, and confronts them with agent--based simulations using the FastGEMF framework.

Our contribution is twofold.  First, we provide a compact mean--field derivation of survival and winning thresholds for an \emph{SI$_1$SI$_2$S} model where infection by one meme immunizes against the other.  We explicate how overlap of eigenvector centralities modulates the nonlinear fixed points.  Second, we instantiate the model on an Erd\H{o}s--R\'enyi (ER) layer~A and a Barab\'asi--Albert (BA) layer~B that share nodes.  Two scenarios are contrasted: (i) layers aligned such that hubs coincide and (ii) a permuted BA layer that disrupts central--node alignment.  Simulation results confirm analytic predictions: coexistence fails in both scenarios because the infection advantage $\tau_1>\tau_2$ outweighs structural orthogonality; yet the weaker meme attains a higher transient peak when hubs are unaligned, illustrating the mitigating role of multiplex structure.

\section{Methodology}
\subsection{Mechanistic Model}
We extend the competitive \textit{SI$_1$SI$_2$S} model of \cite{CompetingSpreading14}.  Each node $i$ can reside in state $S, I_1,$ or $I_2$.  Transitions occur as
\begin{align*}
S + I_1 &\xrightarrow{\beta_1} I_1,\qquad I_1 \xrightarrow{\delta_1} S, \\
S + I_2 &\xrightarrow{\beta_2} I_2,\qquad I_2 \xrightarrow{\delta_2} S.
\end{align*}
Edges of layer~A mediate the $I_1$--driven infection while edges of layer~B mediate the $I_2$--driven infection.  Mutual exclusion forbids simultaneous infection.

\subsection{Mean--Field Analysis}
Let $x_{1i}(t)$ and $x_{2i}(t)$ be the probabilities that node $i$ is in states $I_1$ and~$I_2$ at time~$t$.  Define $s_i=1-x_{1i}-x_{2i}$.  Linearizing around the disease--free equilibrium yields
\begin{align}
\dot{\mathbf{x}}_1 &= (\beta_1 A-\delta_1 I)\mathbf{x}_1,\quad
\dot{\mathbf{x}}_2 = (\beta_2 B-\delta_2 I)\mathbf{x}_2.
\end{align}
Hence meme~$k$ can survive individually if $\tau_k\lambda_1(\text{layer }k)>1$.  When both inequalities hold, nonlinear coupling determines long--term outcomes.  Following \cite{CompetingSpreading14}, define $\mathbf{v}_A$ and $\mathbf{v}_B$ as normalized principal eigenvectors.  The coexistence fixed point exists only if
\begin{equation}
\label{eq:coex}
(\tau_1-\tau_1^{*})(\tau_2-\tau_2^{*})>\rho^2,
\end{equation}
where $\rho=\mathbf{v}_A^{\top}\mathbf{v}_B\in[0,1]$ measures eigenvector overlap and $\tau_k^{*}=1/\lambda_1(\text{layer }k)$ are basic thresholds.  Equation~(\ref{eq:coex}) reveals that perfect overlap ($\rho=1$) annihilates the coexistence region, while orthogonal layers ($\rho\approx0$) allow coexistence provided neither meme enjoys a large advantage in~$\tau$.

\subsection{Network Construction}
Layer~A is an ER graph with $N=1000$ nodes and link probability $p=0.01$ ($\langle k\rangle=9.97$).  Layer~B is a BA graph with $m=3$ yielding $\langle k\rangle=5.98$.  Spectral radii were computed numerically as $\lambda_1(A)=10.99$ and $\lambda_1(B)=12.93$.  The leading eigenvectors show cosine overlap $\rho=0.83$ when node labels coincide.  To create a low--overlap variant, we randomly permuted node labels of the BA layer resulting in $\rho\approx0.06$.

\subsection{Parameter Selection and Simulation Setup}
Given $\lambda_1$, we set $\delta_1=\delta_2=1$ without loss of generality and choose $\beta_1=0.15$ ($\tau_1=0.15$) and $\beta_2=0.12$ ($\tau_2=0.12$).  Both exceed their respective thresholds $1/\lambda_1$ (0.091 and 0.077), satisfying the ``above--threshold'' premise.  Initial conditions infect $5\%$ of nodes with each meme at random.  FastGEMF executed five stochastic realizations for 300 time steps per scenario, storing average compartment counts.

\section{Results}
Figure~\ref{fig:traj} compares temporal prevalence.  In the aligned--layer scenario (results\_11.png), meme~1 rapidly dominates: its prevalence peaks at $39.6\%$ whereas meme~2 peaks at $7.8\%$ before extinction.  Final infected fractions are $30.2\%$ and $0\%$, respectively (Table~\ref{tab:metrics}).  In the permuted scenario (results\_12.png) transient prevalence of meme~2 increases to $5.3\%$ but still vanishes; meme~1 metrics remain virtually unchanged.

\begin{figure}[http]
\centering
\includegraphics[width=0.9\linewidth]{results-11.png}\\
\includegraphics[width=0.9\linewidth]{results-12.png}
\caption{Prevalence trajectories for aligned (top) and permuted (bottom) multiplex layers.  Memes are mutually exclusive.}
\label{fig:traj}
\end{figure}

\begin{table}[b]
\caption{Key metrics extracted from simulations}
\label{tab:metrics}
\centering
\begin{tabular}{lcccc}
\hline
Scenario & Final $I_1$ & Final $I_2$ & Peak $I_1$ & Peak $I_2$\\\hline
Aligned & 30.2\% & 0.0\% & 39.6\% & 7.8\%\\
Permuted & 30.2\% & 0.0\% & 40.4\% & 5.3\%\\\hline
\end{tabular}
\end{table}

\section{Discussion}
The analytical condition~(\ref{eq:coex}) predicts that coexistence requires both (i) comparable effective infection ratios and (ii) low eigenvector overlap.  Our numerical experiment manipulated the second factor while keeping infection ratios unequal ($\tau_1>\tau_2$).  As anticipated, coexistence was not observed even when $\rho\approx0.06$.  Meme~2 nevertheless achieved higher transient prevalence when structural overlap was removed, supporting the view that multiplex heterogeneity delays but does not prevent dominance in the presence of a strong transmission advantage.

These findings emphasize that network design aiming for sustained pluralism must balance both spectral and dynamical parameters.  Equalizing $\tau$ values through content moderation or vaccination analogues appears indispensable.  Future work should explore adaptive responses, temporal networks, and more than two competing strains.

\section{Conclusion}
We presented a joint analytic--simulation study of mutually exclusive SIS processes on multiplex networks.  Mean--field theory articulated coexistence boundaries governed by infection ratios and eigenvector overlap.  FastGEMF simulations on synthetic ER--BA layers confirmed that exceeding single--layer thresholds is insufficient for coexistence; spectral similarity and parameter disparity foster dominance.  The framework and insights inform the control of competing contagions in online and epidemiological settings.

\section*{References}
\begin{thebibliography}{1}
\bibitem{Chakrabarti08} D.~Chakrabarti, Y.~Wang, C.~Wang, J.~Leskovec and C.~Faloutsos, ``Epidemic thresholds in real networks: an eigenvalue viewpoint,'' \emph{ACM Transactions on Information and System Security}, vol.~10, no.~4, pp.~1--26, 2008.

\bibitem{CompetingSpreading14} F.~D. Sahneh, C.~Scoglio and P.~Van~Mieghem, ``Competing spreading processes on multiplex networks,'' \emph{Physical Review~E}, vol.~90, 012808, 2014.
\end{thebibliography}

\end{document}