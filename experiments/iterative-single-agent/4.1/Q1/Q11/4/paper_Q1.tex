\documentclass[10pt,conference]{IEEEtran}
\usepackage{graphicx}
\usepackage{amsmath}
\usepackage{multirow}
\usepackage{hyperref}
\usepackage{booktabs}

% Title
\title{Impact of Degree-Heterogeneous Network Structure on SEIR Disease Dynamics: Deterministic and Stochastic Perspectives}
\author{Anonymous}

\begin{document}

\maketitle

% Abstract
\begin{abstract}
Understanding the effect of network structural heterogeneity on epidemic outcomes is critical for effective epidemic control. We investigate how incorporating degree-heterogeneous contact structures into the SEIR (Susceptible-Exposed-Infectious-Recovered) model alters epidemic dynamics relative to a homogeneous-mixing assumption, comparing both deterministic analytical thresholds and stochastic network-based simulation results. We construct and simulate the SEIR process on two archetypal static networks of identical population size and similar mean degree: (i) an Erdős–Rényi random graph (homogeneous), and (ii) a configuration model with a power-law degree distribution (heterogeneous). Analyses focus on epidemic size, peak prevalence, timing, and duration. We supplement simulation insights with a review of relevant analytical theory from the literature, highlighting the reduced epidemic threshold and more nuanced transmission characteristics in degree-heterogeneous settings. Simulation findings demonstrate stark contrasts between outbreak likelihood, peak size, and final epidemic magnitude, emphasizing the implications for intervention policies.
\end{abstract}

% Introduction
\section{Introduction}
Mathematical modeling of epidemic dynamics is foundational for public health planning. Classical compartmental models, such as SEIR, traditionally assume homogeneous mixing—where each individual is equally likely to contact any other. However, real-world populations exhibit heterogeneous contact networks, often with heavy-tailed (e.g., power-law) degree distributions. Substantial research has explored how such degree heterogeneity alters key epidemic features, including the basic reproduction number ($R_0$), threshold conditions, outbreak probability, and final epidemic size \cite{Aletti2021,BoSong2025}.

Here, we investigate the effect of degree-heterogeneous static network structure on disease spread by contrasting SEIR model dynamics on Erdős–Rényi (ER, homogeneous) and power-law degree (configuration) networks. Specifically, we address: (1) How do deterministic threshold predictions differ between homogeneous and heterogeneous settings? (2) How does observed epidemic severity and trajectory differ in stochastic simulations? (3) What quantitative or qualitative implications for interventions arise from these models?

\section{Methodology}
\subsection{Network Construction}
Two undirected static networks, each with $N=2500$ nodes and mean degree approximately $\langle k \rangle \approx 8$, were synthesized. The first is an Erdős–Rényi graph ($G_{ER}$), modeling homogeneous mixing. The second is a configuration model network ($G_{PL}$) with a power-law degree sequence ($P(k) \sim k^{-2.7}$, minimum degree $2$, maximum $50$), representing high degree heterogeneity. Degree distributions are shown in Fig.~\ref{fig:degree-dist}. Network metrics: $\langle k \rangle_{ER}=7.95$, $\langle k^2 \rangle_{ER}=70.92$; $\langle k \rangle_{PL}=2.36$, $\langle k^2 \rangle_{PL}=10.66$.

\begin{figure}[htb]
    \centerline{\includegraphics[width=0.48\textwidth]{output/degree-dist.png}}
    \caption{Degree distributions for ER (homogeneous) and Configuration model (power-law) networks. Note the heavy tail and abundance of high-degree nodes in the power-law network.}
    \label{fig:degree-dist}
\end{figure}

\subsection{SEIR Model and Parameterization}
The SEIR compartmental model tracks susceptibles (S), exposed (E), infectious (I), and recovered (R) individuals. We simulate transitions: $S + I \xrightarrow{\beta}$ $E + I$ (network-mediated), $E \xrightarrow{\sigma} I$, $I \xrightarrow{\gamma} R$ with rates:
\begin{itemize}
    \item Latent period $=3$ days ($\sigma = 1/3$), infectious period $=5$ days ($\gamma = 1/5$)
    \item Target $R_0 = 2.5$, matched on both networks via $\beta_{net} = R_0 \cdot \gamma / q$, where $q=(\langle k^2 \rangle - \langle k \rangle)/\langle k \rangle$
\end{itemize}
For $G_{ER}$, $\beta_{ER}=0.0631$; for $G_{PL}$, $\beta_{PL}=0.1422$. Initial conditions: $10$ randomly chosen exposed ($E$); all others susceptible.

\subsection{Deterministic Analytical Framework}
Classical deterministic SEIR models produce identical progression on both network types if mean degree and $R_0$ are matched. However, analytical network theory modifies epidemic thresholds $\beta_c$ in heterogeneous settings:\newline
\textbf{Homogeneous (mean-field):} $\beta_c = \gamma/R_0$, threshold at $R_0=1$\newline
\textbf{Heterogeneous (network):} $\beta_c = \gamma \langle k \rangle /( \langle k^2 \rangle - \langle k \rangle )$\newline
Thus, for heavy-tailed networks with large $\langle k^2 \rangle$, epidemic thresholds decrease, making outbreaks more likely and less intervention-effective \cite{Aletti2021}.

\subsection{Stochastic Simulation}
We used the fastGEMF simulator, initialized as above, running five stochastic realizations per network for $200$ days. Output: compartment populations and epidemic curves (see Results).

\section{Results}
\begin{figure}[htb]
    \centerline{\includegraphics[width=0.48\textwidth]{output/seir-compare.png}}
    \caption{SEIR epidemic curves: Infected (I) and Exposed (E) populations on ER and power-law (PL) networks.}
    \label{fig:seir-compare}
\end{figure}
\begin{table}[htb]
\centering
\caption{Key epidemic metrics comparing homogeneous (ER) and heterogeneous (Power-law) networks}
\begin{tabular}{lcccccc}
\toprule
Network & Peak I & Peak Time & Final Size & Duration & Infectious Area\\
\midrule
ER  & 289 & 52.2 & 1929 & 5776 & 9569.2 \\
PL  & 39 & 50.1 & 191 & 560 & 898.7 \\
\bottomrule
\end{tabular}
\label{tab:metrics}
\end{table}
Key findings (Table~\ref{tab:metrics}): On the ER network, the epidemic produces a pronounced peak in infected ($\sim$289) and a large final size (1929). By contrast, on the power-law network, the epidemic is far less severe: the infected population peaks lower ($\sim$39) and total cases are much reduced (191).

\section{Discussion}
Our investigation—supported by both simulation and analytical literature—demonstrates that incorporating degree heterogeneity into epidemic models changes disease dynamics profoundly. Deterministic theory predicts lower epidemic thresholds for heterogeneous networks, suggesting greater susceptibility to outbreaks with the same parameters. However, stochastic network simulations here reveal that, with a fixed number of seeds and strong degree variability but modest mean degree, the disease may fail to reach most of the low-degree population. Super-spreaders (nodes with high degree) are critical: if not initially infected, stochastic fadeout can occur before extensive transmission. This contrasts with the mean-field scenario, where all individuals are equally likely to become infected, leading to larger, sharper outbreaks (higher peak, shorter duration).

Our analysis also highlights: (i) The configuration model’s larger variance in final size and more frequent fadeout, (ii) Final epidemic size remains much lower for this parameter regime, (iii) Implications for targeted interventions: focusing on highly connected nodes may be disproportionately effective.

Analytical studies—including \cite{Aletti2021,BoSong2025}—show that adding group or degree heterogeneity significantly decreases epidemic thresholds and alters control policy effectiveness. However, real-world stochastic effects may limit this if outbreaks fail to ignite among super-spreaders, as observed here. The gap between deterministic predictions and actual epidemic realization underscores the need for simulation-based risk assessments, particularly for heterogeneous populations.

\section{Conclusion}
Degree heterogeneity in contact networks substantially influences SEIR epidemic dynamics. Analytically, it reduces epidemic thresholds, implying higher risk and more challenging control. Stochastic simulations, however, reveal that unless outbreaks take hold in high-degree nodes, final epidemic sizes and peaks can be much smaller than homogeneously mixed models predict. For effective control and risk estimation, policy must balance both deterministic and stochastic insights, with special attention to the structure of contact heterogeneity.

\section*{References}
\begin{thebibliography}{99}

\bibitem{Aletti2021} G. Aletti, A. Benfenati, G. Naldi, "Graph, Spectra, Control and Epidemics: An Example with a SEIR Model," Mathematics, vol. 9, no. 22, 2021, DOI: 10.3390/math9222987.

\bibitem{BoSong2025} B. Song, L. Shi, Z. Ma, "An Assessment of Shipping Network Resilience Under the Epidemic Transmission Using a SEIR Model," Journal of Marine Science and Engineering, 2025, DOI: 10.3390/jmse13061166.

\bibitem{Kiouach2023} D. Kiouach, S. El Azami El‐idrissi, Y. Sabbar, "The impact of Lévy noise on the threshold dynamics of a stochastic susceptible‐vaccinated‐infected‐recovered epidemic model with general incidence functions," Math. Methods Appl. Sci., vol. 47, pp. 297–317, 2023, DOI: 10.1002/mma.9655.

\bibitem{Taylor2011} M. Taylor, T. Taylor, I. Kiss, "Epidemic threshold and control in a dynamic network," Phys. Rev. E, vol. 85, no. 1, p. 016103, 2011, DOI: 10.1103/PhysRevE.85.016103.

\bibitem{Boguna2013} M. Boguñá, C. Castellano, R. Pastor-Satorras, "Nature of the epidemic threshold for the susceptible-infected-susceptible dynamics in networks," Phys. Rev. Lett., vol. 111, no. 6, p. 068701, 2013, DOI: 10.1103/PhysRevLett.111.068701.

\bibitem{Kuga2022} K. Kuga, J. Tanimoto, "Effects of void nodes on epidemic spreads in networks," Scientific Reports, vol. 12, 2022, DOI: 10.1038/s41598-022-07985-9.

\end{thebibliography}

\section*{Appendix}
\subsection*{Additional Figures}
\begin{figure}[htb]
\centerline{\includegraphics[width=0.48\textwidth]{output/degree-dist.png}}
\caption{Degree distribution histograms for ER and power-law networks.}
\end{figure}
\begin{figure}[htb]
\centerline{\includegraphics[width=0.48\textwidth]{output/seir-compare.png}}
\caption{SEIR epidemic curves (Infected and Exposed) for both networks.}
\end{figure}

\subsection*{Code Availability}
The code and simulation data supporting these findings are available upon request or stored at the output directory referenced in the main text.
\end{document}
