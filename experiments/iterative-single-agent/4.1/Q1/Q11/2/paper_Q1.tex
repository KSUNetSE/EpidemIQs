\documentclass[10pt,twocolumn]{IEEEtran}
\usepackage{graphicx}
\usepackage{amsmath}
\usepackage{url}
\title{Epidemic Spread Analysis: SEIR Model on Degree-Heterogeneous versus Homogeneous Networks}
\author{AI Epidemic Modeling Group}

\begin{document}
\maketitle
\begin{abstract}
This paper investigates how degree-heterogeneous (scale-free) network structure alters the disease dynamics of the SEIR model compared to a homogeneous-mixing (Erdős–Rényi, ER) network. We model both scenarios with an identical basic reproduction number (R0), construct both network types with matched average degree, and use calibrated transmission/recovery rates. Stochastic simulations (fastgemf) and analytical arguments reveal profound differences in peak time, peak size, and final epidemic size between these contact topologies. Degree-heterogeneous (scale-free) networks yield smaller final epidemic sizes, delayed and reduced peaks, and longer epidemic duration for the same R0, a result that is explained both numerically and from theory. Implications for epidemic prediction and intervention are discussed.
\end{abstract}

\section{Introduction}
Modeling the spread of infectious diseases is vital for both fundamental epidemiology and public health planning. Traditional epidemic models, such as SEIR, frequently rely on the homogeneous mixing assumption—every individual is equally likely to contact every other individual\cite{Keeling2005,Eames2005}. However, real-world social, sexual, and mobility networks are highly heterogeneous, often exhibiting hubs or a heavy-tailed degree distribution\cite{Barabasi2003,MayLloyd2001}. Numerous studies have established that network topology can strongly influence threshold conditions, epidemic size, and epidemic predictability\cite{Newman2002,Nature19,SciencedirectSEIR}. Yet, most deterministic analyses and classic control strategies are derived for well-mixed models. This study aims to quantify and explain the consequences of incorporating degree heterogeneity into SEIR dynamics, using both analytical insight and stochastic simulation, to highlight major shifts in epidemic outcomes and implications for intervention strategies.

\section{Methodology}
\subsection{Model Overview and Rationale}
We compared SEIR dynamics on two network types: (a) homogeneous-mixing (modeled via Erdős–Rényi random networks), and (b) degree-heterogeneous scale-free networks (Barabási–Albert model). Both networks contained $N=2000$ nodes and matched in mean degree $\langle k \rangle \approx 12$. Simulations and parameterization were designed such that the basic reproduction number $R_0$ was identical in both cases ($R_0=2.5$) to isolate network structure effects.

\subsection{Network Construction}
For the homogeneous scenario, we generated an ER graph with connection probability chosen to yield $\langle k \rangle \approx 12$ ($p = 12/(N-1)$). For the heterogeneous scenario, a Barabási–Albert scale-free network with $m=6$ ensured a comparable average degree. Degree distributions were extracted and plotted (see Figs. \ref{fig:ERdegree} and \ref{fig:BAdegree}). The ER network exhibited a narrow Poisson-like degree distribution, while the BA graph showed a heavy-tailed pattern with few high-degree hubs.

Mean and second moment of the degree distribution were as follows:
 \begin{itemize}
   \item ER: $\langle k \rangle = 12.05$, $\langle k^2 \rangle = 157.33$
   \item BA: $\langle k \rangle = 11.96$, $\langle k^2 \rangle = 315.54$
 \end{itemize}

\begin{figure}[!ht]
  \centering
  \includegraphics[width=0.48\textwidth]{ER_degree_dist.png}
  \caption{Degree distribution of homogeneous-mixing (ER) network.}
  \label{fig:ERdegree}
\end{figure}
\begin{figure}[!ht]
  \centering
  \includegraphics[width=0.48\textwidth]{BA_degree_dist.png}
  \caption{Degree distribution of scale-free (BA) degree-heterogeneous network.}
  \label{fig:BAdegree}
\end{figure}

\subsection{SEIR Model and Parameter Calibration}
The SEIR network model comprised four compartments: Susceptible (S), Exposed (E), Infectious (I), and Recovered (R). The transitions were as follows:
\begin{align*}
\textrm{S-(I)$\rightarrow$E:} & \quad \beta \\
\textrm{E$\rightarrow$I:} & \quad \sigma = 0.25\ \textrm{day}^{-1} \\
\textrm{I$\rightarrow$R:} & \quad \gamma = 0.1667\ \textrm{day}^{-1}
\end{align*}
\noindent
$R_0$ for network SEIR is $\beta / \gamma \cdot q$, where $q = (\langle k^2 \rangle - \langle k \rangle)/\langle k \rangle$. We fixed $R_0=2.5$ and calculated $\beta$ separately for each network:
\begin{itemize}
\item ER: $\beta_{ER} = 0.03455$
\item BA: $\beta_{BA} = 0.01642$
\end{itemize}
Initial condition: $10$ infected, $1990$ susceptible.

\subsection{Stochastic Simulation Protocol}
Stochastic simulations (10 runs per scenario) were carried out using the fastgemf package, allowing realistic, discrete-event SEIR networked epidemic modeling. We saved the full epidemic trajectory in each network.

\section{Results}
Table~\ref{tab:results} summarizes key epidemic metrics from stochastic simulations:

\begin{table}[ht]
\centering
\caption{Epidemic outcome metrics for both networks}
\begin{tabular}{lcccc}
\hline
Network & Peak I & Peak Time & Final R & Duration \\
\hline
ER & 248 & 51.6 & 1630 & 129.1 \\
BA & 70 & 91.3 & 614 & 181.9 \\
\hline
\end{tabular}
\label{tab:results}
\end{table}

\begin{figure}[!ht]
  \centering
  \includegraphics[width=0.48\textwidth]{results-10.png}
  \caption{SEIR epidemic curves, homogeneous-mixing (ER) network.}
\end{figure}
\begin{figure}[!ht]
  \centering
  \includegraphics[width=0.48\textwidth]{results-11.png}
  \caption{SEIR epidemic curves, scale-free (BA) network.}
\end{figure}

\noindent
Salient findings:
\begin{itemize}
  \item The ER network (homogeneous) showed a much higher, earlier peak in the number of infected ($\sim$248 vs 70) and a final epidemic size nearly triple that of the BA.
  \item The BA (scale-free) network had a substantially longer epidemic duration, with a later, lower peak and much of the population never infected ($\sim$614 recovered vs $\sim$1630 in ER).
\end{itemize}

\section{Discussion}
Our results confirm and quantify the profound effect that degree heterogeneity exerts on SEIR epidemic dynamics, even when $R_0$ is matched across network structures:\cite{Nature19,SciencedirectSEIR}. For the scale-free BA network, the majority of nodes possess very few connections, while a small minority act as super-spreaders. This hub-driven structure paradoxically slows overall epidemic progress and causes most infections to be concentrated in, or radiate out from, these well-connected individuals\cite{Barabasi2003,MayLloyd2001}. The deterministic mean-field SEIR model (homogeneous) overestimates final epidemic size and peak prevalence when faced with real-world contact heterogeneity.

Analytically, this can be explained using network-based $R_0$ expressions; while the epidemic thresholds may be similar, the probability of stochastic fade-out and the concentration of vulnerability in hubs both increase in BA-like topologies. More of the community remains uninfected due to local saturation effects near the hubs early in the epidemic, and residual clusters are protected by bottlenecks in connectivity\cite{Keeling2005,Eames2005,Newman2002}.

These findings underscore the limitations of classical well-mixed epidemic models for prediction and intervention planning in modern social networks. The need for targeted interventions (e.g., vaccinating hubs or high-degree nodes) in degree-heterogeneous populations becomes clear\cite{Barabasi2003,MayLloyd2001,SciencedirectSEIR}.

\section{Conclusion}
Incorporating degree-heterogeneous structure in epidemic network models drastically alters SEIR outcomes. While homogeneous-mixing models predict rapid, population-wide outbreaks, scale-free networks with identical mean $R_0$ exhibit lower and delayed peaks, reduced final size, and longer persistence. These differences arise from the topological bottlenecks and over-dispersion of connectivity intrinsic to scale-free structures. Modeling efforts, as well as practical public health strategies, must therefore recognize and adapt to population contact heterogeneity for optimal prediction and mitigation.

\section*{References}
\begin{thebibliography}{10}

\bibitem{Keeling2005}
M. J. Keeling and K. T. D. Eames, "Networks and epidemic models," \emph{J. R. Soc. Interface}, 2, 295–307, 2005.

\bibitem{Barabasi2003}
A.-L. Barabási and E. Bonabeau, "Scale-Free Networks," \emph{Scientific American}, vol. 288, no. 5, pp. 60–69, 2003.

\bibitem{MayLloyd2001}
R. M. May and A. L. Lloyd, "Infection dynamics on scale-free networks," \emph{Phys. Rev. E}, 64, 066112, 2001.

\bibitem{Newman2002}
M. E. J. Newman, "Spread of epidemic disease on networks," \emph{Phys. Rev. E}, 66, 016128, 2002.

\bibitem{Nature19}
Motta, F. C., et al., "Comparing the effects of non-homogeneous mixing patterns on infectious disease dynamics," \emph{Scientific Reports}, 9, 2019.

\bibitem{Eames2005}
K. T. D. Eames and M. J. Keeling, "Models of Epidemic Management and Control on Networks," \emph{Lancet Infect Dis}, 5: 528–536, 2005.

\bibitem{SciencedirectSEIR}
Wang, Y., et al., "Epidemic spreading of an SEIRS model in scale-free networks," \emph{Mathematical and Computer Modelling}, 53.7-8: 1449–1460, 2011.

\end{thebibliography}

\end{document}
