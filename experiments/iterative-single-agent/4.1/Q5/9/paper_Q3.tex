\documentclass[10pt,conference]{IEEEtran}
\usepackage{graphicx}
\usepackage{amsmath,amssymb}
\title{Epidemic Spread Analysis of the SIR Model on Barabási-Albert Networks}

\begin{document}

\maketitle

\begin{abstract}
This work presents a thorough investigation into the spread of infectious diseases over static networks, focusing on the Susceptible-Infected-Recovered (SIR) model, with COVID-19 as a representative disease. Employing a Barabási-Albert (BA) scale-free network to mimic social contacts among individuals, we quantify disease dynamics using effective network-specific parameters. Extensive stochastic simulations generate population trajectories across model compartments, from which we extract key epidemic metrics such as peak infection, epidemic duration, and final outbreak size. This work aims to improve our mechanistic understanding of epidemic propagation in heterogeneous contact structures and offer insights for designing disease control strategies tailored to network topologies.
\end{abstract}

\section{Introduction}
The characterization of epidemic spread in populations with complex contact structures is a major challenge in mathematical epidemiology. Most classical models assume well-mixed populations, but real-world interpersonal interactions are best represented as networks with heterogeneous connectivity patterns \cite{PastorSatorras2015,Eubank2004}. Understanding how network architecture affects disease dynamics is crucial for accurate epidemic forecasting and the development of efficient intervention strategies \cite{Barrat2008,Noh2009}.

In this study, we focus on static network structures to isolate the impact of topology on disease spread, explicitly modeling the SIR dynamics relevant to airborne and droplet-transmitted diseases such as COVID-19. We select the Barabási-Albert model due to its resemblance to real social networks, capturing the hub structure and heavy-tailed degree distribution observed in human contact patterns \cite{Barabasi1999}.

Our goal is to quantify how epidemic metrics such as final outbreak size, peak infection, and epidemic duration are governed by underlying network properties and disease parameters. We perform detailed stochastic simulations to provide insights into the typical epidemic course within such heterogeneous populations.
\section{Methodology}
\subsection{Epidemic Scenario and Mechanistic Model}
We model the spread of a COVID-19-like disease using the SIR compartmental framework, which partitions the population into Susceptible ($S$), Infected ($I$), and Recovered ($R$) states. Transitions are governed by:
\begin{itemize}
    \item $S -(I) \rightarrow I$
    \item $I \rightarrow R$
\end{itemize}
The rates are denoted as $\beta$ (infection rate) and $\gamma$ (recovery rate). The network-driven dynamics require these to be tailored to the topology: for the Barabási-Albert network, we compute the infection probability per edge using $\beta = R_0\gamma/q$, where $q = (\langle k^2 \rangle - \langle k \rangle)/\langle k \rangle$ and $R_0$ is the basic reproduction number \cite{Miller2009}.

\subsection{Network Structure}
A Barabási-Albert (BA) network with 1000 nodes and mean degree $\langle k \rangle \approx 4$ is generated to reflect social heterogeneity. The degree distribution and high clustering typical of the BA model are visualized, and exact moments $\langle k \rangle$ and $\langle k^2 \rangle$ are calculated:
\begin{itemize}
    \item $\langle k \rangle = 4.003$
    \item $\langle k^2 \rangle = 24.37$
\end{itemize}
\begin{figure}[!tb]
    \centering
    \includegraphics[width=0.42\textwidth]{output/degree_distribution.png}
    \caption{Degree distribution of the Barabási-Albert network.}
    \label{fig-degree}
\end{figure}

\subsection{Parameterization}
Based on $R_0 = 2.75$ for COVID-19 \cite{Liu2020}, and a recovery rate of $\gamma = 1/7$ days$^{-1}$, we set $\beta$ using the computed $q$ from the network moments. Initial conditions place 10 randomly selected nodes in the infected state ($I$), with all others susceptible.

\subsection{Simulation Setup}
We employ a stochastic event-driven simulator (FastGEMF) to propagate SIR dynamics over the BA contact network. Simulations run up to 365 days or until extinction. Results---compartment sizes over time---are saved for analysis.

\section{Results}
\subsection{Epidemic Dynamics}
Figure\,\ref{fig-epicurve} displays the time series for $S$, $I$, and $R$ compartments, averaged over multiple stochastic realizations. The infected population peaks at day 43, with a maximum of 218 individuals; the epidemic dies out by day 191. The final attack rate (fraction ever infected) is 79.6\%.
\begin{figure}[!tb]
    \centering
    \includegraphics[width=0.42\textwidth]{output/results-11.png}
    \caption{Stochastic SIR epidemic trajectory: population in each compartment over time (means over 5 runs).}
    \label{fig-epicurve}
\end{figure}

\subsection{Key Metrics}
\begin{table}[!tb]
    \centering
    \caption{Extracted Epidemic Metrics}
    \begin{tabular}{ll}
    \hline
    Metric             & Value               \\
    \hline
    Epidemic duration  & 191 days            \\
    Peak infected      & 218 individuals     \\
    Peak time          & 43 days             \\
    Final attack rate  & 79.6\%              \\
    $R_0$              & 2.75                \\
    Mean degree $\langle k \rangle$  & 4.00        \\
    Second degree moment $\langle k^2 \rangle$ & 24.37 \\
    \hline
    \end{tabular}
    \label{tab-metrics}
\end{table}

\section{Discussion}
The simulation results confirm that the epidemic unfolds rapidly in scale-free networks due to their highly connected hubs, which act as super-spreaders. The peak infection size and the high final attack rate are consistent with theoretical predictions and empirical findings for COVID-19 in heterogeneous populations, with $R_0$ and the degree distribution strongly dictating epidemic outcomes \cite{PastorSatorras2015,Meyers2007}.

Compared to random or homogeneous networks, the BA topology accelerates initial spread, causes higher peak burden, and results in fewer uninfected individuals at epidemic's end. Targeted interventions---such as hub immunization or contact reduction in high-degree nodes---would substantially alter these statistics and should be prioritized in resource-constrained settings \cite{PastorSatorras2002}.

Stochastic variability among realizations was observed but the qualitative epidemic curves remained robust. These results validate network explicit modeling for COVID-19-like diseases and encourage tailored control measures based on topology.
\section{Conclusion}
This study demonstrates that epidemic outcomes in static populations are heavily influenced by network structure. SIR dynamics on Barabási-Albert networks produce rapid, large-scale outbreaks typified by high peak prevalence and final infection size, primarily driven by network hubs. Accurate modeling of disease transmission, including proper rate calibration to network statistics, is crucial for forecasting and control. Future work should examine dynamic or temporal networks and the effect of mitigation strategies.

\begin{thebibliography}{99}
\bibitem{PastorSatorras2015} R. Pastor-Satorras, C. Castellano, P. Van Mieghem, A. Vespignani, "Epidemic processes in complex networks," Reviews of Modern Physics, vol. 87, no. 3, pp. 925-979, 2015.
\bibitem{Eubank2004} S. Eubank et al., "Modelling disease outbreaks in realistic urban social networks," Nature, vol. 429, pp. 180-184, 2004.
\bibitem{Barrat2008} A. Barrat, M. Barthélemy, A. Vespignani, "Dynamical Processes on Complex Networks," Cambridge University Press, 2008.
\bibitem{Noh2009} J. D. Noh, "Susceptible-infected-recovered model on complex networks with degree correlations," Physical Review E, vol. 80, no. 6, p. 066113, 2009.
\bibitem{Barabasi1999} A.-L. Barabási, R. Albert, "Emergence of scaling in random networks," Science, vol. 286, no. 5439, pp. 509-512, 1999.
\bibitem{Miller2009} J. C. Miller, "Spread of infectious disease through clustered populations," Journal of the Royal Society Interface, vol. 6, no. 41, pp. 1121-1134, 2009.
\bibitem{Liu2020} Y. Liu, A. A. Gayle, A. Wilder-Smith, J. Rocklöv, "The reproductive number of COVID-19 is higher compared to SARS coronavirus," Journal of Travel Medicine, vol. 27, no. 2, 2020.
\bibitem{Meyers2007} L. A. Meyers, "Contact network epidemiology: Bond percolation applied to infectious disease prediction and control," Bulletin of the American Mathematical Society, vol. 44, no. 1, pp. 63-86, 2007.
\bibitem{PastorSatorras2002} R. Pastor-Satorras, A. Vespignani, "Immunization of complex networks," Physical Review E, vol. 65, no. 3, p. 036104, 2002.
\end{thebibliography}

\appendices
\section*{Appendix: Supplementary Figures and Data}
\begin{figure}[!tb]
    \centering
    \includegraphics[width=0.42\textwidth]{output/degree_distribution.png}
    \caption{Degree distribution of the Barabási-Albert network used in the simulations.}
\end{figure}

\begin{figure}[!tb]
    \centering
    \includegraphics[width=0.42\textwidth]{output/results-11.png}
    \caption{Epidemic trajectory: time evolution of S, I, R compartments.}
\end{figure}

\end{document}
