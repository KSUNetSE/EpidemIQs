\title{Temporal Structure Matters: SIR Epidemic Spread in Activity-Driven Networks Versus Time-Aggregated Static Networks}

\begin{document}

\maketitle

\begin{abstract}
Understanding how temporal features of contact networks shape epidemic dynamics is crucial for reliable risk assessment and effective intervention. In this study, we systematically compare the propagation of a susceptible-infected-recovered (SIR) infectious disease with a basic reproduction number $R_0 = 3$ on (i) an activity-driven temporal network of $N=1000$ individuals, where each node activates with probability $\alpha=0.1$ per time step to create $m=2$ transient random connections, and (ii) its corresponding time-aggregated static network, in which edge weights reflect interaction frequencies observed over the same period. We construct both the original temporal network and its static aggregate, calibrate SIR parameters for each to maintain matched $R_0$, and run extensive stochastic simulations. Our results demonstrate that temporal connectivity imposes strong restrictions on the propagation of infectious agents. Specifically, while the static aggregate predicts a rapid and extensive outbreak involving over $90\%$ of the population, the temporal network yields a smaller, slower epidemic with a final size near $32\%$. Temporal ordering and link turnover raise effective epidemic thresholds and disrupt transmission chains, leading to a protracted, lower-magnitude outbreak. We further discuss the implications for outbreak prediction, public health response, and network epidemiology model selection when aiming to capture realistic epidemic risks in dynamically interacting populations.
\end{abstract}

\section{Introduction}

The accurate prediction of epidemic size, peak, and duration is a foundational question in the study of infectious disease dynamics. Traditional models represent host populations as static networks or well-mixed compartments, assuming unchanged contact patterns throughout the epidemic. However, in real-world social systems, individual interactions are transient, and network structures co-evolve with the progression of contagion \cite{Karsai2014,Valdano2015,Masuda2016}. This recognition has led to considerable interest in temporal network models that reflect time-varying connectivity.

Recent theoretical and empirical work has highlighted substantial differences in epidemic outcomes depending on whether the underlying contact network is modeled as temporally resolved or time-aggregated \cite{Holme2012,Masuda2016,Nadini2020}. In particular, activity-driven networks (ADNs) have become a central paradigm for investigating such questions \cite{Perra2012,Karsai2014}. In an ADN, each node is characterized by an activity rate specifying its probability of initiating contacts per unit time. When a node becomes active, it creates $m$ transient edges to randomly selected others, after which all links are removed before the next step. The resulting structure captures essential features of mesoscopic social dynamics—namely, fluctuating contact opportunities and the ephemeral nature of individual ties.

Despite increasing adoption of temporal network frameworks, the majority of epidemiological simulations and network-based policy recommendations are still grounded in static time-aggregated networks. Such aggregation is often achieved by summing or averaging observed contacts over a given observational window, yielding weighted or unweighted static graphs. Although these static representations facilitate tractable analysis, they do not preserve the order of contacts or the actual paths through which transmission may occur \cite{Holme2015,Kim2019,Zino2021}. This loss of temporal information can have profound effects on key epidemic metrics such as the basic reproduction number $R_0$, epidemic thresholds, peak prevalence, and the final outbreak size \cite{Liu2014,Holme2012}.

In this study, we address the following central research question: In an activity-driven network with $N=1000$ nodes, $\alpha=0.1$ activation probability, $m=2$ edges per activation, and matching SIR $R_0=3$, how does the temporal structure of the network affect the spread of an infectious disease compared to its time-aggregated static version? Through carefully parameterized stochastic simulations and quantitative analysis, we seek to elucidate the differences in epidemic trajectories and outcomes between these two network models, and to establish recommendations for modeling and public health intervention based on our findings.