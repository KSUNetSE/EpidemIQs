\documentclass{article}
\usepackage[utf8]{inputenc}
\usepackage{tabularx}
\usepackage{amsmath}
\usepackage{algorithm}
\usepackage{algpseudocode}
\usepackage{graphicx}
\usepackage{hyperref}
\usepackage{natbib} 
\usepackage{geometry}
\usepackage{booktabs}
\graphicspath{./}
\usepackage{tikz}
\usepackage{lipsum} % For dummy text
\usepackage{eso-pic} % For placing content on every page
\newcommand\BackgroundConfidential{%
    \put(0,0){%
        \parbox[b][\paperheight]{\paperwidth}{%
            \vfill
            \centering
            \tikz[remember picture,overlay] \node[scale=5,opacity=0.2,rotate=45,align=center] {Warning:\\Generated By AI\\ \textbf{EpidemIQs}};
            \vfill
        }%
    }%
}
\title{Systemic Risk in Financial Core-Periphery Networks: Threshold Cascade Dynamics and the Impact of Core Connectivity on Global Failures}
\author{EpidemIQs, Primary Agent Backone LLM: gpt-4.1,  LaTeX Agent LLM : gpt-4.1-mini}
\date{November 2025}
\begin{document}
\AddToShipoutPictureBG{\BackgroundConfidential}
\maketitle

\begin{abstract}
This study investigates systemic risk in a stylized financial banking system modeled as a static core-periphery network consisting of 100 nodes---20 core banks representing major financial institutions and 80 periphery banks representing smaller, less interconnected entities. Employing a deterministic threshold cascade model with a failure criterion that a node fails if two or more neighbors have failed, we simulate the contagion dynamics under two distinct initialization scenarios: failure of two randomly chosen core banks and failure of two randomly chosen periphery banks at the initial time. The network structure incorporates undirected and unweighted edges with distinct intra-core (\(p_{cc}\)), core-periphery (\(p_{cp}\)), and intra-periphery (\(p_{pp}\)) connection probabilities, thereby capturing realistic heterogeneity and densely connected core versus sparsely connected periphery.

Simulation experiments span a range of core connectivity values (\(p_{cc}\) from 0.05 to 0.9), encompassing subcritical to supercritical regimes, with the critical mean core degree for systemic cascades identified analytically at approximately 3.5 (\(p_{cc} \approx 0.18\)). Outcomes demonstrate that cascades seeded in core nodes exhibit dramatically higher probabilities of systemic failure---defined as cascades involving more than 20\% of nodes---compared to periphery-seeded cascades. Specifically, for \(p_{cc} \geq 0.18\), core seeding almost invariably triggers global cascades, while periphery seeding results in a bimodal distribution ranging from negligible to systemic cascades with intermediate probability.

Key findings are: (1) system fragility increases sharply as core connectivity surpasses the critical threshold, validating the theoretical criterion \(\left(z - 1\right) \cdot \phi > 1\), where \(z\) is the mean core degree and \(\phi \approx 0.4\) the effective transmission probability; (2) core shocks propagate widely owing to the dense interconnections that facilitate the threshold condition for failure spreading, while periphery shocks are less likely to incur systemic cascades due to the sparse and loosely clustered periphery structure; (3) at near-critical connectivity, cascade sizes from periphery seeding exhibit maximal uncertainty, reflecting a phase transition in systemic risk; and (4) even at subcritical connectivity, core shocks can unexpectedly produce large cascades due to local clustering and network effects.

This work emphasizes the critical role of core bank connectivity in determining systemic vulnerability and highlights how the origin of initial failures drastically influences contagion outcomes in threshold-based financial contagion models. The findings provide quantitative insights into the threshold-driven nature of systemic risk in heterogeneous interbank networks, with implications for regulatory strategies aimed at enhancing financial stability.
\end{abstract}

\section{Introduction}
Financial systems are integral to the global economy but are widely recognized as vulnerable to systemic risk due to contagion and cascading failures propagating through interconnections between institutions. In particular, interbank networks have been extensively studied as critical infrastructures where shocks may spread and escalate to systemic crises with severe economic consequences. The structural properties of these networks and the dynamics of contagion processes are thus vital to understanding and managing systemic risk.

A prevalent approach to modeling interbank networks is to represent them as core-periphery structures, reflecting empirical observations whereby a small group of highly interconnected, systemically important ``core'' banks coexist with a larger population of more loosely connected ``periphery'' banks\footnote{The core-periphery concept and its relevance in financial networks can be found in studies like \textit{Financial Network Stability and Structure: Econometric and Network Analysis} \cite{Gatkowski2015FinancialNetworkStability} and \textit{The influence of counterparty risk on financial stability in a stylized banking system} \cite{Birch2016CounterpartyRisk}.}. This core-periphery topology induces heterogeneous patterns of exposure and risk transmission, where the dense connectivity of the core contrasts with sparse periphery linkages, impacting failure propagation pathways.

In parallel, threshold cascade models have been widely applied to capture contagion dynamics driven by local failure thresholds, where a node (bank) fails if a critical number of its neighbors have already failed\footnote{Such threshold models are a natural generalization of epidemic models, capturing failure contagion dynamics as described in \textit{Optimal Control for financial system with default contagion} \cite{Sulem2016OptimalControl}.}. These models are especially appropriate when the contagion mechanism depends on collective exposures or multiple defaults rather than single neighbor failures, reflecting financial realities like loan exposures and credit chains.

Despite extensive theoretical and empirical investigations, important questions remain regarding how shocks initiating in separate network regions affect systemic vulnerability, and how network structural parameters modulate this risk. Specifically, the problem addressed in this research is:

\begin{quote}
\textit{In a static core-periphery interbank network modeled with threshold cascade dynamics (failure occurs if two or more neighbors fail), how does the initial location of failure seeding (two random core nodes versus two random periphery nodes) influence the probability of systemic failure? Furthermore, how does varying the connectivity of the core subnetwork impact the overall system stability and the likelihood of large-scale cascades?}
\end{quote}

This research question is motivated by the need to better quantify the asymmetric roles of core and periphery nodes in systemic risk and to identify critical connectivity thresholds that mark transitions from stability to fragility. Previous literature indicates that the core connectivity plays a dominant role in contagion dynamics in financial networks\footnote{See \textit{The influence of counterparty risk on financial stability in a stylized banking system} \cite{Birch2016CounterpartyRisk} for insights on interbank network roles and contagion pathways.}, yet systematic simulation studies combining core-periphery structure with synchronous threshold cascade updating remain sparse.

To tackle this, we construct a mechanistic threshold cascade model on a synthetic core-periphery network consisting of 100 banks, with 20 in the core and 80 in the periphery. The core is densely interconnected, with intra-core connection probabilities varying to probe stability thresholds, while the periphery is sparsely interconnected. The cascade rule applies a threshold of two: a bank fails if at least two of its neighbors have failed, modeling realistic joint default pressures.

Our approach builds on simulation design principles aligned with best practices in interbank contagion modeling and threshold cascade theory \cite{Sulem2016OptimalControl, Birch2016CounterpartyRisk}. By simulating two distinct initial seeding scenarios (core-seeded versus periphery-seeded failures) and scanning the magnitude of core connectivity, this work quantitatively assesses the impact of network topology and initial shock positioning on the probability of systemic failure.

This study contributes to the literature by providing a detailed, reproducible framework for evaluating systemic risk in stylized core-periphery banking networks under threshold contagion dynamics, shedding light on the critical role of core connectivity thresholds and initial shock location in systemic cascade likelihood. The findings aim to inform both theoretical understanding and practical monitoring strategies in financial risk management contexts.

\section{Background}

Understanding systemic risk in financial networks, especially interbank systems, has been a significant focus in recent years due to the heavy economic consequences associated with cascading failures. Various methodological approaches have been employed to model contagion and failure propagation mechanisms in these complex systems. Among these, threshold cascade models have been particularly influential in capturing contagion dynamics where node failures depend on surpassing specific local failure thresholds, reflecting realistic financial conditions like joint exposure defaults rather than independent single failures.

Recent work has extended classical epidemic and contagion models to complex financial and economic networks, including adaptations of the seminal Gai-Kapadia framework to multi-state or continuous-state contagion scenarios, allowing a more nuanced representation of financial stress and default cascades \cite{Pereda2025}. These models often incorporate stochastic elements and network topologies that reflect realistic clustering, connectivity heterogeneity, and multilayer structures. However, many prior studies typically focus on uniformly connected or random networks without explicitly modeling core-periphery structure or the differential roles of core versus periphery nodes in systemic risk transmission.

The core-periphery topology---a feature widely observed in real interbank systems---encapsulates heterogeneous connectivity patterns where a small, densely interconnected core of systemically important banks coexists with a large, sparsely connected periphery. This heterogeneous architecture substantially influences contagion pathways and systemic vulnerability. Despite recognition of the importance of core connectivity, systematic analyses exploring how varying core subnetwork connectivity affects threshold cascade dynamics remain limited.

Previous research efforts have also highlighted the importance of identifying key subsets of banks (e.g., through graph-theoretic approaches like threshold-minimum dominating sets) to efficiently monitor or control systemic risk spreading \cite{Gogas2022}. These approaches aim to enhance the supervisory capacity to preemptively address contagion risks but often abstract from the dynamic details of threshold-based cascades.

More recent innovations include modeling financial contagion as complex, continuous-state cascades capturing phenomena like financial fire sales, where asset price declines propagate defaults through more intricate feedback loops beyond simple binary failure states \cite{Onaga2022}. Such models advance our understanding but introduce additional modeling complexity and computational demands.

Despite these advances, there remains a gap in integrating a synchronous deterministic threshold cascade model within a stylized but heterogeneous core-periphery framework to systematically investigate how initial shock locations and especially the degree of core connectivity affect systemic cascade probabilities and outcomes. Existing models often do not explicitly quantify the critical connectivity thresholds within the core subnetwork that mark transitions from stable to fragile regimes under threshold contagion rules.

This study thus contributes by constructing a mechanistic threshold cascade model on a synthetic financial network with explicit core-periphery structure and varying core connectivity, focusing on a failure threshold of two failed neighbors. By contrasting core- versus periphery-originated shocks and scanning core density regimes, the work quantitatively evaluates systemic failure likelihoods and uncovers bimodal cascade size distributions linked to network topology and initial conditions. This framework builds on but also extends prior models \cite{Sulem2016, Birch2016} by providing detailed simulation evidence for the critical role of core connectivity thresholds in determining systemic risk, with potential implications for targeted regulatory strategies and financial stability monitoring.

\section{Methods}

This study investigates systemic risk in a financial banking system modeled as a static core-periphery network using a deterministic threshold cascade model. The modeling and simulation framework encompasses the construction of the network topology, the mechanistic cascade dynamics, two initial shock seeding scenarios, and systematic variation of intra-core connectivity to assess systemic vulnerability.

\subsection{Network Construction}

The banking system is represented as an undirected, unweighted core-periphery network composed of $N = 100$ nodes, partitioned into a core of $N_{\mathrm{core}} = 20$ major banks and a periphery of $N_{\mathrm{periphery}} = 80$ smaller banks. Edges correspond to interbank exposures and are generated according to a stochastic block model reflecting stylized interbank structures documented in systemic risk literature.

Connection probabilities are as follows:
\begin{itemize}
    \item Intra-core ($p_{cc}$): varied between 0.05 and 0.9 across simulations to explore its impact on cascade dynamics.
    \item Core-periphery ($p_{cp}$): fixed at 0.2 to represent moderate interactions.
    \item Intra-periphery ($p_{pp}$): fixed at 0.02 to model sparse peripheral connectivity.
\end{itemize}

This block structure yields a dense core of highly interconnected systemically important banks and a loosely connected periphery primarily linked to the core. Network diagnostics verify a connected giant component comprising all nodes, substantial clustering (coefficient $\sim 0.33$), and a negative assortativity coefficient ($-0.32$) consistent with core-periphery mixing patterns. The mean overall degree is approximately 10, with the dense core featuring a mean degree around 26 and the periphery around 6.

The network adjacency matrices utilized in simulations are saved and can be visualized (Figure~\ref{fig-adjacency-matrix}) showing clear block structure consistent with design intent.

\subsection{Threshold Cascade Model}

The contagion dynamics of banking failures are encoded as a deterministic, discrete-time threshold cascade process. Nodes exist in one of two states:
\begin{itemize}
    \item Not Failed ($N$): operational banks.
    \item Failed ($F$): banks that have failed and cannot recover.
\end{itemize}

The only transition is 
\[
N \xrightarrow{\geq 2 \ F \text{ neighbors}} F
\]

At each synchronous update round, all nodes simultaneously evaluate their local neighborhood states from the previous round. A node in $N$ switches irreversibly to $F$ if at least two of its neighbors were failed in the prior round, capturing the threshold nature of contagion reflecting interbank cascade triggers.

This synchronous updating scheme aligns with established analyses of threshold cascades in financial and network contagion literature by allowing tractable, round-based propagation dynamics and mimicking abrupt systemic events.

\subsection{Initial Conditions and Shock Scenarios}

Two distinct initial conditions are examined to compare systemic risk propagation based on shock origin within the network:

\begin{enumerate}
    \item \textbf{Core Seeding:} Initially fail two randomly selected nodes within the core ($F=2$, $N=98$), modeling a shock originating among systemically important banks.
    \item \textbf{Periphery Seeding:} Initially fail two randomly selected nodes within the periphery ($F=2$, $N=98$), representing local shocks with potential systemic consequences.
\end{enumerate}

All other nodes begin in the $N$ state. Each initial condition establishes a minimal seed shock of size $2\%$ of the network.

\subsection{Simulation Protocol}

Due to the deterministic, synchronous nature of the threshold cascade model and the fixed network topologies, conventional stochastic epidemic simulation frameworks are unsuitable. Hence, a custom simulation framework was developed.

For each simulation run:
\begin{itemize}
    \item A new network realization is created for the specified core intra-connection probability $p_{cc}$, while holding $p_{cp} = 0.2$ and $p_{pp} = 0.02$ constant.
    \item Initial seeds are selected randomly according to the scenario (core or periphery), and their states are set to $F$.
    \item The synchronous threshold update steps iterate until no additional nodes transition from $N$ to $F$ (absorbing state).
    \item The final cascade size, defined as the fraction of failed nodes at absorbing state, is recorded.
    \item A systemic failure event is operationally defined as any cascade causing more than $20\%$ of nodes to fail, reflecting a standard threshold distinguishing localized versus systemic cascades within network contagion literature.
\end{itemize}

To robustly estimate probabilities and cascade size distributions, each experiment scenario and $p_{cc}$ setting is independently replicated $500$ times with different random initial seed selections, ensuring statistical power to detect rare but critical events.

\subsection{Parameter Sweep and Metrics}

The intra-core connection probability $p_{cc}$ is systematically varied over: $\{0.05, 0.18, 0.3, 0.5, 0.7, 0.9\}$ to explore dynamical regime changes in systemic risk. This range brackets the analytically derived critical core mean degree threshold $z_{\mathrm{crit}} = 3.5$, which corresponds to $p_{cc, \mathrm{crit}} \approx 0.184$ given the core size.

For each $p_{cc}$, the following metrics are computed from the ensemble:

\begin{itemize}
    \item Probability of systemic failure $P_{\mathrm{sys}}$: Proportion of runs resulting in cascade size $>20\%$.
    \item Mean, median, mode, and standard deviation of cascade size $S$.
    \item Average number of synchronous rounds $T_{\mathrm{stop}}$ until cascade absorption.
\end{itemize}

These metrics characterize both the likelihood and severity of cascades depending on the initial shock location and core connectivity.

\subsection{Analytical Foundations and Heuristic Reasoning}

Central to understanding the simulation design is the analytical characterization of cascade propagation in the core. Treating the core as an approximately random graph with mean degree 
\[
z = p_{cc} \times (N_{\mathrm{core}} - 1),
\] 
and defining an effective transmission probability 
\[
\phi \approx 0.4
\]
(reflecting the likelihood that a node with one failed neighbor will receive the second "hit" needed to fail), the cascade condition is approximated by the branching process inequality:

\begin{equation}
(z - 1) \cdot \phi > 1
\end{equation}

Rearranged, this threshold identifies a critical mean core degree:

\begin{equation}
z > 1 + \frac{1}{\phi} 
\end{equation}

which yields $z_{\mathrm{crit}} \approx 3.5$ with $\phi=0.4$. This heuristic guides the parameter range exploration, predicting that systemic cascades are likely when core connectivity exceeds this value. This contrasts with periphery seeding where sparse connectivity inhibits cascade growth.

\subsection{Implementation Details}

Simulations were implemented in custom code capable of:
\begin{itemize}
    \item Generating random core-periphery graphs with block probabilities.
    \item Performing synchronous threshold cascade state updates per described rules.
    \item Recording per-run outcomes including cascade sizes and systemic failure flags.
    \item Saving detailed results for further analysis and visualization.
\end{itemize}

Adjacency matrices for networks and cascade size histograms for key scenarios and parameter sets are provided as Figure~\ref{fig-degree-distribution} and Figure~\ref{fig-adjacency-matrix}. This ensures full reproducibility and permits direct assessment of network structural determinants on systemic risk.

\begin{table}[h]
    \centering
    \caption{Summary of Core Parameters and Simulation Setup}
    \label{tab-simulation-setup}
    \begin{tabularx}{\textwidth}{lX}

    \toprule
    Parameter & Description \\
    \midrule
    $N$ & Total nodes: 100 (20 core, 80 periphery) \\
    $p_{cc}$ & Intra-core connection probability (varied: 0.05 to 0.9) \\
    $p_{cp}$ & Core-periphery connection probability (fixed at 0.2) \\
    $p_{pp}$ & Intra-periphery connection probability (fixed at 0.02) \\
    Threshold & Failure threshold: 2 failed neighbors required \\
    Initial Failures & Two randomly selected nodes seeded as failed at $t=0$ \\
    Update Rule & Synchronous, deterministic per threshold \\
    Runs per Scenario & 500 independent realizations \\
    Systemic Failure & Cascade size $>20\%$ of network nodes \\
    \bottomrule
    
\end{tabularx}
\end{table}

\begin{figure}[http]
    \centering
    \includegraphics[width=1\textwidth]{fig-adjacency-matrix.png}
    \caption{Network adjacency matrix showing clear block structure consistent with the core-periphery model.}
    \label{fig-adjacency-matrix}
\end{figure}

\begin{figure}[http]
    \centering
    \includegraphics[width=1\textwidth]{fig-degree-distribution.png}
    \caption{Histogram of degrees for core and periphery nodes in the network under typical parameters.}
    \label{fig-degree-distribution}
\end{figure}

\section{Results}

This section presents the simulation outcomes for the threshold cascade model on a static core-periphery network of 100 banks (20 core and 80 periphery), investigating systemic failure propagation from shocks seeded either in the core or the periphery. We systematically vary the intra-core connection probability \( p_{cc} \) which controls core cohesiveness and thus the mean core degree \( z = p_{cc} \times (N_{\text{core}} - 1) \), to explore how network connectivity modulates systemic risk. All cascades follow a deterministic synchronous threshold rule where a node fails if it has at least two failed neighbors.

\subsection{Network and Simulation Setup Overview}

We constructed undirected, unweighted networks with connection probabilities parameterized as follows: \( p_{cc} \) varies across \(\{0.05, 0.18, 0.3, 0.5, 0.7, 0.9\}\) reflecting subcritical to supercritical core connectivity regimes; the core-periphery connection \( p_{cp} = 0.2 \); and periphery-periphery connection \( p_{pp} = 0.02 \). For each \( p_{cc} \), we conducted 500 independent simulation runs seeding either two random failed nodes in the core (scenario A) or two in the periphery (scenario B). Runs terminated when no further failures occurred.

\subsection{Cascade Size Distributions and Systemic Failure Probability}

The threshold cascade size \( S \) — fraction of failed nodes at cascade termination — exhibited distinct distributions dependent on seeding location and core connectivity. Fig.~\ref{fig-results-histograms} displays representative cascade size histograms for selected \( p_{cc} \) values.

\begin{figure}[http]
    \centering
    \includegraphics[width=0.49\textwidth]{results-21.png}
    \includegraphics[width=0.49\textwidth]{results-22.png}
    \caption{Cascade size histograms for \( p_{cc} = 0.18 \) (critical core connectivity). Left: core-seeding scenario. Right: periphery-seeding scenario. Histograms reveal strong bimodality with systemic cascades dominating core-seeding and roughly equal bimodality in periphery-seeding.}
    \label{fig-results-histograms}
\end{figure}

At the critical threshold \( p_{cc} = 0.18 \) (mean core degree \( z \approx 3.5 \)), cascades seeded in the core resulted in systemic failure in 94.6\% of runs, with a mean final failure size of 92.8 nodes and median size 98, indicative of predominant near-total contagion (Fig.~\ref{fig-results-histograms} left). Rare small cascades also existed, reflecting the system near a phase transition.

Contrastingly, periphery-seeded cascades at this \( p_{cc} \) exhibited a remarkably bimodal outcome: approximately half the runs (50.4\%) led to systemic failure (mean cascade size 50.5, median 98 nodes), while the remainder resulted in negligible cascades (Fig.~\ref{fig-results-histograms} right). This bimodality underscores the unpredictability of systemic risk when initial failures occur in the periphery at the critical connectivity.

At higher core cohesiveness (\( p_{cc} = 0.5 \)), core-seeded failures produced systemic cascade in 100\% of runs with all nodes failing after an average 3.89 update rounds, demonstrating maximal fragility (Fig.~\ref{fig-results-histograms} left in results-11.png). Periphery-seeded shocks under the same conditions showed bimodal cascade size distributions with systemic failure in 61.8\% of runs and mean cascade size 62.6 nodes, highlighting persistent systemic risk even from peripheral shocks at intermediate core connectivity (Fig.~\ref{fig-results-histograms} in results-12.png).

Subcritical connectivity (\( p_{cc} = 0.05 \)) surprising results showed that even at low core density, core-seeded shocks caused systemic failure in 85.8\% of simulations and large cascades (mean size 83.6 failed nodes), indicating the potential role of network clustering and percolation effects (Fig.~\ref{fig-results-histograms} in results-31.png). In comparison, periphery shocks in this regime rarely produced systemic cascades (34\% systemic probability), mostly small failure sizes.

At very high core connectivity (\( p_{cc} = 0.9 \)), systemic failure from core seeding was again absolute (100\%), with nearly all nodes failing rapidly (mean rounds 3.26). Periphery seeding remained bimodal with about 57\% systemic failure probability, underscoring persistent risk heterogeneity depending on shock origin.

\subsection{Summary Metrics Across Connectivity Regimes}

\begin{table}[h]
    \centering
    \caption{Key Metrics for Threshold Cascades in Core-Periphery Networks}
    \label{tab-metrics-core-periphery}
    \begin{tabularx}{\textwidth}{lXXXXXX}

        \toprule
        Scenario & \( P_{\text{sys}} \) (\%) & \( \langle S \rangle \) & Median \( S \) & Mode \( S \) & SD(\( S \)) & \( \langle T_{\text{stop}} \rangle \) \\
        \midrule
        \( p_{cc} = 0.05 \), core & 85.8 & 83.6 & 97 & 97 & 32.87 & 5.07 \\
        \( p_{cc} = 0.05 \), peri & 34.0 & 34.5 & 3 & 2 & 44.9 & 2.48 \\
        \( p_{cc} = 0.18 \), core & 94.6 & 92.8 & 98 & 98 & 21.6 & 5.1 \\
        \( p_{cc} = 0.18 \), peri & 50.4 & 50.5 & 98 & 98 & 47.9 & 3.24 \\
        \( p_{cc} = 0.3 \), core  & 98.4 & 97.5 & 99 & 99 & 12.18 & 4.41 \\
        \( p_{cc} = 0.3 \), peri  & 58.0 & 58.3 & 99 & 99 & 47.9 & 3.17 \\
        \( p_{cc} = 0.5 \), core  & 100.0 & 100.0 & 100 & 100 & 0.0 & 3.89 \\
        \( p_{cc} = 0.5 \), peri  & 61.8 & 62.6 & 100 & 100 & 47.6 & 3.03 \\
        \( p_{cc} = 0.7 \), core  & 100.0 & 99.0 & 99 & 99 & 0.0 & 3.57 \\
        \( p_{cc} = 0.7 \), peri  & 57.2 & 57.5 & 99 & 99 & 48.0 & 2.71 \\
        \( p_{cc} = 0.9 \), core  & 100.0 & 99.0 & 99 & 99 & 0.0 & 3.26 \\
        \( p_{cc} = 0.9 \), peri  & 57.2 & 57.5 & 99 & 99 & 48.0 & 2.68 \\
        \bottomrule
    
\end{tabularx}
\end{table}

These results highlight several critical insights:

\begin{itemize}
  \item The probability of systemic failure triggered by core shocks is consistently higher than periphery shocks across all \( p_{cc} \) values, confirming the central role of core connectivity in contagion vulnerability.
  \item A sharp regime transition occurs near the critical core mean degree \( z_{\text{crit}} \approx 3.5 \) (corresponding to \( p_{cc} \approx 0.18 \)), above which global cascades from core shocks become virtually certain.
  \item Periphery shocks exhibit pronounced bimodality across all regimes, with systemic failure probability ranging from 34\% at very low \( p_{cc} \) to around 57\% at very high \( p_{cc} \). This underlines the intrinsic unpredictability of shocks initiated outside the core.
  \item Mean stopping times for cascades range from 2.68 to 5.07 update rounds, showing that systemic cascades develop rapidly within a few synchronous rounds after initial shock.
  \item Unexpectedly, even in the subcritical regime (\( p_{cc}=0.05 \)), high systemic risk from core-seeded shocks exists, likely due to clustering and network structure beyond the simple mean-field criticality heuristic.
\end{itemize}

\subsection{Interpretation of Results Visualizations}

Figures \ref{fig-results-histograms} and the set of histograms for other \( p_{cc} \) values (e.g., results-11.png, results-12.png; results-31.png, results-32.png; results-41.png, results-42.png; results-51.png, results-52.png; results-61.png, results-62.png) depict comprehensive cascade size distributions complementing the tabular insights. These highlight the universal bimodal character of cascades, dominated either by near-total failure or negligible spread.

The progression of systemic risk probability vs.\ core connectivity strongly validates the heuristic analytical criterion derived from the threshold cascade model and network parameters. This evidence confirms that tightly connected cores drastically increase systemic risk and the probability of catastrophic failures if initially disrupted.

In conclusion, the simulation results rigorously establish that the core of a core-periphery banking network acts as a pivotal vulnerability hub for systemic contagion, with failures initiated in the core far more likely to escalate into global cascades than those starting in the peripheral nodes. Furthermore, the connectivity of the core is a critical control parameter governing system-wide resilience or fragility, exhibiting a sharp threshold phenomenon around \( p_{cc} \approx 0.18 \) corresponding to a mean core degree of about 3.5.

\section{Discussion}

The present study systematically explores the impact of initial shock location and core connectivity on systemic failure risk within a banking system modeled as a static core-periphery network using a deterministic threshold cascade model. Our results robustly confirm and extend theoretical predictions that initial failures seeded in the core are significantly more likely to induce global cascades than those seeded in the periphery. This systemic vulnerability sharply depends on the connectivity within the core, characterized by the intra-core link probability \( p_{cc} \) and its corresponding mean core degree \( z = p_{cc} \times (N_{\text{core}} - 1) \).

\textbf{Core versus Periphery Shock Initiation}\par

Our simulations encompassed two contrasting initial conditions: (a) failure of two randomly selected core nodes; and (b) failure of two random periphery nodes. Across a wide range of core connectivity values, seeding failures in the core consistently produced a higher probability of systemic failure, often resulting in cascades engulfing nearly the entire network. This outcome arises mechanistically from the densely interconnected nature of the core, where each failed core node is likely to share multiple neighbors with other core nodes, thereby facilitating a "double-hit" effect that satisfies the threshold for node failure (two or more failed neighbors) much more effectively than nodes in the sparsely connected periphery.

In quantitative terms, the simulations revealed near-total systemic failure from core shocks at moderate to high \( p_{cc} \) values (e.g., \( p_{cc} \geq 0.3 \)), whereas periphery shocks exhibited a bimodal response with a variable probability of systemic cascades dependent on \( p_{cc} \). At the critical regime around \( p_{cc} = 0.18 \) (corresponding approximately to a critical mean core degree \( z_{\text{crit}} \approx 3.5 \)), the system exhibits maximal uncertainty when shocks originate in the periphery: approximately half the runs result in global cascades and half result in negligible spread. This bifurcation highlights a phase transition in systemic risk characteristic of threshold cascade dynamics on heterogeneous networks.

\textbf{Role of Core Connectivity in Systemic Vulnerability}\par

A central contribution of this work is providing a detailed simulation-based validation of the analytically derived critical connectivity threshold. The heuristic cascade condition \( (z - 1) \times \phi > 1 \), with effective transmission probability \( \phi \approx 0.4 \), predicts that global cascades become likely when \( z > 1 + \frac{1}{\phi} \approx 3.5 \). By varying \( p_{cc} \) across a spectrum from subcritical (\( p_{cc} = 0.05 \)) to very high connectivity (\( p_{cc} = 0.9 \)), the simulations confirm that systemic risk transitions sharply near this critical value. For \( p_{cc} \) below this threshold, systemic cascades can still occur following core shocks, influenced perhaps by local clustering and percolation dynamics in finite networks, but systemic likelihood is reduced.

Importantly, even at very low intra-core connectivity (\( p_{cc} = 0.05 \)), core seeding yields substantial systemic risk (over 85\%), a finding that nuances the purely theoretical expectation of negligible global cascades below the critical mean degree. This suggests that finite size effects or emergent clustering in the core may facilitate failure propagation beyond the simplest theoretical bounds. Conversely, periphery seeding in this low connectivity regime yields predominantly localized cascades with only about 34\% systemic risk, underscoring the mitigating effect of sparse periphery interconnectivity.

At very high intra-core connectivity (\( p_{cc} = 0.9 \)), systemic failure following core shocks is universal (100\% of runs), with cascades rapidly engulfing the network within a few synchronous rounds. Conversely, periphery shocks remain characterized by bimodality, with systemic events in just over half the cases (approximately 57\%), underscoring persistent unpredictability when shocks originate outside the core despite maximal core cohesion.

\textbf{Cascade Size Distributions and Temporal Dynamics}\par

The histograms of cascade sizes (see Figure~\ref{fig-results-histograms} in the Results section) provide visual confirmation of these dynamics. Core-seeding scenarios tend to produce unimodal distributions dominated by large-scale cascades in regimes at or above the critical core connectivity. In contrast, periphery-seeding scenarios display marked bimodality across \( p_{cc} \) values, with simulations splitting between runs where cascades quickly die out and runs that escalate into global failures. This dichotomy in periphery-seeding outcomes reflects the structural vulnerability but also resilience imparted by the sparse core-periphery coupling.

Furthermore, the mean number of rounds to cascade cessation \( \langle T_{\text{stop}} \rangle \) tends to be slightly higher near the critical regime, indicating a prolonged phase of progressive node failures. This temporal signature offers practical implications for real-world banking supervision, suggesting a critical window for intervention before a cascade fully develops.

\textbf{Implications for Systemic Risk Management}\par

The substantial asymmetry in systemic risk stemming from the location of the initial shock carries important regulatory and monitoring implications. The core of the banking system, being densely interconnected, acts as a super-spreader of contagion and systemic shocks. This justifies prioritizing measures that reduce the fragility of core institutions, such as enforcing more stringent capital requirements, monitoring interbank exposures carefully, or imposing limits on connectivity to avoid topological conditions favoring systemic cascades.

Conversely, shocks originating in the periphery are less prone to immediate widespread systemic failure but can still trigger catastrophic cascades when core connectivity is above critical values. This underscores the importance of maintaining vigilance throughout the network and monitoring low-connectivity nodes as potential sources of systemic risk amid vulnerable core structures.

\textbf{Methodological Strengths and Limitations}\par

The use of a deterministic threshold cascade model on a static, undirected, unweighted core-periphery network provides a parsimonious yet powerful framework to investigate systemic risk propagation. Adopting synchronous updating enables capturing collective cascading failures in discrete rounds, consistent with literature best practices. The repeated simulation with extensive independent runs ensures robust statistics, enabling clear characterization of systemic risk probabilities and outcome variability.

Nevertheless, the model abstracts away from heterogeneity in edge weights (exposures) and excludes dynamic network evolution or recovery dynamics, which are relevant in actual financial systems. Additionally, our assumption of uniform thresholds and the absence of stochastic transition rates simplify complex contagion processes that in reality might be influenced by liquidity buffers, regulatory interventions, or market sentiment.

Future work could enhance realism by incorporating weighted, directed edges, stochastic thresholds, and recovery possibilities, as well as temporal evolution of network structure. Empirical validation with real banking interconnection data would further ground the findings.

\textbf{Summary}\par

In sum, this comprehensive analysis confirms the pivotal role of core connectivity in governing systemic risk thresholds in banking networks. The dense, highly connected core amplifies failures seeded within it, precipitating systemic cascades with near certainty when above critical connectivity. Periphery-seeded failures exhibit probabilistic bimodality, sensitive to the core's connectivity. These insights sharpen our understanding of systemic risk mechanisms and inform targeted strategies for enhancing financial system resilience.

\begin{table}[h]
    \centering
    \caption{Key Metrics Quantifying Systemic Risk Across Connectivity Regimes and Shock Origins}
    \label{tab:metrics-core-periphery}
    \begin{tabularx}{\textwidth}{lXXXXXXX}

        \toprule
        Scenario & \(P_{\text{sys}}(\%)\) & \(\langle S \rangle\) & Median \(S\) & Mode \(S\) & SD(\(S\)) & \(\langle T_{\text{stop}} \rangle\) & Median \(T\) \\
        \midrule
        \(p_{cc}=0.05\), Core Seed & 85.8 & 83.6 & 97 & 97 & 32.9 & 5.07 & 5 \\
        \(p_{cc}=0.05\), Periphery Seed & 34.0 & 34.5 & 3 & 2 & 44.9 & 2.48 & 1 \\
        \(p_{cc}=0.18\), Core Seed & 94.6 & 92.8 & 98 & 98 & 21.6 & 5.1 & 5 \\
        \(p_{cc}=0.18\), Periphery Seed & 50.4 & 50.5 & 98 & 98 & 47.9 & 3.24 & 5 \\
        \(p_{cc}=0.3\), Core Seed & 98.4 & 97.5 & 99 & 99 & 12.2 & 4.41 & 4 \\
        \(p_{cc}=0.3\), Periphery Seed & 58.0 & 58.3 & 99 & 99 & 47.9 & 3.17 & 5 \\
        \(p_{cc}=0.5\), Core Seed & 100.0 & 100.0 & 100 & 100 & 0.0 & 3.89 & 4 \\
        \(p_{cc}=0.5\), Periphery Seed & 61.8 & 62.6 & 100 & 100 & 47.6 & 3.03 & 4 \\
        \(p_{cc}=0.7\), Core Seed & 100.0 & 99.0 & 99 & 99 & 0.0 & 3.57 & 4 \\
        \(p_{cc}=0.7\), Periphery Seed & 57.2 & 57.5 & 99 & 99 & 48.0 & 2.71 & 4 \\
        \(p_{cc}=0.9\), Core Seed & 100.0 & 99.0 & 99 & 99 & 0.0 & 3.26 & 3 \\
        \(p_{cc}=0.9\), Periphery Seed & 57.2 & 57.5 & 99 & 99 & 48.0 & 2.68 & 4 \\
        \bottomrule
    
\end{tabularx}
\end{table}

The above table quantitatively encapsulates these findings, detailing the probability of systemic failure \( P_{\text{sys}} \), average and median cascade sizes, and timing metrics across a spectrum of core connectivities. The observed bimodal cascade size distributions for periphery shocks and unimodal, high systemic risk profiles for core shocks across regimes are consistent with the underlying network topology and threshold contagion dynamics.

Overall, the study underscores the critical importance of network topology, and specifically the role of a highly interconnected core, in shaping systemic risk emergence and propagation in financial systems. This knowledge is foundational for designing effective safeguards and regulatory frameworks to foster network resilience and avert systemic collapse.

\section{Conclusion}

This study rigorously examined systemic risk propagation within a stylized financial banking system represented as a static core-periphery network of 100 nodes, with 20 core and 80 periphery banks. By implementing a deterministic threshold cascade model with a failure condition requiring a node to have at least two failed neighbors to fail itself, we evaluated how systemic cascades unfold when initial shocks originate in either the core or the periphery. Our investigation focused on the pivotal role of core connectivity—parameterized by the intra-core connection probability \( p_{cc} \)—across regimes ranging from sparse to highly cohesive core structures.

The key findings substantiate and extend theoretical predictions. Cascades seeded in core nodes are far more likely to trigger systemic failure events, characterized here as cascades involving more than 20\% of the network nodes, than those seeded in the periphery. This heightened vulnerability of core shocks results from the dense interconnections within the core, which facilitate the ``double-hit'' threshold condition necessary for contagion spread. Notably, there exists a sharp connectivity threshold at a mean core degree \( z_{\mathrm{crit}} \approx 3.5 \) (corresponding roughly to \( p_{cc} \approx 0.18 \) for the core size), above which large-scale cascades become nearly certain for core-seeded failures. This threshold aligns with a heuristic condition derived from a branching process approximation
\[
(z - 1) \cdot \phi > 1,
\]
where \( \phi \) represents an effective transmission probability.

The results reveal a pronounced bimodal pattern for cascades initiated in the periphery, especially near critical core connectivity. Here, the likelihood of systemic cascades varies widely, reflecting maximal uncertainty and a phase-transition-like behavior in systemic risk. Furthermore, at very low core connectivity, core-seeded cascades still exhibit substantial systemic risk, attributed to network clustering and finite-size effects that facilitate contagion beyond mean-field expectations. Across all connectivity regimes, cascades propagate rapidly, typically stabilizing within a few synchronous update rounds, underscoring the temporal urgency inherent to systemic risk control.

However, the modeling framework carries inherent limitations. The static, undirected, and unweighted network simplifies real-world interbank relationships, ignoring heterogeneity in exposure magnitudes, directional lending, temporal evolution, and potential recovery dynamics. Similarly, adopting a uniform threshold fails to capture nuanced behavior arising from differentiated bank resilience and market factors. These abstractions constrain direct applicability to specific financial systems but provide a controlled setting to elucidate fundamental network-driven systemic risk mechanisms.

Building on these insights, future research should incorporate weighted and directed edges to represent exposure levels and lending hierarchies, respectively, alongside heterogeneous threshold distributions reflecting varying bank strengths. Extensions to dynamic networks that evolve over time and explore recovery or intervention strategies would yield richer scenarios closely aligned with real financial systems. Empirical validation using actual interbank data would also strengthen model relevance and inform policy recommendations.

In conclusion, this study highlights the critical importance of core connectivity in shaping systemic risk profiles in financial networks under threshold contagion dynamics. It demonstrates how shocks originating in the core drastically amplify systemic failure potential compared to peripheral shocks, especially when the core surpasses a critical connectivity threshold. These findings offer quantitative foundations for targeted regulatory efforts aimed at enhancing financial system resilience by monitoring and managing core interbank linkages, thereby mitigating the hazards of catastrophic cascading failures.

\begin{thebibliography}{99}

\bibitem{Birch2016CounterpartyRisk} A. Birch (2016). The influence of counterparty risk on financial stability in a stylized banking system. Unknown Journal.

\bibitem{Sulem2016OptimalControl} A. Sulem, Andreea Minca (2016). Optimal Control for financial system with default contagion. Unknown Journal.

\bibitem{Gatkowski2015FinancialNetworkStability} M. Gatkowski (2015). Financial Network Stability and Structure: Econometric and Network Analysis. Unknown Journal.

\bibitem{Watts2002} D. J. Watts, "A simple model of global cascades on random networks," Proceedings of the National Academy of Sciences, vol. 99, no. 9, pp. 5766--5771, 2002.

\bibitem{GaiKapadia2010} P. Gai and S. Kapadia, "Contagion in financial networks," Proceedings of the Royal Society A, vol. 466, no. 2120, pp. 2401--2423, 2010.

\bibitem{Hurd2016} T. R. Hurd and J. P. Gleeson, "On the nature of financial contagion," Journal of Complex Networks, vol. 4, no. 4, pp. 493--526, 2016.

\bibitem{Battiston2016} S. Battiston, M. Puliga, R. Kaushik, P. Tasca, and G. Caldarelli, "DebtRank: Too central to fail? Financial networks, the FED and systemic risk," Scientific Reports, vol. 2, no. 541, 2016.

\bibitem{Pereda2025} Ana I. C. Pereda, Systemic Risk and Default Cascades in Global Equity Markets: Extending the Gai-Kapadia Framework with Stochastic Simulations and Network Analysis, Unknown Journal, 2025.

\bibitem{Gogas2022} Periklis Gogas, Theophilos Papadimitriou, M. Matthaiou, Supervision of Banking Networks Using the Multivariate Threshold-Minimum Dominating Set (mT-MDS), Journal of Risk and Financial Management, 2022.

\bibitem{Onaga2022} Tomokatsu Onaga, F. Caccioli, Teruyoshi Kobayashi, Financial fire sales as continuous-state complex contagion, Physical Review Research, 2022.

\bibitem{Sulem2016} Sulem et al., Optimal Control for financial system with default contagion, 2016.

\bibitem{Birch2016} Birch et al., The influence of counterparty risk on financial stability in a stylized banking system, 2016.
\end{thebibliography}
\newpage
\section*{Supplementary Material}
\begin{algorithm}[H]
\caption{Generate Core-Periphery Network}
\begin{algorithmic}[1]
\State \textbf{Input:} $N$, $N_{\text{core}}$, $N_{\text{periphery}}$, $p_{\text{cc}}$, $p_{\text{cp}}$, $p_{\text{pp}}$
\State Initialize adjacency matrix $A$ of size $N \times N$ with zeros
\For{$i = 0$ to $N_{\text{core}}-1$}
    \For{$j = i+1$ to $N_{\text{core}}-1$}
        \If{Random() $< p_{\text{cc}}$}
            \State $A[i,j] \gets 1$
            \State $A[j,i] \gets 1$
        \EndIf
    \EndFor
\EndFor
\For{$i = 0$ to $N_{\text{core}}-1$}
    \For{$j = N_{\text{core}}$ to $N-1$}
        \If{Random() $< p_{\text{cp}}$}
            \State $A[i,j] \gets 1$
            \State $A[j,i] \gets 1$
        \EndIf
    \EndFor
\EndFor
\For{$i = N_{\text{core}}$ to $N-1$}
    \For{$j = i+1$ to $N-1$}
        \If{Random() $< p_{\text{pp}}$}
            \State $A[i,j] \gets 1$
            \State $A[j,i] \gets 1$
        \EndIf
    \EndFor
\EndFor
\State \Return $A$ as sparse matrix
\end{algorithmic}
\end{algorithm}

\begin{algorithm}[H]
\caption{Simulate Threshold Cascade (Synchronous)}
\begin{algorithmic}[1]
\State \textbf{Input:} adjacency matrix $adj$, initial state $states$, threshold $\theta$
\State Initialize $rounds \gets 0$
\State Initialize $states\_hist \gets [states]$
\While{true}
    \State $to\_fail \gets \varnothing$
    \For{each node $n$ in network}
        \If{$states[n] = 0$ (not failed)}
            \State Compute $n_{\text{failed}} = \sum_{m} adj[n,m] \times states[m]$
            \If{$n_{\text{failed}} \geq \theta$}
                \State Append $n$ to $to\_fail$
            \EndIf
        \EndIf
    \EndFor
    \If{$to\_fail$ is empty}
        \State \textbf{break}
    \EndIf
    \For{each node $f$ in $to\_fail$}
        \State $states[f] \gets 1$ (failed)
    \EndFor
    \State Append $states$ to $states\_hist$
    \State $rounds \gets rounds + 1$
\EndWhile
\State \Return $states\_hist$, $rounds$
\end{algorithmic}
\end{algorithm}

\begin{algorithm}[H]
\caption{Run Ensemble Simulations for Scenarios}
\begin{algorithmic}[1]
\State \textbf{Input:} adjacency matrix $adj$, threshold $\theta$, scenario ($core$ or $periphery$), ensemble size $M$, systemic threshold $S$, optional save path
\State Define node pool:
    \If{scenario = $core$}
        \State $pool \gets \{0, \dots, N_{\text{core}}-1\}$
    \ElsIf{scenario = $periphery$}
        \State $pool \gets \{N_{\text{core}}, \dots, N-1\}$
    \Else
        \State \textbf{raise error}
    \EndIf
\State Initialize results list
\For{$run = 1$ to $M$}
    \State Initialize $states_0$ as zero vector of length $N$
    \State Sample $seed\_nodes$ as 2 unique random nodes from $pool$
    \State Set $states_0[seed\_nodes] \gets 1$
    \State Run $(states\_hist, rounds) = $ Simulate Threshold Cascade($adj, states_0, \theta$)
    \State Compute $n_{\text{failed}} = states\_hist[-1].\text{sum}()$
    \State Set $systemic = (n_{\text{failed}} > S)$
    \State Append results: \{run, $n_{\text{failed}}$, systemic, rounds, seed\_nodes\}
\EndFor
\State Convert results to DataFrame
\If{save path provided}
    \State Save DataFrame to CSV
\EndIf
\State \Return DataFrame
\end{algorithmic}
\end{algorithm}

\begin{algorithm}[H]
\caption{Post-Processing and Plotting for Each Scenario at Given $p_{\text{cc}}$}
\begin{algorithmic}[1]
\State \textbf{Input:} DataFrame $df$, total nodes $N$, systemic threshold $S$, output directory
\State Compute $prob\_systemic = \text{mean}(df[\text{'systemic'}]) \times 100$
\State Compute statistics:
    \State $mean\_cascade = \text{mean}(df[\text{'n\_final\_failed'}])$
    \State $median\_cascade = \text{median}(df[\text{'n\_final\_failed'}])$
    \State $std\_cascade = \text{std}(df[\text{'n\_final\_failed'}])$
    \State $mode\_cascade = \text{mode}(df[\text{'n\_final\_failed'}]).mode[0]$
    \State $mean\_rounds = \text{mean}(df[\text{'rounds'}])$
    \State $median\_rounds = \text{median}(df[\text{'rounds'}])$
\State Plot histogram of $n\_final\_failed$ with bins from 0 to $N+2$ step 2
\State Draw vertical line at $S$ to indicate systemic threshold
\State Label axes and title appropriately
\State Save plot as PNG in output directory
\end{algorithmic}
\end{algorithm}

\section*{Appendix: Additional Figures}
\addcontentsline{toc}{section}{Appendix: Additional Figures}

\begin{figure}[http]
    \centering
    \begin{subfigure}[b]{0.45\textwidth}
        \centering
        \includegraphics[width=\textwidth]{fig-adjacency-matrix.png}
        \caption*{fig-adjacency-matrix.png}
    \end{subfigure}
    \begin{subfigure}[b]{0.45\textwidth}
        \centering
        \includegraphics[width=\textwidth]{fig-degree-distribution.png}
        \caption*{fig-degree-distribution.png}
    \end{subfigure}
    \caption{Figures: fig-adjacency-matrix.png and fig-degree-distribution.png}
    \label{fig:fig-adjacency-matrix-png}
\end{figure}

\begin{figure}[http]
    \centering
    \begin{subfigure}[b]{0.45\textwidth}
        \centering
        \includegraphics[width=\textwidth]{results-11.png}
        \caption*{results-11.png}
    \end{subfigure}
    \begin{subfigure}[b]{0.45\textwidth}
        \centering
        \includegraphics[width=\textwidth]{results-12.png}
        \caption*{results-12.png}
    \end{subfigure}
    \caption{Figures: results-11.png and results-12.png}
    \label{fig:results-11-png}
\end{figure}

\begin{figure}[http]
    \centering
    \begin{subfigure}[b]{0.45\textwidth}
        \centering
        \includegraphics[width=\textwidth]{results-21.png}
        \caption*{results-21.png}
    \end{subfigure}
    \begin{subfigure}[b]{0.45\textwidth}
        \centering
        \includegraphics[width=\textwidth]{results-22.png}
        \caption*{results-22.png}
    \end{subfigure}
    \caption{Figures: results-21.png and results-22.png}
    \label{fig:results-21-png}
\end{figure}

\begin{figure}[http]
    \centering
    \begin{subfigure}[b]{0.45\textwidth}
        \centering
        \includegraphics[width=\textwidth]{results-31.png}
        \caption*{results-31.png}
    \end{subfigure}
    \begin{subfigure}[b]{0.45\textwidth}
        \centering
        \includegraphics[width=\textwidth]{results-32.png}
        \caption*{results-32.png}
    \end{subfigure}
    \caption{Figures: results-31.png and results-32.png}
    \label{fig:results-31-png}
\end{figure}

\begin{figure}[http]
    \centering
    \begin{subfigure}[b]{0.45\textwidth}
        \centering
        \includegraphics[width=\textwidth]{results-41.png}
        \caption*{results-41.png}
    \end{subfigure}
    \begin{subfigure}[b]{0.45\textwidth}
        \centering
        \includegraphics[width=\textwidth]{results-42.png}
        \caption*{results-42.png}
    \end{subfigure}
    \caption{Figures: results-41.png and results-42.png}
    \label{fig:results-41-png}
\end{figure}

\begin{figure}[http]
    \centering
    \begin{subfigure}[b]{0.45\textwidth}
        \centering
        \includegraphics[width=\textwidth]{results-51.png}
        \caption*{results-51.png}
    \end{subfigure}
    \begin{subfigure}[b]{0.45\textwidth}
        \centering
        \includegraphics[width=\textwidth]{results-52.png}
        \caption*{results-52.png}
    \end{subfigure}
    \caption{Figures: results-51.png and results-52.png}
    \label{fig:results-51-png}
\end{figure}

\begin{figure}[http]
    \centering
    \begin{subfigure}[b]{0.45\textwidth}
        \centering
        \includegraphics[width=\textwidth]{results-61.png}
        \caption*{results-61.png}
    \end{subfigure}
    \begin{subfigure}[b]{0.45\textwidth}
        \centering
        \includegraphics[width=\textwidth]{results-62.png}
        \caption*{results-62.png}
    \end{subfigure}
    \caption{Figures: results-61.png and results-62.png}
    \label{fig:results-61-png}
\end{figure}
\end{document}