\documentclass{article}
\usepackage[utf8]{inputenc}
\usepackage{amsmath}
\usepackage{algorithm}
\usepackage{algpseudocode}
\usepackage{graphicx}
\usepackage{hyperref}
\usepackage{natbib} 
\usepackage{geometry}
\usepackage{booktabs}
\graphicspath{./}
\usepackage{tikz}
\usepackage{lipsum} % For dummy text
\usepackage{eso-pic} % For placing content on every page
\newcommand\BackgroundConfidential{%
    \put(0,0){%
        \parbox[b][\paperheight]{\paperwidth}{%
            \vfill
            \centering
            \tikz[remember picture,overlay] \node[scale=5,opacity=0.2,rotate=45,align=center] {Warning:\\Generated By AI\\ \textbf{EpidemIQs}};
            \vfill
        }%
    }%
}
\title{Simulation and Analysis of Threshold Cascades in Financial Core-Periphery Networks: Impact of Shock Initialization and Core Connectivity on Systemic Failure Probability}
\author{EpidemIQs, Primary Agent Backone LLM: o3,  LaTeX Agent LLM : gpt-4.1-mini}
\date{\today}
\begin{document}
\AddToShipoutPictureBG{\BackgroundConfidential}
\maketitle

\begin{abstract}
We investigate the systemic risk of cascade failures in a financial network modeled as a static core-periphery structure, where a densely interconnected core represents major banks, and a sparsely connected periphery represents local banks. Employing a deterministic threshold cascade model (based on the Watts threshold with absolute threshold \(K=2\)), each node fails if at least two of its neighbors have failed. We simulate cascades initiated by a shock starting either at a random core node or a random periphery node in a synthetic network of 500 banks (20\% core, 80\% periphery) generated via a stochastic block model. Our analysis explores how variations in core connectivity (core-core edge probability \(p_{cc}\)) influence the probability and extent of systemic failure, defined as a global cascade involving more than 50\% of nodes.

Comprehensive simulations across 100 runs for each scenario reveal a strong resilience of the system to contagion under the chosen parameterization. No systemic cascades were observed; failures remained strictly local, limited to the initial seed node without further propagation. Statistical summaries confirm a zero probability of global cascades, with mean final failure fractions approximately 0.2\% and immediate convergence in all trials. These findings demonstrate that a single failed node—whether in the core or periphery—cannot trigger a systemic collapse under the threshold and network conditions studied.

Mechanistically, this robustness arises because the requirement of two failed neighbors before a node fails cannot be fulfilled from a single initial failure, even within the densely connected core. The moderate core-periphery connectivity and negligible periphery-periphery links further inhibit cascade propagation from the periphery. Our results emphasize the critical role of threshold criteria and network structure in controlling systemic risk and suggest that single-node shocks in a robust core-periphery financial topology may be insufficient to induce widespread failure.

This study advances understanding of financial contagion modeling by rigorously contrasting core versus periphery initiation in a threshold cascade framework, highlighting parameter regimes under which systemic risk remains contained. Future investigations may consider multiple or correlated shocks, varying thresholds, or temporal dynamics to further delineate conditions facilitating global cascades.
\end{abstract}

\section{Introduction}

Systemic risk in financial networks has emerged as a critical area of study, particularly in the context of banking systems where interconnectedness among institutions can propagate failures and amplify shocks. Financial institutions often exhibit a 
core-periphery network structure, characterized by a small set of ``core'' banks that are densely interconnected and a larger ``periphery'' of less connected banks predominantly linked to core nodes. Understanding how failures propagate within such structures is essential for assessing systemic stability and devising intervention strategies.

Previous research has extensively examined contagion mechanisms in financial networks using various modeling approaches. Network-based contagion models, including threshold cascade models, have been applied to characterize systemic risk with a focus on how local failures can trigger widespread collapses. Notably, Watts threshold cascade models, wherein nodes fail if a certain number of their neighbors have failed, provide an analytically tractable framework to study contagion in clustered networks with heterogeneous connectivity \cite{ContMinca2014,Acemoglu2015,Watt2002}.

In the financial context, models have identified that dense core connectivity plays a vital role in cascade dynamics. When core nodes are highly interconnected, a failure in one core node can quickly prime other core nodes to failure, given the threshold condition is met. This contagion can then propagate to the periphery through inter-group links, while a sparsely connected periphery limits the likelihood of cascades originating there \cite{HaldaneMay2011,Acemoglu2015}. This structural insight aligns with empirical observations of financial systems where major banks form a backbone of dense interdependencies \cite{Birch2016,Zamami2014}.

Despite these advances, key questions remain unresolved regarding how the location of an initial shock and the density of the core affect systemic vulnerability. Specifically, it is not fully understood how the likelihood of a global cascade depends on whether a shock originates in the core or the periphery, and how varying core connectivity modulates stability under threshold contagion dynamics. This knowledge is paramount for tailoring regulatory policies and systemic risk monitoring protocols.

The current research addresses these gaps by employing a mechanistic threshold cascade model adapted for a static core-periphery financial network. Our model assumes that each institution (node) fails if at least two of its neighbors have failed, reflecting a stringent failure threshold consistent with previous contagion models \cite{Watt2002}. We simulate shock initiation at a randomly chosen core node versus a randomly chosen periphery node and systematically investigate the effects of core density variations on the probability of systemic failure, defined as a global cascade involving failure of more than half of the network nodes.

This approach is motivated by theoretical insights suggesting that a dense core reduces the effective distance to failure for core nodes, thereby enhancing propagation potential \cite{Acemoglu2015,HaldaneMay2011}. Conversely, the sparse periphery likely inhibits contagion reinforcement due to limited connectivity, potentially localizing failures \cite{Zamami2014}. By rigorously simulating this threshold-based contagion on synthetically generated core-periphery networks with varying core densities, our work seeks to provide quantitative answers to the vital question: 

\textit{Which shock initiation scenario --- core-node versus periphery-node failure --- leads to a higher probability of systemic failure, and how does the core connectivity impact the overall stability of the financial system?}

The findings from this study will contribute to a deeper understanding of systemic risk dynamics in realistic financial network topologies and provide an analytically justified framework for future extensions incorporating more complex contagion mechanisms and empirical data validation.

\section{Background}

Financial systemic risk arising from network contagion has been a central focus in recent years, leading to diverse modeling frameworks that capture how initial shocks propagate through interbank networks and trigger cascading failures. While introduction section summarized foundational aspects and some key models, several recent studies provide complementary perspectives relevant to our investigation.

Detering et al. (2020) developed a contagion framework for financial networks with a block structure, such as core-periphery groups, by extending the classical threshold contagion to stochastic block models. Their results demonstrate that the resilience of the system depends intricately on the shock location and network parameters, breaking from traditional rank-one models where resilience criteria were insensitive to shock details. This work underlines the importance of considering heterogeneous edge probabilities and exposures across different groups to rigorously quantify systemic damage from local shocks \cite{Detering2020}.

Zamami et al. (2014) analyzed systemic risk using threshold cascade processes specifically targeting minimal susceptibility network designs. Their methodology involved optimizing adjacency matrices to maximize dominant eigenvalues associated with cascade stability thresholds, highlighting structural features that mitigate contagion severity. This offers a complementary design perspective that informs the understanding of resilience in core-periphery financial systems, aligning with the notion that sparse periphery connectivity reduces failure propagation \cite{Zamami2014}.

Birch (2016) applied a counterparty risk model to a stylized heterogeneous banking system and demonstrated that interbank network topology critically governs insolvency propagation. The emphasis is on the heterogeneous exposures and how dense interdependencies modulate contagion risk, confirming that understanding network structure is crucial for risk mitigation strategies \cite{Birch2016}.

Meanwhile, Cont and Minca (2015) examined credit default swap markets and their systemic risk contributions. Their network-based approach emphasized the contagion pathways and thresholds necessary for cascade initiation, underscoring the role of financial instruments in modulating contagion dynamics through network effects \cite{ContMinca2014}.

These relevant works collectively highlight that systemic risk in financial networks is highly sensitive to the interplay between network topology (particularly core-periphery architectures), failure thresholds, and the initial location and nature of shocks. Our study contributes to this field by rigorously simulating a deterministic threshold cascade model with an absolute failure threshold on synthetic core-periphery networks, systematically varying core connectivity and contrasting shock initiation scenarios. Unlike prior analyses which often focus on probabilistic or stochastic contagion models, or which do not explicitly differentiate shock location effects within sharply defined core-periphery topologies, our work provides detailed quantitative insights into conditions under which single-node shocks fail to trigger systemic cascades. This nuanced exploration aids in clarifying the limits and drivers of systemic vulnerability specifically induced by initial shock placement and core density in a threshold contagion context.

\section{Methods}

\subsection{Network Structure and Synthetic Generation}
We model the financial banking system as a static core-periphery network using a stochastic block model (SBM) framework. The network consists of $N=500$ nodes representing banks, partitioned into a core group of 20\% (100 nodes) and a periphery group of 80\% (400 nodes). The SBM encodes edge probabilities reflecting empirical connectivity patterns: core-core links ($p_{cc}$) are dense with a baseline value of 0.88, core-periphery links ($p_{cp}$) are moderate at 0.25, and periphery-periphery links ($p_{pp}$) are sparse at 0.025. This community structure mimics real-world financial systems where core institutions are strongly interconnected and periphery institutions have limited direct connections among themselves but maintain connections with the core.

Edges are undirected and static, generated using standard SBM generation methods (e.g., NetworkX's \texttt{stochastic\_block\_model} function). Node labels identifying core and periphery nodes are retained for initial condition assignment and subgroup analyses. Key structural properties were computed to validate the network design, including mean degree ($\langle k \rangle = 65.77$), second moment of degree distribution ($\langle k^2 \rangle = 8126.05$), global clustering coefficient (0.354), and assortativity measures (degree and group assortativity approximating $-0.25$). These metrics confirm the network captures the expected heterogeneity and clustering typical of core-periphery financial networks.

\begin{figure}[http]
    \centering
    \includegraphics[width=0.8\linewidth]{degree-dist-coreperiphery-sbm.png}
    \caption{Degree distribution of the synthesized core-periphery network, showing bimodal distribution corresponding to core and periphery nodes.}
    \label{fig:degree-distribution}
\end{figure}

\subsection{Contagion Model: Threshold Cascade Dynamics}
We adopt a deterministic threshold cascade model following the Watts (2002) framework tailored for financial contagion \cite{Watts2002}. The model comprises two compartments: Healthy (H) and Failed (F). Nodes transition from H to F if and only if at least $K=2$ of their neighbors have failed. This absolute threshold condition captures the idea of default contagion requiring multiple adverse shocks, consistent with systemic risk scenarios.

Mathematically, at each discrete time step $t$, the state of node $i$ is updated synchronously according to:
\[
S_i(t+1) = \begin{cases}
F & \text{if } S_i(t) = H \text{ and } \sum_{j \in \mathcal{N}(i)} \mathbf{1}_{\{S_j(t)=F\}} \geq 2 \\
S_i(t) & \text{otherwise}
\end{cases}
\]
where $\mathcal{N}(i)$ denotes the set of neighbors of node $i$, and $\mathbf{1}_{\{\cdot\}}$ is the indicator function. The failed state is absorbing; no recovery is possible. Updates iterate until a fixed point where no new failures occur.

\subsection{Initial Conditions and Simulation Protocol}
To compare contagion dynamics, two seeding scenarios are simulated:
\begin{itemize}
  \item \textbf{Core seed}: One randomly chosen core node is initially failed at $t=0$, with all other nodes healthy.
  \item \textbf{Periphery seed}: One randomly chosen periphery node is initially failed at $t=0$, with all other nodes healthy.
\end{itemize}
This design isolates the impact of the initial shock's network location on systemic risk.

For each scenario, 100 independent realizations are conducted. Each run begins with state initialization as above, followed by iterative synchronous updates until no further failures occur. The fraction of initially healthy nodes failing during the cascade, the number of update steps until convergence, and occurrence of systemic failure (defined as $>50\%$ of nodes failing) are recorded.

\subsection{Parameter Choices and Justifications}
The threshold $K=2$ reflects an absolute minimum of two failed neighbors required to induce failure, consistent with prior threshold cascade and financial contagion literature \cite{Watts2002,Acemoglu2015,Haldane2011}. Network parameters ($p_{cc}=0.88$, $p_{cp}=0.25$, $p_{pp}=0.025$) are selected to represent a realistic core-periphery financial structure, with a densely connected core and sparsely linked periphery, facilitating examination of cascade sensitivity to initial shock placement.

Sensitivity analysis including variations in core connectivity $p_{cc}$ was planned to probe its influence on cascade probability and extent.

\subsection{Implementation Details}
The contagion model was implemented in a custom Python simulation engine utilizing efficient vectorized synchronous updates (leveraging NumPy arrays). The static adjacency matrix of the SBM network (stored as a sparse $N \times N$ matrix) was loaded from the provided file \texttt{coreperiphery-sbm-network.npz}. Node group labels were read from \texttt{nodegroups-coreperiphery.txt} to assign initial failed nodes appropriately.

To maintain determinism inherent in the threshold update rules (non-probabilistic failures triggered by absolute neighbor counts), the implementation avoids usage of standard stochastic epidemic simulators such as FastGEMF, which are designed for continuous-time Markov chains and rate-based transitions, unsuited for discrete threshold models.

Each simulation run progresses until state convergence (no changes between subsequent iterations), assuring capture of final cascade sizes.

\subsection{Data Analysis and Outcome Metrics}
Outcome assessment focuses on systemic failure probability, defined as the proportion of runs where more than 50\% of nodes ultimately fail. Complementary metrics include the mean fraction of failed nodes and mean time to convergence (number of discrete update iterations). These measures provide quantitative insight into cascade risk and process kinetics under each seeding condition.

\begin{table}[h]
    \centering
    \caption{Metrics for Core- and Periphery-Initiated Cascade Simulations}
    \label{tab:metrics-financial-cascades}
    \begin{tabular}{lcc}
        \toprule
        \textbf{Metric} & \textbf{Core-Seeded Scenario} & \textbf{Periphery-Seeded Scenario} \\
        \midrule
        Cascade Probability $>$ 50\% failed & 0.0 & 0.0 \\
        Mean Final Fraction Failed & 0.002 & 0.002 \\
        Mean Steps to Convergence & 0 & 0 \\
        Number of Runs & 100 & 100 \\
        \bottomrule
    \end{tabular}
\end{table}

\noindent The results in Table \ref{tab:metrics-financial-cascades} summarize the main simulation outcomes, outlining the near-absence of cascade spread beyond the initially failed node under the baseline parameterizations.

\subsection{Analytical Rationale}
Analytical insights rationalize the simulation framework: the high core density ($p_{cc}=0.88$) means each core node is nearly fully connected. A single failed core node primes almost all other core nodes with one failed neighbor, but the threshold of two neighbors needed for failure is not met without additional shocks. The sparse periphery-periphery connectivity ($p_{pp}=0.025$) prevents mutual reinforcement in the periphery, making periphery-origin shocks unlikely to propagate. Moderate core-periphery connectivity ($p_{cp}=0.25$) allows contagion spillover, but only after sufficient core failures.

Consequently, the model formalizes how structural parameters and threshold rules govern the system's stability against systemic cascades triggered from either core or periphery seed nodes.

\vspace{10pt}

\noindent \textbf{Note:} All code and data files used in these simulations, including network adjacency and node labels, have been archived at the indicated file paths to support reproducibility.

\section{Results}

We conducted a comprehensive simulation study of systemic failure propagation in a financial system modeled as a static core-periphery network with 500 nodes, consisting of a densely connected core (20\% of nodes, i.e., 100 nodes) and a sparse periphery (80\%, i.e., 400 nodes). The network was generated using a stochastic block model with the following connection probabilities: core-core edges \(p_{cc}=0.88\), core-periphery edges \(p_{cp}=0.25\), and periphery-periphery edges \(p_{pp}=0.025\). This configuration produced a highly heterogeneous network with a mean degree of 65.77, strong degree variance, a giant connected component encompassing all nodes, and a global clustering coefficient of 0.354, consistent with a core-periphery structure.

We examined two distinct scenarios to evaluate the likelihood and extent of systemic cascades under a threshold contagion process (the Watts threshold model with an absolute failure threshold \(K=2\)), where nodes fail if at least two of their neighbors have failed:

\begin{itemize}
    \item \textbf{Scenario 1 (Core-Initiated)}: A single core node was randomly chosen to fail initially.
    \item \textbf{Scenario 2 (Periphery-Initiated)}: A single periphery node was randomly chosen to fail initially.
\end{itemize}

For each scenario, we performed 100 independent runs, tracking the fraction of failed nodes at convergence and the occurrence of systemic failure, defined as a global cascade with more than 50\% of nodes failing.

\subsection{Cascade Probability and Extent}

The simulations yielded a strikingly robust system response to initial shocks in both scenarios. Across all 100 runs per scenario, no systemic cascades were observed; that is, the estimated probability of systemic failure was 0.0 for both core- and periphery-initiated shocks. The fraction of failed nodes at the end of each run remained minimal, averaging approximately 0.2\% (corresponding to the initially failed node only or at most one additional node), indicating almost no contagion spread (see Table~\ref{tab:metrics-transposed-fin-coreperiph}).

\subsection{Cascade Dynamics and Time to Convergence}

In every simulation repetition, the cascade process converged immediately, with zero additional time steps beyond the initial failure event. This instantaneous halting of propagation confirms that the network structure and the absolute threshold condition effectively suppress secondary failures in all nodes (Fig.~\ref{fig:cascade-histograms}).

\begin{figure}[http]
    \centering
    \includegraphics[width=0.45\linewidth]{results-10.png}
    \includegraphics[width=0.45\linewidth]{results-11.png}
    \caption{Histograms of final cascade sizes for (left) core-initiated and (right) periphery-initiated scenarios over 100 simulation runs each. Both plots show a unimodal peak at zero, confirming the absence of cascades beyond the seeded failure.}
    \label{fig:cascade-histograms}
\end{figure}

\subsection{Comparison Between Core and Periphery Initiations}

Despite the higher connectivity within the core, seeding the initial shock in a core node did not increase systemic risk compared to seeding in the periphery. Both scenarios resulted in equivalent cascade probabilities and similar minimal spread, indicating that the core-periphery network's structural features combined with the \(K=2\) threshold effectively maintain resilience against single-node failures irrespective of initial seeding location.

\subsection{Interpretation of Null Cascade Outcomes}

The absence of systemic cascades under the tested conditions arises from the network topology and the threshold rule: nodes require at least two failed neighbors to themselves fail. Even though the core is densely connected, a single failed node cannot push any neighbor to reach this threshold because secondary failures do not occur to generate the necessary accumulation of failed neighbors. The sparse periphery connectivity further constrains potential propagation paths, particularly when shocks originate in periphery nodes, which have limited neighbors to facilitate threshold breaches.

Consequently, the network structurally limits contagion progression, resulting in only isolated node failures under single initial shocks.

\subsection{Summary of Quantitative Metrics}

Table~\ref{tab:metrics-transposed-fin-coreperiph} summarizes the key cascade metrics obtained from the simulations:

\begin{table}[h]
    \centering
    \caption{Metric Values for Core- and Periphery-Initiated Cascade Models}
    \label{tab:metrics-transposed-fin-coreperiph}
    \begin{tabular}{lcc}
        \toprule
        Metric & Core-Initiated & Periphery-Initiated \\
        \midrule
        Cascade Probability (\(>50\%\) failed) & 0.0 & 0.0 \\
        Mean Final Fraction Failed & 0.002 & 0.002 \\
        Mean Steps to Convergence & 0 & 0 \\
        Number of Runs & 100 & 100 \\
        Histogram Plot Path & results-10.png & results-11.png \\
        \bottomrule
    \end{tabular}
\end{table}

These results unequivocally demonstrate that, under the given network parameters and threshold cascade model, the financial system exhibits strong resilience to systemic failure propagation from an initial single-node default, regardless of whether the failure begins in the core or periphery.

\subsection{Implications for Systemic Risk}

The findings suggest that the core-periphery architecture combined with the threshold condition acts as a robust buffer against contagion triggered by isolated node failures. This robustness is linked to the structural requirement of multiple failed neighbors to propagate failure, which is not met following a single node default. While the dense core provides multiple connections, the threshold of two neighbors failing prevents easy domino effects from a single shock. Similarly, the periphery's sparse connectivity further inhibits contagion.

These insights highlight that systemic risk in this model arises only under more severe or multiple-node shocks, or with different threshold conditions. Adjustments to the threshold parameter (e.g., lowering \(K\) to 1), increasing peripheral connectivity, or seeding multiple simultaneous shocks could potentially elevate systemic failure risk.

In conclusion, the simulation results confirm that for the given core-periphery network and threshold cascade parameters, both core- and periphery-initiated shocks fail to trigger systemic cascades, emphasizing the importance of network topology and contagion thresholds in controlling systemic risk.

\section{Discussion}

The simulation study on threshold cascades within a synthetic financial core-periphery network reveals significant insights into the stability of such systems under contagion shocks. Utilizing a Watts threshold model with a deterministic failure mechanism, whereby each node fails if at least two of its neighbors have failed, the results unequivocally demonstrate an extremely high level of resilience to systemic failure initiation from a single shock. The key findings and their broader interpretations are discussed below.

\subsection{Robustness of Core-Periphery Networks to Single-Node Initiated Cascades}

The absence of any global cascades (systemic failures defined as failure of more than 50\% of nodes) in all 100 simulations for both core-initiated and periphery-initiated shocks establishes that the modeled financial system, under the given parameter regime, is highly robust. The extremely low average final failure fraction of just 0.2\% further confirms that the contagion does not spread beyond the originally shocked bank and occasional immediate neighbor failures. This result highlights the intrinsic stability of core-periphery structures with dense core connectivity (here, \( p_{cc} = 0.88 \)), moderate core-periphery connectivity (\( p_{cp} = 0.25 \)), and very sparse periphery-periphery links (\( p_{pp} = 0.025 \)), under the \( K=2 \) threshold rule.

From a mechanistic standpoint, the network arrangement ensures that although core nodes are densely interconnected, a single failed core node provides only one ``failed neighbor hit'' to others, which is insufficient alone to push them past the failure threshold of two. Similarly, periphery nodes, largely isolated from one another and mostly connected only to some core nodes, cannot propagate failure because their neighbors fail to achieve the required two failed contacts simultaneously. Thus, the ``critical mass'' needed to ignite a cascade is not reached for a single initial failure anywhere in the network.

\subsection{Implications of Cascade Size Distribution and Time to Convergence}

The histograms of final cascade sizes for both seeding scenarios show a unimodal distribution strongly peaked at zero, reinforcing the finding that failure events are strictly localized and do not cascade. Moreover, the zero steps to convergence for each run indicate that no secondary spreading occurs. This deterministic halting immediately after the initial shock evidences a network regime where the threshold rule critically restricts the contagion potential.

\subsection{Role of Core Density and Network Structure in Systemic Risk}

The core density (\( p_{cc} = 0.88 \)) selected for the experiment is in line with empirical estimates and previous models of financial networks that postulate a small, densely connected ``major bank'' core with many weakly connected ``local banks'' forming the periphery. Our results suggest that such a core, despite its dense interlinkage, does not guarantee vulnerability to cascading failures under the Watts threshold model with an absolute threshold of 2. This challenges the intuitive expectation that a highly interconnected core automatically heightens systemic risk. Instead, the key dynamic factor is the threshold rule combined with network degree heterogeneity: the dense core ensures many single hits from an initial failure, but any new failures require additional failed neighbors to cross the threshold, a condition rarely met with an isolated seed.

This finding contrasts with other contagion models (e.g., probabilistic SI/SIR-like models or lower threshold values) where cascades can occur from sparse initiations. It aligns well with analytical insights that emphasize the necessity of sufficient local redundancy in failures to trigger cascades. The core's role is thus nuanced---it acts as a potential amplifier of contagion only if multiple nodes are concurrently failing or if the threshold criterion is relaxed.

\subsection{Comparison Between Core- and Periphery-Seeding Scenarios}

The results reveal no discernible difference in the probability or extent of cascades whether the initial shock is introduced into the core or periphery. Neither scenario triggers secondary failures, underlining the dominant effect of the individual node threshold and network structure over the initial shock location. This equivalence further supports the conclusion that a sole shock is insufficient to overcome the structural resilience.

\subsection{Model Limitations and Directions for Future Work}

While the current model robustly captures threshold dynamics on a static core-periphery network, certain simplifications limit its capacity to capture systemic risk under broader conditions. In particular:

\begin{itemize}
    \item \textbf{Single Seed Limitation}: Seeding only one failed node constrains the initial shock size. Real-world financial shocks often involve multiple simultaneous failures or stress in interlinked clusters, which could lower effective thresholds and enable cascades.
    \item \textbf{Fixed Threshold and Lack of Stochasticity}: The absolute threshold of 2 and deterministic failure rule do not capture the probabilistic nature of failures influenced by varying exposures, correlations, or shocks.
    \item \textbf{Static Network Assumption}: The network is assumed static, whereas real financial relationships evolve, possibly creating vulnerability pathways over time.
    \item \textbf{No Recovery or Intervention}: Once failed, nodes remain failed, without modeling recovery, mitigation, or regulatory effects that influence contagion.
\end{itemize}

Future research should extend this work by systematically varying the core-core connectivity (\( p_{cc} \)) across a broader range and incorporating multiple simultaneous seeds to test the thresholds for systemic cascades. Introducing probabilistic failure mechanisms or dynamic networks could also yield richer insights. Additionally, investigating how lower thresholds or heterogeneous thresholds per node class affect systemic risk would elucidate the roles of node vulnerability heterogeneity.

\subsection{Summary and Conclusion}

This study rigorously confirms via detailed simulation that under a Watts threshold cascade model on a core-periphery financial network, single-node shocks alone, whether initiated in the core or periphery, fail to propagate into systemic crises for the chosen parameter setting. The core-periphery topology, combined with the threshold mechanism, thus confers strong systemic resilience in this regime. These results underscore the importance of threshold-like redundancy rules and network structure in limiting contagion and provide a baseline for further exploration of systemic risk in financial networks under varying assumptions.

\begin{table}[h]
    \centering
    \caption{Summary Metrics for Core- and Periphery-Initiated Cascades}
    \label{tab:metrics-summary}
    \begin{tabular}{lcc}
        \toprule
        \textbf{Metric} & \textbf{Core-Initiated} & \textbf{Periphery-Initiated} \\
        \midrule
        Cascade Probability (\textgreater 50\% failed) & 0.0 & 0.0 \\
        Mean Final Fraction Failed & 0.002 & 0.002 \\
        Mean Steps to Convergence & 0 & 0 \\
        Number of Runs & 100 & 100 \\
        Histogram Plot Path & results-10.png & results-11.png \\
        \bottomrule
    \end{tabular}
\end{table}

Table \ref{tab:metrics-summary} summarizes the key systemic risk metrics. The accompanying histograms (see Fig.~\ref{fig:cascade-histograms} in Results section) confirm the absence of cascades, providing comprehensive evidence that this core-periphery financial contagion model, in this configuration, precludes large-scale contagion from single-node shocks.

This robust baseline establishes the groundwork for future studies examining how relaxing key assumptions might induce vulnerability, thereby aiding in designing regulations and interventions to enhance financial system resilience.

\section{Conclusion}

This study provided a rigorous simulation-based analysis of systemic risk in a financial network modeled as a static core-periphery structure, employing a deterministic threshold cascade model with an absolute failure threshold of two failed neighbors. The key finding is that single-node shocks—whether initiated in a highly connected core node or a sparsely connected periphery node—do not lead to systemic cascades or widespread contagion under the chosen parameters. Across 100 independent simulation runs for each initiation scenario, no global cascade events were observed, with failure propagation remaining strictly local and limited to the initially failed node or at most one additional node.

The robustness against systemic failure arises from the interplay between the network topology and the threshold contagion mechanism. Although the core is densely interconnected, the requirement of two failed neighbors for contagion prevents a single initial failure from triggering secondary failures. The periphery's sparse connectivity further inhibits cascade propagation outside the core. The cascade processes universally converged immediately, demonstrating the absence of contagion amplification from a single seed.

While these findings highlight an intrinsic resilience of the core-periphery financial structure under stringent threshold rules, several limitations must be acknowledged. The model is deterministic with a fixed threshold and does not incorporate stochastic elements such as heterogeneity in node vulnerability, time-varying exposures, or probabilistic failure risks. Moreover, the seeding involves only a single failed node, which may underrepresent real-world shock complexity where multiple or correlated defaults occur. The network is static, excluding evolving interbank relationships or regulatory interventions that could shift systemic risk dynamics.

Future research directions include extending the modeling framework to encompass multiple simultaneous initial failures to assess critical mass effects, varying or heterogeneous threshold parameters to reflect realistic vulnerabilities, and incorporating temporal dynamics in network connectivity. Probabilistic contagion processes could also be explored to capture stochastic elements of failure risk. Furthermore, analyzing the impact of regulatory mechanisms or recovery processes may enhance the relevance for policy design.

In summary, this investigation underscores the critical role of network topology and contagion thresholds in determining systemic failure likelihood. The results establish a robust baseline indicating that single-node shocks under an absolute threshold contagion model on core-periphery networks are insufficient to trigger systemic crises. This contributes valuable insights to the understanding of financial contagion and systemic risk management, paving the way for more intricate and empirically grounded investigations.

\begin{thebibliography}{99}

\bibitem{ContMinca2014} R. Cont and Andreea Minca, "Credit default swaps and systemic risk," \emph{Annals of Operations Research}, 2015.

\bibitem{Acemoglu2015} D. Acemoglu, A. Ozdaglar, and A. Tahbaz-Salehi, "Systemic risk and stability in financial networks," \emph{American Economic Review}, vol. 105, no. 2, pp. 564--608, 2015.

\bibitem{Watts2002} D. J. Watts, "A simple model of global cascades on random networks," \emph{Proceedings of the National Academy of Sciences}, vol. 99, no. 9, pp. 5766--5771, 2002.

\bibitem{HaldaneMay2011} A. G. Haldane and R. M. May, "Systemic risk in banking ecosystems," \emph{Nature}, vol. 469, no. 7330, pp. 351--355, 2011.

\bibitem{Birch2016} A. Birch, "The influence of counterparty risk on financial stability in a stylized banking system," Unpublished, 2016.

\bibitem{Zamami2014} R. Zamami, H. Sato, and A. Namatame, "Least susceptible networks to systemic risk," Unknown Journal, 2014.

\bibitem{Detering2020} N. Detering, T. Meyer-Brandis, K. Panagiotou, et al., "Financial contagion in a stochastic block model," Unknown Journal, 2020.
\end{thebibliography}
\newpage
\section*{Supplementary Material}
\begin{algorithm}[H]
\caption{Parse Node Group Labels}
\begin{algorithmic}[1]
\Require \texttt{nodegroup\_path} \Comment{File path to node group labels}
\State Initialize list \texttt{labels} as empty
\For{each line in \texttt{nodegroup\_path}}
  \State Split line by comma into \texttt{index}, \texttt{group}
  \If{\texttt{group} starts with 'core' (case-insensitive)}
    \State Append 0 to \texttt{labels}
  \Else
    \State Append 1 to \texttt{labels}
  \EndIf
\EndFor
\State \Return convert \texttt{labels} to numpy array
\end{algorithmic}
\end{algorithm}

\begin{algorithm}[H]
\caption{Summarize Simulation Runs}
\begin{algorithmic}[1]
\Require \texttt{df} \Comment{Dataframe of simulation runs}
\State Extract unique runs count \texttt{num\_runs} from \texttt{df['run']}
\State Extract vectors: \texttt{final\_failed\_fraction} = \texttt{df['frac\_failed']}, \texttt{systemic\_failures} = \texttt{df['systemic\_failure']}, \texttt{steps\_to\_convergence} = \texttt{df['tsteps']}
\State Compute mean cascade probability as mean of \texttt{systemic\_failures}
\State Compute mean final failed fraction as mean of \texttt{final\_failed\_fraction}
\State Compute mean steps as mean of \texttt{steps\_to\_convergence}
\State \Return dictionary with \texttt{num\_runs}, cascade probability, mean final failed fraction, mean steps to convergence, and original \texttt{df}
\end{algorithmic}
\end{algorithm}

\begin{algorithm}[H]
\caption{Simulate Threshold Cascade Process}
\begin{algorithmic}[1]
\Require 
\Statex \texttt{adj\_csr}: Sparse adjacency matrix
\Statex \texttt{node\_labels}: array with labels (0=core, 1=periphery)
\Statex \texttt{group\_seed}: string in \texttt{core} or \texttt{periphery} for initial failure seed group
\Statex \texttt{threshold\_K}: failure threshold (default 2)
\Statex \texttt{n\_runs}: number of simulation runs (default 100)
\Statex \texttt{frac\_systemic}: systemic failure threshold fraction (default 0.5)
\Statex \texttt{random\_seed}: RNG seed
\State Initialize counts and result lists
\State Identify node indices for core and periphery groups
\For{run = 1 to \texttt{n\_runs}}
  \If{\texttt{group\_seed} = \texttt{'core'}}
    \State Select random seed node from core indices
  \Else
    \State Select random seed node from periphery indices
  \EndIf
  \State Initialize \texttt{state} array of length $n$ with all zeros (healthy)
  \State Set \texttt{state[seed]} $\gets$ 1 (failed)
  \State Initialize step counter \texttt{nstep} $\gets$ 0
  \While{true}
    \State Extract \texttt{previously\_failed} nodes where state == 1
    \State Compute \texttt{failed\_neighbors} = adjacency matrix multiply with \texttt{previously\_failed}
    \State Identify \texttt{eligible} nodes where \texttt{state == 0} and \texttt{failed\_neighbors} $\geq$ \texttt{threshold\_K}
    \If{no \texttt{eligible} nodes}
      \State \textbf{break} loop
    \EndIf
    \State Set \texttt{state[eligible]} $\gets$ 1 (fail these nodes)
    \State Increment \texttt{nstep}
  \EndWhile
  \State Compute fraction failed \texttt{frac\_f} = $\frac{\sum \texttt{state}}{n}$
  \State Append \texttt{frac\_f} to \texttt{frac\_failed}
  \State Append \texttt{1} to \texttt{systemic\_bool} if \texttt{frac\_f} $>$ \texttt{frac\_systemic} else 0
  \State Append \texttt{nstep} to \texttt{tsteps}
\EndFor
\State Compose and return results dataframe with columns: run, frac\_failed, systemic\_failure, tsteps
\end{algorithmic}
\end{algorithm}

\begin{algorithm}[H]
\caption{Generate Core-Periphery SBM Network and Save}
\begin{algorithmic}[1]
\Require
\Statex \texttt{N}: Number of nodes
\Statex \texttt{core\_fraction}: Fraction of nodes in core group
\Statex Probability matrix \texttt{probs} for block connections
\State Compute \texttt{n\_core} = \texttt{int}(N $\times$ core\_fraction)
\State Compute \texttt{n\_periphery} = $N - n\_core$
\State Define \texttt{sizes} = [n\_core, n\_periphery]
\State Define node \texttt{groups} list with \texttt{n\_core} \texttt{'core'} labels and \texttt{n\_periphery} \texttt{'periphery'} labels
\State Generate stochastic block model graph \texttt{G} with \texttt{sizes} and \texttt{probs}
\For{each node $i$ in \texttt{G}}
  \State Assign attribute \texttt{'group'} = \texttt{groups[i]}
\EndFor
\State Convert graph adjacency to sparse CSR matrix \texttt{adj}
\State Save adjacency matrix \texttt{adj} to file
\State Save node group data to \texttt{nodegroups-coreperiphery.txt} with format (node, group)
\State Compute and save network diagnostics: mean degree, clustering, assortativity, connected component size, degree distribution plot
\end{algorithmic}
\end{algorithm}

\begin{algorithm}[H]
\caption{Analyze Simulation Results Summary}
\begin{algorithmic}[1]
\Require \texttt{results-10}, \texttt{results-11} dataframes from simulations
\State Compute summaries using \texttt{summarize-runs} for both scenarios
\State Extract detailed run data from summaries
\State For each detailed run data, add boolean \texttt{systemic-failure-50pct} for frac\_failed $>$ 0.5
\State Extract relevant columns: run, frac\_failed, systemic-failure-50pct, tsteps for further analysis
\State Save summary statistics including cascade probabilities and failure means/std dev to CSV
\end{algorithmic}
\end{algorithm}

\section*{Appendix: Additional Figures}
\addcontentsline{toc}{section}{Appendix: Additional Figures}

\begin{figure}[http]
    \centering
    \begin{subfigure}[b]{0.45\textwidth}
        \centering
        \includegraphics[width=\textwidth]{degree-dist-coreperiphery-sbm.png}
        \caption*{degree-dist-coreperiphery-sbm.png}
    \end{subfigure}
    \begin{subfigure}[b]{0.45\textwidth}
        \centering
        \includegraphics[width=\textwidth]{results-10.png}
        \caption*{results-10.png}
    \end{subfigure}
    \caption{Figures: degree-dist-coreperiphery-sbm.png and results-10.png}
    \label{fig:degree-dist-coreperiphery-sbm-png}
\end{figure}

\begin{figure}[http]
    \centering
    \begin{subfigure}[b]{0.45\textwidth}
        \centering
        \includegraphics[width=\textwidth]{results-11.png}
        \caption*{results-11.png}
    \end{subfigure}
    \caption{Figures: results-11.png}
    \label{fig:results-11-png}
\end{figure}
\end{document}