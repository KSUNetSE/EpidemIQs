\documentclass{article}
\usepackage[utf8]{inputenc}
\usepackage{amsmath}
\usepackage{algorithm}
\usepackage{algpseudocode}
\usepackage{graphicx}
\usepackage{hyperref}
\usepackage{natbib} 
\usepackage{geometry}
\usepackage{booktabs}
\graphicspath{./}
\usepackage{tikz}
\usepackage{lipsum} % For dummy text
\usepackage{eso-pic} % For placing content on every page
\newcommand\BackgroundConfidential{%
    \put(0,0){%
        \parbox[b][\paperheight]{\paperwidth}{%
            \vfill
            \centering
            \tikz[remember picture,overlay] \node[scale=5,opacity=0.2,rotate=45,align=center] {Warning:\\Generated By AI\\ \textbf{EpidemIQs}};
            \vfill
        }%
    }%
}
\title{Network-Induced Oscillations in a Memory-Augmented SIRS Model for Viral Internet Trend Propagation on Watts-Strogatz Small-World Networks}
\author{EpidemIQs, Primary Agent Backone LLM: gpt-4.1,  LaTeX Agent LLM : gpt-4.1-mini}
\begin{document}
\AddToShipoutPictureBG{\BackgroundConfidential}
\maketitle

\begin{abstract}
This study investigates the dynamics of a viral internet trend modeled as a susceptible-infectious-recovered-susceptible (SIRS)-like process on a Watts-Strogatz small-world network. Users transition through three states: Unaware (U), Posting (P) that corresponds to active viral spread, and Fatigued (F), representing temporary immunity or loss of interest. A critical extension includes a forgetting mechanism, whereby Fatigued users revert to the Unaware state at a slow rate \(\xi\), introducing the potential for cyclic resurgence of posting activity. We analyze if this forgetting cycle induces persistent oscillations—stable waves of popularity—or leads to convergence to a steady endemic state.

The model is parameterized with network properties representative of realistic social platforms (\(N=10{,}000\) nodes, mean degree \(k=8\), rewiring probability \(p=0.1\)), supporting both local clustering and short path lengths. Transition rates \(\beta\) (posting initiation), \(\gamma\) (fatigue onset), and \(\xi\) (forgetting) are systematically varied and simulated using stochastic continuous-time Markov processes. Simulations on large networks reveal a rapid initial surge in posting prevalence followed by monotonic decay to a near-zero equilibrium, consistent with classical well-mixed SIRS theory, exhibiting no sustained oscillations.

Downsized simulations on smaller networks (\(N=1{,}000\)) enable detailed parameter scanning, revealing two dominant dynamic regimes. First, oscillatory regimes manifest as damped recurring waves of Posting prevalence, persisting over extended periods before eventual decay. These oscillations are synchronized by network structural properties and supported by moderate transmission rates, lower forgetting rates, and intermediate fatigue rates. Second, monotonic regimes feature a single pronounced Posting peak followed by decay without subsequent resurgence. Intermediate noisy regimes with marginal damped oscillations are also observed under certain parameter configurations.

Spectral analysis corroborates periodicities in the Posting prevalence time series, with oscillation periods on the order of 40--55 days. The presence or absence of oscillations depends critically on the interplay of the forgetting rate with transmission and fatigue parameters, as well as the underlying network topology.

Overall, our results demonstrate that while the classical deterministic SIRS model precludes stable limit-cycle oscillations, embedding the process on a realistic small-world contact network with forgetting enables network-induced damped oscillations in viral trend activity. These findings elucidate mechanisms underpinning the cyclic popularity of social media phenomena and highlight the importance of network topology and memory loss in contagion dynamics.
\end{abstract}

\section{Introduction}

The study of contagion processes on complex networks has been of paramount interest in recent years, driven by the need to understand the dynamics of infectious diseases, information diffusion, and social behavior propagation. Epidemic modeling has evolved from the classical compartmental differential equation models assuming homogeneous mixing to more intricate network-based models that incorporate heterogeneity and structure in the contact patterns among individuals \cite{EVolz2007SIRdynamics,Apolloni2013AgeSpecific}. The development and application of Susceptible-Infected-Recovered-Susceptible (SIRS)-type frameworks have been extended beyond biological pathogens to model social contagions such as the viral spread of internet memes or trends, which exhibit similar dynamics but with distinct features, including memory loss and repeated susceptibility \cite{TangWu2019Fragility}.

Traditional mean-field SIRS models describe a population partitioned into compartments representing health or awareness states, typically assuming uniform mixing and constant transition rates between states. Although such models capture the overall macroscopic behavior, they fail to represent the complexities arising from social network structures, individual heterogeneity, and temporal correlations in contacts \cite{RahmandadSterman2004Heterogeneity,Rocha2017Sampling}. Notably, in the classical SIRS model with constant rates, oscillatory dynamics invariably relax to a stable endemic equilibrium without sustained limit cycles \cite{TangWu2019Fragility}. However, incorporating heterogeneity and network topology, particularly small-world structures characterized by high clustering and short path lengths, can fundamentally alter system behavior by inducing persistent oscillations and asynchronous waves of infection or posting prevalence \cite{Baek2007Oscillatory,TangWu2019Fragility,Ki2007Oscillatory}.

Small-world networks, as conceptualized by Watts and Strogatz, aptly describe many real-world social systems, combining local clustering with random long-range shortcuts that facilitate rapid global spread and maintain community structures \cite{LaurentBadel2020}. Such topologies are conducive to the emergence of synchronized behavior due to recurrent excitation in clustered groups, separated by shortcuts which promote non-local spread. The role of network topology is critical in contagion processes, where the degree distribution, clustering coefficient, and path length influence both the epidemic threshold and eventual outbreak size \cite{EVolz2011Effects,Huang2005SmallWorld}.

Recent studies have extended epidemic modeling to incorporate information dynamics on networks, introducing compartments mapping to user states such as Unaware, Posting (actively spreading), and Fatigued (immune or uninterested), with transitions that include forgetting or loss of immunity, thereby completing a cycle analogous to SIRS epidemiological models \cite{Yamakou2024InverseStochastic,Ki2007Oscillatory,TangWu2019Fragility}. This model framework captures the lifecycle of a viral internet trend, where individuals can repeatedly become susceptible after losing interest and later re-engage. The dynamics on small-world networks demonstrate a complex interplay of infection (posting) rates, fatigue (loss of enthusiasm), and forgetting (loss of immunity), where certain parameter regimes yield persistent oscillations or waves of popularity. These oscillations, often damped but recurrent due to network effects, contrast starkly with the monotonic convergence to steady state in homogeneous well-mixed populations \cite{Baek2007Oscillatory,RahmandadSterman2004Heterogeneity}.

The introduction of a forgetting rate \( \beta_e \) (transition from Fatigued back to Unaware) is essential to capture the scenario of loss of immunity or waning interest, allowing the system to cycle through states repeatedly. The presence of a non-zero forgetting rate in SIRS dynamics on small-world contact networks has been shown to precipitate network-induced synchronization phenomena, manifesting as stable or damped oscillations in the fraction of Posting individuals \cite{TangWu2019Fragility,Baek2007Oscillatory}. However, the existence, stability, and characteristics of such oscillations depend sensitively on the transmission rates \( \beta_2 \), fatigue rates \( \beta_3 \), forgetting rates \( \beta_e \), and network topology parameters such as rewiring probability and mean degree.

Despite extensive theoretical and numerical work on epidemic and information spreading on networks, open questions remain concerning the mechanisms and parameter regimes that enable persistent viral popularity waves in social media-like systems. In particular, understanding how the cyclic transitions of social contagion states on small-world networks yield either persistent oscillations representing periodic resurges of interest or convergence to steady-state endemic levels is critical for predicting and controlling the spread of information.

This research addresses the question: 

\textit{Does introducing a non-zero forgetting rate in an SIRS-like model of a viral internet trend on a Watts-Strogatz small-world network induce stable (persistent) oscillations in the fraction of Posting individuals (waves of popularity), or does it lead to a steady endemic state? Furthermore, how do the transition rates (transmission \( \beta_2 \), fatigue \( \beta_3 \), forgetting \( \beta_e \)) and network structural parameters (mean degree \( k \), rewiring probability \( p \)) influence these dynamic regimes?}

To answer this, we construct and analyze a mechanistic SIRS model for social contagion where individuals cycle through Unaware (U), Posting (P), and Fatigued (F) compartments with stochastic transitions. The model is embedded on a large Watts-Strogatz static small-world network reflecting social media connectivity patterns. We employ both analytical stability analyses of the corresponding well-mixed ODE system and extensive stochastic simulations on the small-world network to explore how network structure and model parameters shape the emergence of oscillations and endemic equilibria. 

Our approach builds upon and integrates insights from prior studies examining epidemic thresholds and transmission dynamics on dynamically evolving and heterogeneous networks \cite{EVolz2007SIRdynamics,Apolloni2014Metapopulation,RahmandadSterman2004Heterogeneity}, the impact of network clustering and contact heterogeneity on information spread \cite{Volz2011Effects,Yamakou2024InverseStochastic}, and empirical evidence of social contagion oscillations driven by memory and forgetting mechanisms. By characterizing the parameter and network regimes that yield persistent posting waves versus steady endemic states, this work contributes to a deeper understanding of viral dynamics in online social contexts and informs potential intervention strategies to foster or mitigate trend propagation.

\section{Background}

The modeling of contagion dynamics using epidemic frameworks has been extensively applied to understand not only biological pathogens but also social contagion phenomena such as information and viral internet trend propagation. While classical compartmental models like Susceptible-Infected-Recovered-Susceptible (SIRS) provide a baseline for capturing the cyclical nature of susceptibility and immunity, their mean-field assumptions of homogeneous mixing limit their ability to capture complex temporal and spatial patterns observed in realistic social systems.

Recent advances emphasize the critical role of network topology in shaping contagion processes. In particular, small-world networks, characterized by high local clustering and short average path lengths, reflect realistic social connectivity patterns and influence the spread and temporal evolution of viral phenomena. Embedding SIRS-type models on small-world networks can induce richer dynamics, including oscillatory and asynchronous waves of infections or posting activity, which classical well-mixed models fail to generate stably \cite{Baek2007Oscillatory,TangWu2019Fragility}.

An important extension to social contagion modeling is the incorporation of memory effects and forgetting mechanisms that allow individuals to lose immunity (or interest) and return to susceptibility. This factor closes the SIRS cycle and permits repeated reinfection or re-engagement, critical for modeling repeated viral trend resurgences on social media platforms. Studies indicate that including a forgetting rate in SIRS dynamics on small-world networks can give rise to network-induced synchronization phenomena such as stable or damped oscillations in the prevalence of active spreaders \cite{TangWu2019Fragility,Baek2007Oscillatory}.

Beyond classical SIRS frameworks, contemporary work explores enhanced propagation models integrating additional states and refinement of transition processes to better capture real-world dynamics. For example, approaches that integrate dynamic node influence with local network structure improve identification of influential spreaders, relevant for optimizing viral dissemination or suppression \cite{Houshiyu2025}. Additionally, models that incorporate forgetting or waning immunity states have been used to simulate the decay of cognitive immunity and intervention strategies such as prebunking against misinformation in social networks \cite{Raij2025}.

These efforts collectively highlight that the convergence or oscillation outcomes of social contagion depend critically on the interplay between transmission rates, fatigue or recovery rates, forgetting rates, and the underlying contact network architecture. While prior work reveals the emergence of oscillations under specific parameter and network conditions, the existence and stability of such oscillations in the context of memory-augmented SIRS models on Watts-Strogatz small-world networks remains an area requiring further systematic exploration.

In this study, by embedding a memory-augmented SIRS-like model on large-scale Watts-Strogatz small-world networks, we aim to clarify how the forgetting mechanism interacts with network topology and transition parameters to induce or suppress oscillations in viral trend posting prevalence. Compared to prior models, our focus on the mechanistic roles of stochastic Markov transitions and detailed parameter sweeps over realistic network structures enables a more nuanced understanding of the parameter regimes governing oscillatory versus monotonic viral dynamics.

\section{Methods}

In this study, we investigate the dynamics of a viral internet trend propagation modeled as a susceptible-infected-recovered-susceptible (SIRS) type process over a social contact network. The compartments used are: Unaware (U), Posting (P), and Fatigued (F), corresponding to users unaware of the trend, actively posting and spreading the trend, and fatigued users who have lost interest or immunity to the trend, respectively. This setup constitutes a cyclic SIRS-like contagion with memory loss, modeled by the chain of transitions, \( U \to P \to F \to U \), with transition rates \(\beta\), \(\gamma\), and \(\xi\) respectively. 

\subsection{Mathematical Model}

The model is formalized as follows. In the classical well-mixed ordinary differential equation (ODE) version, the dynamics are captured by:
\begin{equation}\label{eq:model-odes}
\frac{dU}{dt} = \xi F - \beta U P, \quad
\frac{dP}{dt} = \beta U P - \gamma P, \quad
\frac{dF}{dt} = \gamma P - \xi F,
\end{equation}
with the mass conservation constraint \( U + P + F = 1 \). Here:

\begin{itemize}
  \item \(\beta\) is the transmission rate of the trend per contact between a Posting and an Unaware individual,
  \item \(\gamma\) is the fatigue rate at which Posting individuals become Fatigued,
  \item \(\xi\) is the forgetting rate at which Fatigued individuals return to the Unaware state.
\end{itemize}

Theoretical analysis of this system reveals that it admits an endemic equilibrium with damped oscillations, but no stable limit cycles (i.e., no persistent oscillations) due to absence of Hopf bifurcations in the classical ODE SIRS setting with constant rates.

\subsection{Network Model}

To capture realistic social connectivity and the impact of network topology on the contagion dynamics, we embed the SIRS model on a static Watts-Strogatz small-world network. The network is constructed with the following parameters:

\begin{itemize}
  \item Number of nodes: \(N = 10,000\), representing individual users,
  \item Mean degree: \(k = 8\), representing average social ties per user,
  \item Rewiring probability: \(p = 0.1\), producing a balance between clustering and shortcuts,
  \item The network exhibits strong small-world properties including high clustering coefficient (0.47) and low average shortest path length (approximately 7).
\end{itemize}

This topology is well-suited to represent the rapid global spread of information alongside community echo chambers typical in online social media platforms.

The contact network is fully connected with a giant connected component (GCC) of size 10,000 nodes, and a narrow degree distribution centered at \(k=8\), minimizing heterogeneity bias in transmission.

\subsection{Stochastic Simulation Framework}

The SIRS process is simulated as a continuous-time Markov chain (CTMC) on the Watts-Strogatz network, capturing the stochastic nature of user interactions.

Transitions are defined as:
\begin{itemize}
  \item Edge-based transmission \(U \to P\): an Unaware node becomes Posting at a rate \(\beta\) multiplied by the number of Posting neighbors,
  \item Node-based fatigue transition \(P \to F\): Posting nodes become Fatigued at rate \(\gamma\),
  \item Forgetting transition \(F \to U\): Fatigued nodes forget the trend and revert to Unaware at rate \(\xi\).
\end{itemize}

For each simulation scenario:

\begin{itemize}
  \item Initial conditions are set with 99\% Unaware, 1\% Posting, and 0\% Fatigued nodes, with Posting nodes randomly distributed.
  \item Parameter scans are performed over sets of \(\beta\), \(\gamma\), and \(\xi\) covering values \(\beta \in \{0.15, 0.3, 0.5\}\), \(\gamma \in \{0.08, 0.15\}\), and \(\xi \in \{0.005, 0.015\}\).
  \item Network size for detailed parameter scans is \(N=1000\) due to computational constraints, while a baseline simulation at full size \(N=10,000\) is run for diagnostics.
  \item Each scenario is replicated with six stochastic realizations to estimate mean dynamics and variability.
  \item Simulations progress up to \(t_{\max} = 120\) time units to capture multiple potential oscillatory cycles.
\end{itemize}

\subsection{Simulation Implementation and Data Collection}

Simulations are implemented leveraging state-of-the-art CTMC network simulators capable of handling large sparse graphs stored in compressed sparse formats (e.g., .npz files). The Watts-Strogatz graph is loaded from compressed files, and the adjacency structure is used for efficient evaluation of infection pressure on Unaware nodes.

Time-series are recorded for each compartment's population count at fine temporal resolution. For each scenario, the following outputs are saved:

\begin{itemize}
  \item Mean and 90\% confidence interval (CI) envelopes of susceptible (U), infected (P), and fatigued (F) fraction over time,
  \item Key epidemic metrics including peak Posting prevalence, peak time, episode duration (time with \(P > 1\%\) of population), and final compartment sizes,
  \item Quantification of oscillatory behavior by counting peaks and computing oscillation periods via spectral analysis.
\end{itemize}

\subsection{Analytical Approaches}

Complementing simulations, analytic reasoning is applied to evaluate stability and oscillatory regimes. The Jacobian of the ODE system at endemic equilibrium is linearized to determine eigenvalue spectra, identifying parameter regimes allowing oscillatory transients. However, it is noted that the classical SIRS with constant rates does not permit stable limit cycles, consistent with simulation findings.

For the networked system, no closed-form analytic formula for oscillations exists due to complex interplay between network topology, stochasticity, and local interactions. Thus, spectral methods and time series from simulations are used to characterize emergent oscillations and their dependence on parameters \(\beta, \gamma, \xi\), mean degree \(k\), and rewiring \(p\).

\subsection{Summary of Model Parameters and Simulation Protocol}

\begin{table}[h]
\centering
\caption{Summary of Model Parameters and Network Characteristics}
\label{tab:model-params}
\begin{tabular}{l c p{7cm}}
\toprule
Parameter & Value & Description \\
\midrule
\(N\) & 10,000 (baseline) / 1,000 (scans) & Number of nodes in Watts-Strogatz network \\
\(k\) & 8 & Mean node degree \\
\(p\) & 0.1 & Watts-Strogatz network rewiring probability \\
\(\beta\) & Varied (0.15, 0.3, 0.5) & Transmission rate per P neighbor \\
\(\gamma\) & Varied (0.08, 0.15) & Fatigue rate from Posting to Fatigued \\
\(\xi\) & Varied (0.005, 0.015) & Forgetting rate from Fatigued back to Unaware \\
Initial Conditions & 99\% U, 1\% P, 0\% F & Randomly seeded initial infected \\
Simulation time \(t_{\max}\) & 120 units & Duration of each simulation run \\
Replications & 6 & Number of independent stochastic realizations per parameter set \\
\bottomrule
\end{tabular}
\end{table}

This methods framework ensures that the question of whether the forgetting mechanism induces persistent oscillations in Posting prevalence or leads to steady endemic states is addressed by combining rigorous analytic characterization and comprehensive stochastic simulations over realistic network structures, leveraging mechanistic insights and empirical parameter ranges.

Figures illustrating network diagnostics (degree distribution and clustering vs.\ path length) and example time series traces are provided in the Results section (Figures ws-degree-dist.png, ws-clustering-path.png, results-00.png).

\section{Results}

This section presents a comprehensive analysis of the simulation outcomes of the SIRS-like \( (U \rightarrow P \rightarrow F \rightarrow U) \) viral internet trend propagation model executed on Watts-Strogatz small-world networks. The primary focus is on the dynamics of the Posting (P) population, aiming to elucidate whether the inclusion of a forgetting rate \( \xi \) induces stable oscillations or leads to a steady endemic state.

\subsection{Network Characterization}

The simulations employed Watts-Strogatz networks with \( N=10,000 \) nodes for the baseline scenario and \( N=1,000 \) nodes for the detailed parameter scans. The network had a mean degree \( k=8 \) and rewiring probability \( p=0.1 \), producing a structure with strong small-world characteristics such as high clustering coefficient (0.471) and low mean shortest path length (6.99), which are consistent with empirical social networks facilitating viral spreading (see Fig.~\ref{fig:degree-dist} and Fig.~\ref{fig:clustering-path}).

\begin{figure}[http]
    \centering
    \includegraphics[width=0.48\textwidth]{ws-degree-dist.png}
    \caption{Degree distribution of the Watts-Strogatz network used in simulations. The narrow distribution confirms a homogeneous social contact structure with mean degree \( k=8 \).}
    \label{fig:degree-dist}
\end{figure}

\begin{figure}[http]
    \centering
    \includegraphics[width=0.48\textwidth]{ws-clustering-path.png}
    \caption{Structural diagnostics of the Watts-Strogatz network showing the clustering coefficient and mean shortest path length, supporting both local echo-chambers and rapid global information diffusion.}
    \label{fig:clustering-path}
\end{figure}

\subsection{Baseline Scenario on Large Network (N=10,000)}

In the baseline simulation (Scenario 00) with parameters \( \beta = 0.15 \), \( \gamma = 0.08 \), and \( \xi = 0.005 \), the Posting population exhibited a rapid rise, peaking sharply around day 12 at approximately 65\% of the population. This was followed by a monotonic decline to negligible levels, with no evidence of sustained oscillations (Fig.~\ref{fig:results-00}). The final Fatigued population stabilized at a high level, indicating lasting immunity or loss of interest.

Quantitative metrics for this scenario include:

\begin{itemize}
    \item Peak Posting: 6,400 individuals
    \item Peak Time: 12 days
    \item Episode Duration (P $>$ 1\%): 45 days
    \item Final Posting \( (t_{\max}) \): near zero
    \item Doubling Time (early exponential growth): 0.8 days
\end{itemize}

This behavior conforms to mean-field expectations where oscillations are damped due to homogeneous mixing in a large population.

\begin{figure}[http]
    \centering
    \includegraphics[width=0.6\textwidth]{results-00.png}
    \caption{Time series of compartment prevalences in the baseline scenario (N=10,000) showing a single peak in Posting (P) followed by monotonic decay to steady state.}
    \label{fig:results-00}
\end{figure}

\subsection{Parameter Scan Results on Medium Network (N=1,000)}

Simulations varying \( \beta \), \( \gamma \), and \( \xi \) on smaller networks revealed two distinct dynamic regimes:

\paragraph{Oscillatory Regime} 
Scenarios 03, 05, 07, 09, and 12 displayed prominent damped oscillations in the Posting population with 3 to 6 peaks spaced approximately 40 to 55 days apart (Fig.~\ref{fig:results-03} to Fig.~\ref{fig:results-12}). Initial peak prevalences ranged between 40\% and 83\%, with oscillation amplitudes decaying over time. Increasing stochastic effects were observed late in the simulations, revealed by widening confidence intervals.

Key spectral analyses confirmed dominant oscillation periods aligning with visual periodicity (\(\sim 45\) days). These oscillations result from network-induced synchronization facilitated by small-world clustering and the interplay of fatigue and forgetting rates.

\paragraph{Monotonic Regime}
Scenarios 04, 06, 08, 10 and partly 11 exhibited a rapid single Posting peak followed by a smooth decline without further oscillations. Peak prevalences were similar to the oscillatory group but lacked recurrent outbreaks (Fig.~\ref{fig:results-04} to Fig.~\ref{fig:results-11}). The confidence intervals were narrower, indicating more deterministic dynamics.

Scenario 11 showed marginal oscillations with significant stochasticity, representing a transition between regimes.

\begin{figure*}[ht]
    \centering
    \includegraphics[width=0.32\textwidth]{results-03.png}
    \includegraphics[width=0.32\textwidth]{results-05.png}
    \includegraphics[width=0.32\textwidth]{results-07.png}
    \caption{Oscillatory scenarios exemplifying damped Posting waves over time with visible periodic peaks and 90\% confidence intervals.}
    \label{fig:results-03}
\end{figure*}

\begin{figure*}[ht]
    \centering
    \includegraphics[width=0.32\textwidth]{results-09.png}
    \includegraphics[width=0.32\textwidth]{results-12.png}
    \includegraphics[width=0.32\textwidth]{results-11.png}
    \caption{Additional oscillatory and marginal oscillatory scenarios demonstrating Posting prevalence dynamics. Scenario 11 shows ambiguous oscillations with higher stochastic noise.}
    \label{fig:results-12}
\end{figure*}

\begin{figure*}[ht]
    \centering
    \includegraphics[width=0.32\textwidth]{results-04.png}
    \includegraphics[width=0.32\textwidth]{results-06.png}
    \includegraphics[width=0.32\textwidth]{results-08.png}
    \caption{Monotonic scenarios showing single Posting peak and decay patterns with minimal secondary peaks.}
    \label{fig:results-04}
\end{figure*}

The detailed metrics across all scenarios are tabulated in Table~\ref{tab:metrics-transposed}, comparing peak prevalences, timing, episode durations, oscillatory peaks and periods, and related epidemiological measures.

\begin{table*}[ht]
    \centering
    \caption{Summary of Key Metrics for SIRS-UPF Network Simulations across Parameter Scenarios}
    \label{tab:metrics-transposed}
    \begin{tabular}{lccccccccccc}
        \toprule
        Metric & 00 & 03 & 04 & 05 & 06 & 07 & 08 & 09 & 10 & 11 & 12 \\
        \midrule
        Network Size \( N \) & 10000 & 1000 & 1000 & 1000 & 1000 & 1000 & 1000 & 1000 & 1000 & 1000 & 1000 \\
        \( \beta \) & 0.15 & 0.15 & 0.15 & 0.3 & 0.3 & 0.3 & 0.3 & 0.5 & 0.5 & 0.5 & 0.3 \\
        \( \gamma \) & 0.08 & 0.08 & 0.15 & 0.08 & 0.08 & 0.15 & 0.15 & 0.08 & 0.08 & 0.15 & 0.1 \\
        \( \xi \) & 0.005 & 0.005 & 0.015 & 0.015 & 0.015 & 0.005 & 0.015 & 0.015 & 0.015 & 0.005 & 0.01 \\
        Peak Posting & 6400 & 440 & 750 & 830 & 630 & 850 & 800 & 830 & 600 & 150 & 140 \\
        Peak Time (days) & 12 & 12 & 8 & 10 & 10 & 12 & 10 & 12 & 10 & 10 & 12 \\
        Epidemic Duration (days) & 45 & 96 & 48 & 120 & 44 & 116 & 50 & 110 & 47 & 65 & 108 \\
        Final Posting \( (t_{\max}) \) & \(<1\) & 20 & 0 & 30 & 23 & 40 & 7 & 35 & 17 & 17 & 19 \\
        Final Fatigued (F) & 7600 & 670 & 790 & 770 & 620 & 780 & 740 & 760 & 690 & 720 & 690 \\
        Doubling Time (days) & 0.8 & 0.6 & 0.3 & 0.5 & 0.3 & 0.4 & 0.4 & 0.5 & 0.4 & 0.8 & 0.7 \\
        Oscillation Peaks & 1 & 4 & 1 & 5 & 1 & 5 & 1 & 4 & 1 & 3 & 4 \\
        Oscillation Period (days) & -- & 44 & -- & 46 & -- & 51 & -- & 48 & -- & 50 & 47 \\
        Amplitude (1st peak) & 6400 & 440 & 750 & 830 & 630 & 850 & 800 & 830 & 600 & 150 & 140 \\
        90\% CI Width (peak P) & 150 & 31 & 51 & 33 & 42 & 41 & 39 & 34 & 30 & 70 & 54 \\
        90\% CI Width (final P) & \(<3\) & 6 & \(<1\) & 5 & 3 & 5 & 2 & 6 & 3 & 11 & 10 \\
        \bottomrule
    \end{tabular}
\end{table*}

\subsection{Interpretation of Results}

The simulations demonstrate that the viral trend SIRS model on a static Watts-Strogatz network can exhibit both damped oscillations and monotonic decay depending on parameters: 

\begin{itemize}
    \item \textbf{Oscillatory behavior} occurs chiefly when \( \beta \) is moderate to high, \( \gamma \) is moderate to high, and \( \xi \) is low, supporting sustained synchronization of Posting waves across the clustered network topology.
    \item \textbf{Monotonic dynamics} arise at parameter combinations where higher forgetting rates and lower fatigue combine to suppress recurrent outbreaks, consistent with classical ODE predictions.
    \item \textbf{Network size} strongly influences observed dynamics; large populations suppress oscillations, conforming to mean-field behavior, while smaller populations allow quasi-synchronized recurrent Posting outbreaks.
\end{itemize}

Confidence intervals widen during later cycles reflecting increasing stochastic variability as susceptible numbers deplete. The oscillation periods (approximately 40--55 days) resonate with theoretical predictions regarding delays induced by fatigue and forgetting processes.

Overall, the findings validate theoretical assertions that network structure and demographic stochasticity can induce persistent oscillatory features absent in well-mixed models. This highlights the importance of considering realistic social network topology in modeling information propagation dynamics.

\section*{Summary}

The introduced forgetting mechanism \( \xi \) in the SIRS viral trend model, when combined with network structures of intermediate clustering and path length typical of Watts-Strogatz small-world networks, can induce network-synchronized Posting prevalence oscillations. However, these oscillations are damped and strongly parameter-dependent, disappearing in larger well-mixed-like populations.

The detailed parameter scan reveals clear regimes separating oscillatory and monotonic dynamics, with quantitative metrics provided for reproducibility and further investigation.

\noindent\textbf{Note:} The data and figures referenced in this section correspond to simulation outputs fully documented and stored as per the simulation logs and are reproducible following the file mappings specified earlier.

\section{Discussion}

The present study investigates the dynamics of a viral internet trend modeled through a SIRS-like network contagion process on a Watts-Strogatz small-world network. Individuals in the network transition from an Unaware state (U) to Posting (P), then to Fatigued (F), before eventually forgetting and returning to the Unaware state with a slow rate $\xi$. By embedding this SIRS-like cycle on a social network with realistic small-world properties and examining a range of parameter regimes, the study reveals insights into whether such a forgetting mechanism can lead to persistent oscillations in Posting prevalence or whether the system settles into a steady endemic state.

\subsection{Role of Network Structure and Model Dynamics}
The Watts-Strogatz network used in simulations exhibits key features of social networks: high clustering coefficient (0.47), moderate mean degree (8), and short average path length ($\sim 7$) facilitating both local echo chambers and global connectivity. This structure introduces nontrivial spatial correlations absent in well-mixed populations and consequently shapes the spreading dynamics profoundly. As corroborated by the diagnostics (Figure~\ref{fig:degree-dist} and Figure~\ref{fig:clustering-path}), the network is homogeneously connected but retains enough local clustering to enable the emergence of complex epidemic patterns due to network-induced correlation effects.

When implemented as per-edge and node-specific Markov transitions (with $\beta$, $\gamma$, and $\xi$ rates governing $U \to P$, $P \to F$, and $F \to U$ transitions respectively), the model captures three critical mechanisms of social contagion with memory loss -- transmission, fatigue, and forgetting. This minimal yet complete compartmentalization aligns with canonical SIRS dynamics, extended here by a networked context and the forgetting mechanism that closes the cycle.

\subsection{Comparison Between Well-Mixed and Networked Dynamics}
The classical well-mixed SIRS model with constant rates is known to produce at most damped oscillations approaching an endemic equilibrium, without stable limit cycles. This analytical understanding is reinforced by the large-network ($N=10,000$) simulation results (scenario 00), where the Posting population exhibits a single prominent peak followed by monotonic decay to near zero steady state (Figure~\ref{fig:results-00}). This confirms that in the limit of large, homogeneously mixed populations or nearly random networks (high rewiring), oscillations induced by the forgetting process are transient and not sustainable.

Contrastingly, simulations on smaller but still sufficiently large networks ($N=1000$) exhibit two distinct regimes, governed primarily by parameter combinations ($\beta$, $\gamma$, and $\xi$), illustrating the crucial role of network topology and finite size in enabling or suppressing oscillatory regimes. The presence of local clustering and intermediate rewiring ($p=0.1$) fosters network synchronization effects, creating conditions for transient yet recurrent Posting waves.

\subsection{Characterizing Oscillatory and Monotonic Regimes}
Results across scenarios 03 to 12 manifest a bifurcation-like pattern in model behavior: oscillatory Posting prevalence characterized by multiple damped waves in some parameter regimes, and monotonic single-peak epidemic curves in others.

The oscillatory regime, exemplified by scenarios 03, 05, 07, 09, and 12, reveals multiple Posting peaks spaced approximately 40 to 55 days apart with diminishing amplitude (Figures~\ref{fig:results-03} and \ref{fig:results-12}). The persistence of these oscillations over time, albeit damped, indicates the emergence of network-induced quasi-cycles where wave synchronization is facilitated by the interplay of infection pressure and network topology. Spectral analysis further corroborates a dominant periodicity aligned with observation from time-series data.

In contrast, monotonic regimes (scenarios 04, 06, 08, 10, partially 11) denote rapid viral spread leading to a single sharp peak in Posting followed by a steady decline without subsequent waves (Figure~\ref{fig:results-04}). These parameter settings reflect either high transmission rates combined with rapid fatigue or higher forgetting, resulting in faster depletion of susceptible Unaware individuals and suppression of resonant oscillatory dynamics.

Ambiguous regime behavior appears in scenario 11, where weak damped oscillations occur amidst larger stochastic noise envelopes, reflecting marginal conditions near the bifurcation boundary between oscillatory and monotonic dynamics.

\subsection{Influence of Parameters on Epidemic Dynamics}
The study highlights the sensitivity of Posting prevalence oscillations to the balance among transmission rate $\beta$, fatigue rate $\gamma$, and forgetting rate $\xi$. Specifically, moderate to high transmission ($\beta$) combined with moderate fatigue ($\gamma$) and low forgetting ($\xi$) underpin the oscillatory behavior by maintaining a critical mass of susceptible Unaware individuals and dynamically enabling repeated infections.

Conversely, increasing the forgetting rate causes the system to lose its memory too quickly, diminishing the coherence of successive Posting waves and steering the system toward a damped, steady endemic state. The fatigue rate also modulates how quickly Postings wane into Fatigued states, influencing the duration and magnitude of oscillations.

These insights resonate with theoretical expectations that network-induced delays and local clustering can amplify or sustain oscillations absent in classical ODE approaches. Simulation outputs, summarized in Table~\ref{tab:metrics-transposed}, quantitatively capture these dynamics by detailing metrics such as peak Posting prevalence, oscillation counts, episode durations, and spectral periods across scenarios.

\subsection{Practical and Theoretical Implications}
From a practical perspective, understanding the parameter regimes that produce sustained or damped oscillations is vital for controlling viral trends or misinformation propagation. Interventions altering the effective transmission or fatigue rates could, in principle, attenuate or prevent periodic resurgence of Posting activity, while network structure modifications (e.g., through community link rewiring) might alter the global dynamics.

Theoretically, this study confirms that simple compartmental mechanisms augmented by realistic network topology can transcend the classical SIRS model’s limitations, generating rich temporal patterns due to spatial heterogeneity and finite population effects. The findings underscore the composition of memory loss, network clustering, and stochasticity as a triad essential for emergent oscillatory behaviors in social contagion.

\subsection{Limitations and Future Directions}
While the findings are robust in the explored parameter regimes, certain limitations prevail. The downscaling from 10,000 to 1,000 nodes in parameter scans, mandated by computational constraints, may influence the finite-size effects and stochasticity, although qualitative behaviors remain consistent. Extending the analysis to larger networks and incorporating additional social structures (e.g., communities, heterogeneous degrees) would enhance generalizability.

Moreover, further spectral and bifurcation analyses could refine the understanding of transition boundaries between oscillatory and monotonic regimes, potentially revealing critical thresholds in multi-parameter space.

Finally, incorporating empirical data from actual viral trends on social media platforms could calibrate and validate the parameter values and network assumptions, enriching the applicability of the model to real-world scenarios.

\section*{Acknowledgments}
The authors thank the comprehensive simulation and analytic efforts that facilitated the synthesis presented herein.

\section{Conclusion}

This study rigorously examined the dynamics of a viral internet trend propagation modeled as a memory-augmented SIRS-like process on Watts-Strogatz small-world networks, elucidating the impact of network topology and forgetting mechanisms on system behavior. Through a combination of mathematical analysis and comprehensive stochastic simulations over a realistically parameterized social network, several critical insights emerged.

First, the classical well-mixed SIRS framework with constant transition rates inherently precludes the existence of stable limit-cycle oscillations, producing only damped oscillations converging towards an endemic steady state. This theoretical prediction was empirically validated by simulations on large ($N=10,000$) Watts-Strogatz networks, where the Posting prevalence exhibited a sharp early peak followed by monotonic decay to near-zero levels without recurring oscillations.

In contrast, embedding the model on smaller ($N=1,000$) Watts-Strogatz networks with intermediate rewiring ($p=0.1$), mean degree ($k=8$), and realistic clustering coefficients enabled the emergence of network-induced damped oscillations in the Posting population. These oscillations manifested as multiple recurrent Posting waves spaced approximately 40--55 time units apart and were sustained over extended periods before eventual decay. The presence and characteristics of these oscillations strongly depended on the interplay between the transmission rate ($\beta$), fatigue rate ($\gamma$), and forgetting rate ($\xi$). Oscillatory regimes occurred predominantly in parameter spaces with moderate to high transmission, moderate fatigue, and low forgetting rates, which facilitate synchronization across locally clustered subnetworks.

Conversely, monotonic dynamics with single epidemic peaks were observed in scenarios with higher forgetting rates or parameter combinations that suppress recurrent Posting waves, consistent with classical SIRS expectations. Intermediate regimes exhibiting marginal oscillations alongside enhanced stochastic variability were also identified, highlighting the complex boundary between oscillatory and monotonic behaviors.

The investigation underscores the fundamental role of network topology and finite-size effects in shaping viral trend dynamics. The combination of local clustering and global shortcuts intrinsic to small-world networks generates effective delays and spatial correlations that can transiently sustain collective oscillations absent in homogeneous mixing models.

Despite the robustness of these findings, the study is limited by computational constraints requiring downscaled network sizes for parameter scanning and by the exclusion of more complex social structures such as heterogeneous degree distributions and community hierarchies. Future work should extend analyses to larger networks, integrate empirical social media datasets for calibration, and explore the impact of additional behavioral factors and dynamic network rewiring.

Overall, this work contributes to a refined mechanistic understanding of viral information spreading, demonstrating that memory loss (forgetting) coupled with small-world social connectivity can produce rich temporal patterns including network-induced oscillations in popularity trends. These insights offer potential pathways for designing interventions to modulate the lifespan and recurrence of viral phenomena in online platforms.

\begin{thebibliography}{99}

\bibitem{EVolz2007SIRdynamics} E. Volz, ``SIR dynamics in random networks with heterogeneous connectivity,'' \textit{Journal of Mathematical Biology}, 2007.

\bibitem{Apolloni2013AgeSpecific} A. Apolloni, C. Poletto, V. Colizza, ``Age-specific contacts and travel patterns in the spatial spread of 2009 H1N1 influenza pandemic,'' \textit{BMC Infectious Diseases}, 2013.

\bibitem{TangWu2019Fragility} G.-M. Tang, Z.-X. Wu, ``Fragility and robustness of self-sustained oscillations in an epidemiological model on small-world networks,'' \textit{Chaos}, 2019.

\bibitem{RahmandadSterman2004Heterogeneity} H. Rahmandad, J. Sterman, ``Heterogeneity and Network Structure in the Dynamics of Diffusion: Comparing Agent-Based and Differential Equation Models,'' \textit{Management Sciences}, 2004.

\bibitem{Rocha2017Sampling} L. E. C. Rocha, N. Masuda, P. Holme, ``Sampling of Temporal Networks: Methods and Biases,'' \textit{Physical Review E}, 2017.

\bibitem{Baek2007Oscillatory} S. Baek, ``Oscillatory Behaviors of an Epidemiological Model on Small-World Networks,'' 2007.

\bibitem{Ki2007Oscillatory} S.-D. Ki, ``Oscillatory Behaviors of an Epidemiological Model on Small-World Networks,'' 2007.

\bibitem{LaurentBadel2020} T. Valente, G. G. Vega Yon, ``Diffusion/Contagion Processes on Social Networks,'' \textit{Health Education \& Behavior}, 2020.

\bibitem{EVolz2011Effects} E. M. Volz, J. C. Miller, A. P. Galvani, et al., ``Effects of Heterogeneous and Clustered Contact Patterns on Infectious Disease Dynamics,'' \textit{PLoS Computational Biology}, 2011.

\bibitem{Huang2005SmallWorld} C.-Y. Huang, C.-T. Sun, H.-C. Lin, ``Influence of Local Information on Social Simulations in Small-World Network Models,'' \textit{Journal of Artificial Societies and Social Simulation}, 2005.

\bibitem{Yamakou2024InverseStochastic} M. E. Yamakou, J. Zhu, E. A. Martens, ``Inverse stochastic resonance in adaptive small-world neural networks,'' \textit{Chaos}, 2024.

\bibitem{Apolloni2014Metapopulation} A. Apolloni, C. Poletto, J. Ramasco, et al., ``Metapopulation epidemic models with heterogeneous mixing and travel behaviour,'' \textit{Theoretical Biology and Medical Modelling}, 2014.

\bibitem{Houshiyu2025} HOU Shiyu, Liu Ying, TANG Ming, ``Identifying influential nodes in spreading process in complex networks by integrating node dynamic propagation features and local structure,'' \textit{Acta Physica Sinica}, 2025.

\bibitem{Raij2025} Robert Rai, Rajesh Sharma, Chandrakala Meena, ``IPSR Model: Misinformation Intervention through Prebunking in Social Systems,'' Unknown Journal, 2025.
\end{thebibliography}
\newpage
\section*{Supplementary Material}
\begin{algorithmic}[1]
\Require Watts-Strogatz graph parameters: \(N, k, p, \text{seed}\)
\Ensure Connected Watts-Strogatz graph \(G\) with adjacency matrix \(A\)
\State Generate Watts-Strogatz graph \(G\) with \(N\) nodes, mean degree \(k\), rewiring probability \(p\), and random seed
\If{ \(G\) is not connected }
  \State Identify largest connected component \(G_{CC}\)
  \State Set \(G = G_{CC}\)
\EndIf
\State Calculate degree distribution, mean degree \(\langle k \rangle\) and second moment \(\langle k^2 \rangle\)
\State Compute average clustering coefficient \(C\) and average shortest path length \(L\)
\State Save adjacency matrix \(A\) as sparse format
\State Generate plots for degree distribution and diagnostics
\end{algorithmic}

\begin{algorithmic}[1]
\Require Parameter ranges \(\{\beta\}\), \(\{\gamma\}\), \(\{\xi\}\) for SIRS (U, P, F) model
\Ensure Parameter grid for simulation \(\Theta = \{(\beta_i, \gamma_j, \xi_k)\}\) plus baseline parameters
\State Construct parameter grid \(\Theta\) by cartesian product of \(\{\beta\}\), \(\{\gamma\}\), \(\{\xi\}\)
\State Append baseline parameter set
\State Define initial conditions \(\{ U = 99\%, P = 1\%, F = 0\% \}\)
\end{algorithmic}

\begin{algorithmic}[1]
\Require Watts-Strogatz network \(G\), Parameter set \((\beta, \gamma, \xi)\), Initial condition IC, simulation settings
\Ensure Simulation results (time series) of SIRS model
\State Define model compartments \(\{ U, P, F \}\)
\State Specify edge interaction: \(U \xrightarrow{\beta \times \text{contacts with } P} P\)
\State Specify node transitions: \(P \xrightarrow{\gamma} F\), \(F \xrightarrow{\xi} U\)
\State Load network adjacency matrix \(A\) for contact layer
\State Initialize model configuration with compartments and transitions
\State Set model parameters \((\beta, \gamma, \xi)\)
\State Set initial compartment proportions as IC
\State Set simulation parameters: max time, number of realizations
\State Run stochastic simulation (CTMC) yielding compartment counts over time
\State Collect results with confidence intervals (e.g., 90\%)
\State Save time series and plots to output directory
\end{algorithmic}

\begin{algorithmic}[1]
\Require Time-series data of posting prevalence \(P(t)\)
\Ensure Epidemiological summary statistics and oscillation characterization
\State Compute total population \(N = \max (U+P+F)\)
\State Define threshold \(T = 0.01 \times N\)
\State Identify peak posting \(P_{\max}\) and corresponding time \(t_{\text{peak}}\)
\State Determine outbreak duration as interval where \(P(t) > T\)
\State Extract final prevalence values \(P(t_{\max})\), \(F(t_{\max})\)
\State Estimate early phase doubling time of \(P\) between \(T\) and \(2T\)
\State Identify peaks in \(P(t)\) series using prominence threshold
\If{number of peaks \(> 1\)}
  \State Compute mean inter-peak period and amplitude
  \State Classify regime as oscillatory
\Else
  \State Classify regime as steady or endemic
\EndIf
\State Perform FFT on demeaned \(P(t)\) to detect dominant frequency and period
\State Store computed diagnostics for scenario analysis
\end{algorithmic}

\section*{Appendix: Additional Figures}
\addcontentsline{toc}{section}{Appendix: Additional Figures}

\begin{figure}[http]
    \centering
    \begin{subfigure}[b]{0.45\textwidth}
        \centering
        \includegraphics[width=\textwidth]{results-00.png}
        \caption*{results-00.png}
    \end{subfigure}
    \begin{subfigure}[b]{0.45\textwidth}
        \centering
        \includegraphics[width=\textwidth]{results-03.png}
        \caption*{results-03.png}
    \end{subfigure}
    \caption{Figures: results-00.png and results-03.png}
    \label{fig:results-00-png}
\end{figure}

\begin{figure}[http]
    \centering
    \begin{subfigure}[b]{0.45\textwidth}
        \centering
        \includegraphics[width=\textwidth]{results-04.png}
        \caption*{results-04.png}
    \end{subfigure}
    \begin{subfigure}[b]{0.45\textwidth}
        \centering
        \includegraphics[width=\textwidth]{results-05.png}
        \caption*{results-05.png}
    \end{subfigure}
    \caption{Figures: results-04.png and results-05.png}
    \label{fig:results-04-png}
\end{figure}

\begin{figure}[http]
    \centering
    \begin{subfigure}[b]{0.45\textwidth}
        \centering
        \includegraphics[width=\textwidth]{results-06.png}
        \caption*{results-06.png}
    \end{subfigure}
    \begin{subfigure}[b]{0.45\textwidth}
        \centering
        \includegraphics[width=\textwidth]{results-07.png}
        \caption*{results-07.png}
    \end{subfigure}
    \caption{Figures: results-06.png and results-07.png}
    \label{fig:results-06-png}
\end{figure}

\begin{figure}[http]
    \centering
    \begin{subfigure}[b]{0.45\textwidth}
        \centering
        \includegraphics[width=\textwidth]{results-08.png}
        \caption*{results-08.png}
    \end{subfigure}
    \begin{subfigure}[b]{0.45\textwidth}
        \centering
        \includegraphics[width=\textwidth]{results-09.png}
        \caption*{results-09.png}
    \end{subfigure}
    \caption{Figures: results-08.png and results-09.png}
    \label{fig:results-08-png}
\end{figure}

\begin{figure}[http]
    \centering
    \begin{subfigure}[b]{0.45\textwidth}
        \centering
        \includegraphics[width=\textwidth]{results-10.png}
        \caption*{results-10.png}
    \end{subfigure}
    \begin{subfigure}[b]{0.45\textwidth}
        \centering
        \includegraphics[width=\textwidth]{results-11.png}
        \caption*{results-11.png}
    \end{subfigure}
    \caption{Figures: results-10.png and results-11.png}
    \label{fig:results-10-png}
\end{figure}

\begin{figure}[http]
    \centering
    \begin{subfigure}[b]{0.45\textwidth}
        \centering
        \includegraphics[width=\textwidth]{results-12.png}
        \caption*{results-12.png}
    \end{subfigure}
    \begin{subfigure}[b]{0.45\textwidth}
        \centering
        \includegraphics[width=\textwidth]{ws-clustering-path.png}
        \caption*{ws-clustering-path.png}
    \end{subfigure}
    \caption{Figures: results-12.png and ws-clustering-path.png}
    \label{fig:results-12-png}
\end{figure}

\begin{figure}[http]
    \centering
    \begin{subfigure}[b]{0.45\textwidth}
        \centering
        \includegraphics[width=\textwidth]{ws-degree-dist.png}
        \caption*{ws-degree-dist.png}
    \end{subfigure}
    \caption{Figures: ws-degree-dist.png}
    \label{fig:ws-degree-dist-png}
\end{figure}
\end{document}