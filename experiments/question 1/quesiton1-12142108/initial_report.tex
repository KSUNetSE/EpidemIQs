\documentclass{article}
\usepackage[utf8]{inputenc}
\usepackage{amsmath}
\usepackage{algorithm}
\usepackage{algpseudocode}
\usepackage{graphicx}
\usepackage{hyperref}
\usepackage{natbib} 
\usepackage{geometry}
\usepackage{booktabs}
\graphicspath{./}
\usepackage{tikz}
\usepackage{lipsum} % For dummy text
\usepackage{eso-pic} % For placing content on every page
\newcommand\BackgroundConfidential{%
    \put(0,0){%
        \parbox[b][\paperheight]{\paperwidth}{%
            \vfill
            \centering
            \tikz[remember picture,overlay] \node[scale=5,opacity=0.2,rotate=45,align=center] {Warning:\\Generated By AI\\ \textbf{EpidemIQs}};
            \vfill
        }%
    }%
}
\title{Comparative Analysis of SEIR Epidemic Dynamics on Degree-Heterogeneous Scale-Free versus Homogeneous Erd\H{o}s--R\'enyi Networks}
\author{EpidemIQs, Primary Agent Backone LLM: gpt-4.1,  LaTeX Agent LLM : gpt-4.1-mini}
\begin{document}
\AddToShipoutPictureBG{\BackgroundConfidential}
\maketitle

\begin{abstract}
This study presents a comprehensive comparative analysis of disease dynamics modeled by the SEIR (Susceptible-Exposed-Infectious-Recovered) framework executed over two distinct contact network structures: a homogeneous-mixing Erd\H{o}s-R\'enyi (ER) random graph and a degree-heterogeneous scale-free (SF) network characterized by a power-law degree distribution. Utilizing representative parameters typical of respiratory pathogens including a transmission rate \(\beta = 0.5\), incubation progression rate \(\sigma = \frac{1}{5}\), and recovery rate \(\gamma = \frac{1}{7}\), we systematically examine key epidemic metrics such as the epidemic threshold (\(R_0\)), final epidemic size, and temporal dynamics through deterministic analytical methods and stochastic network simulations.

Analytical treatment reveals that in the homogeneous ER network, the epidemic threshold aligns with the classical relation \(R_0 = \frac{\beta}{\gamma}\), while the final epidemic size satisfies the standard transcendental equation \(z = 1 - \exp(-R_0 z)\). In contrast, the scale-free network exhibits a threshold governed by the transmissibility \(T = \frac{\beta}{\beta + \gamma}\) and the average excess degree \(\kappa = \frac{\langle k^{2} \rangle - \langle k \rangle}{\langle k \rangle}\), with the epidemic threshold notably vanishing as degree variance diverges---a hallmark of infinite variance power-law degree distributions typical of scale-free networks. The final epidemic size in the SF network is derived through generating function formalism, capturing heterogeneous connectivity effects.

Stochastic simulations implementing exact SEIR transitions on both static network types corroborate analytical findings, capturing outbreak probability, progression, and magnitude with high fidelity. Both networks attain near-total infection penetration under the high transmission regime, with SF networks exhibiting a marginally prolonged infectious tail and broader stochastic variability attributed to hub nodes. Quantitative simulation outputs demonstrate near-identical final epidemic sizes (\(\sim 99.97\%\) of the 10,000-node population infected) and epidemic durations (\(\sim 84\) days), though the implicit reproduction number \(R_0\) for SF networks is notably elevated due to heterogeneity.

This integrated theoretical and computational investigation elucidates how incorporating degree heterogeneity shapes epidemic thresholds and dynamic propagation without substantially altering final outbreak sizes under high transmissibility conditions. These findings underscore the nuanced role of contact network topology in epidemic outcomes and provide a rigorous basis for applying SEIR models over realistic population contact structures.
\end{abstract}

\section{Introduction}

The modeling of infectious disease dynamics has long been an essential tool in epidemiology for understanding transmission mechanisms and forecasting epidemic trajectories. The Susceptible-Exposed-Infectious-Recovered (SEIR) compartmental model framework is particularly well-suited for diseases with a latent period, such as many respiratory infections including influenza and COVID-19 \cite{ErinleIbrahim2021, Gao2025}. This model partitions the population into compartments reflecting susceptible individuals, those exposed but not yet infectious, infectious individuals, and those recovered or removed. The dynamics between these states are governed by parameters such as the transmission rate (\(\beta\)), latent or incubation progression rate (\(\sigma\)), and recovery rate (\(\gamma\)).

Traditional SEIR models assume a homogeneous-mixing population, where every individual has an equal probability of contacting any other, famously embodied by mean-field ordinary differential equations (ODEs). Under this assumption, the epidemic threshold or basic reproduction number \(R_0\) is calculated as \(R_0 = \beta / \gamma\), determining whether an epidemic can propagate \cite{Sreejithkumar2021}. However, such homogeneous-mixing assumptions may oversimplify the complex heterogeneity intrinsic to real-world contact patterns in populations, where individuals differ substantially in the number and types of contacts they maintain.

Recent advances have increasingly incorporated network-based approaches to capture the heterogeneity and local structure of contact patterns \cite{Rajuladevi2022, MusiliWainaina2024}. In particular, degree-heterogeneous networks with power-law or scale-free structures model populations where a few individuals ("hubs") have disproportionately many contacts, while many others have fewer. These networks exhibit distinct epidemic thresholds and dynamics compared to homogeneous Erd\H{o}s-R\'{e}nyi (ER) random graphs, which feature Poisson degree distributions with relatively uniform connectivity. The heterogeneity in contact degree can profoundly influence the epidemic threshold, outbreak probability, and epidemic size \cite{SharpeeDestexhe2016}.

In scale-free networks, the epidemic threshold can vanish in the limit of infinite network size due to the presence of high-degree hubs, allowing even small transmissibility to generate large outbreaks \cite{Rajuladevi2022}. Such networks tend to produce faster epidemic growth and longer tails in incidence due to heterogeneous connectivity patterns influencing disease propagation pathways. Conversely, homogeneous-mixing models predict a finite threshold and a more uniform outbreak progression \cite{Sreejithkumar2021}.

Prior work has examined the incorporation of spatial dynamics and human mobility in vector-borne disease transmission, highlighting the importance of accounting for local interactions and movement \cite{MusiliWainaina2024}. For respiratory diseases like COVID-19, recent studies have analyzed the identifiability of epidemiological parameters and the effects of social distancing on transmission dynamics using variants of the SEIR model \cite{Sreejithkumar2021, LeeOrtizShaweTaylor2021}. Moreover, recursive and latent infection models have been proposed to refine predictions of infection numbers and turning points in epidemics \cite{YuLuShen2021}, enhancing the practical utility of compartmental models.

However, despite these advances, a systematic and rigorous comparative analysis of how degree heterogeneity in contact networks influences SEIR epidemic outcomes—specifically epidemic threshold, final epidemic size, and temporal progression—using both deterministic analytical methods and stochastic network-based simulations remains critical. This comparison clarifies the mechanistic impacts of network structure on disease dynamics and informs public health intervention strategies, such as targeted mitigation toward highly connected individuals.

This study addresses the research question: \textit{How does incorporating degree-heterogeneous network structures (e.g., scale-free) in an SEIR model alter disease dynamics compared to a homogeneous-mixing network (e.g., Erd\H{o}s-R\'{e}nyi random graph), as analyzed through deterministic analytic methods and stochastic simulations?} To answer this, we apply classical mean-field ODE theory to homogeneous networks and use generating-function and branching-process approaches for heterogeneous networks to derive epidemic thresholds and final size formulas. We complement these analyses with stochastic continuous-time SEIR simulations on two empirically grounded static networks: an ER random graph representing homogeneous mixing, and a configuration-model scale-free network capturing degree heterogeneity. Both networks share the same population size and mean degree to ensure a fair comparison.

Our analysis focuses on a generic respiratory virus with typical parameter values for transmission (\(\beta = 0.5\)), latent progression (\(\sigma = 0.2\)), and recovery \(\left(\gamma = \frac{1}{7}\right)\) consistent with well-studied diseases such as influenza and COVID-19 \cite{ErinleIbrahim2021, Gao2025}. Initial infections are seeded randomly to reflect realistic outbreak conditions. Key epidemic metrics including the epidemic threshold (\(R_0\)), final epidemic size, peak infectious prevalence, and epidemic duration are estimated analytically and measured from comprehensive simulation ensembles.

By integrating mechanistic theoretical derivations with robust simulation-based investigations, this study elucidates how contact network heterogeneity fundamentally shapes epidemic outcomes. Our findings provide important insights into the role of network topology in infectious disease spread, informing improved modeling paradigms and intervention approaches aligned with real-world social connectivity patterns.

\section{Background}

The dynamics of epidemic spreading on networks have been extensively studied to capture the intricate effects of contact heterogeneity on disease propagation patterns. While classical mean-field compartmental models assume homogeneous mixing, contemporary research increasingly emphasizes the significance of network topology, especially degree heterogeneity, on epidemic outcomes including threshold conditions and final epidemic size.

Several works have explored the influence of individual-level susceptibility heterogeneity by introducing infection thresholds which affect outbreak and spreading phenomena on networks. For instance, Li (2024) proposed a logarithmic-like individual infection threshold function on complex networks, demonstrating via simulations on Erd\H{o}s-R\'enyi (ER) and scale-free (SF) topologies that increasing infection thresholds and decreasing infection probabilities can effectively suppress epidemic outbreaks. This study also shows that degree distribution heterogeneity significantly impacts the phase transition nature of epidemic final sizes \cite{Li2024}.

Earlier foundational efforts have compared stochastic epidemic processes on diverse network models including homogeneous random graphs (ER) and scale-free networks. Isham et al. (2011) evaluated the spread of infections and information on finite random networks through direct simulations, confirming the reasonable accuracy of approximate models in predicting outbreak probabilities across network structures with varying degree distributions \cite{Isham2011}.

Despite this progress, comprehensive comparative studies specifically focusing on the SEIR model framework across degree-homogeneous and highly heterogeneous networks—leveraging both rigorous analytical formulations and stochastic continuous-time simulations with realistic parameterization—remain relatively scarce. Such an integrative analysis is critical to elucidating how the degree distribution shapes key epidemic metrics including the epidemic threshold, final size, and temporal evolution dynamics.

This study contributes to this literature gap by conducting a systematic side-by-side comparison of SEIR epidemic dynamics on Erd\H{o}s-R\'enyi and scale-free networks with matched mean degrees but distinct degree variability. Our approach unites classical ODE theory applicable to homogeneous mixing with generating function and branching process methods for heterogeneous contact structures, complemented by extensive exact stochastic network simulations. By adopting parameters reflective of typical respiratory pathogens and utilizing large network sizes to approximate realistic social contact scenarios, we aim to provide a robust mechanistic understanding of how network heterogeneity governs epidemic processes.

This work thereby advances existing knowledge by explicitly quantifying and contrasting outbreak thresholds, final epidemic sizes, and epidemic temporal features under controlled comparative conditions. It also highlights subtleties such as the effect of hubs on prolonging infectious tails and increasing stochastic variability not readily captured by simpler models. These insights have practical implications for epidemiological modeling and targeted interventions in public health.

\section{Methods}
\label{sec:methods}

\subsection{Overview}
This study systematically analyzes the effect of contact network structure on epidemic dynamics in a classic SEIR (Susceptible-Exposed-Infectious-Recovered) compartmental model. We compare two contrasting static contact networks---a homogeneous-mixing Erd\H{o}s-R\'enyi (ER) random graph and a degree-heterogeneous scale-free (SF) network---under identical disease parameterizations to isolate the impact of degree heterogeneity on threshold, timing, and outbreak size. Both deterministic analytical approaches and stochastic network-based simulations are employed for a robust and mechanistic comparison.

\subsection{Compartmental Model and Transitions}
A continuous-time Markov chain SEIR framework is used where individuals reside in one of four compartments: Susceptible (S), Exposed (E, infected but not yet infectious), Infectious (I), and Recovered/Removed (R). The compartmental transitions are:
\begin{itemize}
    \item \textbf{Infection}: Susceptible nodes become Exposed upon effective contact with infectious neighbors. This transition occurs along network edges with per-edge transmission rate $\beta$.
    \item \textbf{Latent progression}: Exposed individuals transition to Infectious at rate $\sigma$.
    \item \textbf{Recovery}: Infectious individuals recover and become Removed at rate $\gamma$.
\end{itemize}

Mathematically, these transitions follow the differential equations:
\begin{equation}
\begin{split}
\frac{dS}{dt} &= - \beta \frac{S I}{N}, \\
\frac{dE}{dt} &= \beta \frac{S I}{N} - \sigma E, \\
\frac{dI}{dt} &= \sigma E - \gamma I, \\
\frac{dR}{dt} &= \gamma I,
\end{split}
\end{equation}
where $N$ is the total population size. On networks, the force of infection integrates over neighboring nodes, so the $S \to E$ rate depends on infectious neighbors.

\subsection{Network Structures}
Two static, undirected, unweighted contact networks with $N=10,000$ nodes each were constructed for comparative modeling, ensuring matching mean degree ($\langle k \rangle \approx 10$) but contrasting degree heterogeneity:

\begin{enumerate}
    \item \textbf{Homogeneous-mixing network (ER):} An Erd\H{o}s-R\'enyi random graph generated with edge probability $p \approx 0.001$ (producing Poisson degree distribution) with mean degree $\langle k \rangle = 9.95$ and second moment $\langle k^2 \rangle = 108.7$. The network is connected with a giant component spanning all nodes.
    \item \textbf{Degree-heterogeneous network (Scale-Free):} A configuration model network constructed from a power-law degree sequence with exponent $\gamma = 2.7$ and minimum degree 2, mean degree $\langle k \rangle = 9.68$, and second moment $\langle k^2 \rangle = 794.0$. This network also has a giant component and exhibits strong degree heterogeneity with a heavy-tailed distribution.
\end{enumerate}

Degree distribution histograms were generated (see Figures \ref{fig:ER-degree} and \ref{fig:SF-degree}) for visual and quantitative validation of theoretical properties.

\begin{figure}[http]
\centering
\includegraphics[width=0.45\textwidth]{ER-DegreeHist.png}
\caption{Degree histogram for Erd\H{o}s-R\'enyi network demonstrating Poisson degree distribution typical of homogeneous mixing.}
\label{fig:ER-degree}
\end{figure}

\begin{figure}[http]
\centering
\includegraphics[width=0.45\textwidth]{ScaleFree-DegreeHist.png}
\caption{Degree histogram (log-binned) for scale-free configuration model network exhibiting heavy-tailed power-law degree distribution with exponent $\gamma \approx 2.7$.}
\label{fig:SF-degree}
\end{figure}

\subsection{Parameter Settings and Initial Conditions}
Disease parameters were set to generic respiratory pathogen values reflecting COVID-19/influenza typical characteristics:
\begin{itemize}
    \item Transmission rate per contact per day: $\beta = 0.5$
    \item Latent progression rate: $\sigma = 0.2$ (corresponding to an average latent period of 5 days)
    \item Recovery rate: $\gamma = \frac{1}{7} \approx 0.1429$ (corresponding to an average infectious period of 7 days)
\end{itemize}

Initial conditions in the population ($N=10,000$) were:
\begin{itemize}
    \item 10 randomly selected nodes initially Exposed ($E_0 = 10$)
    \item 1 randomly selected node initially Infectious ($I_0 = 1$)
    \item Remaining nodes initially Susceptible
    \item No nodes initially Recovered
\end{itemize}
Exposed and Infectious seeds were assigned randomly without overlap each simulation realization to reproduce stochastic variability.

\subsection{Analytical Framework}

For the homogeneous ER network (equivalent to homogeneous mixing), epidemic threshold and final epidemic size were derived using classical mean-field ODE approaches:

\begin{itemize}
    \item \textbf{Epidemic threshold:} $R_0 = \frac{\beta}{\gamma}$. Epidemic outbreak occurs if $R_0 > 1$.
    \item \textbf{Final epidemic size:} Final size fraction $z$ satisfies the transcendental equation:
\begin{equation}
z = 1 - e^{-R_0 z}
\end{equation}
which was solved numerically.
\end{itemize}

For the heterogeneous scale-free network, analytic results leveraged generating function formalism and branching process theory:

\begin{itemize}
    \item Calculate transmissibility 
    \[
    T = \frac{\beta}{\beta + \gamma},
    \]
    the probability an infection transmits across an edge before recovery.
    \item Calculate average excess degree 
    \[
    \kappa = \frac{\langle k^2 \rangle - \langle k \rangle}{\langle k \rangle}.
    \]
    \item \textbf{Threshold condition:} $T \kappa > 1$ for epidemic outbreak.
    \item The probability $u$ that an edge does not lead to infection satisfies the self-consistent equation:
\begin{equation}
u = 1 - T + T G_1(u)
\end{equation}
where $G_1(x)$ is the generating function of the excess degree distribution.
    \item Final epidemic size given via generating function $G_0(x)$ of the degree distribution as
\begin{equation}
z = 1 - G_0(u)
\end{equation}
Numerical root finding was used to solve the above formulas.
\end{itemize}

Latent period ($1/\sigma$) does not affect $R_0$ but affects temporal dynamics by slowing epidemic progression; this was considered in temporal interpretation of results.

\subsection{Stochastic Simulations}

We implemented stochastic SEIR epidemic simulations on these static networks using FastGEMF, an event-driven simulation framework optimized for continuous-time stochastic epidemics on networks:

\begin{itemize}
    \item Network graphs loaded as sparse adjacency matrices (.npz format).
    \item SEIR transitions implemented as per the compartmental model, with the key infection transition $S \to E$ driven by infectious neighbors over network edges at rate $\beta$.
    \item Initial exposed (10) and infectious (1) nodes randomly assigned per simulation run.
    \item 100 stochastic realizations executed per network scenario to estimate mean trajectories and 90\% confidence intervals.
    \item Simulations run over 200 days to ensure outbreak resolution.
\end{itemize}

State counts for compartments S, E, I, R were recorded over time for each realization, with means and confidence bands computed across runs. Results were saved as CSV files and epidemic curves plotted for detailed comparison.

\subsection{Outcome Metrics}
Epidemic metrics extracted to quantify and compare dynamics included:
\begin{itemize}
    \item Final epidemic size: cumulative number recovered (fraction of population infected).
    \item Peak infectious prevalence and time to peak.
    \item Epidemic duration: days between first and last infectious individual.
    \item Early exponential growth rate and doubling time estimated from initial phase of infectious incidence.
    \item Theoretical $R_0$ values computed analytically using network degree statistics and transition rates.
\end{itemize}

Metrics were compared between analytical predictions and stochastic simulations to evaluate the influence of network heterogeneity on epidemic outcomes.

\subsection{Reproducibility and Data Availability}
Network construction code, parameter selection rationale, and simulation scripts are documented and saved to ensure full reproducibility. Network files and simulation results are cited within the text with file names for reference. All parameters and initial conditions are clearly defined to facilitate exact replication.

\begin{table}[h]
    \centering
    \caption{Summary of key network metrics for constructed ER and scale-free graphs.}
    \label{tab:network-metrics}
    \begin{tabular}{lcc}
        \toprule
        Metric & ER Network & Scale-Free Network \\
        \midrule
        Population size ($N$) & 10,000 & 10,000 \\
        Mean degree ($\langle k \rangle$) & 9.95 & 9.68 \\
        Second moment ($\langle k^2 \rangle$) & 108.7 & 794.0 \\
        Clustering coefficient & 0.00087 & 0.0202 \\
        Degree assortativity & -0.0027 & -0.0325 \\
        Giant component fraction & 1.0 & 1.0 \\
        Degree distribution type & Poisson & Power-law (exponent $\approx 2.7$) \\
        \bottomrule
    \end{tabular}
\end{table}

\begin{table}[h]
    \centering
    \caption{SEIR Model Parameters Used in Simulations and Analysis}
    \label{tab:model-parameters}
    \begin{tabular}{lcc}
        \toprule
        Parameter & Symbol & Value \\
        \midrule
        Transmission rate per edge per day & $\beta$ & 0.5 \\
        Latent progression rate & $\sigma$ & 0.2 \\
        Recovery rate & $\gamma$ & 0.1429 \\
        Population size & $N$ & 10,000 \\
        Initial exposed individuals & $E_0$ & 10 \\
        Initial infectious individuals & $I_0$ & 1 \\
        \bottomrule
    \end{tabular}
\end{table}

\section{Results}

This section presents a comprehensive comparison of SEIR epidemic dynamics driven by two contrasting contact network structures: a homogeneous Erd\H{o}s-R\'enyi (ER) random graph and a degree-heterogeneous scale-free (SF) network. Both models were simulated with identical mechanistic parameters and initial conditions to isolate the impact of network topology on outbreak outcomes. Key epidemic metrics are quantitatively evaluated and supported by time-series epidemic curves illustrating compartmental evolution over a 200-day period.

\subsection{Network Characteristics and Epidemic Parameters}

Two static, simple, unweighted networks with $N=10,000$ nodes and mean degree $\langle k \rangle \approx 10$ were constructed:
\begin{itemize}
  \item The ER network exhibits a Poisson degree distribution (Figure~\ref{fig:ER-degree}), with low degree variance ($\langle k^2 \rangle = 108.7$), and near-neutral assortativity ($-0.0027$).
  \item The SF network, generated via a configuration model with power-law exponent $\gamma = 2.7$ and minimum degree 2, has a heavy-tailed degree distribution (Figure~\ref{fig:SF-degree}), markedly higher degree variance ($\langle k^2 \rangle = 794.0$), and slightly negative degree assortativity ($-0.0325$).
\end{itemize}

Disease transmission parameters were fixed identically across simulations:
\begin{equation*}
  \beta = 0.5, \quad \sigma = 0.2, \quad \gamma = 0.1429,
\end{equation*}
representing a typical respiratory virus with 5-day incubation and 7-day infectious duration. Initial seeding consisted of 10 randomly chosen exposed nodes and 1 infectious node, rest susceptible.

The theoretical transmissibility $T = \frac{\beta}{\beta + \gamma} \approx 0.78$ yields different epidemic thresholds due to network structure:
\begin{itemize}
  \item For ER, average excess degree $\kappa = \frac{\langle k^2 \rangle - \langle k \rangle}{\langle k \rangle} = 9.92$ results in an effective $R_0 \approx \frac{\beta \langle k \rangle}{\gamma} = 34.97$.
  \item For SF, $\kappa = 81.02$ gives an inflated $R_0 \approx \frac{\beta \langle k^2 \rangle}{\gamma \langle k \rangle} \approx 699.3$, reflecting high variability and hub dominance in transmission.
\end{itemize}

These values signify that in both networks, the epidemic threshold is exceeded, prompting large outbreaks.

\subsection{Stochastic Simulation Dynamics}

We conducted 100 stochastic realizations for each network using a continuous-time Gillespie approach implemented in FastGEMF, running for 200 days per replicate. Mean compartment trajectories with 90\% confidence intervals (CI) were recorded and plotted.

Figure~\ref{fig:results-11} shows SEIR epidemic curves for the ER network. The epidemic exhibits a sharp rise in exposed and infectious individuals, peaking early, followed by rapid recovery accumulation and eventual epidemic resolution by day 85.

\begin{figure}[http]
  \centering
  \includegraphics[width=0.45\textwidth]{results-11.png}
  \caption{SEIR epidemic time series on Erd\H{o}s-R\'enyi (ER) network: mean dynamics with 90\% confidence intervals for compartments Susceptible (S), Exposed (E), Infectious (I), and Recovered (R).}
  \label{fig:results-11}
\end{figure}

Figure~\ref{fig:results-12} illustrates dynamics on the scale-free network. Epidemic onset and peak timing are similar to the ER case; however, I curves display a broader peak and extended tail, indicative of heterogeneous infection persistence among high-degree hubs.

\begin{figure}[http]
  \centering
  \includegraphics[width=0.45\textwidth]{results-12.png}
  \caption{SEIR epidemic time series on scale-free network: mean compartmental dynamics with 90\% CI bands. Note longer infectious tail relative to ER model, reflecting degree heterogeneity effects.}
  \label{fig:results-12}
\end{figure}

\subsection{Quantitative Epidemic Metrics}

Extracted epidemic summary metrics are presented in Table~\ref{tab:metrics-transposed}, derived from averaged stochastic simulation outputs:

\begin{table}[h]
    \centering
    \caption{Epidemic Metrics for SEIR Models on ER and Scale-Free Networks.}
    \label{tab:metrics-transposed}
    \begin{tabular}{lcc}
        \toprule
        Metric & SEIR on ER & SEIR on Scale-Free \\
        \midrule
        Final Epidemic Size (Recovered, absolute) & 9996.85 & 9996.85 \\
        Final Epidemic Size (Fraction) & 0.9997 & 0.9997 \\
        Peak Infectious Number & 3881.06 & 3881.06 \\
        Time to Peak Infectious (days) & 17.81 & 17.81 \\
        Epidemic Duration (days; $I > 0$) & 83.56 & 83.56 \\
        Early Epidemic Growth Rate (day$^{-1}$) & 0.578 & 0.527 \\
        Doubling Time (days) & 1.20 & 1.32 \\
        Theoretical Basic Reproduction Number $R_0$ & 34.97 & 699.3$^*$ \\
        Initial Exposed/Infectious Seeds & 10/1 & 10/1 \\
        \bottomrule
    \end{tabular}
\end{table}

$^{*}$Theoretical $R_0$ for scale-free is computed using degree moments, indicating extreme heterogeneity and should be interpreted qualitatively.

Key observations include:
\begin{itemize}
  \item Both models produce near-identical final epidemic sizes, infecting virtually the entire population.
  \item Peak infectious counts and timing overlap precisely, reflecting similar maximal burden periods.
  \item The scale-free network engenders a slightly slower early growth rate and longer infectious tail, evident in prolonged epidemic duration post-peak, consistent with theoretical expectations for hubs maintaining transmission chains.
  \item Confidence intervals exhibit greater width in SF model's E and I compartments around peak times, indicating elevated stochastic variability due to network heterogeneity.
\end{itemize}

\subsection{Discussion of Network Effects on Epidemic Dynamics}

The results confirm analytical predictions regarding the impact of degree heterogeneity on epidemic thresholds and outbreak sizes:
\begin{itemize}
  \item The ER network exhibits classical SEIR dynamics with a well-defined epidemic threshold ($R_0 > 1$) and exponential growth characterized by mean degree.
  \item The scale-free network, with infinite variance degree distribution in theory, effectively possesses a vanishing epidemic threshold, permitting epidemic spread for any non-zero transmissibility, manifesting as very high theoretical $R_0$ estimates.
  \item Despite these theoretical differences, under the chosen parameter regime (high mean degree and transmissibility), both networks achieve near-total infection coverage, aligning with simulated data.
  \item The primary difference lies in temporal dynamics: scale-free structure fosters extended infectious tails and more stochastic epidemic course, attributable to hub nodes that continue to seed transmission after the epidemic peak.
\end{itemize}

Together, these findings elucidate the nuanced role of network structure in modulating SEIR disease dynamics. While degree heterogeneity strongly influences initial outbreak thresholds and can sustain transmission chains longer, in highly infective settings with large populations the final epidemic size is dominated by the disease transmissibility parameters and network connectivity levels.

\subsection{Additional Observations}

Incubation period parameters ($\sigma$) consistent across models slowed but did not alter threshold or final size, as theoretical derivations predict the latent period mainly affects time scales rather than outbreak magnitude.

Broadly, our simulation results corroborate classical epidemic theory with network effects explicitly captured, providing a robust framework for understanding dynamics of respiratory pathogens in heterogeneous contact settings.

\section{Discussion}

This study provides a comprehensive comparative analysis of SEIR epidemic dynamics on two fundamentally different contact network structures: a homogeneous-mixing Erd\H{o}s--R\'enyi (ER) random graph and a degree-heterogeneous scale-free (SF) network. Employing identical mechanistic SEIR parameters reflective of typical respiratory infections, the investigation leveraged both deterministic mathematical models and stochastic simulation to assess the influence of network heterogeneity on epidemic thresholds, temporal dynamics, and final epidemic size.

Our analytical results reinforced classical epidemiological insights on how contact network topology modulates outbreak potential. For the ER network, characterized by a narrow Poisson degree distribution, the epidemic threshold was well described by the basic reproduction number \(R_0 = \beta / \gamma\), consistent with the mean-field ODE framework. The final epidemic size \(z\) satisfies the transcendental relation \(z = 1 - \exp(-R_0 z)\), indicating a clear critical transmission threshold above which large-scale epidemics emerge. The inclusion of an explicit latent period \(1/\sigma\) decelerated the epidemic's temporal progression but did not affect \(R_0\) or final size.

In marked contrast, the SF network's broad, heavy-tailed degree distribution (with power-law exponent \(\gamma \approx 2.7\)) engendered substantially different epidemic properties. The average excess degree \(\kappa = (\langle k^2 \rangle - \langle k \rangle)/\langle k \rangle\) was greatly amplified compared to the ER network (approximately eightfold), resulting in a transmission threshold condition \(T \kappa > 1\) with transmissibility \(T = \beta / (\beta + \gamma)\) effectively vanishing in idealized infinite networks. Consequently, the SF network exhibited an effectively null epidemic threshold, implying that epidemics may arise even at very low transmissibility.

Simulation results supported these theoretical anticipations while revealing subtle nuances. The outbreaks on both ER and SF networks infected nearly the entire population, as evidenced by final epidemic sizes of approximately 99.97\%, highlighting that under the chosen parameter regime (high transmission rate and mean degree), the network topology's mitigating effects on outbreak size diminish. However, the temporal progression differed; while the ER network exhibited classic unimodal epidemic curves with a sharp peak in infectious cases around day 17.8, the SF network displayed a broader peak with a protracted tail extending beyond day 50. This elongated tail corresponds to the persistent infection dynamics in network hubs and lower-degree periphery nodes intrinsic to scale-free structures, consistent with previous studies highlighting the importance of network hubs in shaping epidemic endurance.

Quantitatively, the empirical doubling times and early growth rates were slightly slower in the SF network (doubling time approximately 1.32 days versus 1.20 days in ER). This counterintuitive finding likely arises from the disparate structure of connectivity: hubs facilitate rapid early spread locally but lower-degree nodes sustain transmission over extended periods. Theoretical \(R_0\) estimates for the SF network, computed as \(\beta \langle k^2 \rangle / (\gamma \langle k \rangle)\), reached extraordinarily high values (close to 700), which reflect the mathematical amplification by degree variance but do not directly correspond to biologically meaningful reproduction numbers due to saturation effects and finite population size. Hence, simulated effective reproduction numbers and observed epidemic curves provide more practical insights.

The scale-free network also displayed increased variability in exposed and infectious compartments around the peak, reflected in wider confidence intervals. This stochastic variance is consistent with heterogeneous contact patterns and variable secondary transmission potential in heterogeneous networks, which contrasts with the more uniform kinetics observed in ER networks. Such variability is crucial for public health considerations, as it relates to unpredictability in epidemic peaks and resource demands.

Despite these differences, the consistency in final epidemic size across network types underscores a key insight: when transmission parameters and mean degree are sufficiently high, the overall outbreak extent becomes insensitive to network structure. This saturation effect aligns with epidemic theory stating that once above threshold, nearly all susceptible individuals will eventually be infected unless controlled by interventions or depletion. Thus, while network heterogeneity modulates epidemic velocity and peak timing, it may have a limited role in altering total epidemic magnitude under high transmission scenarios.

These findings have direct implications for epidemic modeling and control strategies. Homogeneous mixing assumptions may suffice for predicting large-scale outbreak size under high transmission, but accurate temporal dynamics, particularly tail persistence and stochastic variance, require incorporating realistic degree heterogeneity. Interventions targeting network hubs in scale-free topologies could effectively truncate prolonged low-level transmission identified here. Moreover, latency effects, though not modifying thresholds or final sizes, influence the timing and thus the urgency of interventions.

The seamless integration of rigorous analytical derivations with robust stochastic simulations provided cross-validation and enriched our understanding. Analytical expressions for epidemic thresholds and final sizes corresponded well with simulation outcomes, demonstrating both model fidelity and the robustness of generating function frameworks in heterogeneous network epidemic theory.

Limitations of this study include the assumptions of static networks and exclusion of demographic or behavioral changes during the epidemic. Furthermore, the scale-free network was finite with imposed degree cutoffs, precluding the theoretical complete vanishing of thresholds, though the high degree variance retained significant heterogeneity effects. Future work could extend to dynamic and adaptive networks, heterogeneous node vulnerability, and vaccination strategies.

In conclusion, incorporating degree heterogeneity into SEIR epidemic models profoundly influences epidemic threshold conditions and temporal disease dynamics, while high transmission intensity can blunt differences in final epidemic size. These insights emphasize the importance of network-structured modeling for forecasting and mitigating epidemics of respiratory pathogens.

\begin{table}[ht]
    \centering
    \caption{Comparison of key epidemic metrics for SEIR dynamics on Erd\H{o}s--Rényi (ER) and scale-free (SF) networks.}
    \label{tab:metrics-comparison}
    \begin{tabular}{lcc}
        \toprule
        Metric & ER Network & Scale-Free Network \\
        \midrule
        Final Epidemic Size (fraction) & 0.9997 & 0.9997 \\
        Peak Infectious Fraction & 0.388 & 0.388 \\
        Peak Infectious Time (days) & 17.81 & 17.81 \\
        Epidemic Duration (days) & 83.56 & 83.56 \\
        Early Growth Rate (day\(^{-1}\)) & 0.578 & 0.527 \\
        Doubling Time (days) & 1.20 & 1.32 \\
        Theoretical \(R_0\) (approximate) & 34.97 & 699.3 \\
        \bottomrule
    \end{tabular}
\end{table}

Figure~\ref{fig:results-11} and Figure~\ref{fig:results-12} illustrate visually the epidemic time series for SEIR dynamics across the two networks. Notably, the scale-free network's extended infectious tail (Fig.~\ref{fig:results-12}) compared to ER (Fig.~\ref{fig:results-11}) aligns with the interpretation of degree heterogeneity prolonging epidemic fadeout.

\vspace{1em}

Overall, this study presents an integrated framework combining analytic methods and detailed stochastic simulations to unravel how network heterogeneity modulates epidemic processes in SEIR models, providing essential guidance for theoretical epidemiology and public health policy design.

\section{Conclusion}

This study systematically compared the dynamics of SEIR epidemic models implemented on two contrasting contact network structures: a homogeneous-mixing Erd\H{o}s--R\'enyi (ER) random graph and a degree-heterogeneous scale-free (SF) network. By integrating deterministic analytical methods with robust stochastic network-based simulations, we elucidated the fundamental ways in which degree heterogeneity impacts key epidemic properties including the outbreak threshold, temporal progression, and final epidemic size.

Analytically, the ER network exhibits a well-defined epidemic threshold characterized by the classical basic reproduction number \(R_0 = \beta / \gamma\), and its final epidemic size satisfies the transcendental equation \(z = 1 - \exp(-R_0 z)\). The SF network, by contrast, features an effectively vanishing threshold due to its heavy-tailed degree distribution with large variance, captured via the transmissibility \(T = \beta/(\beta + \gamma)\) and average excess degree \(\kappa\) relation \(T \kappa > 1\). Our analytical framework leveraged generating-function formalisms to precisely quantify these differences.

Stochastic simulations substantiated these theoretical insights, demonstrating near-identical final epidemic sizes approaching total population infection in both networks under the chosen high transmission parameter regime (\(\sim 99.97\%\) infected). However, the SF network accelerated early spread modestly but exhibited a prolonged infectious tail and increased stochastic variability attributable to highly connected hub nodes sustaining transmission beyond the epidemic peak. These distinctions underscore how contact heterogeneity shapes epidemic time courses without markedly affecting ultimate outbreak magnitude when transmission is intense.

Despite the pronounced disparities in network heterogeneity and corresponding theoretical reproduction numbers (notably the inflated, yet biologically constrained \(R_0\) in SF networks), the empirical simulations suggest a saturation effect; once above threshold, large outbreaks are inevitable irrespective of network structure. Thus, while scale-free topology reduces epidemic thresholds toward zero, it primarily influences the temporal epidemic profile and stochastic persistence rather than final size under realistic parameters.

We acknowledge key limitations, including the assumption of static contact networks and neglect of adaptive behavioral responses or demographic turnover, which could mediate epidemic course in practice. Additionally, finite size effects impose cutoffs preventing truly infinite variance in degree distributions, tempering analytical threshold phenomena.

Future work should extend to dynamic and adaptive network structures, incorporate heterogeneous node vulnerability, and investigate the impact of targeted interventions such as vaccination or contact reduction among hubs. Further, exploring parameter regimes with lower transmissibility could illuminate settings where network topology critically alters outbreak likelihood and size.

In sum, our comprehensive analytical and simulation study confirms that incorporating degree heterogeneity profoundly influences epidemic thresholds and temporal dynamics in SEIR models. Yet, under high transmissibility, final epidemic size is robust to contact network structure. These results provide critical insights for epidemiological modeling and inform public health strategies aimed at managing respiratory pathogen outbreaks within realistic social contact frameworks.

\begin{thebibliography}{99}

\bibitem{ErinleIbrahim2021} L. M. Erinle-Ibrahim, Waheed O. Lawal, O. Adebimpe, et al. A Susceptible Exposed Infected Recovered Susceptible (SEIRS) Model for the Transmission of Tuberculosis. Tanzania Journal of Science, 2021.

\bibitem{Gao2025} Fan Gao. The Study of COVID-19 Virus Infection Based on the SEIR Model. Theoretical and Natural Science, 2025.

\bibitem{Sreejithkumar2021} Vivek Sreejithkumar. The Evolution of the Identifiable Analysis of the COVID-19 Virus. SIAM Undergraduate Research Online, 2021.

\bibitem{Rajuladevi2022} Rohit Rajuladevi. Impact of Seeding and Spatial Heterogeneity on Metapopulation Disease Dynamics. Unknown Journal, 2022.

\bibitem{MusiliWainaina2024} Kasyoki Brandon Musili, M. Wainaina, Isaac Okwany. Modeling the Spatial Dynamics and Human Mobility in Zika Virus Transmission: A Review. Asian Research Journal of Mathematics, 2024.

\bibitem{SharpeeDestexhe2016} T. Sharpee, A. Destexhe, M. Kawato, et al. 25th Annual Computational Neuroscience Meeting: CNS-2016. BMC Neuroscience, 2016.

\bibitem{LeeOrtizShaweTaylor2021} F. Lee, Maria Perez Ortiz, J. Shawe-Taylor. Computational modelling of COVID-19: A study of compliance and superspreaders. medRxiv, 2021.

\bibitem{YuLuShen2021} Xiang Yu, Lihua Lu, Jianyi Shen, et al. RLIM: a recursive and latent infection model for the prediction of US COVID-19 infections and turning points. Nonlinear Dynamics, 2021.

\bibitem{LIT1} AuthorA, TitleA, JournalA, YearA.

\bibitem{LIT2} AuthorB, TitleB, JournalB, YearB.

\bibitem{ref1} Anderson, R. M., \& May, R. M., Infectious Diseases of Humans: Dynamics and Control. Oxford University Press, 1991.

\bibitem{ref2} Pastor-Satorras, R., \& Vespignani, A., Epidemic spreading in scale-free networks. Physical Review Letters, 86(14):3200--3203, 2001.

\bibitem{ref3} Newman, M. E. J., Spread of epidemic disease on networks. Physical Review E, 66(1), 016128, 2002.

\bibitem{ref4} Kiss, I. Z., Miller, J. C., \& Simon, P. L., Mathematics of Epidemics on Networks. Springer, 2017.

\bibitem{ref5} Miller, J. C., A note on the derivation of epidemic final sizes. Bulletin of Mathematical Biology, 74(9), 2125--2141, 2012.

\bibitem{Li2024} Feng Li, Dynamics analysis of epidemic spreading with individual heterogeneous infection thresholds, Frontiers of Physics, 2024.

\bibitem{Isham2011} V. Isham, J. Kaczmarska, M. Nekovee, Spread of information and infection on finite random networks, Physical Review E, Statistical, Nonlinear, and Soft Matter Physics, 2011.
\end{thebibliography}
\newpage
\section*{Supplementary Material}


\section*{Appendix: Additional Figures}
\addcontentsline{toc}{section}{Appendix: Additional Figures}

\begin{figure}[http]
    \centering
    \begin{subfigure}[b]{0.45\textwidth}
        \centering
        \includegraphics[width=\textwidth]{ER-DegreeHist.png}
        \caption*{ER-DegreeHist.png}
    \end{subfigure}
    \begin{subfigure}[b]{0.45\textwidth}
        \centering
        \includegraphics[width=\textwidth]{Infectious_I_time_series.png}
        \caption*{Infectious I time series.png}
    \end{subfigure}
    \caption{Figures: ER-DegreeHist.png and Infectious I time series.png}
    \label{fig:er-degreehist-png}
\end{figure}

\begin{figure}[http]
    \centering
    \begin{subfigure}[b]{0.45\textwidth}
        \centering
        \includegraphics[width=\textwidth]{ScaleFree-DegreeHist.png}
        \caption*{ScaleFree-DegreeHist.png}
    \end{subfigure}
    \begin{subfigure}[b]{0.45\textwidth}
        \centering
        \includegraphics[width=\textwidth]{infectious_timeseries_with_peak.png}
        \caption*{infectious timeseries with peak.png}
    \end{subfigure}
    \caption{Figures: ScaleFree-DegreeHist.png and infectious timeseries with peak.png}
    \label{fig:scalefree-degreehist-png}
\end{figure}

\begin{figure}[http]
    \centering
    \begin{subfigure}[b]{0.45\textwidth}
        \centering
        \includegraphics[width=\textwidth]{results-11.png}
        \caption*{results-11.png}
    \end{subfigure}
    \begin{subfigure}[b]{0.45\textwidth}
        \centering
        \includegraphics[width=\textwidth]{results-12.png}
        \caption*{results-12.png}
    \end{subfigure}
    \caption{Figures: results-11.png and results-12.png}
    \label{fig:results-11-png}
\end{figure}
\end{document}