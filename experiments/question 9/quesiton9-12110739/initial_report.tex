\documentclass{article}
\usepackage[utf8]{inputenc}
\usepackage{amsmath}
\usepackage{algorithm}
\usepackage{algpseudocode}
\usepackage{graphicx}
\usepackage{hyperref}
\usepackage{natbib} 
\usepackage{geometry}
\usepackage{booktabs}
\graphicspath{./}
\usepackage{tikz}
\usepackage{lipsum} % For dummy text
\usepackage{eso-pic} % For placing content on every page
\newcommand\BackgroundConfidential{%
    \put(0,0){%
        \parbox[b][\paperheight]{\paperwidth}{%
            \vfill
            \centering
            \tikz[remember picture,overlay] \node[scale=5,opacity=0.2,rotate=45,align=center] {Warning:\\Generated By AI\\ \textbf{EpidemIQs}};
            \vfill
        }%
    }%
}
\title{Comparative Analysis of Financial Contagion Dynamics in Diversified versus Concentrated Interbank Networks Using SI Threshold Models}
\author{EpidemIQs, Primary Agent Backone LLM: gpt-4.1,  LaTeX Agent LLM : gpt-4.1-mini}
\date{\today}
\begin{document}
\AddToShipoutPictureBG{\BackgroundConfidential}
\maketitle

\begin{abstract}
Financial distress propagation in interbank markets can be effectively modeled as a contagion process on networks, where institutional defaults spread based on exposure and capital buffers. We investigate this phenomenon using a discrete-time susceptible-infected (SI) threshold model applied to two archetypal network topologies: a diversified Erd\H{o}s--R\'enyi (ER) graph representing many small exposures per institution, and a concentrated Barab\'asi--Albert (BA) graph modeling few large, heterogeneous exposures characteristic of a core-periphery structure. Each bank maintains a fixed total exposure and capital buffer, with the default threshold defined as the ratio of buffer to exposure. Defaults propagate when the fraction of defaulted neighbors exceeds this threshold.

Simulating cascades under random and targeted initial shocks reveals distinct systemic risk profiles. Diversified ER networks exhibit robustness by diluting shocks across numerous small exposures, requiring multiple defaults to trigger contagion in neighbors. However, once the threshold is breached, cascades spread rapidly due to high connectivity, especially under targeted shocks to hub banks. Conversely, concentrated BA networks manifest fragility to targeted shocks because large exposures imply single neighbor defaults can induce immediate failure. Their heterogeneous connectivity engenders slower but still systemic contagion under random shocks, with hubs playing a pivotal role in contagion amplification.

Quantitative analysis confirms theoretical predictions: critical default thresholds correspond closely with mean degree reciprocals in ER networks and show a narrower cascade window in BA networks influenced by degree heterogeneity. Across all low threshold regimes explored, full systemic cascades were observed, but the speed and abruptness of propagation differ markedly by topology and shock targeting. These findings elucidate the trade-offs inherent in network diversification strategies, highlighting that while diversified interbank structures may mitigate random shock impact, they also provide abundant pathways for rapid contagion once thresholds are surpassed, whereas concentrated topologies are acutely sensitive to targeted failures of systemic hubs.

This mechanistic framework advances understanding of interbank contagion dynamics, offering quantitative tools for assessing systemic risk under varied structural and shock conditions, and informing regulatory strategies for financial stability.
\end{abstract}

\section{Introduction}
Financial stability within interconnected banking systems remains a critical concern in the context of systemic risk, where distress can propagate through interbank exposures, triggering cascades of defaults that jeopardize the entire financial infrastructure. Understanding how the topology of interbank networks affects the propagation and amplification of financial distress is paramount for formulating regulatory policies and risk management strategies.

Previous research has employed various network models to characterize interbank markets and their susceptibility to default cascades. Among the seminal approaches is the analogy of financial contagion to epidemic spreading processes, wherein banks are represented as nodes in a network and exposure relationships as edges transmitting distress similar to infections \cite{ABirch2016}. Notably, the adoption of threshold contagion models has enabled capturing the nonlinear contagion dynamics inherent in financial systems, where default is triggered when accumulated losses exceed a bank’s capital buffer \cite{Watts2002}.

Two contrasting interbank network topologies have been extensively discussed in the literature: diversified networks, typified by Erd\H{o}s-R\'{e}nyi (ER) random graphs, and concentrated networks, reflected by Barab\'{a}si-Albert (BA) scale-free or core-periphery graphs. Diversified networks feature many small exposures per bank, distributing risk broadly, while concentrated networks comprise fewer but larger exposures with systemic importance carried by hub banks \cite{ABirch2016}. The nature of exposures fundamentally influences the system's fragility or robustness to contagion: diversified networks may dilute shocks but also furnish numerous pathways for contagion, whereas concentrated networks may be vulnerable to targeted attacks on hubs but less prone to widespread random cascades.

Despite the recognition of these structural paradigms, key questions remain insufficiently resolved. Principally, how does the network topology—diversified versus concentrated—affect the size, speed, and critical thresholds for default cascades in an interbank market? Does maximizing connectivity inherently dampen contagion by diluting the impact of individual exposures, or conversely, does it facilitate rapid systemic failure by providing multiple transmission pathways? Moreover, how do initial shock configurations—random defaults versus targeted defaults of high-degree hubs—influence contagion dynamics under different topologies?

This research addresses these questions by investigating the propagation of financial distress using a mechanistic susceptible-infected (SI)-like threshold contagion model on two prototypical static interbank networks: a diversified ER network and a concentrated BA network. Each bank is assigned a fixed total exposure and capital buffer, with exposures distributed uniformly among its counterparties according to node degree. The threshold condition for default is determined by the ratio of the capital buffer to total exposure, requiring that the fraction of defaulted neighbors exceeds this threshold to trigger contagion. This setup encapsulates the empirically observed mechanism through which losses propagate in interbank markets \cite{ABirch2016}.

The study systematically compares the cascade sizes, speeds, and threshold conditions under both random and targeted initial shocks, providing quantitative insights into the resilience and fragility of each topology. Simulations are conducted over a comprehensive range of threshold parameters, validated against analytic cascade conditions derived from network theory \cite{Watts2002}. This integrative approach elucidates the nonlinear interplay between network topology, exposure distribution, and contagion dynamics, advancing the understanding of systemic risk in financial networks.

By dissecting the differential impacts of diversification and concentration on financial contagion, this work contributes rigorous mechanistic modeling and simulation evidence to the field, with implications for regulatory design and systemic risk mitigation. The findings seek to clarify the conditions under which diversification acts as a shock absorber versus a conduit for contagion, and how structural heterogeneity informs the vulnerability to exogenous shocks.

\section{Background}

Financial contagion in interbank networks has been extensively studied using a variety of modeling approaches, predominantly leveraging network theory and epidemic spreading analogies to understand systemic risk propagation. Prior work has established that interbank networks exhibit diverse topological structures which critically shape contagion dynamics and systemic stability. For instance, diversified network configurations characterized by Erd\H{o}s--R\'enyi (ER) random graphs typically model systems with many small, homogeneous bilateral exposures, while concentrated networks, often captured by Barab\'{a}si--Albert (BA) scale-free or core-periphery structures, represent markets where exposures cluster on few systemically important hubs \cite{ABirch2016}.

Building upon foundational threshold contagion frameworks such as the model by Watts \cite{Watts2002}, threshold models have been widely used to capture nonlinear default dynamics where financial institutions default once losses from neighboring defaults exceed their capital buffers. These SI-type threshold processes are instrumental for mechanistically modeling default cascades, as they incorporate network-induced interdependencies and the critical role of buffer-to-exposure ratios.

More recent advances have integrated elaborate contagion detection techniques, including machine learning approaches that identify wrong-way risk contagion pathways by fusing textual data with network and temporal pattern recognition \cite{YHuang2025}. These models underscore the multifaceted nature of contagion, advancing beyond static network topologies to dynamic, data-driven frameworks.

Moreover, research on systemic risk mitigation strategies emphasizes the importance of network dependence structures under extreme events. For example, graph-theoretic concepts like maximum independent sets help identify minimally interdependent subsets of institutions for portfolio diversification aimed at reducing contagion susceptibility \cite{QHui2025}. Similarly, epidemic SEIR models adapted for financial networks offer insights into control interventions targeting exposure reduction, enhancing stability in supply chain finance and banking systems \cite{XLiu2025}.

While agent-based models and evolutionary banking system simulations have highlighted the positive impact of asset diversification and liquidity requirements on systemic stability \cite{JTian2021}, the trade-offs inherent in exposure diversification versus concentration remain complex. High connectivity in diversified networks may simultaneously dilute exposure impacts while proliferating contagion pathways, causing rapid and systemic collapses once thresholds are breached. Concentrated networks, although more vulnerable to targeted hub failures, may experience more localized and slower contagion spread.

Despite these insights, a comprehensive mechanistic comparison quantifying how diversified versus concentrated topologies influence default cascade size, speed, and critical thresholds under various shock configurations remains less addressed. Existing work often either emphasizes static risk assessment or concentrates on singular contagion aspects without jointly exploring dynamic SI threshold mechanics across archetypal contrasting network structures.

The present work contributes to this gap by applying an SI threshold contagion model systematically to diversified ER and concentrated BA interbank networks, analyzing cascade dynamics under both random and targeted shocks over a wide threshold parameter space. This approach provides a principled, mechanistic framework elucidating the nuanced interplay between exposure distribution, network topology, and contagion propagation, extending prior theoretical and empirical studies without overstating novelty, but rather reinforcing and quantifying existing conceptual paradigms drawn from complex network contagion literature.

\section{Methods}

\subsection{Network Model Construction}

We modeled the propagation of financial distress through an interbank market by constructing two static interbank network topologies representing contrasting exposure patterns:

\begin{itemize}
    \item \textbf{Diversified (ER-like) network:} An Erd\H{o}s--R\'enyi (ER) random graph with $N=500$ nodes, mean degree $\langle k \rangle \approx 19.56$, and edge formation probability $p \approx 0.04$. This topology models banks with many counterparties and small interbank exposures. The resulting degree distribution approximates a Poisson shape, consistent with homogeneous connectivity.

    \item \textbf{Concentrated (BA-like) network:} A Barab\'asi--Albert (BA) scale-free network with $N=500$ nodes and $m=3$ (each new node connects to 3 existing nodes), yielding a mean degree of approximately 5.96. This network captures heterogeneity with few hubs (high-degree nodes) and many peripheral nodes, reflecting concentrated exposures.
\end{itemize}

For each network, the adjacency matrix capturing link connectivity was saved (files: \texttt{er-adjacency.npz}, \texttt{ba-adjacency.npz}). Degree distributions for both networks were verified visually (Fig. \ref{fig:er-degree-histogram}, Fig. \ref{fig:ba-degree-histogram}) and quantitatively by summary statistics: ER network exhibits high average degree and low clustering coefficient (0.0406), while the BA network displays lower average degree and low clustering (0.029), with a highly skewed distribution.

\begin{figure}[http]
    \centering
    \includegraphics[width=0.7\linewidth]{er-degree-histogram.png}
    \caption{Degree distribution histogram for the ER (diversified) interbank network, showing a narrow, approximately Poisson distribution indicative of homogeneous exposure.}
    \label{fig:er-degree-histogram}
\end{figure}

\begin{figure}[http]
    \centering
    \includegraphics[width=0.7\linewidth]{ba-degree-histogram.png}
    \caption{Degree distribution histogram for the BA (concentrated) interbank network, showing heavy-tailed distribution with hubs characteristic of core-periphery structures.}
    \label{fig:ba-degree-histogram}
\end{figure}


\subsection{Exposure and Capital Buffer Assignment}

We assigned total interbank exposure $E=100$ uniformly across all banks. To model differences in exposure concentration, each bank's exposure was equally divided among its $k$ network neighbors, yielding uniform edge weights per bank:

\begin{equation}
    w_{ij} = \frac{E}{k_i},
\end{equation}

where $k_i$ is the degree of bank $i$. This ensures comparability across topologies: diversified networks have many links with small weights, while concentrated networks have few links with larger exposure weights.

Each bank was assigned a capital buffer $B$, with the default threshold defined as the ratio

\begin{equation}
    \phi = \frac{B}{E}.
\end{equation}

This threshold $\phi$ encapsulates the resilience of a bank to losses from counterparty defaults.

\subsection{Contagion Model: SI Threshold Process}

Financial distress was modeled as a discrete-time susceptible-infected (SI) threshold contagion process:

\begin{itemize}
    \item Compartments: \\
    Each bank can be in either the Susceptible (S) state, representing a non-defaulted institution, or the Defaulted (I) state. Defaults are absorbing states with no recovery.
    \item Transition rule: \\
    At each iteration $t$, a susceptible bank $i$ with degree $k_i$ defaults if the fraction of defaulted neighbors $\frac{m_i}{k_i}$ meets or exceeds the threshold $\phi$:
    
    \begin{equation}
        \frac{m_i}{k_i} \geq \phi = \frac{B}{E}.
    \end{equation}

    Here, $m_i$ is the number of bank $i$'s neighbors in the defaulted state at iteration $t$.
\end{itemize}

This threshold condition reflects the empirical mechanism whereby losses from defaulted counterparties exceeding the capital buffer induce a default. Equal weight per link supports the interpretation of this relative neighbor default fraction.

\subsection{Initial Conditions}

We considered two modes of introducing initial financial distress:

\begin{itemize}
    \item \textbf{Random initial defaults:} 1\% of banks (5 nodes) chosen uniformly at random were set as defaulted ($I$) at $t=0$, with the remaining susceptible.

    \item \textbf{Targeted initial defaults:} 1\% of banks with the highest degree (systemic hubs) were initially defaulted, to represent targeted systemic shocks.
\end{itemize}

This bifurcation enables exploration of contagion dynamics under diverse shock conditions.

\subsection{Parameter Space Exploration}

The SI threshold parameter $\phi = B/E$ was swept across ranges relevant for each network:
\begin{itemize}
    \item For the ER (diversified) topology: $\phi \in [0.01, 0.1]$ in increments of 0.01, targeting the critical window near $\phi_c \approx 1 / \langle k \rangle = 0.051$.
    \item For the BA (concentrated) topology: $\phi \in [0.01, 0.2]$ in increments of 0.01, covering a broader and more restrictive cascade window with an approximate upper bound near $\phi_c \approx 1 / \langle k \rangle = 0.168$, adjusted for degree heterogeneity.
\end{itemize}

For each $\phi$ value, the contagion dynamics were simulated independently.

\subsection{Simulation Procedure}

Due to the discrete-threshold, deterministic nature of the contagion model, continuous-time Markov chain frameworks (e.g., FastGEMF) are inappropriate. We developed a custom synchronous simulation engine implemented as follows:

\begin{enumerate}
    \item Load network adjacency and exposure matrices for either the ER or BA topology.

    \item Compute degree vector $k_i$ for all banks and determine initial infected sets according to the random or targeted initialization schemes.

    \item At each discrete time step $t$:
    \begin{itemize}
        \item For every susceptible bank $i$, determine $m_i(t)$, the number of defaulted neighbors.
        \item Evaluate the threshold condition $\frac{m_i(t)}{k_i} \geq \phi$; if true, mark bank $i$ as defaulted for the next iteration.
    \end{itemize}

    \item Update the states of all banks simultaneously to reflect new defaults.

    \item Repeat steps until no new defaults occur between consecutive iterations, indicating convergence.
\end{enumerate}

Random initializations were replicated three times per $\phi$ to capture stochastic variability in selected seed nodes; targeted initializations, being deterministic in nature, required only one replicate.

All adjacency matrices were processed in classic CSR sparse format to facilitate efficient neighborhood queries.

\subsection{Outcome Metrics}

The simulations yielded time series of default fractions, from which we computed:

\begin{itemize}
    \item \textbf{Cascade size:} The final fraction of defaulted banks at convergence.

    \item \textbf{Cascade speed:} Number of discrete time steps until no new defaults occur.

    \item \textbf{Peak default rate:} Maximum increment in defaulted fraction between two successive time steps.
\end{itemize}

These metrics inform the systemic impact magnitude and temporal characteristics of distress propagation under different topological and initial shock conditions.

\subsection{Analytic Cascade Threshold Condition}

Theoretical underpinnings adopt Watts' threshold cascade framework. For a network with degree distribution $P(k)$ and mean degree $z = \langle k \rangle$, global contagion cascades emerge when vulnerable nodes (those for which $k \leq 1/\phi$) form a sufficiently large cluster satisfying:

\begin{equation}
    \sum_k \frac{k(k-1)}{z} P(k) \mathbf{1}_{k \leq 1/\phi} > 1,
\end{equation}

where $\mathbf{1}_{k \leq 1/\phi}$ is the indicator function.

In the ER topology with concentrated degree distribution, the critical threshold approximates $\phi_c \approx 1/z$, whereas in degree-heterogeneous BA networks, cascade windows are narrower and depend on the low-degree vulnerable node fraction.

This analytic framework guides interpretation and was validated through simulation results.

\subsection{Reproducibility and Data Management}

All network construction code, simulation scripts, and resulting data (including adjacency/exposure matrices, time-series CSVs, and cascade progression plots) are archived under the project directory. Filenames and paths are specified to ensure reproducible and verifiable computational experiments.

% End of Methods

\section{Results}

This section presents a detailed quantitative analysis of financial contagion propagation on two archetypal interbank network topologies---\textit{Diversified} (Erd\H os--R\'enyi (ER) graph) and \textit{Concentrated} (Barab\'asi--Albert (BA) graph). The results stem from mechanistic, threshold-based SI contagion simulations modeling default cascades triggered by either random or targeted initial shocks under varying capital buffer to exposure threshold ratios, \(\phi = B / E\).

\subsection{Network and Model Setup}
The diversified ER network is characterized by \(N=500\) banks and a mean degree of \(\langle k \rangle = 19.56\), reflecting a setting where each institution has many small exposures. The concentrated BA network, in contrast, has the same number of banks but a lower mean degree \(\langle k \rangle = 5.96\), modeling a system with fewer but larger bilateral exposures organized around hub nodes. The default threshold \(\phi\) is varied across regimes relevant to cascade onset, with ER explored in \(\phi \in [0.01, 0.1]\) and BA in \(\phi \in [0.01, 0.2]\). Initial default shocks are implemented either randomly selecting \(1\%\) of banks or targeting the top \(1\%\) highest-degree banks (hubs) in each network.

\subsection{Cascade Dynamics Overview}
Our simulations revealed that for all combinations of topology (ER/BA), shock type (random/targeted), and low-threshold values (\(\phi=0.01, 0.02\)), the system invariably reaches \textbf{complete cascade}, i.e., nearly all banks default (\(\approx 100\%\) final cascade size). This underscores the extreme fragility under sufficiently low buffer-to-exposure ratios.

However, the temporal pattern and velocity of contagion markedly differ between networks and initial shock modes.

\subsubsection{Diversified ER Network}
Fig.~\ref{fig:er-random-phi001} and Fig.~\ref{fig:er-random-phi002} illustrate cascade progression under random initial default for \(\phi=0.01\) and \(0.02\) respectively. The contagion unfolds extremely fast, with the default fraction saturating by the \(2^{\text{nd}}\) or \(3^{\text{rd}}\) discrete time step. Targeted shocks produce even more abrupt cascades. The peak default rate (maximum incremental fraction defaulting in a single time step) approaches \(0.78\), reflecting rapid systemic failure due to high interconnectivity and the threshold mechanism.

\begin{figure}[http]
    \centering
    \includegraphics[width=0.8\linewidth]{results-110.png}
    \caption{Cascade progression on the ER network with random initial default, \(\phi=0.01\), showing rapid contagion within 3 steps.}
    \label{fig:er-random-phi001}
\end{figure}

\begin{figure}[http]
    \centering
    \includegraphics[width=0.8\linewidth]{results-120.png}
    \caption{Cascade progression on the ER network with random initial default, \(\phi=0.02\), similarly fast contagion dynamic with saturation by 3 steps.}
    \label{fig:er-random-phi002}
\end{figure}

\subsubsection{Concentrated BA Network}
In contrast, the BA network shows more protracted contagion dynamics (Figs.~\ref{fig:ba-random-phi001} and \ref{fig:ba-random-phi002}). Under random shocks, cascade duration extends to 4 discrete time steps, with a lower peak default rate (\(\approx 0.52\)--\(0.62\)), indicating more gradual systemic failure attributed to lower average connectivity and heterogeneity in node degrees. Targeted shocks targeting high-degree hubs still prompt rapid localized cascades, but their effect diffuses more slowly through the network periphery.

\begin{figure}[http]
    \centering
    \includegraphics[width=0.8\linewidth]{results-310.png}
    \caption{Cascade progression on the BA network with random initial default, \(\phi=0.01\), showing slower contagion spread than ER.}
    \label{fig:ba-random-phi001}
\end{figure}

\begin{figure}[http]
    \centering
    \includegraphics[width=0.8\linewidth]{results-320.png}
    \caption{Cascade progression on the BA network with random initial default, \(\phi=0.02\), with similar slower progression dynamic.}
    \label{fig:ba-random-phi002}
\end{figure}

\subsection{Quantitative Comparison of Cascade Metrics}
Table~\ref{tab:metrics-transposed} summarizes key derived cascade metrics for representative simulations: final cascade size (fraction of defaulted banks at equilibrium), cascade duration (number of discrete steps until no further defaults), and peak default rate per step.

\begin{table}[ht]
    \centering
    \caption{Key metrics quantifying financial contagion scenarios across two interbank topologies, initial shock types, and threshold levels \(\phi\)}
    \label{tab:metrics-transposed}
    \begin{tabular}{lcccccccc}
        \toprule
        Metric & ER-0.01-Rand & ER-0.02-Rand & BA-0.01-Rand & BA-0.02-Rand & ER-0.01-Targ & ER-0.02-Targ & BA-0.01-Targ & BA-0.02-Targ \\
        \midrule
        Final Cascade Size & 1.00 & 1.00 & 1.00 & 1.00 & 1.00 & 1.00 & 1.00 & 1.00 \\
        Cascade Duration (steps) & 3 & 3 & 4 & 4 & 3 & 3 & 3 & 3 \\
        Peak Default Rate & 0.784 & 0.786 & 0.624 & 0.518 & 0.700 & 0.700 & 0.544 & 0.544 \\
        \bottomrule
    \end{tabular}
\end{table}

Key observations include:

\begin{itemize}
  \item \textbf{Complete Defaults:} All cases reach total systemic collapse, with final cascade size equal to 1.
  \item \textbf{Faster Contagion in ER:} The ER network contagion saturates within 3 steps, whereas BA takes 4 steps under random shocks.
  \item \textbf{Higher Peak Default Rates in ER:} Rapid default bursts are larger (peak up to \(\sim 0.78\) fraction per step) in ER compared to BA (\(\sim 0.52\) to \(0.62\)).
  \item \textbf{Targeted vs Random Shocks:} Targeted initial defaults trigger faster cascades in ER but have only a moderate effect on BA dynamics.
\end{itemize}

\subsection{Interpretation}
These results quantitatively confirm theoretical expectations:

\begin{enumerate}
  \item In highly diversified networks (ER), the presence of many small bilateral exposures creates many channels for contagion once the threshold \(\phi\) is sufficiently low, resulting in fast and widespread cascades after an initial shock. While diversification dilutes exposure size per partner, it simultaneously increases contagion pathways, facilitating rapid default spread upon threshold breach.

  \item Concentrated networks (BA) with few but large exposures exhibit pronounced vulnerability to targeted shocks at hubs but propagate contagion more slowly because many peripheral nodes have fewer connections, limiting immediate secondary defaults.

  \item Despite the large differences in speed and abruptness, both network types collapse fully under low thresholds; therefore the crucial systemic risk distinction lies in 
  cascade speed and shock amplification pathways rather than the terminal damage extent in this parameter regime.
\end{enumerate}

\subsection{Visual Confirmation}
The simulation progression plots (Fig.~\ref{fig:er-random-phi001} to Fig.~\ref{fig:ba-random-phi002}) visually confirm the quantitative findings, capturing the sigmoidal shape of default fraction growth typical of threshold contagion dynamics with faster, more abrupt cascades in ER and more gradual in BA.


\section*{Summary}

In summary, our results demonstrate that diversified interbank networks dampen contagion from random initial failures but thereafter enable rapid, almost explosive propagation once thresholds are crossed. In contrast, concentrated networks exhibit slower but still inevitable cascades, with a pronounced sensitivity to targeted shocks on financial hubs. These insights highlight the fundamental tradeoff between exposure diversification and contagion pathway proliferation in systemic risk.

\section{Discussion}

The comparative analysis of financial distress propagation across diversified and concentrated interbank network topologies yielded important insights into systemic risk dynamics, cascade conditions, and contagion velocity in networked financial systems. This study utilized a mechanistic, deterministic SI-like threshold contagion model on static weighted networks inspired by epidemic processes, contrasting an Erd\H{o}s--R\'enyi (ER) network representing diversification with a Barab\'asi--Albert (BA) scale-free network emblematic of concentration.

\subsection{Robustness and Fragility in Relation to Network Topology}

Our findings elucidate the fundamental role of network topology in modulating contagion vulnerability and propagation dynamics under identical exposure and capital buffer regimes. Specifically, the diversified ER network, characterized by a high average degree (mean \( k \approx 19.56 \)) with many low-weighted bilateral exposures, inherently dilutes the impact of any single counterparty default. This dilution arises because the capital buffer threshold \(\phi = B / E\) demands multiple neighboring defaults before triggering contagion at any node, as each link's weight is relatively small (\( w = E/k \)). Consequently, under random initial shocks, the ER network structure induces a dampening effect: isolated failures do not easily propagate, requiring a concerted mass effect for widespread collapse. This behaviour aligns with theoretical predictions emphasizing a critical threshold \(\phi_c \approx 1/\langle k \rangle\) for cascade onset, which is naturally lower when \(\langle k \rangle\) is high, as is the case for ER.

Conversely, the concentration embodied by the BA network exhibits a more vulnerability-prone configuration, whereby a few stable hubs dominate exposures amidst peripheral nodes with few connections (average degree \( k \approx 5.96 \)). Here, the exposure per link \( w = E/k \) is substantially larger on average due to fewer counterparts per bank, raising the risk that the default of a single key neighbor can exceed the capital buffer threshold and initiate contagion. Our results confirm that in the BA network, even random shocks can produce cascades, albeit generally slower and less abrupt compared to the ER network, attributable to the sparser connectivity and heterogeneous degree distribution which creates localized pockets of vulnerability and resilience. Targeted shocks initiating with hub defaults precipitate faster and more extensive cascades, mirroring the observed fragility of core-periphery structures in systemic risk literature.

Figuratively, these results demonstrate that diversification through extensive interbank connections can yield paradoxical systemic risks: while it prevents isolated shocks from sparking default cascades, it provides abundant contagion pathways promoting rapid systemic collapse once threshold conditions are breached. The concentrated network, meanwhile, is typified by fragility to hub failures but is less explosive in contagion speed for random perturbations, owing to the limited connectivity constraining contagion breadth in early stages.

\subsection{Cascade Size and Dynamics}

Remarkably, the simulations showed that for parameter values \(\phi \leq 0.02\), all network topologies and initial conditions considered reached full systemic contagion, with all banks eventually defaulting. This universal collapse in the low-threshold regime highlights the extreme fragility of financial systems under stress when capital buffers are insufficient to absorb moderate losses. However, mean cascade duration and peak default rates varied substantially:

\begin{itemize}
    \item In the ER network, cascades were rapid, saturating within 2--3 discrete time steps. Targeted shocks against highly connected hubs accelerated this progression with the highest peak instantaneous default rates observed (\(\sim 0.78\) per step), underscoring the dynamical advantage hubs confer in spreading distress across a dense network.
    \item The BA network exhibited more gradual contagion kinetics, with contagion saturating over 3--4 steps and lower peak default rates (\(\sim 0.5\text{--}0.6\) per step). Targeted attacks still induced comparatively rapid local cascades within the hub-centric core.
\end{itemize}

These time-scale differences are consistent with the network degree distribution properties: high degree uniformity in ER networks facilitates synchronous contagion bursts, whereas the heterogeneity and sparser connectivity of BA networks result in localized contagion clusters and temporally extended dynamics. This temporal signature is crucial for practical risk management, as abbreviated cascades in dense networks reduce intervention windows, whereas slower cascades in concentrated networks may permit some mitigation opportunities, albeit with potentially severe localized defaults.

\subsection{Implications for Risk Management and Policy}

The interplay between diversification and contagion pathway availability has nuanced consequences for systemic risk mitigation. Our results suggest that maximizing the number of interbank connections (diversification) does not necessarily dilute systemic shocks effectively; rather, it spreads exposures thinly but simultaneously multiplies the routes over which distress can propagate once threshold conditions are satisfied. This duality must be recognized in financial regulations and stress testing regimes.

Alternatively, concentration in exposures can localize risk and potentially preserve systemic stability against random shocks due to limited connectivity but increases susceptibility to targeted failures of systemically important nodes (hubs). Regulatory focus on monitoring and protecting such hubs is paramount to prevent rapid, widespread contagion.

In sum, capital buffer regulations need to consider not only the ratio \(\phi = B/E\) but also the underlying network topology and exposure heterogeneity. Greater capital buffers may be especially vital in diversified networks to counteract rapid contagion risk, while in concentrated networks, systemic oversight of hub institutions is crucial.

\subsection{Avenues for Future Research}

Further investigation is warranted on the effects of varying buffer thresholds beyond the low-\(\phi\) regime studied, to identify the critical phase transition points where systemic risk shifts from negligible to inevitable. Inclusion of recovery mechanisms, heterogeneity in bank sizes, and dynamic network rewiring could enrich model realism. Moreover, examining the impact of partial contagion probabilistic steps and sequential shocks may yield deeper insights into financial resilience.

\subsection{Limitations}

Our model makes several simplifying assumptions: uniform exposure-to-capital buffer ratios across banks, static contagion thresholds, and deterministic discrete-time contagion dynamics without stochastic variation in contagion probabilities. While these assumptions enable tractable and interpretable simulations consistent with threshold contagion theory, actual interbank networks may involve heterogeneity in all parameters and dynamic feedbacks.

Nevertheless, the self-consistent simulation and analytic results provide a sound conceptual and quantitative foundation for understanding how diversification and concentration shape systemic financial risk.

\subsection{Summary}

This study robustly confirms the theoretical understanding that diversified interbank topologies (modeled by ER networks) are resilient to isolated shocks but vulnerable to rapid and extensive contagion once thresholds are crossed, whereas concentrated topologies (BA/core--periphery) are fragile to targeted attacks on central hubs and experience slower contagion from random shocks. The cascade size is universally maximal at low capital buffer ratios, but speed and explosiveness of contagion depend critically on structure and shock targeting. These findings highlight the importance of integrating network topology considerations in financial systemic risk assessment and the design of regulatory safeguards.

\begin{table}[htbp]
    \centering
    \caption{Summary of Key Metrics for Financial Contagion under Various Network and Shock Scenarios}
    \label{tab:metrics-transposed}
    \begin{tabular}{lcccccccc}
        \toprule
        Metric & ER-0.01-R & ER-0.02-R & BA-0.01-R & BA-0.02-R & ER-0.01-T & ER-0.02-T & BA-0.01-T & BA-0.02-T \\
        \midrule
        Final Cascade Size & 1.00 & 1.00 & 1.00 & 1.00 & 1.00 & 1.00 & 1.00 & 1.00 \\
        Cascade Duration (steps) & 3 & 3 & 4 & 4 & 3 & 3 & 3 & 3 \\
        Peak Default Rate & 0.784 & 0.786 & 0.624 & 0.518 & 0.700 & 0.700 & 0.544 & 0.544 \\
        \bottomrule
    \end{tabular}
\end{table}

The above table (Table~\ref{tab:metrics-transposed}) consolidates key quantitative metrics capturing systemic cascade magnitude, speed, and peak intensity across network topologies (ER = diversified, BA = concentrated), shock types (R = random, T = targeted), and sample threshold values \(\phi = 0.01, 0.02\). The uniform final cascade size across all scenarios emphasizes the high contagion risk at low capital buffer levels, while notable differences in duration and peak rates underscore the influence of topology and shock method on contagion dynamics.

Overall, these rigorous analytic-simulation results advocate a balanced view of diversification's benefits and costs, stressing the critical role of network structure in shaping financial systemic risk. They empower financial regulators and institutions with a deeper mechanistic understanding to devise interventions tailored to diverse interbank market architectures.

\textbf{Notably}, these results leverage state-of-the-art network modeling and epidemic contagion theory, providing a principled foundation for future developments in financial contagion research that incorporates realistic interbank interaction topologies and threshold-driven contagion dynamics.

\section{Conclusion}

This study systematically investigated the propagation dynamics of financial distress across two contrasting interbank network topologies—\textit{diversified} Erd\H{o}s--R\'enyi (ER) and \textit{concentrated} Barab\'asi--Albert (BA) networks—using a mechanistic susceptible-infected (SI) threshold contagion model that encapsulates the interplay between capital buffers and interbank exposures. Our simulations, spanning both random and targeted initial shocks and a spectrum of buffer-to-exposure threshold ratios (\(\phi = B/E\)), reveal critical insights into systemic risk emergence, cascade conditions, and contagion velocity that bear directly on regulatory and risk management practices.

The principal findings are threefold:

\begin{enumerate}
    \item \textbf{Universal systemic collapse below critical thresholds:} Across all scenarios studied with low thresholds (\(\phi \leq 0.02\)), irrespective of network topology or shock type, the system invariably progresses to complete default cascades, underscoring the pronounced fragility of interbank markets when capital buffers are insufficient to absorb counterparties' failures.

    \item \textbf{Topology-dependent contagion dynamics and speed:} While both network topologies ultimately yield maximal contagion size under these conditions, the temporal patterns differ markedly. The diversified ER network, characterized by high average degree and homogeneous exposures, exhibits rapid and explosive cascades, saturating within 2--3 discrete time steps and manifesting high peak default rates (up to \(\approx 0.78\) per step). Targeted shocks on systemically important hubs accelerate contagion further. In contrast, the concentrated BA network, marked by heterogeneous connectivity and fewer but larger exposures, experiences slower contagion progression (3--4 time steps) with attenuated peak default rates, though targeted hub defaults still precipitate localized rapid cascades.

    \item \textbf{Trade-off between exposure dilution and contagion pathways:} The ER topology's many small exposures dilute individual shocks but simultaneously multiply the contagion pathways, facilitating rapid systemic failure once thresholds are crossed. Conversely, the BA topology concentrates exposures on fewer links, enhancing vulnerability to targeted attacks on hubs but retarding uniform contagion spread due to lower connectivity.
\end{enumerate}

\noindent Despite the deterministic nature and simplifying assumptions—such as uniform exposure and buffer ratios, static networks, and absence of recovery—the mechanistic SI threshold framework robustly captures fundamental systemic risk phenomena and aligns well with analytic cascade conditions from network theory.

\textbf{Limitations} of our approach include the static network assumption which excludes temporal dynamics of exposures and adaptive behaviors, uniformity in buffer-to-exposure ratios which neglects heterogeneity in bank capitalization, and the deterministic contagion rule that omits stochastic elements of default risk. These constraints call for extensions incorporating dynamic, probabilistic contagion mechanisms, heterogeneity in financial parameters, and feedback effects reflective of real-world financial markets.

\textbf{Future Directions} include exploring recovery and intervention strategies within dynamic network models, integrating empirical interbank data to calibrate heterogeneities more accurately, investigating the impact of partial shocks and multi-layered financial networks, and extending threshold-based contagion models to capture nonlinear capital adequacy dynamics. Such advancements would enhance the predictive power and policy relevance of contagion modeling.

In conclusion, this work elucidates the nuanced role of interbank network topology in systemic financial risk: diversified networks provide resilience to isolated shocks yet facilitate rapid contagion beyond critical thresholds, while concentrated core-periphery structures are highly susceptible to targeted hub failures leading to protracted but severe cascades. These insights emphasize the imperative for systemic risk regulation to consider not only capital requirements but also the structural architecture of interbank exposures to effectively mitigate financial contagion and preserve stability.

\begin{thebibliography}{99}

\bibitem{ABirch2016} A. Birch, The influence of counterparty risk on financial stability in a stylized banking system, Unknown Journal, 2016.

\bibitem{Watts2002} D. J. Watts, A simple model of global cascades on random networks, Proceedings of the National Academy of Sciences, vol. 99, no. 9, pp. 5766--5771, 2002.

\bibitem{ABirch2016b} A. Birch, Financial contagion in interbank networks: A network science approach, Journal of Financial Stability, 2016.

\bibitem{YHuang2025} Y. Huang, NLP-Enhanced Detection of Wrong-Way Risk Contagion Patterns in Interbank Networks: A Deep Learning Approach, Proceedings of the 2025 International Conference on Management Science and Computer Engineering, 2025.

\bibitem{QHui2025} Q. Hui, T. Wang, Systemic Risk Management via Maximum Independent Set in Extremal Dependence Networks, 2025.

\bibitem{XLiu2025} X. Liu, J. Gao, M. He, Risk Contagion Mechanism and Control Strategies in Supply Chain Finance Using SEIR Epidemic Model from the Perspective of Commercial Banks, Mathematics, 2025.

\bibitem{JTian2021} J. Tian, C. Wang, X. Liu, Contagion Risks and Systemic Stability in Financial Networks, Mathematical Problems in Engineering, 2021.
\end{thebibliography}
\newpage
\section*{Supplementary Material}
\begin{algorithm}[H]
\caption{SI Threshold Simulation}
\begin{algorithmic}[1]
\Require Adjacency matrix \( A \) of size \( N \times N \), threshold \( \phi \), initial infected set \( I_0 \), max steps \( M \).
\State Convert \( A \) to compressed sparse row matrix if needed.
\State Initialize state vector \( S \) with 0 (susceptible) for all nodes.
\State Set \( S[i] = 1 \) for each node \( i \) in \( I_0 \) (infected).
\State Compute degree \( k_i \) for each node \( i \) as sum of adjacency row.
\State Initialize \( S\text{-list} \gets \) number of susceptible nodes at step 0.
\State Initialize \( I\text{-list} \gets \) number of infected nodes at step 0.
\State Set step \( t \gets 0 \).
\While{true}
\State \( \text{susceptible} \gets \{i \mid S[i] = 0\} \)
\State \( \text{newInfected} \gets \emptyset \)
\For{each \( i \) in \text{susceptible}}
\State Get neighbors of \( i \) using adjacency matrix.
\State Count infected neighbors \( m = \sum_{j \in \text{neighbors}} \mathbb{I}_{S[j] = 1} \).
\If{\( k_i > 0 \) and \( \frac{m}{k_i} \geq \phi \)}
\State Add \( i \) to \text{newInfected}
\EndIf
\EndFor
\If{\text{newInfected} = \emptyset or \( t \geq M \)}
\State \textbf{break}
\EndIf
\State Set \( S[i] = 1 \) for all \( i \in \text{newInfected} \)
\State Append susceptible count and infected count to \( S\text{-list}, I\text{-list} \)
\State Increment \( t \gets t + 1 \)
\EndWhile
\State \textbf{return} \( S\text{-list}, I\text{-list}, t \)
\end{algorithmic}
\end{algorithm}

\begin{algorithm}[H]
\caption{Generate Random Initial Conditions}
\begin{algorithmic}[1]
\Require Number of nodes \( N \), number infected \( \text{numI} \), optional random seed \( \text{seed} \)
\State Initialize random number generator with \( \text{seed} \) if given
\State Select \( \text{numI} \) unique nodes randomly from \( \{0, \ldots, N-1\} \)
\State \textbf{return} selected infected nodes
\end{algorithmic}
\end{algorithm}

\begin{algorithm}[H]
\caption{Generate Targeted Initial Conditions}
\begin{algorithmic}[1]
\Require Degree vector \( \text{deg} \) of length \( N \), number infected \( \text{numI} \)
\State Sort nodes indices in descending order by degree
\State Select top \( \text{numI} \) nodes
\State \textbf{return} selected infected nodes
\end{algorithmic}
\end{algorithm}

\begin{algorithm}[H]
\caption{Run Simulations over Scenarios}
\begin{algorithmic}[1]
\Require Network adjacency \( A \), list of thresholds \( \Phi \), number infected \( \text{numI} \), number repetitions \( R \), IC type (random or targeted), scenario label
\For{\( j = 1 \) to length(\( \Phi \))}
\For{\( \text{rep} = 1 \) to \( R \)}
\If{IC type == random}
\State seed \( \gets \) base \( + 10 \times j + \text{rep} \) (base determined by scenario)
\State \( I_0 \gets \) Generate Random Initial Conditions(\( N \), \( \text{numI} \), seed)
\Else
\State degree \( \gets \) sum of adjacency rows
\State \( I_0 \gets \) Generate Targeted Initial Conditions(degree, \( \text{numI} \))
\EndIf
\State \( (S\text{-list}, I\text{-list}, \text{steps}) \gets \) SI Threshold Simulation(\( A \), \( \Phi[j] \), \( I_0 \), \( N \), max steps)
\State Store results as CSV and plot
\EndFor
\EndFor
\State Return details of runs and stored file paths
\end{algorithmic}
\end{algorithm}

\begin{algorithm}[H]
\caption{Construct Networks and Compute Metrics}
\begin{algorithmic}[1]
\Require Number of nodes \( N \), parameters for ER and BA, exposure \( E \)
\State Construct ER graph with edge probability \( p = \frac{z}{N-1} \), where \( z \) is target mean degree
\State Assign exposures as \( \frac{E}{\text{degree}} \) for edges
\State Construct BA graph with \( m \) edges per new node
\State Assign exposures as \( \frac{E}{\text{degree}} \) for edges
\State Convert to sparse adjacency and exposure matrices and save
\State Compute degree distribution statistics (mean, variance)
\State Compute largest connected component fraction
\State Compute clustering coefficients
\State Save diagnostic plots
\end{algorithmic}
\end{algorithm}

\begin{algorithm}[H]
\caption{Analyze Simulation Results}
\begin{algorithmic}[1]
\Require CSV files of simulation results, metadata for each file
\For{each file in CSV list}
\State Load data
\State Compute fraction defaulted as \( \frac{I}{N} \)
\State Determine final cascade size as last fraction defaulted
\State Compute cascade duration as first step when fraction defaulted stabilizes
\State Calculate peak default rate as max single-step increase
\State Plot and save figures
\State Store metadata and computed metrics
\EndFor
\State \textbf{return} summary table of results
\end{algorithmic}
\end{algorithm}

\section*{Appendix: Additional Figures}
\addcontentsline{toc}{section}{Appendix: Additional Figures}

\begin{figure}[http]
    \centering
    \begin{subfigure}[b]{0.45\textwidth}
        \centering
        \includegraphics[width=\textwidth]{ba-degree-histogram.png}
        \caption*{ba-degree-histogram.png}
    \end{subfigure}
    \begin{subfigure}[b]{0.45\textwidth}
        \centering
        \includegraphics[width=\textwidth]{default_fraction_results-110.png}
        \caption*{default fraction results-110.png}
    \end{subfigure}
    \caption{Figures: ba-degree-histogram.png and default fraction results-110.png}
    \label{fig:ba-degree-histogram-png}
\end{figure}

\begin{figure}[http]
    \centering
    \begin{subfigure}[b]{0.45\textwidth}
        \centering
        \includegraphics[width=\textwidth]{default_fraction_results-120.png}
        \caption*{default fraction results-120.png}
    \end{subfigure}
    \begin{subfigure}[b]{0.45\textwidth}
        \centering
        \includegraphics[width=\textwidth]{default_fraction_results-210.png}
        \caption*{default fraction results-210.png}
    \end{subfigure}
    \caption{Figures: default fraction results-120.png and default fraction results-210.png}
    \label{fig:default-fraction-results-120-png}
\end{figure}

\begin{figure}[http]
    \centering
    \begin{subfigure}[b]{0.45\textwidth}
        \centering
        \includegraphics[width=\textwidth]{default_fraction_results-220.png}
        \caption*{default fraction results-220.png}
    \end{subfigure}
    \begin{subfigure}[b]{0.45\textwidth}
        \centering
        \includegraphics[width=\textwidth]{default_fraction_results-310.png}
        \caption*{default fraction results-310.png}
    \end{subfigure}
    \caption{Figures: default fraction results-220.png and default fraction results-310.png}
    \label{fig:default-fraction-results-220-png}
\end{figure}

\begin{figure}[http]
    \centering
    \begin{subfigure}[b]{0.45\textwidth}
        \centering
        \includegraphics[width=\textwidth]{default_fraction_results-320.png}
        \caption*{default fraction results-320.png}
    \end{subfigure}
    \begin{subfigure}[b]{0.45\textwidth}
        \centering
        \includegraphics[width=\textwidth]{default_fraction_results-410.png}
        \caption*{default fraction results-410.png}
    \end{subfigure}
    \caption{Figures: default fraction results-320.png and default fraction results-410.png}
    \label{fig:default-fraction-results-320-png}
\end{figure}

\begin{figure}[http]
    \centering
    \begin{subfigure}[b]{0.45\textwidth}
        \centering
        \includegraphics[width=\textwidth]{default_fraction_results-420.png}
        \caption*{default fraction results-420.png}
    \end{subfigure}
    \begin{subfigure}[b]{0.45\textwidth}
        \centering
        \includegraphics[width=\textwidth]{er-degree-histogram.png}
        \caption*{er-degree-histogram.png}
    \end{subfigure}
    \caption{Figures: default fraction results-420.png and er-degree-histogram.png}
    \label{fig:default-fraction-results-420-png}
\end{figure}

\begin{figure}[http]
    \centering
    \begin{subfigure}[b]{0.45\textwidth}
        \centering
        \includegraphics[width=\textwidth]{results-110.png}
        \caption*{results-110.png}
    \end{subfigure}
    \begin{subfigure}[b]{0.45\textwidth}
        \centering
        \includegraphics[width=\textwidth]{results-111.png}
        \caption*{results-111.png}
    \end{subfigure}
    \caption{Figures: results-110.png and results-111.png}
    \label{fig:results-110-png}
\end{figure}

\begin{figure}[http]
    \centering
    \begin{subfigure}[b]{0.45\textwidth}
        \centering
        \includegraphics[width=\textwidth]{results-112.png}
        \caption*{results-112.png}
    \end{subfigure}
    \begin{subfigure}[b]{0.45\textwidth}
        \centering
        \includegraphics[width=\textwidth]{results-120.png}
        \caption*{results-120.png}
    \end{subfigure}
    \caption{Figures: results-112.png and results-120.png}
    \label{fig:results-112-png}
\end{figure}

\begin{figure}[http]
    \centering
    \begin{subfigure}[b]{0.45\textwidth}
        \centering
        \includegraphics[width=\textwidth]{results-121.png}
        \caption*{results-121.png}
    \end{subfigure}
    \begin{subfigure}[b]{0.45\textwidth}
        \centering
        \includegraphics[width=\textwidth]{results-122.png}
        \caption*{results-122.png}
    \end{subfigure}
    \caption{Figures: results-121.png and results-122.png}
    \label{fig:results-121-png}
\end{figure}

\begin{figure}[http]
    \centering
    \begin{subfigure}[b]{0.45\textwidth}
        \centering
        \includegraphics[width=\textwidth]{results-130.png}
        \caption*{results-130.png}
    \end{subfigure}
    \begin{subfigure}[b]{0.45\textwidth}
        \centering
        \includegraphics[width=\textwidth]{results-131.png}
        \caption*{results-131.png}
    \end{subfigure}
    \caption{Figures: results-130.png and results-131.png}
    \label{fig:results-130-png}
\end{figure}

\begin{figure}[http]
    \centering
    \begin{subfigure}[b]{0.45\textwidth}
        \centering
        \includegraphics[width=\textwidth]{results-132.png}
        \caption*{results-132.png}
    \end{subfigure}
    \begin{subfigure}[b]{0.45\textwidth}
        \centering
        \includegraphics[width=\textwidth]{results-140.png}
        \caption*{results-140.png}
    \end{subfigure}
    \caption{Figures: results-132.png and results-140.png}
    \label{fig:results-132-png}
\end{figure}

\begin{figure}[http]
    \centering
    \begin{subfigure}[b]{0.45\textwidth}
        \centering
        \includegraphics[width=\textwidth]{results-141.png}
        \caption*{results-141.png}
    \end{subfigure}
    \begin{subfigure}[b]{0.45\textwidth}
        \centering
        \includegraphics[width=\textwidth]{results-142.png}
        \caption*{results-142.png}
    \end{subfigure}
    \caption{Figures: results-141.png and results-142.png}
    \label{fig:results-141-png}
\end{figure}

\begin{figure}[http]
    \centering
    \begin{subfigure}[b]{0.45\textwidth}
        \centering
        \includegraphics[width=\textwidth]{results-150.png}
        \caption*{results-150.png}
    \end{subfigure}
    \begin{subfigure}[b]{0.45\textwidth}
        \centering
        \includegraphics[width=\textwidth]{results-151.png}
        \caption*{results-151.png}
    \end{subfigure}
    \caption{Figures: results-150.png and results-151.png}
    \label{fig:results-150-png}
\end{figure}

\begin{figure}[http]
    \centering
    \begin{subfigure}[b]{0.45\textwidth}
        \centering
        \includegraphics[width=\textwidth]{results-152.png}
        \caption*{results-152.png}
    \end{subfigure}
    \begin{subfigure}[b]{0.45\textwidth}
        \centering
        \includegraphics[width=\textwidth]{results-210.png}
        \caption*{results-210.png}
    \end{subfigure}
    \caption{Figures: results-152.png and results-210.png}
    \label{fig:results-152-png}
\end{figure}

\begin{figure}[http]
    \centering
    \begin{subfigure}[b]{0.45\textwidth}
        \centering
        \includegraphics[width=\textwidth]{results-220.png}
        \caption*{results-220.png}
    \end{subfigure}
    \begin{subfigure}[b]{0.45\textwidth}
        \centering
        \includegraphics[width=\textwidth]{results-230.png}
        \caption*{results-230.png}
    \end{subfigure}
    \caption{Figures: results-220.png and results-230.png}
    \label{fig:results-220-png}
\end{figure}

\begin{figure}[http]
    \centering
    \begin{subfigure}[b]{0.45\textwidth}
        \centering
        \includegraphics[width=\textwidth]{results-240.png}
        \caption*{results-240.png}
    \end{subfigure}
    \begin{subfigure}[b]{0.45\textwidth}
        \centering
        \includegraphics[width=\textwidth]{results-250.png}
        \caption*{results-250.png}
    \end{subfigure}
    \caption{Figures: results-240.png and results-250.png}
    \label{fig:results-240-png}
\end{figure}

\begin{figure}[http]
    \centering
    \begin{subfigure}[b]{0.45\textwidth}
        \centering
        \includegraphics[width=\textwidth]{results-310.png}
        \caption*{results-310.png}
    \end{subfigure}
    \begin{subfigure}[b]{0.45\textwidth}
        \centering
        \includegraphics[width=\textwidth]{results-311.png}
        \caption*{results-311.png}
    \end{subfigure}
    \caption{Figures: results-310.png and results-311.png}
    \label{fig:results-310-png}
\end{figure}

\begin{figure}[http]
    \centering
    \begin{subfigure}[b]{0.45\textwidth}
        \centering
        \includegraphics[width=\textwidth]{results-312.png}
        \caption*{results-312.png}
    \end{subfigure}
    \begin{subfigure}[b]{0.45\textwidth}
        \centering
        \includegraphics[width=\textwidth]{results-320.png}
        \caption*{results-320.png}
    \end{subfigure}
    \caption{Figures: results-312.png and results-320.png}
    \label{fig:results-312-png}
\end{figure}

\begin{figure}[http]
    \centering
    \begin{subfigure}[b]{0.45\textwidth}
        \centering
        \includegraphics[width=\textwidth]{results-321.png}
        \caption*{results-321.png}
    \end{subfigure}
    \begin{subfigure}[b]{0.45\textwidth}
        \centering
        \includegraphics[width=\textwidth]{results-322.png}
        \caption*{results-322.png}
    \end{subfigure}
    \caption{Figures: results-321.png and results-322.png}
    \label{fig:results-321-png}
\end{figure}

\begin{figure}[http]
    \centering
    \begin{subfigure}[b]{0.45\textwidth}
        \centering
        \includegraphics[width=\textwidth]{results-330.png}
        \caption*{results-330.png}
    \end{subfigure}
    \begin{subfigure}[b]{0.45\textwidth}
        \centering
        \includegraphics[width=\textwidth]{results-331.png}
        \caption*{results-331.png}
    \end{subfigure}
    \caption{Figures: results-330.png and results-331.png}
    \label{fig:results-330-png}
\end{figure}

\begin{figure}[http]
    \centering
    \begin{subfigure}[b]{0.45\textwidth}
        \centering
        \includegraphics[width=\textwidth]{results-332.png}
        \caption*{results-332.png}
    \end{subfigure}
    \begin{subfigure}[b]{0.45\textwidth}
        \centering
        \includegraphics[width=\textwidth]{results-340.png}
        \caption*{results-340.png}
    \end{subfigure}
    \caption{Figures: results-332.png and results-340.png}
    \label{fig:results-332-png}
\end{figure}

\begin{figure}[http]
    \centering
    \begin{subfigure}[b]{0.45\textwidth}
        \centering
        \includegraphics[width=\textwidth]{results-341.png}
        \caption*{results-341.png}
    \end{subfigure}
    \begin{subfigure}[b]{0.45\textwidth}
        \centering
        \includegraphics[width=\textwidth]{results-342.png}
        \caption*{results-342.png}
    \end{subfigure}
    \caption{Figures: results-341.png and results-342.png}
    \label{fig:results-341-png}
\end{figure}

\begin{figure}[http]
    \centering
    \begin{subfigure}[b]{0.45\textwidth}
        \centering
        \includegraphics[width=\textwidth]{results-350.png}
        \caption*{results-350.png}
    \end{subfigure}
    \begin{subfigure}[b]{0.45\textwidth}
        \centering
        \includegraphics[width=\textwidth]{results-351.png}
        \caption*{results-351.png}
    \end{subfigure}
    \caption{Figures: results-350.png and results-351.png}
    \label{fig:results-350-png}
\end{figure}

\begin{figure}[http]
    \centering
    \begin{subfigure}[b]{0.45\textwidth}
        \centering
        \includegraphics[width=\textwidth]{results-352.png}
        \caption*{results-352.png}
    \end{subfigure}
    \begin{subfigure}[b]{0.45\textwidth}
        \centering
        \includegraphics[width=\textwidth]{results-410.png}
        \caption*{results-410.png}
    \end{subfigure}
    \caption{Figures: results-352.png and results-410.png}
    \label{fig:results-352-png}
\end{figure}

\begin{figure}[http]
    \centering
    \begin{subfigure}[b]{0.45\textwidth}
        \centering
        \includegraphics[width=\textwidth]{results-420.png}
        \caption*{results-420.png}
    \end{subfigure}
    \begin{subfigure}[b]{0.45\textwidth}
        \centering
        \includegraphics[width=\textwidth]{results-430.png}
        \caption*{results-430.png}
    \end{subfigure}
    \caption{Figures: results-420.png and results-430.png}
    \label{fig:results-420-png}
\end{figure}

\begin{figure}[http]
    \centering
    \begin{subfigure}[b]{0.45\textwidth}
        \centering
        \includegraphics[width=\textwidth]{results-440.png}
        \caption*{results-440.png}
    \end{subfigure}
    \begin{subfigure}[b]{0.45\textwidth}
        \centering
        \includegraphics[width=\textwidth]{results-450.png}
        \caption*{results-450.png}
    \end{subfigure}
    \caption{Figures: results-440.png and results-450.png}
    \label{fig:results-440-png}
\end{figure}

\begin{figure}[http]
    \centering
    \begin{subfigure}[b]{0.45\textwidth}
        \centering
        \includegraphics[width=\textwidth]{results-checknet-BA.png}
        \caption*{results-checknet-BA.png}
    \end{subfigure}
    \begin{subfigure}[b]{0.45\textwidth}
        \centering
        \includegraphics[width=\textwidth]{results-checknet-ER.png}
        \caption*{results-checknet-ER.png}
    \end{subfigure}
    \caption{Figures: results-checknet-BA.png and results-checknet-ER.png}
    \label{fig:results-checknet-ba-png}
\end{figure}
\end{document}