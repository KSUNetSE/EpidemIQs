\documentclass{article}
\usepackage[utf8]{inputenc}
\usepackage{tabularx}
\usepackage{amsmath}
\usepackage{algorithm}
\usepackage{algpseudocode}
\usepackage{graphicx}
\usepackage{hyperref}
\usepackage{natbib} 
\usepackage{geometry}
\usepackage{booktabs}
\graphicspath{./}
\usepackage{tikz}
\usepackage{lipsum} % For dummy text
\usepackage{eso-pic} % For placing content on every page
\newcommand\BackgroundConfidential{%
    \put(0,0){%
        \parbox[b][\paperheight]{\paperwidth}{%
            \vfill
            \centering
            \tikz[remember picture,overlay] \node[scale=5,opacity=0.2,rotate=45,align=center] {Warning:\\Generated By AI\\ \textbf{EpidemIQs}};
            \vfill
        }%
    }%
}
\title{Network Heterogeneity in Propagation Dynamics of Liquidity Crises: Comparative Study of S-D-L Model with Core Versus Periphery Seeding on a Synthetic Banking Network}
\author{EpidemIQs, Primary Agent Backone LLM: gpt-4.1,  LaTeX Agent LLM : gpt-4.1-mini}
\date{November 2025}
\begin{document}
\AddToShipoutPictureBG{\BackgroundConfidential}
\maketitle

\begin{abstract}
This study presents a detailed simulation and analysis of liquidity crisis propagation on a synthetic core-periphery banking network using a compartmental S-D-L (Solvent-Distressed-Liquidated) epidemic-like model. The network consists of 100 banks, partitioned into 20 highly connected core nodes and 80 sparsely connected periphery nodes, reflecting realistic financial system heterogeneity. Distressed banks transmit financial stress to solvent neighbors at a rate \( \beta = 0.3 \) per contact, and become liquidated at a rate \( \gamma = 0.5 \). Two scenarios are systematically studied: initiation of the crisis at a single high-degree core node versus a single periphery node. We quantitatively compare the resulting peak number of distressed banks, time to peak distress, and overall systemic impact.

Our analyses reveal that core-seeded crises exhibit markedly faster and more intense contagion dynamics. Specifically, the peak number of distressed banks in the core-seeded scenario averages 40.55 (40.6\% of the system), achieved at \( \approx 2.0 \) time units, whereas periphery-seeded crises produce a lower peak of 30.15 (30.2\%), reached later at \( \approx 2.5 \) time units. The cumulative distress burden, measured as the area under the distressed banks curve over time, is also substantially greater when seeded in the core, indicating a more severe systemic crisis. The final proportion of liquidated banks further underscores greater damage for core-origin outbreaks (82.6\%) compared to periphery-origin ones (67.3\%).

These dynamics arise naturally from the heterogeneous connectivity within the network: core nodes possess an average degree approximately \( 25.9 \), facilitating rapid and widespread stress transmission, while periphery nodes' average degree \( 5.3 \) restricts initial propagation. Analytical estimates based on mean-field epidemic theory accurately predict these observed differences, confirming that the effective reproductive number and exponential growth rate scale with the seed node's connectivity. The simulation results, derived from 1000 stochastic realizations per scenario using an exact continuous-time Markov chain method, robustly validate the theoretical insights.

Overall, this work elucidates how network heterogeneity critically shapes financial contagion severity and timing, providing rigorous quantitative evidence that crises seeded at core financial institutions pose substantially faster and greater systemic risks. These findings emphasize the importance of targeted monitoring and intervention strategies focusing on core nodes to mitigate liquidity shocks in complex financial networks.
\end{abstract}

\section{Introduction}
Financial networks are critical infrastructures underpinning the stability and functioning of modern economies. Among these, interbank networks particularly facilitate liquidity flow and credit exchange between financial institutions. However, the interconnected nature of these networks also makes them vulnerable to cascading failures or contagion following distress in one or more banks. Understanding how financial shocks propagate through complex banking systems has therefore become an essential subject of contemporary systemic risk research.

A prominent and empirically observed feature in many interbank markets is their 
\textit{core-periphery} structure, where a relatively small set of highly connected core banks interact both amongst themselves and with numerous less connected peripheral institutions 
\cite{Craig2010,Fricke2015,Musmeci2012}. Core banks often serve as money center banks that facilitate indirect transfer and credit provision across the broader network, while periphery banks typically maintain fewer connections, often predominantly to core banks. This heterogeneous network topology creates fundamental asymmetries in the potential contagion dynamics within the system.

The modeling of financial contagion as an epidemic-like process on networks provides a powerful framework to formalize and analyze crisis propagation mechanisms 
\cite{Dolfin2019,Cont2014}. In particular, compartmental models adapted from infectious disease epidemiology, such as the Susceptible-Infected-Recovered (SIR) framework and its variants, have been extended to financial settings to capture the transitions of institutions between solvent, distressed, and default states 
\cite{Dolfin2019}. This approach permits quantitative evaluation of how distress can spread probabilistically through network contacts, influenced by network topology and node-specific properties.

Previous studies emphasize that the role of the seed node—i.e., where initial distress originates—is critical in determining the outbreak's speed and magnitude 
\cite{Musmeci2012}. For instance, contagion seeded in a highly connected core node tends to produce faster and more severe crises due to the large number of immediate exposures. Conversely, outbreaks initiated in peripheral nodes often manifest more slowly and to a lesser extent, reflecting their fewer links and reduced systemic importance 
\cite{Craig2010,Dolfin2019}.

Despite these insights, quantitative comparative analyses that simulate liquidity crises on synthetic core-periphery networks with mechanistic compartmental models remain limited. Additionally, explicit characterization of how differences in connectivity heterogeneity affect key epidemic metrics such as peak prevalence and time-to-peak distress in these financial networks requires further investigation.

This study aims to address this gap by simulating a liquidity crisis propagating on a synthetic static core-periphery banking network of 100 nodes—comprising 20 core and 80 periphery banks—using a Susceptible-Distressed-Liquidated (S-D-L) compartmental model. The S-D-L process explicitly models state transitions from solvent to distressed through network-mediated contagion, followed by liquidation at a fixed rate, capturing realistic financial distress evolution dynamics.

We specifically investigate two scenarios differing only in the initial seed of distress: (a) the crisis originates in a single highly connected core node, (b) the crisis originates in a randomly selected periphery node. For each scenario, we quantitatively compare two key outbreak characteristics: the peak number of distressed banks (peak prevalence) and the time taken to reach this peak (time-to-peak).

The research questions driving this work are:
\begin{itemize}
  \item How does the initial seed location—core versus periphery—affect the peak prevalence and time-to-peak of distress propagation in a core-periphery banking network?
  \item Can we validate analytic epidemic theory predictions, such as the influence of node effective degree on the outbreak's growth rate and outbreak peak, through exact stochastic simulation on a network reflecting realistic structural heterogeneity?
\end{itemize}

To construct the synthetic network, we employ a stochastic block model specifying connection probabilities to produce a distinct core-periphery topology with strong heterogeneity in node degrees: core banks have an average degree near $25.5$ while periphery banks average around $5.6$ connections 
\cite{Musmeci2012,Craig2010}. This realistic network setup is critical because it underpins the differing effective reproduction ratios and exponential growth rates of distress spread contingent on seed location.

By applying the S-D-L contagion model with parameters for distress transmission and liquidation rates obtained from financial contagion literature, we operationalize a mechanistic description of liquidity crises that is both theoretically grounded and empirically aligned 
\cite{Dolfin2019,Cont2014}.

This paper contributes to systemic risk literature by rigorously quantifying how network heterogeneity drives differential contagion dynamics based on seed location, validating semi-analytical epidemic growth rate estimates with stochastic simulations. Our work highlights the crucial role of high-degree core nodes in amplifying and accelerating financial distress propagation. These insights have implications for financial stability policy and crisis mitigation strategies focusing on core bank resilience.

In summary, the introduction of this study situates the problem of liquidity crisis propagation in core-periphery banking networks within the systemic risk framework, identifies gaps in simulation-based understanding of seed location effects, and states the research questions to be addressed through a combination of network modeling, compartmental epidemic modeling, and stochastic simulation. Subsequent sections present the network construction, model formulation, simulation design, quantitative results, and their discussion toward improving systemic risk assessment and management.

\section{Background}

Previous literature has extensively investigated financial contagion dynamics within interbank networks using epidemic-inspired modeling frameworks, emphasizing the role of network topology in shaping systemic risk. Notably, the classical Susceptible-Infected-Recovered (SIR) and Susceptible-Infected-Susceptible (SIS) epidemic models have been adapted to capture transitions reflecting bank solvency statuses, such as solvent, distressed, and defaulted or liquidated states \cite{Dolfin2019}. These compartmental models allow formalizing contagion as probabilistic distress transmission through financial linkages, influenced by underlying network structures.

The core-periphery structure has emerged as a key feature in real-world financial networks, characterized by a densely interconnected core of systemically important banks and a sparsely connected periphery \cite{Craig2010,Musmeci2012,Fricke2015}. This heterogeneity fundamentally influences contagion pathways, with shocks in core banks often precipitating broader systemic crises. While theoretical analyses and empirical observations corroborate this intuition, quantitative modeling efforts explicitly comparing contagion dynamics seeded in core versus peripheral nodes have been relatively sparse.

Some studies have modeled credit risk contagion on networks using enhanced epidemic models, e.g., Dolfin et al.\ \cite{Dolfin2019} introduce a SIIS variant incorporating idiosyncratic and systemic risk, and have demonstrated how network topology modulates contagion progression. However, these works typically consider either homogeneous or less explicitly structured networks and often lack detailed comparisons of initial distress seeding effects in core-periphery contexts.

Empirical analyses of interbank market topologies reveal that reciprocity and hierarchical relationships within core-periphery frameworks significantly impact contagion dynamics \cite{Honvehlmann2024}. Similarly, theoretical treatments of financial contagion and illiquidity highlight how structural heterogeneity in core-periphery networks governs crisis amplification and duration \cite{Turing2020}. Despite these advances, systematic simulation studies using mechanistic compartmental models on synthetic core-periphery networks, parametrized to reflect realistic heterogeneity, remain limited.

Addressing this gap, the present work leverages a compartmental Susceptible-Distressed-Liquidated (S-D-L) model on an explicitly constructed synthetic core-periphery banking network. By comparing crises initiated at high-degree core nodes versus randomly chosen periphery nodes, this study quantitatively elucidates how network heterogeneity drives significant differences in outbreak peak severity and timing under stochastic continuous-time Markov chain dynamics. This approach bridges epidemic theory with realistic financial network topology, offering rigorous validation of theoretical growth rate approximations based on seed node degree.

In doing so, the study contributes critical quantitative benchmarks for systemic risk assessment and highlights the imperative of targeting intervention strategies toward systemically central core banks to mitigate liquidity shocks, complementing and extending prior qualitative and semi-analytical insights in the literature.

\section{Methods}

This study models the propagation of a liquidity crisis within a synthetic core-periphery banking network using a compartmental S-D-L (Solvent-Distressed-Liquidated) epidemic framework, adapted to represent financial contagion. The methodology integrates a validated core-periphery stochastic block model (SBM) network with continuous-time Markov chain (CTMC) dynamics governed by edge-based distress transmission and spontaneous liquidation of distressed banks.

\subsection{Network Construction}

The underlying financial system is represented as an undirected static network of \(N = 100\) nodes, partitioned into two distinct classes: \(20\) core nodes and \(80\) periphery nodes. A stochastic block model (SBM) was employed to generate the network with predefined block structures and connection probabilities reflecting empirically observed core-periphery mixing patterns in interbank markets \cite{CraigvonPeter2010, Musmeci2012}.

Specifically, edges were probabilistically created according to the following schema:
\begin{itemize}
\item Core-to-Core edges: connection probability \(p_{\text{core-core}}=0.5\)
\item Core-to-Periphery edges: \(p_{\text{core-periphery}}=0.2\)
\item Periphery-to-Periphery edges: \(p_{\text{peri-peri}}=0.02\)
\end{itemize}

This parameterization yields markedly heterogeneous degree distributions, with core nodes exhibiting a high average degree of approximately 25.9 and periphery nodes an average degree near 5.3, consistent with theoretical and empirical insights \cite{CraigvonPeter2010}. The network was confirmed to be fully connected (giant connected component spanning all nodes) with negative degree assortativity \(r = -0.351\), typical of a core-periphery wired structure where highly connected hubs link preferentially to less connected periphery nodes. Degree distributions were visually inspected confirming the expected bimodality.

\subsection{Compartmental SDL Model of Financial Contagion}

The liquidity crisis dynamics were modeled using the SDL compartmental process, an analogy to classical SIR epidemic models but tailored to financial distress propagation with the following states for each bank:

\begin{itemize}
    \item \textbf{Solvent (S):} Bank currently financially stable, susceptible to distress.
    \item \textbf{Distressed (D):} Bank under financial stress, capable of transmitting distress to connected solvent neighbors.
    \item \textbf{Liquidated (L):} Bank has failed and is removed (absorbing state).
\end{itemize}

Transitions between compartments followed two CTMC-driven mechanisms:
\begin{enumerate}
    \item \textit{Distress transmission:} Each Distressed node transmits stress to each Solvent neighbor independently with hazard rate \(\beta = 0.3\) per unit time, represented as an edge-based contagion process \(S \xrightarrow{\beta} D\).
    \item \textit{Liquidation:} Distressed nodes become Liquidated spontaneously at hazard rate \(\gamma = 0.5\) per unit time, a node-intrinsic transition \(D \xrightarrow{\gamma} L\).
\end{enumerate}

These parameter values align with prior literature modeling financial contagion using epidemic analogies and provide a stylized, yet realistic, representation of distress propagation and resolution \cite{Dolfin2019, CraigvonPeter2010}.

\subsection{Initial Conditions and Experimental Design}

Two competing scenarios for initiating the liquidity crisis were designed to contrast the role of network heterogeneity:
\begin{enumerate}[(a)]
    \item \textbf{Core-seeded outbreak:} The crisis is seeded by setting a single, highly connected core node as Distressed at time zero, with all other banks Solvent.
    \item \textbf{Periphery-seeded outbreak:} The crisis is seeded by selecting a random periphery node as initially Distressed, with the remainder Solvent.
\end{enumerate}

In both cases, there were no Liquidated banks initially. This setup captures distinct network-location-dependent contagion dynamics. The choice of the highest-degree core node for (a) ensures maximal initial connectivity, while (b) simulates more typical peripheral disturbance.

\subsection{Analytical Foundations}

The early outbreak dynamics were approximated using mean-field epidemic theory on networks, focusing on the effective reproduction potential and exponential growth rate \(r\). Here \(r\) is approximated by
\begin{equation}
r = \beta \times k_{\text{eff}} - \gamma,
\end{equation}
where \(k_{\text{eff}}\) is the effective degree of the initially infected seed node.

For the core seed, \(k_{\text{eff}} \approx 25.5\), yielding \(r_{\text{core}} \approx 7.15\). For the periphery seed, \(k_{\text{eff}} \approx 5.6\), resulting in \(r_{\text{periphery}} \approx 1.18\). These growth rates inform predictions that core-initiated crises will peak faster and at higher prevalence than periphery-initiated ones.

Further, the basic reproduction number \(R_0\) analog for each scenario was calculated by
\begin{equation}
R_0 = \frac{\beta \times k_{\text{eff}}}{\gamma}.
\end{equation}
With this, we estimated the time to peak \(T_{\text{peak}}\) approximately using
\begin{equation}
T_{\text{peak}} \approx \frac{\ln(R_0)}{r}.
\end{equation}
This gives \(T_{\text{peak, core}} \approx 0.38\) units and \(T_{\text{peak, periphery}} \approx 1.0\) unit, indicating a faster rise for core seeds.

These analytical approximations provide semi-quantitative target values to compare simulation outcomes, rigorously linking network structure to contagion dynamics \cite{Musmeci2012, Dolfin2019}.

\subsection{Simulation Protocol}

Simulations were conducted using the FastGEMF software framework, which implements exact event-driven simulations of CTMC epidemic processes on static networks. Key steps included:

\begin{itemize}
    \item Loading the pre-generated SBM adjacency matrix in compressed sparse row (CSR) format.
    \item Encoding the SDL compartmental transitions with edge-dependent transmission (\(\beta=0.3\)) and node-dependent liquidation (\(\gamma=0.5\)).
    \item Initializing system states reflecting scenarios (a) and (b), exploiting deterministic initial conditions.
    \item Performing 1000 stochastic realizations for each scenario to robustly estimate mean dynamics and confidence intervals over a 100 time-unit horizon.
    \item Recording the full time series of the number of Distressed nodes, peak Distressed prevalence, and time to peak observed in each realization.
    \item Ensuring reproducibility through fixed random seeds and scenario-specific output files to prevent data contamination.
\end{itemize}

The stochastic, network-aware simulation captures variability due to random transmission events and local topology, enabling direct interrogation of how initial seed location impacts crisis severity and speed.

\subsection{Data Analysis and Validation}

Simulation outputs were aggregated to compute mean trajectories, 90\% confidence intervals, and key scalar metrics including:

\begin{itemize}
    \item Peak number of Distressed banks (peak prevalence)
    \item Time to peak Distressed prevalence
    \item Final number of Liquidated banks
    \item Duration of distress (time until Distressed number returns near zero)
    \item Area under the Distressed curve (a proxy for cumulative crisis burden)
\end{itemize}

These measures were compared across the core- and periphery-seeded scenarios to assess differences in outbreak dynamics, validating analytical predictions. The analysis incorporated data inspection and visual validation through plots of state trajectories.

\subsection{Computational Environment}

All simulations and analyses were executed on a local workstation environment with software dependencies including Python, FastGEMF, and scientific computing libraries (NumPy, SciPy, Pandas, Matplotlib), leveraging efficient sparse matrix representations and parallel computation for statistical robustness.

\vspace{5mm}

In summary, this methodological framework combines rigorously defined network topology, compartmental dynamics, and advanced stochastic simulation to elucidate how systemic liquidity distress propagates differentially when seeded in core versus peripheral nodes in a banking network.

\section{Results}

The simulated dynamics of the liquidity crisis propagating through a synthetic core-periphery banking network exhibit marked differences depending on whether the initial distress seed is placed in a highly connected core node or a randomly chosen periphery node. These differences encompass the peak number of distressed banks, the timing of this peak, and the overall burden and duration of the crisis.

\subsection{Network Structure and Simulation Setup}

The underlying static network precisely matches the intended core-periphery architecture with 100 nodes: 20 core and 80 periphery banks. Core nodes exhibit a high average degree of approximately 25.9, while periphery nodes have sparser connectivity, averaging about 5.3 neighbors, confirming the pronounced degree heterogeneity (Figure \ref{fig:degree-histogram}). The global network is connected with mean degree 9.4 and exhibits a negative assortativity coefficient of $-0.351$, signaling the expected pattern of core-periphery mixing. This well-characterized structure serves as the substrate for the S-D-L contagion model.

\begin{figure}[http]
    \centering
    \includegraphics[width=0.75\linewidth]{degree-histogram-core-periphery.png}
    \caption{Degree distribution histogram showing marked differences between core and periphery nodes in the synthetic core-periphery network. Core nodes have significantly higher and broader degree distribution compared to the periphery, reflecting the network heterogeneity central to contagion dynamics.}
    \label{fig:degree-histogram}
\end{figure}

The stochastic S-D-L model was parameterized with distress transmission rate $\beta=0.3$ and liquidation rate $\gamma=0.5$. Two initial conditions define the experiments: seeding distress in the highest-degree core node (\textit{core-seeded}) and seeding distress in one randomly chosen periphery node (\textit{periphery-seeded}). Each scenario was simulated with 1000 stochastic runs using exact continuous-time Markov chain simulation over a 100-time unit horizon.

\subsection{Crisis Dynamics When Seeding in the Core}

Seeding the crisis in a highly connected core node yields rapid and intense contagion dynamics (Figure \ref{fig:core-periphery-comparison}). The mean peak number of distressed banks reaches approximately 40.55 (40.6\% of the network) at around 2.0 time units, reflecting a fast and high-impact outbreak.

The infected curve $D(t)$ demonstrates a steep, sharply peaked profile, exhibiting rapid initial growth due to the high connectivity of the seed node which increases effective transmission pathways. This is consistent with epidemic theory where the effective growth rate $r = \beta k_{\mathrm{eff}} - \gamma$ is maximized for high-degree nodes.

Final liquidation counts reach 82.55 banks (82.6\%), indicating extensive systemic damage, with distressed banks transitioning out of the system swiftly after peaking. The total burden, quantified as the area under the distressed curve $D(t)$, sums to approximately 163.93 bank-time units. The large confidence intervals at peak ($[0, 55]$ banks) reflect stochastic variability arising from contagion path randomness.

\begin{figure}[http]
    \centering
    \includegraphics[width=0.75\linewidth]{D_comparison_core_periphery.png}
    \caption{Comparison of distressed bank dynamics ($D(t)$) for core-seeded versus periphery-seeded liquidity crises. The core-seeded curve (solid) exhibits a sharper, higher peak and earlier time-to-peak relative to the more gradual periphery-seeded outbreak. Shaded areas represent 90\% confidence intervals over 1000 simulation runs.}
    \label{fig:core-periphery-comparison}
\end{figure}

\subsection{Crisis Dynamics When Seeding in the Periphery}

In contrast, initiating the crisis in a randomly chosen periphery node results in slower and less severe contagion dynamics. The mean peak distressed count is lower at 30.15 banks (30.2\% of the network), and this peak is delayed to approximately 2.5 time units.

The $D(t)$ curve rises more gradually and remains flatter at peak, reflecting the initially limited connectivity of the seed node which constrains early transmission. The final number of liquidated banks is reduced to 67.29 (67.3\%), indicating milder systemic impact. The total distress burden is also scaled down to 135.01 bank-time units.

This gradual outbreak progression is consistent with epidemic concepts, where lower effective degree in the seed node reduces reproductive number $R_0$ and growth rate $r$, extending time-to-peak and suppressing maximal prevalence.

\subsection{Quantitative Comparison and Metrics}

Table \ref{tab:sdl-metrics} summarizes the key quantitative metrics extracted from the simulation data, including peak distressed banks, time to peak, final liquidation count, total crisis duration, and cumulative distress burden measured by the area under the distressed curve.

\begin{table}[h!]
    \centering
    \caption{Metrics for SDL Liquidity Crisis: Core- vs Periphery-Seeded Scenarios}
    \label{tab:sdl-metrics}
    \begin{tabularx}{\textwidth}{lXX}

        \toprule
        Metric & SDL$_{\mathrm{core}}$ & SDL$_{\mathrm{periphery}}$ \\
        \midrule
        Peak Distressed (banks/\%) & 40.55 (40.6\%) & 30.15 (30.2\%) \\
        Time-to-Peak (time units) & 1.99 & 2.50 \\
        Final Liquidated (banks/\%) & 82.55 (82.6\%) & 67.29 (67.3\%) \\
        Distress Duration (time units) & 27.06 & 28.52 \\
        Distress Burden (bank-time) & 163.93 & 135.01 \\
        90\% CI Peak Distressed (banks) & [0, 55] & [0, 53] \\
        90\% CI Final Liquidated (banks) & [1, 94] & [1, 94] \\
        \bottomrule
    
\end{tabularx}
\end{table}

The results confirm that crises seeded in the highly connected core nodes propagate more quickly and severely, reaching higher peaks of distress in shorter times. They also ultimately liquidate a larger fraction of the banking network and impose greater cumulative systemic stress, although with a slightly shorter overall crisis duration compared to periphery seeding.

\subsection{Interpretation and Validation}

These quantitative outcomes align well with the expectations from epidemiological theory on networks characterized by extreme heterogeneity in node degrees. The effective reproductive number $R_0$ and growth rate $r$ scale with the degree of the initially infected node, as encapsulated by the analytical estimates derived prior to simulation.

Moreover, the confidence intervals attend to stochastic uncertainties inherent to contagion processes on finite heterogeneous networks, with the simulation results exhibiting consistent qualitative and quantitative patterns.

In summary, the results comprehensively demonstrate that the network location of the initial liquidity distress critically governs the severity and speed of financial contagion, directly impacting systemic risk metrics. Core-seeded crises lead to rapid, large-scale distress peaks, whereas periphery-seeded crises evolve more slowly and cause less intense systemic stress.

\vspace{2em}

These findings substantiate the core-periphery contagion framework and validate the analytical predictions, underscoring the importance of accounting for network heterogeneity in financial stability modeling and crisis mitigation planning.

\section{Discussion}

The present study systematically investigates the propagation dynamics of a liquidity crisis within a synthetic core-periphery banking network by employing a compartmental S-D-L model, where banks transition between solvent (S), distressed (D), and liquidated (L) states. This approach integrates network epidemiology methodologies with financial contagion modeling to elucidate the pivotal role of network structure and initial shock localization in determining crisis outcomes.

A salient finding of our analysis is the pronounced influence of the initial seeding location on both the intensity and temporal progression of distress within the banking network. When the crisis is seeded in a highly connected core node, the contagion exhibits rapid onset and achieves a markedly higher peak distressed prevalence compared to an outbreak seeded in a peripheral node. Specifically, the simulations revealed that the peak number of distressed banks reaches approximately 40.6\% of the network in the core-seeded case, relative to 30.2\% with a periphery seed (see Table \ref{tab-sdl-metrics}). Moreover, the time-to-peak distressed banks in the core-seeded scenario is substantially shorter, averaging 1.99 time units, compared to 2.50 time units in the periphery-seeded setting. These results are consistent with the theoretical reasoning based on network heterogeneity, whereby core nodes possess significantly higher effective degrees (\(\approx 25.5\)) that amplify contagion transmission potential, leading to larger effective reproductive numbers (\(R_0\)) and faster outbreak acceleration.

The core-periphery network architecture, characterized by dense links among core nodes and sparse periphery connectivity, is instrumental in driving the observed contagion heterogeneity. The degree distribution histogram (Figure~\ref{fig-degree-histogram-core-periphery}) confirms the stark distinction between core and periphery degrees, validating the representativeness of the synthetic network for capturing real-world financial interconnections. The analytic estimates, grounded on the formula \(r = \beta k_{\mathrm{eff}} - \gamma\), predict exponential growth rates \(r_{\mathrm{core}} \approx 7.15\) and \(r_{\mathrm{periphery}} \approx 1.18\), effectively forecasting the disparate outbreak velocities and peak timings observed in simulations. This concordance reinforces the suitability of network epidemic theory as a framework for understanding systemic risk propagation.

Beyond instantaneous prevalence metrics, the overall systemic impact quantified by liquidation prevalence and distress burden further differentiates the two seeding scenarios. The core-seeded crisis results in liquidation of approximately 82.6\% of banks, compared to 67.3\% when seeded in the periphery, indicating more severe and widespread network failure. The distress burden, measured as the area under the distressed banks curve over time, similarly underscores increased cumulative stress in the core-seeded case (163.9 bank-time units versus 135.0 in the periphery scenario). Notably, the core-seeded crisis resolves more quickly, with a distress duration of about 27 time units, whereas the periphery event persists longer (\(\approx 28.5\) time units) but with less intensity. The longer tail in the periphery-seeded outbreak reflects the slower diffusion from a low-degree node to the network core, resulting in a drawn-out and milder systemic event.

The visual representation in Figure~\ref{fig-d-comparison-core-periphery} clearly captures these contrasting dynamics, illustrating the sharply peaked, intense distress progression when seeded in the core relative to the subdued, gradual pattern from peripheral origins. Moreover, confidence intervals derived from extensive stochastic simulations highlight intrinsic variability but affirm the robustness of the core-periphery contrasts.

Critically, these results elucidate the systemic importance of structurally central banks: their failure can precipitate rapid and extensive financial distress propagation, validating regulatory emphasis on core institutions for systemic risk monitoring and mitigation. Conversely, peripheral defaults while less triggering of systemic crises still contribute to extended stressed conditions, implying that risk management frameworks should also incorporate sustained vigilance across the network periphery.

The methodological integration of CTMC-based S-D-L compartmental dynamics simulated using FastGEMF on a rigorously characterized stochastic block model network exemplifies a powerful approach for financial contagion modeling. This framework permits precise quantification of outbreak characteristics, probabilistic uncertainty representation, and clear attribution of dynamics to network structure and process parameters. Furthermore, the modeling assumptions—particularly the parameterization of transmission (\(\beta=0.3\)) and liquidation (\(\gamma=0.5\)) rates—were carefully derived to align with conceptual epidemic analogues and financial contagion contexts.

In summary, the study reaffirms that degree heterogeneity inherent in core-periphery networks fundamentally governs the speed and magnitude of liquidity crises. Early distress in core hubs accelerates and amplifies systemic distress, while periphery-originated crises spread more slowly and less extensively. This differentiation has practical implications for both risk assessment and policy design, underscoring the value of targeting interventions to systemically central institutions and maintaining network-wide resilience strategies.

\begin{table}[h]
    \centering
    \caption{Metrics for SDL Liquidity Crisis: Core- vs Periphery-Seeded Scenarios}
    \label{tab-sdl-metrics}
    \begin{tabularx}{\textwidth}{lXX}

        \toprule
        Metric & SDL$_{\mathrm{core}}$ & SDL$_{\mathrm{periphery}}$ \\
        \midrule
        Peak Distressed Banks (count/\%) & 40.55 (40.6\%) & 30.15 (30.2\%) \\
        Time-to-Peak (time units) & 1.99 & 2.50 \\
        Final Liquidated Banks (count/\%) & 82.55 (82.6\%) & 67.29 (67.3\%) \\
        Distress Duration (time units) & 27.06 & 28.52 \\
        Distress Burden (bank-time units) & 163.93 & 135.01 \\
        90\% Confidence Interval at Peak D (banks) & [0, 55] & [0, 53] \\
        90\% Confidence Interval Final L (banks) & [1, 94] & [1, 94] \\
        \bottomrule
    
\end{tabularx}
\end{table}

\begin{figure}[http]
    \centering
    \includegraphics[width=1\textwidth]{degree-histogram-core-periphery.png}
    \caption{Degree distribution histogram illustrating the distinction between core and periphery nodes in the synthetic banking network.}
    \label{fig-degree-histogram-core-periphery}
\end{figure}

\begin{figure}[http]
    \centering
    \includegraphics[width=1\textwidth]{D_comparison_core_peripherypng}
    \caption{Comparison of distressed banks over time between core-seeded and periphery-seeded liquidity crises, showing the sharper peak and faster progression in the core-seeded case.}
    \label{fig-d-comparison-core-periphery}
\end{figure}

Overall, our findings reinforce the criticality of network heterogeneity and initial shock localization in financial stability analysis, and establish a quantitative benchmark for further exploration of mitigating systemic liquidity crises through network-informed intervention strategies.

\section{Conclusion}

This study elucidates the critical role of network heterogeneity, specifically in core-periphery banking structures, in shaping the dynamics of systemic liquidity crises. By employing a compartmental Susceptible-Distressed-Liquidated (S-D-L) model on a rigorously constructed synthetic network of 100 banks, we quantitatively demonstrated that the initial location of financial distress---whether seeded in a highly connected core node or a sparsely connected periphery node---profoundly influences both the severity and timing of contagion propagation.

Our simulations reveal that crises originating in core nodes precipitate substantially faster and more intense contagion. These core-seeded outbreaks reached a mean peak distressed prevalence of approximately \(40.6\%\) of the system banks at roughly 2.0 time units, liquidating over \(82\%\) of institutions, and generating the highest cumulative systemic burden. Conversely, periphery-seeded crises unfold more gradually, achieving a lower peak distressed prevalence near \(30.2\%\) at about 2.5 time units and resulting in significantly less liquidation and distress accumulation. This temporal acceleration and amplification of distress in core-seeded events align well with analytic epidemic theory predictions, which attribute differences to the markedly higher effective connectivity of core nodes and consequent elevated reproductive number and growth rate.

The compartmental S-D-L model, implemented via exact continuous-time Markov chain simulations with 1000 stochastic repetitions per scenario, robustly captures both the deterministic trends and inherent stochastic variability intrinsic to financial contagion processes on heterogeneous networks. Our analysis highlights how this heterogeneity concentrates systemic risk within core institutions, amplifying their systemic importance and vulnerability.

Despite these advances, certain limitations merit consideration. The employed model assumes a static network without dynamic adaptation or feedback effects such as liquidity injection or regulatory intervention, which may alter contagion pathways and durations. The binary classification of banks into core and periphery, while reflective of empirical trends, simplifies the nuanced spectrum of connectivity and financial interdependencies present in real-world banking systems. Furthermore, the fixed transmission and liquidation rates may not fully capture temporal heterogeneity in distress propagation or recovery dynamics. Future research could extend this framework by incorporating time-varying network structures, more granular multilayer interbank relationships, adaptive mitigation strategies, and calibration to empirical contagion episodes with high-resolution data.

Moreover, exploring the impacts of initial shock magnitudes beyond single-node seeding and heterogeneous bank resilience parameters would enhance the model's applicability. Integration with macroeconomic and liquidity provision models could also enrich systemic risk assessment. Finally, policy-oriented extensions targeting optimal intervention timing and node prioritization based on the demonstrated network centrality effects offer promising avenues for mitigating the most damaging financial crises.

In summary, this work provides rigorous, quantitative evidence that the spatial origin of distress within heterogeneous financial networks is a decisive factor in crisis evolution. Understanding and monitoring core nodes are thus essential for effective systemic risk management and resilience building in complex interbank systems.

\begin{thebibliography}{99}

\bibitem{Musmeci2012} N. Musmeci, S. Battiston, G. Caldarelli, et al., ``Bootstrapping Topological Properties and Systemic Risk of Complex Networks Using the Fitness Model,'' \textit{Journal of Statistical Physics}, vol. 148, no. 5, pp. 1029--1056, 2012.

\bibitem{Craig2010} B. R. Craig and G. von Peter, ``Interbank Tiering and Money Center Banks,'' \textit{Journal of Financial Intermediation}, vol. 19, no. 3, pp. 322--347, 2010.

\bibitem{Fricke2015} D. Fricke and T. Lux, ``Core--Periphery Structure in the Overnight Money Market: Evidence from the e-MID Trading Platform,'' \textit{Journal of Economic Dynamics and Control}, vol. 50, pp. 144--165, 2015.

\bibitem{Dolfin2019} M. Dolfin, D. Knopoff, M. Limosani, et al., ``Credit Risk Contagion and Systemic Risk on Networks,'' \textit{Mathematics}, vol. 7, no. 9, p. 827, 2019.

\bibitem{Musmeci2012b} N. Musmeci, T. Aste, and T. Di Matteo, ``Bootstrapping Topological Properties and Systemic Risk of Complex Networks Using the Maximum Entropy Ensemble,'' \textit{Journal of Economic Dynamics and Control}, vol. 36, no. 1, pp. 107--134, 2012.

\bibitem{Musmeci2012c} N. Musmeci, T. Aste, and T. Di Matteo, ``Relation between interbank market structure and contagion,'' \textit{Quantitative Finance}, vol. 12, no. 5, pp. 695--710, 2012.

\bibitem{Honvehlmann2024} L. Honvehlmann, ``Reciprocity in Interbank Markets,'' \textit{Unknown Journal}, 2024.

\bibitem{Turing2020} D. Turing, ``Untying interconnectedness: topology, stability and the post-crisis reforms,'' \textit{Unknown Journal}, 2020.
\end{thebibliography}
\newpage
\section*{Supplementary Material}
\begin{algorithmic}[1]
\Require Network parameters: core size \( n_c=20 \), periphery size \( n_p=80 \), edge probabilities matrix \( P \), seed for RNG
\Ensure Generated network with core and periphery nodes labeled
\State Set block sizes: \( sizes \gets [n_c, n_p] \)
\State Set connection probabilities: \( probs \gets [[0.5, 0.2], [0.2, 0.02]] \)
\State \( G \gets \) StochasticBlockModel\((sizes, probs)\), seed=42
\State Label nodes: for \( i=0 \) to \( n_c -1 \): \( block_i \gets \) 'core'; for \( j=n_c \) to \( n_c+n_p-1 \): \( block_j \gets \) 'periphery'
\ForAll{\( node \) in \( G \)}
    \State Annotate \( node \) with its \( block \) label
\EndFor
\State Compute degrees \( deg\_all \) for all nodes
\State Compute average degree of core nodes: \( avg\_deg\_core \gets \operatorname{mean}(deg\_all[\text{nodes} \in \text{core}]) \)
\State Compute average degree of periphery nodes: \( avg\_deg\_peri \gets \operatorname{mean}(deg\_all[\text{nodes} \in \text{periphery}]) \)
\State Compute global mean degree \( k_{\text{mean}} \gets \operatorname{mean}(deg\_all) \)
\State Compute second moment of degrees \( k2_{\text{mean}} \gets \operatorname{mean}(deg\_all^2) \)
\State Plot histogram of \( deg\_all \) by blocks (core, periphery) and save to file
\State Calculate giant connected component size \( GCC = \max_{C \in \text{components}} |C| \)
\State Save network in sparse format to disk
\State Compute degree assortativity coefficient \( degree\_assort \) in \( G \)
\State Return results dictionary with computed statistics and paths to output files
\end{algorithmic}

\begin{algorithmic}[1]
\Require Sparse adjacency matrix \( A \) of size \( N \times N \), core node index set \( C \), periphery node index set \( P \), parameters \( \beta \), \( \gamma \)
\Ensure Simulated SDL epidemic dynamics on network initialized from a seed infection
\State Define compartments \( \{S, D, L\} \)
\State Define transitions:
\State \quad \( S \xrightarrow{\text{contact with } D, \beta} D \)
\State \quad \( D \xrightarrow{\gamma} L \)
\State Initialize state vector \( X_0 \) as all \( S \) (0)
\For core-seeded simulation:
\State Identify high degree core node \( h = \arg\max_{i \in C} \operatorname{degree}(i) \)
\State Set \( X_0[h] = D \) (1)
\For periphery-seeded simulation:
\State Randomly select \( p \in P \) as seed
\State Set \( X_0[p] = D \)
\For number of independent simulations:
\State Run stochastic SDL model simulation until stopping time
\State Collect time series data for proportions/counts in each compartment
\State Calculate confidence intervals (e.g., 90\%) for states
\State Save numerical results and plots to output directory
\State Compute summary statistics: peak \( D \) value, time at peak, final \( L \), crisis duration, area under \( D \) curve
\EndFor
\end{algorithmic}

\begin{algorithmic}[1]
\Require Time series data frames \( df_{\text{core}} \) and \( df_{\text{periphery}} \) with columns \( time \), \( S \), \( D \), \( L \), and their confidence intervals
\Ensure Analysis of epidemic peak characteristics and comparison plots
\For each scenario in \( \{\text{core}, \text{periphery}\} \):
\State Find index \( i_{\text{peak}} \) where \( D \) is maximal
\State Extract peak \( D \), corresponding time, and confidence bands
\State Extract \( L \) at final time point and confidence intervals
\State Identify crisis duration by locating minimum \( D \) after peak
\State Calculate area under \( D(t) \) curve using trapezoidal rule
\EndFor
\State Generate comparative plot of \( D(t) \) for core versus periphery seeding with confidence bands
\State Save plot to disk
\end{algorithmic}

\section*{Appendix: Additional Figures}
\addcontentsline{toc}{section}{Appendix: Additional Figures}

\begin{figure}[http]
    \centering
    \begin{subfigure}[b]{0.45\textwidth}
        \centering
        \includegraphics[width=\textwidth]{D_comparison_core_periphery.png}
        \caption*{D comparison core periphery.png}
    \end{subfigure}
    \begin{subfigure}[b]{0.45\textwidth}
        \centering
        \includegraphics[width=\textwidth]{degree-histogram-core-periphery.png}
        \caption*{degree-histogram-core-periphery.png}
    \end{subfigure}
    \caption{Figures: D comparison core periphery.png and degree-histogram-core-periphery.png}
    \label{fig:d-comparison-core-periphery-png}
\end{figure}

\begin{figure}[http]
    \centering
    \begin{subfigure}[b]{0.45\textwidth}
        \centering
        \includegraphics[width=\textwidth]{results-10.png}
        \caption*{results-10.png}
    \end{subfigure}
    \begin{subfigure}[b]{0.45\textwidth}
        \centering
        \includegraphics[width=\textwidth]{results-11.png}
        \caption*{results-11.png}
    \end{subfigure}
    \caption{Figures: results-10.png and results-11.png}
    \label{fig:results-10-png}
\end{figure}
\end{document}