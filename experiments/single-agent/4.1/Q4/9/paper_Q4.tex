\documentclass[conference]{IEEEtran}
\usepackage{graphicx}
\usepackage{booktabs}
\usepackage{multirow}
\title{Epidemic Spread Analysis of SIR Model over Static Random Networks: Impact of Initial Seeding Locations}

\begin{document}
\maketitle

\begin{abstract}
We perform an in-depth computational study of epidemic spread on static Erdős-Rényi (ER) networks using the SIR compartmental model. We investigate how the initial seeding of infections----randomly distributed across the network versus localized at high-degree (hub) nodes----affects outbreak dynamics. Two realistic initial conditions are simulated on a 1000-node random network: (1) three randomly chosen initially infected nodes and (2) the three nodes of highest degree (hubs) infected. Quantitative metrics such as peak infection size, timing, final epidemic size, epidemic duration, and doubling time are extracted and compared. Results demonstrate dramatically increased outbreak size and duration when initial infections seed network hubs, highlighting the importance of targeting high-degree individuals in containment strategies.
\end{abstract}

\section{Introduction}
Mathematical modeling of infectious diseases has long played a central role in understanding epidemic dynamics and informing control strategies. Traditional models such as the Susceptible-Infected-Removed (SIR) framework capture disease progression at the population level, while incorporating network structure enables more realistic representation of interpersonal contacts and transmission pathways\cite{Alota2020,Kamaletal2023,AntulovFantulin2014}. Recently, research has focused on how both network topology and the initial locations of infections (seeding) can dramatically alter epidemic outcomes\cite{Kamaletal2023,AntulovFantulin2014}.

The present work aims to answer the question: How does the initial placement of infections (either random or focused on high-degree hubs) affect epidemic spread in static networks? While previous studies demonstrate that network heterogeneity can create vulnerability to 'super-spreader' scenarios, few systematic simulation studies directly compare random versus hub-based seeding in otherwise identical networks.

Here, we use the stochastic SIR model simulated on an Erdős-Rényi random network (mean degree $\langle k \rangle \approx 10$), with two initial seeding strategies held at three infected nodes. By extracting a suite of outbreak metrics and visualizing the dynamics, we directly quantify the consequences of these distinct initial conditions, offering insight relevant for public health planning and targeted interventions.

\section{Methodology}
\subsection{Network Structure and Generation}
An Erdős-Rényi (ER) random graph with $N=1000$ nodes and edge probability $p=0.01$ was constructed using the NetworkX library. This leads to a Poisson-like degree distribution with a mean degree $\langle k \rangle \approx 10$ and second-degree moment $\langle k^2 \rangle \approx 108.8$. The large but homogeneous network structure provides a controlled baseline for assessing seeding effects. The generated network was saved as a SciPy sparse matrix; its degree distribution is shown in Figure~\ref{fig:degree_dist}.

\begin{figure}[htb]
    \centering
    \includegraphics[width=0.45\textwidth]{degree_distribution.png}
    \caption{Degree distribution of generated Erdős-Rényi (ER) network ($N=1000$, $p=0.01$).}
    \label{fig:degree_dist}
\end{figure}

\subsection{Epidemic Mechanistic Model}
We employ the classical SIR (Susceptible-Infected-Removed) model embedded on the network:\begin{itemize}
  \item 
    \textbf{States:} $S$ (susceptible), $I$ (infected), $R$ (removed/recovered)
  \item 
    \textbf{Transitions:}
    $S \stackrel{\textrm{infected neighbor}}{\longrightarrow} I$ at rate $\beta = 0.00418$
    
    $I \rightarrow R$ at recovery rate $\gamma = 0.04$ (average infectious period 7 days)
\end{itemize}
Transition rates are set so that $R_0=2.5$ in the context of the network, using $\beta = R_0 \cdot \gamma / q$, where $q = (\langle k^2 \rangle-\langle k \rangle)/\langle k \rangle$ is the mean excess degree.

\subsection{Initial Conditions}
Two seeding scenarios are considered:\begin{itemize}
  \item \textbf{Random seeding:} Three infected nodes selected uniformly at random (997 susceptible, 0 removed).
  \item \textbf{Hub seeding:} The three highest-degree nodes are initially infected (997 susceptible, 0 removed). All other nodes begin in the susceptible state.
\end{itemize}

\subsection{Simulation Protocol}
Simulations were performed using the FastGEMF package, allowing discrete-time stochastic realizations of the SIR process over networks. Each scenario was run for 120 days with five stochastic replications. Compartment counts over time were recorded, results averaged, and time series exported as CSV and PNG files for further analysis.

\section{Results}
\subsection{Infection Curves and Dynamics}
The infection trajectories for the two initial conditions differ strikingly (see Figure~\ref{fig:compare_curves}). When seeded randomly, the outbreak stagnates, never exceeding three simultaneous infections and dying out rapidly. In contrast, when seeded in the hubs, the infection steadily grows, peaking at 16 infected at day 56, with a long epidemic tail and a final size of 41 recovereds.

\begin{figure}[htb]
    \centering
    \includegraphics[width=0.45\textwidth]{compare_infected.png}
    \caption{Comparison of infected population over time: random seeding (blue) vs hub seeding (orange).}
    \label{fig:compare_curves}
\end{figure}

Visual inspection shows:
\begin{itemize}
    \item \textbf{Random seeding:} The infection curve displays no epidemic phase---prevalence remains at 2-3 and dies out.\newline
    \item \textbf{Hub seeding:} The infection grows, peaks at $t \approx 57$ days, then declines, with infection persisting past 120 days.
\end{itemize}

\subsection{Epidemic Metrics Summary}
\begin{table}[htb]
\centering
\caption{Summary of epidemic metrics by initial condition}
\begin{tabular}{lccccc}
\toprule
 Seeding & Peaked & Peak Time & Final R & Duration & Doubling T. \\ \midrule
Random & 3 & 0.0 & 3 & 37 & n/a \\
Hub & 16 & 56.7 & 41 & 120+ & 9.0 \\
\bottomrule
\end{tabular}
\end{table}

Quantitatively, when seeded at random, the outbreak does not propagate. When seeded at the hubs, the epidemic is sustained, infecting a substantial fraction of the network---13 times more than in the random scenario. Doubling time is slower than in classical mean-field models, due to local depletion of susceptibles in high-degree neighborhoods, but the extended epidemic duration is pronounced.

\section{Discussion}
Our results confirm and quantify the vulnerability of random networks to targeted outbreaks. When initial infections are placed in high-degree nodes, the effective reproductive number is greatly amplified since these nodes have disproportionate influence on the network. In contrast, random seeding more commonly leads to early stochastic extinction. This aligns with findings in the theoretical literature \cite{Alota2020,AntulovFantulin2014,Rocha2010}, which emphasize the key role of superspreaders and topological features in epidemic propagation.

The simulation design intentionally isolates seeding as the variable of interest, using consistent network topology, parameterization, and mechanistic model. While we used an ER network for simplicity, results are expected to be even more pronounced in heterogeneous networks such as scale-free (Barabási-Albert) topologies.

These findings underscore the public health rationale for early targeting of high-contact individuals or locations for monitoring, isolation, or vaccination. Limiting spread from hubs can prevent large-scale outbreaks, while stochastic die-out is more probable if only peripheral nodes are initially infected.

Limitations include the focus on static, memoryless networks and SIR models without latent periods or reinfection. Future research should systematically examine other network types and control interventions.

\section{Conclusion}
Initial placement of infected individuals plays a critical role in the success or extinction of outbreaks in networked populations. In ER networks, seeding infections at random typically results in early fadeout, but placing just a few infections at hubs can drive a large, persistent epidemic. Our quantitative and visual analyses provide clear guidance for honing surveillance and containment, and motivate extensions to real-world, dynamic contact networks.

\section*{References}

\begin{thebibliography}{9}
\bibitem{Alota2020} Cherrylyn P. Alota, C. P. Pilar-Arceo, A. de los Reyes V, "An Edge-Based Model of SEIR Epidemics on Static Random Networks," Bulletin of Mathematical Biology, vol. 82, 2020, doi:10.1007/s11538-020-00769-0.

\bibitem{Kamaletal2023} E. Pinto, E. G. Nepomuceno, J. Kusak, "EPIGUI: Graphical User Interface for Simulating Epidemics on Networks," Trends in Computational and Applied Mathematics, 2023, doi:10.5540/tcam.2022.024.01.00091.

\bibitem{AntulovFantulin2014} Nino Antulov-Fantulin et al., "Detectability limits of epidemic sources in networks," Phys. Rev. Lett., vol. 114, 248701, 2015.

\bibitem{Rocha2010} L. E. C. Rocha, F. Liljeros, P. Holme, "Simulated Epidemics in an Empirical Spatiotemporal Network of 50,185 Sexual Contacts," PLoS Comput. Biol., vol. 7, 2010, doi:10.1371/journal.pcbi.1001109.
\end{thebibliography}

\appendices
\section*{Appendix A: Network and Simulation Files}
All code, figures, and simulation outputs are available as supplementary files:
\begin{itemize}
    \item \texttt{network.npz}: ER adjacency matrix
    \item \texttt{degree\_distribution.png}: ER degree histogram
    \item \texttt{results-11.csv}, \texttt{results-12.csv}: Compartment time series for both scenarios
    \item \texttt{results-11.png}, \texttt{results-12.png}: Infection dynamics plots
    \item \texttt{compare\_infected.png}: Infection comparison overlay
    \item \texttt{metrics\_table.csv}: Summary metrics table
\end{itemize}

\end{document}
