\documentclass[10pt,conference]{IEEEtran}
\usepackage{graphicx}
\usepackage{amsmath}
\usepackage{amssymb}
\usepackage{hyperref}

% Title, Abstract, and Introduction
\title{Epidemic Spread Analysis of SIR Model Over Static Scale-Free Networks}

\begin{document}

\maketitle

\begin{abstract}
The accurate understanding of epidemic dynamics on static contact networks is crucial for predicting outbreak trajectories and guiding interventions. We examine the stochastic SIR (Susceptible-Infected-Recovered) model over a large-scale Barabási-Albert static network, emphasizing the impact of network topology on epidemic thresholds and final epidemic size. Our work integrates a mechanistically parameterized SIR model---with transmission rates tuned to a network-based reproduction number---with computational simulation using FastGEMF. We extract key epidemic metrics such as peak infection, epidemic duration, and final outbreak size, and contextualize our findings with established literature. The results reinforce the significance of network structure in influencing epidemic dynamics, and provide baseline quantitative references for intervention planning and future research in epidemic forecasting. 
\end{abstract}

\section{Introduction}
Understanding the mechanisms underlying the spread of infectious diseases is a central concern in epidemiology, especially in heterogeneous populations connected by complex contact patterns. Classical compartmental models such as the SIR framework have offered foundational insights, but traditional approaches assume homogeneous mixing, neglecting the realistic, heterogeneous structure of human contacts \cite{Madar2004, Sottile2020, Han2025}.

In reality, individuals interact through networks characterized by broad degree distributions, clustering, and community structure \cite{Newman2002, Keeling2005}. As a result, network-based epidemic models provide a more accurate foundation for predicting epidemic invasions, threshold behavior, and the impact of interventions. In particular, studies have shown that static networks---where connections remain unchanged over the timescale of pathogen spread---are a valid and analytically tractable framework for many scenarios \cite{Sottile2020, Ochab2010, Tizzani2018}.

Prior work has elucidated how network topology, such as the presence of highly connected `hubs' in scale-free networks, can substantially alter epidemic thresholds, the potential for outbreaks, and the sensitivity of epidemic indices to initial conditions and parameters \cite{Sottile2020, Madar2004, Ochab2010}. Furthermore, mechanistic parameterizations leveraging network moments enable the robust estimation of network-adjusted transmission rates and basic reproduction numbers. Recent advances in computational platforms provide access to scalable, agent-based simulations that permit quantitative assessment of epidemic outcomes on arbitrarily large contact networks \cite{Han2025}.

Motivated by these developments, this study aims to quantitatively evaluate the epidemic trajectory of a canonical SIR process on a prototypical scale-free static network. We systematically construct the contact network, calibrate model parameters using established epidemic thresholds, and simulate outbreak dynamics. Leveraging contemporary analysis techniques, we extract and interpret epidemic peak, duration, and size metrics, placing them in direct comparison with literature results and theoretical expectations. The results provide a clear demonstration of the interplay between network topology and epidemic outcomes, and form a basis for future extensions including intervention strategies and additional model complexity.

\section*{References}

\begin{thebibliography}{}

\bibitem{Newman2002} M. E. J. Newman, ``Spread of epidemic disease on networks,'' \textit{Phys. Rev. E}, vol. 66, 2002, 016128. doi:10.1103/PhysRevE.66.016128.
\bibitem{Madar2004} Nilly Madar, T. Kalisky, R. Cohen, et al., ``Immunization and epidemic dynamics in complex networks,'' \textit{The European Physical Journal B}, vol. 38, pp. 269-276, 2004. doi:10.1140/EPJB/E2004-00119-8.
\bibitem{Sottile2020} Sara Sottile, Ozan Kahramanoğulları, M. Sensi, ``How network properties and epidemic parameters influence stochastic SIR dynamics on scale-free random networks,'' \textit{Journal of Simulation}, vol. 18, pp. 206-219, 2020. doi:10.1080/17477778.2022.2100724.
\bibitem{Ochab2010} Jeremi K. Ochab, P. Góra, ``Shift of percolation thresholds for epidemic spread between static and dynamic small-world networks,'' \textit{The European Physical Journal B}, vol. 81, pp. 373-379, 2010. doi:10.1140/epjb/e2011-10975-6.
\bibitem{Han2025} Shuai Han, Lukas Stelz, T. Sokolowski, et al., ``Unifying Physics- and Data-Driven Modeling via Novel Causal Spatiotemporal Graph Neural Network for Interpretable Epidemic Forecasting,'' \textit{arXiv}, abs/2504.05140, 2025. doi:10.48550/arXiv.2504.05140.
\bibitem{Tizzani2018} Michele Tizzani, Simone Lenti, Enrico Ubaldi, et al., ``Epidemic spreading and aging in temporal networks with memory,'' \textit{Phys. Rev. E}, vol. 98, 2018. doi:10.1103/PhysRevE.98.062315.
\bibitem{Keeling2005} M. J. Keeling, K. T. D. Eames, ``Networks and epidemic models,'' \textit{J. R. Soc. Interface}, vol. 2, pp. 295–307, 2005.

\end{thebibliography}

% Additional sections (Methodology, Results, etc.) will be written in the following steps, including figures and appendices as needed.

\end{document}
