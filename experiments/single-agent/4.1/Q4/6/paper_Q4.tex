\title{Epidemic Spread Analysis of SIR Model over Static Networks}
\begin{document}

\maketitle

\begin{abstract}
This paper investigates the dynamics of epidemic spread on two prototypical static network structures: Erdős–Rényi random graphs and Barabási–Albert scale-free networks, using an SIR (Susceptible-Infective-Removed) modeling framework. Employing realistic parameterization (reproduction number $R_0=2.5$, mean infectious duration 5 days, population $N=1000$), we perform direct stochastic simulations using FastGEMF over synthetic but epidemiologically plausible networks. The study quantifies key epidemic metrics such as final outbreak size, peak prevalence, epidemic duration, and doubling time, revealing marked effects of network topology on overall disease impact.
\end{abstract}

\section{Introduction}
The study of infectious disease spread over complex networks has revealed profound effects of network topology on key epidemiological outcomes \cite{Newman2002NetworkSpread,Keeling2005NetworkImplications}. While classical compartmental models assume homogeneous mixing, real populations—whether social, technological, or biological—exhibit heterogeneous connection patterns. Recent theoretical and computational advances allow explicit representation of these contact patterns as static graphs, providing new insight into the threshold behavior and predictability of epidemics. Notably, static networks such as Erdős–Rényi (ER) random graphs and Barabási–Albert (BA) scale-free networks, with their distinct degree distributions, have been widely studied due to their analytical tractability and relevance to real networks. This work follows the framework pioneered by Newman \cite{Newman2002NetworkSpread}, modeling the spread of an acute immunizing infection over both network types and quantifying how topology modulates epidemic risk, peak intensity, and total burden. The questions addressed are: Given identical disease parameters and equivalent mean degree, how do ER and BA networks differ in predicted epidemic course and outcome, and what mechanistic insights can be drawn for epidemic control?

\section{Methodology}
\subsection{Modeling Framework}
We model epidemic dynamics using the SIR (Susceptible-Infected-Recovered) paradigm, appropriate for acute, immunizing diseases such as influenza or COVID-19. The dynamical process is implemented node-wise on two synthetic static networks:

\begin{itemize}
  \item Erdős–Rényi (ER) random graph: $N=1000$ nodes, mean degree $\langle k \rangle=8$. Each pair of nodes is connected independently with probability $p=\langle k \rangle/(N-1)$.
  \item Barabási–Albert (BA) scale-free network: $N=1000$ nodes, each new node attaches to $m=4$ existing nodes; asymptotically $\langle k \rangle \approx 2m = 8$.
\end{itemize}

Both networks are generated using the NetworkX package, with degree distributions shown in Figure~\ref{fig:degdist}.

Disease dynamics proceed according to:

\begin{enumerate}
    \item Infection: Susceptible nodes connect to their Infected neighbors and become infected at rate $\beta$ per Infectious-Susceptible edge.
    \item Recovery: Infected nodes recover at rate $\gamma$ per unit time, entering the Removed class (immune).
\end{enumerate}

Parameters are set to match $R_0=2.5$ using Newman's result for network epidemics:
\[
\beta = \frac{R_0 \cdot \gamma}{\langle k^2 \rangle - \langle k \rangle}/\langle k \rangle
\]
where $\langle k^2 \rangle$ is the second degree moment. For both networks, $\gamma=0.2$, $R_0=2.5$, and $N=1000$. $\beta$ is computed individually for each network using their empirical degree statistics (see Table~\ref{tab:network-stats}).

Initial conditions are set as $1\%$ infected (10 nodes), $99\%$ susceptible, $0\%$ recovered, distributed randomly.

Simulation and analysis are implemented in Python using FastGEMF for SIR process, with results output for five stochastic realizations per scenario, run over 120 days.

\begin{figure}[h!]
\centering
\includegraphics[width=0.45\textwidth]{degdist_ER.png}
\includegraphics[width=0.45\textwidth]{degdist_BA.png}
\caption{Degree distributions for Erdős–Rényi (left) and Barabási–Albert (right) networks.}
\label{fig:degdist}
\end{figure}

\begin{table}[h!]
\caption{Network degree statistics and SIR parameters}
\label{tab:network-stats}
\centering
\begin{tabular}{lccc}
\hline
 & $\langle k \rangle$ & $\langle k^2 \rangle$ & $\beta$ \\
ER & $8.04$ & $72.48$ & $0.062$ \\
BA & $7.97$ & $138.02$ & $0.031$ \\
\hline
\end{tabular}
\end{table}

\section{Results}
Simulation outputs for both networks are summarized in Figure~\ref{fig:epicurves} (time evolution of $S$, $I$, $R$) and Table~\ref{tab:metrics} (quantitative metrics). The following observations are noted:

\begin{itemize}
  \item \textbf{Final epidemic size:} In ER network, $773$ individuals out of $1000$ were eventually infected (entered $R$ class), compared to only $341$ in BA.
  \item \textbf{Peak infection:} Peak prevalence was much higher in ER (185) than in BA (64), and occurred sooner ($18.3$ days vs $21.3$ days).
  \item \textbf{Epidemic duration:} Epidemic lasted \textasciitilde $64$ days in ER and \textasciitilde $59$ days in BA.
  \item \textbf{Doubling time} (early phase): Disease spread twice as fast in ER (1.47 days) than in BA (3.48 days).
\end{itemize}

\begin{figure}[h!]
\centering
\includegraphics[width=0.48\textwidth]{results-11.png}
\includegraphics[width=0.48\textwidth]{results-12.png}
\caption{Compartment trajectories $S$, $I$, $R$ vs time for (left) ER network, (right) BA network, averaged over five stochastic replicates.}
\label{fig:epicurves}
\end{figure}

\begin{table}[h!]
\caption{Key epidemic metrics from simulation}
\label{tab:metrics}
\centering
\begin{tabular}{lcc}
\hline
Metric & ER & BA \\
\hline
Final epidemic size ($R_{final}$) & 773 & 341 \\
Peak infection & 185 & 64 \\
Time of peak & 18.3 days & 21.3 days \\
Epidemic duration & 64.1 days & 59.2 days \\
Doubling time & 1.47 days & 3.48 days \\
\hline
\end{tabular}
\end{table}

\section{Discussion}
The results confirm and quantify strong effects of network topology on the progression and societal burden of epidemics. Despite identical mean degree and $R_0$, the scale-free BA network exhibits substantially lower final epidemic size and peak prevalence than the ER random network. This difference is consistent with theoretical predictions \cite{Newman2002NetworkSpread,Keeling2005NetworkImplications}; in scale-free networks, the presence of highly connected hubs may facilitate rapid local outbreaks but also create bottlenecks that limit large-scale propagation when $R_0$ is moderate. Homogeneity of degree in ER networks supports faster, more predictable epidemic waves with higher peak and larger fraction infected.

The epidemic threshold is determined by the network's degree distribution, particularly the mean excess degree $q=(\langle k^2 \rangle-\langle k \rangle)/\langle k \rangle$. Our calculation of $\beta$ for fixed $R_0$ ensures comparability; nonetheless, the lower $q$ for ER means each infection can reach more new susceptibles, driving larger, sharper outbreaks. In contrast, BA networks exhibit a wider spread of contacts, cushioning the growth.

The implications are practical: for networks with heavy-tailed degree distributions, conventional $R_0$-based interventions may overestimate epidemic risk compared to well-mixed or ER-type populations. Rapid identification and protection of hubs may be especially beneficial in such contexts.

\section{Conclusion}
This comparative study demonstrates that even under identical epidemiological parameters, the underlying contact structure dramatically impacts epidemic burden and kinetics. Static homogeneous (ER) networks produce larger, faster outbreaks compared to scale-free (BA) topologies with identical mean degree. These findings highlight the importance of accurate network modeling in outbreak prediction and suggest tailored strategies for surveillance and control in heterogeneous networks.

\section*{References}
\begin{thebibliography}{99}
\bibitem{Newman2002NetworkSpread} M. E. J. Newman, "The spread of epidemic disease on networks," Phys. Rev. E, vol. 66, no. 1, p. 016128, 2002. Available: https://doi.org/10.1103/PhysRevE.66.016128
\bibitem{Keeling2005NetworkImplications} M.J. Keeling, "Implications of network structure for epidemic dynamics," Theor. Popul. Biol., vol. 67, pp. 1–8, 2005. doi:10.1016/j.tpb.2004.08.002
\end{thebibliography}

\appendix
\section{Supplementary Figures}
\subsection{Degree Distributions}
\begin{figure}[h!]
\centering
\includegraphics[width=0.45\textwidth]{degdist_ER.png}
\includegraphics[width=0.45\textwidth]{degdist_BA.png}
\caption{Degree distributions for ER (left) and BA (right) networks.}
\end{figure}

\subsection{Code Excerpts}
Network and simulation codes are included as supplementary files:\newline
network\_construction.py, parameter\_setting.py, simulation-11.py, simulation-12.py, analysis\_ER.py, analysis\_BA.py.

\end{document}
