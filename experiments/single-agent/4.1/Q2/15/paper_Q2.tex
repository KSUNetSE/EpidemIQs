\documentclass[10pt,twocolumn]{IEEEtran}
\usepackage{amsmath,amssymb,graphicx}
\title{Causes for Breaking the Chain of Transmission: Analysis via SIR Models on Well-mixed and Network Populations}
\author{Epidemic Spread Analysis Project}
\begin{document}
\maketitle
\begin{abstract}
This study rigorously investigates the conditions under which the chain of transmission in an epidemic breaks, addressing whether cessation typically arises due to the decline of infectives or exhaustion of susceptibles. We conduct both analytical and network-based simulations using the SIR model, interpret final size results, and link our findings to established literature. The analysis reveals that, in standard SIR dynamics, transmission almost always breaks due to the depletion of infectives, while a population of susceptibles generally remains. These results are robust across both fully-mixed and static network scenarios and are supported by finalized theoretical and computational evidence.
\end{abstract}

\section{Introduction}
Understanding exactly how and why an epidemic ceases to spread---whether due to a paucity of infectives or to a complete lack of susceptibles---is fundamental for epidemiology and public health planning. Classical infectious disease models, such as the susceptible-infected-recovered (SIR) framework, predict that epidemics terminate when no further transmission is possible. This situation can theoretically occur either when there are no more infectious individuals to propagate the disease, or when no susceptibles remain to be infected. However, the precise answer depends on the interplay of model parameters, network topology, and stochastic effects.\par

Analytical works and simulation approaches ({\it e.g.},\cite{Lin2024-indirect-vacc, Li2019-network-bifurcation}) have examined these questions in both deterministic and network-based SIR models. The final size relation---which links the initial and final proportions of susceptible individuals to the basic reproduction number $R_0$---provides insight into the typical route of epidemic termination. In this paper, we employ both analytical solutions and stochastic simulations on explicit networks to elucidate which phenomenon primarily drives the breaking of the transmission chain.

\section{Methodology}
We employ two complementary approaches: (i) an analytically integrated ODE-based SIR model for a well-mixed population, and (ii) a stochastic network-based implementation using a static Erd\H{o}s--R\'enyi graph (mean degree eight, $N=1000$).\par

For both models, we initialize with a small number (1\%) of infective nodes. The transmission rate ($\beta$ for network, normalized per-contact for ODE) and recovery rate ($\gamma = 0.1$) are set to yield $R_0 = 2.5$ for the network scenario, matching classical threshold conditions\cite{Lin2024-indirect-vacc}. Network-based transmission rate $\beta$ is chosen according to $\beta = R_0 \gamma / q$ with $q = (\langle k^2 \rangle - \langle k \rangle)/\langle k \rangle$. Simulations proceed until all infectives are resolved or a maximum time is reached.

The primary outputs are:\begin{itemize}
    \item Evolution of $S$, $I$, $R$ over time for both models
    \item Final number of susceptibles and fraction removed at termination
    \item Time at which $I$ vanishes, and number of susceptibles at that moment
    \item Comparison to final size relations from epidemic theory
\end{itemize}
Relevant code and CSV/PNG files (\texttt{results-11.csv}, \texttt{results-12.csv}, and figures) document the quantitative workflow.

\section{Results}
\subsection{ODE-based (Well-mixed) Analysis}
As shown in Fig.~\ref{fig:sir_ode}, integrating standard SIR equations for $N=1000$ with initial $I_0=1$ and $R_0=0$ shows that the epidemic terminates with practically all individuals removed ($R_{final}/N \approx 1$), almost no infectives ($I_{final}\approx 10^{-4}$), and only a negligible fraction of susceptibles $(\approx 10^{-14})$ remain. The final size relation from the literature ([Lin2024-indirect-vacc]) is satisfied: $R(\infty)/N$ matches theoretical predictions for $R_0=2.5$ for a large population.\par

\begin{figure}[tb]
    \centering
    \includegraphics[width=0.47\textwidth]{results-11.png}
    \caption{SIR model (ODE): Epidemic trajectory for $N=1000$, $R_0=3$. Transmission ceases as $I\to 0$, only negligible fraction of susceptibles remains.}
    \label{fig:sir_ode}
\end{figure}

\subsection{Network SIR Dynamics}
For the Erd\H{o}s--R\'enyi network, epidemic evolution (Fig.~\ref{fig:network_sir}) shows typically that the infection stops with a substantial fraction (here 20\%) of susceptibles left: $I$ vanishes at $t\approx 149$ with $S\approx 202$. No further infections occur past this point, as there are no infective individuals remaining to transmit. Peak infection occurs at $t\approx 44.5$ (see Table~\ref{tab:metrics}). These results are consistent across repeated simulations.\par

\begin{figure}[tb]
    \centering
    \includegraphics[width=0.47\textwidth]{results-12.png}
    \caption{SIR on Erd\H{o}s--R\'enyi network: Transmission breaks with $20\%$ of nodes still susceptible ($S=202$) when $I$ first drops to zero.}
    \label{fig:network_sir}
\end{figure}

\begin{table}[tb]
\centering
\caption{Key Epidemic Metrics from Simulations}
\begin{tabular}{lccc}
Scenario & $S_{end}$ & $R_{end}$ & $I_{peak}$/time \\
\hline
ODE SIR & $\sim 0$ & $\sim N$ & NA \\
Network SIR & $202$ & $798$ & $203$ at $t=44.5$ \\
\end{tabular}
\label{tab:metrics}
\end{table}

These findings directly reflect theoretical predictions for SIR final size on random networks\cite{Li2019-network-bifurcation}.

\section{Discussion}
Our findings unambiguously demonstrate that, in both analytic and network SIR models with typical parameter ranges ($R_0 > 1$), the chain of transmission almost always breaks due to the decline of infectious individuals ($I \to 0$), not due to complete exhaustion of susceptibles. This is reflected by the pronounced plateau in $S$ once $I$ vanishes (Fig.~\ref{fig:network_sir}), and by solutions to final size relations\cite{Lin2024-indirect-vacc}.\par

Epidemics only run to exhaustion of susceptibles in limiting cases (e.g. $R_0\gg 1$, small/structured populations, or if reinfection is possible, as in SIS models). On realistic networks, $S_{end}$ can be substantial due to finite connectivity and herd immunity effects. These results are in consistent agreement with literature on SIR model bifurcation analyses and stochastic realizations\cite{Li2019-network-bifurcation}. Limitations include that real-world networks can be more heterogeneous, and stochastic fadeout is possible even for $R_0>1$. Practical implications are that the risk of disease recurrence depends strongly on remaining susceptibles, not solely on the exhausting of all possible hosts.

\section{Conclusion}
Both theory and simulation demonstrate that epidemics most typically cease because the pool of infectives drops to zero, leaving a nonzero fraction of susceptibles uninfected. This conclusion is robust to stochastic effects and network structure for standard SIR scenarios, and has direct relevance for epidemic forecasting, control, and planning. Future study should address the impact of network heterogeneity, demographic turnover, and other realistic extensions.

\section*{References}
\begin{thebibliography}{10}

\bibitem{Lin2024-indirect-vacc}
L.~Lin, H.~Hamedmoghadam, and R.~Shorten, ``Quantifying indirect and direct vaccination effects arising in the SIR model,'' \emph{J. R. Soc. Interface}, vol. 21, 2024. doi:10.1098/rsif.2024.0299.

\bibitem{Li2019-network-bifurcation}
C.-H.~Li and A.~Yousef, ``Bifurcation analysis of a network-based SIR epidemic model with saturated treatment function,'' \emph{Chaos}, vol.~29, no.~3, p. 033129, 2019. doi:10.1063/1.5079631.

\end{thebibliography}

\appendices
\section{Code, Data, and Additional Figures}
\noindent
\textsf{All code and simulation artifacts are provided as supplementary files. Key scripts include: \texttt{analytical\_sir\_finalsize.py}, \texttt{simulation\_SIR\_network.py}, and result files: \texttt{results-11.csv}, \texttt{results-12.csv}, \texttt{results-11.png}, \texttt{results-12.png}.}

\end{document}
