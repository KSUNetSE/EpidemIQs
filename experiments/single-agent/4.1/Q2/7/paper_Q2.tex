\section*{Title}
Epidemic Spread Analysis of SIR Model over Static Erdős–Rényi Network

\section*{Abstract}
This work investigates the propagation of infectious diseases through static random contact networks using the classic susceptible-infected-recovered (SIR) mechanistic model. We specifically examine an Erdős–Rényi (ER) contact network comprising 1000 individuals, and apply parameter choices characteristic of moderately transmissible pathogens with basic reproduction number $R_0=2.5$. By configuring the SIR parameters ($\beta=0.025$, $\gamma=0.1$) according to the degree moments of the network and simulating the resulting epidemics using a network-based Gillespie-style simulator, we extract epidemic severity metrics including peak infection, final epidemic size, epidemic duration, and doubling time. Our findings show that, under these conditions, the final epidemic size reaches $835$, the infection peaks at $224$ individuals at $t\approx35.3$, and the infection grows with an approximate doubling time of $6.3$ days. These results are discussed in the context of existing literature demonstrating how network topology influences epidemic outcomes compared to well-mixed and more clustered/heterogeneous network structures.

\section*{Introduction}
Understanding the spread of infectious diseases in structured populations is a fundamental challenge in network epidemiology. Traditional compartmental epidemic models, such as the SIR and SEIR frameworks, fundamentally assume either well-mixed homogeneous contacts or integrate some forms of population structure. However, real-world contact networks present intricate connectivity patterns which deeply modulate the trajectory and impact of epidemic waves.

The Erdős–Rényi random graph is a canonical model for static social mixing, characterized by homogeneous randomness in the presence of edges and absence of local clustering. Epidemic dynamics simulated on such networks provide both a baseline and an upper-bound scenario for widespread transmission, against which the impacts of more realistic contact patterns—clustering, degree heterogeneity, modularity—can be measured \cite{PLoSOne2015, ResearchGate23740104}. The SIR mechanistic model offers a tractable yet powerful framework for understanding the interplay between individual-level transmission events and emergent epidemic patterns.

In this study, we develop and simulate an SIR process over an ER random graph of $N=1000$ nodes, parameterized to yield a moderately contagious outbreak. Our aims are to quantify epidemic severity under these conditions, extract biologically and socially relevant metrics, and critically compare our findings to those in the broader literature on network-driven epidemics.

\section*{Methodology}
\subsection*{Epidemic Scenario and Mechanistic Model}
The studied scenario models the early growth and subsequent course of an acute directly transmitted infection (e.g., COVID-19, influenza-like) in a static population. The SIR model is selected as it faithfully represents the immunizing nature of infection, with individuals transitioning through Susceptible $(S)$, Infected $(I)$, and Recovered $(R)$ compartments, and no possibility of reinfection.

The infection rate $(\beta)$ and recovery rate $(\gamma)$ are selected to yield a basic reproduction number $R_0 = 2.5$, appropriately adjusted using the contact structure of a random network. Transitions are governed by:
\begin{align*}
S + I & \xrightarrow{\beta} 2I \\
I & \xrightarrow{\gamma} R
\end{align*}

\subsection*{Network Structure}
We construct a static Erdős–Rényi random network $(G_{N,p})$ with $N=1000$ nodes, and connection probability $p=0.01$, resulting in a mean degree $\langle k \rangle \approx 10$ and second moment $\langle k^2 \rangle \approx 110$. The edge structure is saved as a sparse adjacency matrix, and the degree distribution (see Figure~\ref{fig:degree-dist}) confirms the expected Poisson behavior.

Initial conditions assign $S=990$, $I=10$, $R=0$; infected individuals are selected as the first ten nodes for deterministic initialization.

\subsection*{Simulation and Parameterization}
Simulations are conducted using the FastGEMF software, employing a Gillespie-style simulation for the SIR process over the static ER network. The infection rate on the network is computed as $\beta = R_0 \gamma / q$, where $q= (\langle k^2 \rangle - \langle k \rangle) / \langle k \rangle$. Recovery rate is fixed at $\gamma=0.1$. The simulation is run for a maximum of $160$ days, or until the process stabilizes, with five realizations to average stochastic effects.

Key output metrics include the population dynamics by compartment (see Figure~\ref{fig:epidemic-curve}), peak and final sizes, epidemic duration, and doubling time in the initial growth phase.

\begin{figure}[h]
    \centering
    \includegraphics[width=0.45\textwidth]{degree_dist.png}
    \caption{Degree distribution for the simulated Erdős–Rényi network ($N=1000$, $p=0.01$).}
    \label{fig:degree-dist}
\end{figure}

\section*{Results}
Figure~\ref{fig:epidemic-curve} depicts the evolution of susceptible, infected, and recovered populations over time. The infection curve exhibits rapid exponential growth up to a peak at $t \approx 35.3$ days ($I = 224$), followed by a decline to extinction by $t\approx 160$. The epidemic lasts approximately $125$ days (duration defined as $I>1$). The final size of the outbreak is $835$ (recovered/immune), corresponding to $83.5\%$ of the initial population. The estimated doubling time in the exponential phase is $6.3$ days.

\begin{figure}[h]
    \centering
    \includegraphics[width=0.55\textwidth]{results-11.png}
    \caption{Time evolution of epidemic compartments (S, I, R) from SIR simulation on static ER network ($N=1000$, $p=0.01$, $\beta=0.025$, $\gamma=0.1$).}
    \label{fig:epidemic-curve}
\end{figure}

These results closely track theoretical and simulation studies on random networks: the infection propagates rapidly, with high final size and high peak infection, illustrating how the lack of clustering and degree heterogeneity amplifies epidemic risk compared to empirical contact structures \cite{PLoSOne2015}.

\section*{Discussion}
The simulated SIR epidemic on an Erdős–Rényi network typifies the consequences of homogeneous random mixing: the final epidemic size and peak infection load are relatively high, matching the predictions for random graphs in contrast to clustered or degree-heterogenous networks. This is consistent with recent literature demonstrating that configuration-type models (ER, configuration model, etc.) generally produce larger epidemic sizes and faster growth than observed in empirical or ERGM-modeled contact networks \cite{PLoSOne2015}.

The implications are especially relevant for public health modeling. While ER networks provide a useful baseline for theoretical study and upper-bound estimation of epidemic severity, more realistic predictions for intervention outcomes and epidemic control necessitate consideration of clustering, modularity, and degree variance found in real contact networks.

The current results reinforce prior analyses, showing that the epidemic threshold, size, and dynamics are substantially shaped by network moments; the selected $R_0$ and $
$ values produce infection kinetics similar to well-mixed models, though real populations typically show reduced transmission due to network structure \cite{ResearchGate23740104}.

\section*{Conclusion}
Simulated SIR epidemics on static Erdős–Rényi random graphs exhibit rapid transmission, high peak infections, and large final outbreak sizes, exceeding those typically observed in empirical network models with realistic contact heterogeneity. Our findings reiterate the crucial role of network structure in epidemic outcomes and validate the utility of ER graphs as useful reference models for epidemic theoreticians, while highlighting their limitations for direct forecasting in real-world populations.

\section*{References}

\begin{thebibliography}{9}

\bibitem{PLoSOne2015} Jenness, S. M., Goodreau, S. M., & Morris, M., 2015. "A Simulation Study Comparing Epidemic Dynamics on Exponential Random Graphs and Random Graphs." PLoS One, 10(11), e0142181.

\bibitem{ResearchGate23740104} Newman, M. E. J. Random Graphs as Models of Networks. ResearchGate. Available: https://www.researchgate.net/publication/23740104_Random_Graphs_as_Models_of_Networks

\end{thebibliography}

\section*{Appendices}
\subsection*{A. Simulation and Network Construction Code}
\begin{verbatim}
# (See supplementary files: network_construction.py, parameter_setting.py, simulation-11.py)
\end{verbatim}

\subsection*{B. Additional Results Tables}
\begin{tabular}{lcc}
Metric & Value \\
\hline
Final Epidemic Size & 835 \\
Peak Infection & 224 \\
Peak Time (days) & 35.3 \\
Epidemic Duration (days) & 125.2 \\
Doubling Time (days) & 6.3 \\
\end{tabular}
