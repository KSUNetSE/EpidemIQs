\section*{Title}
Epidemic Spread Analysis of SIR Model over Erdős-Rényi Static Contact Network

\section*{Abstract}
This study investigates the dynamics of epidemic spread using the SIR (Susceptible-Infectious-Removed) compartmental model over a static Erdős-Rényi (ER) contact network with parameters reflective of real-world respiratory disease transmission, such as COVID-19. We develop a computational framework to simulate disease transmission, parameterize the model based on literature-derived values for transmission and recovery, and analyze critical epidemic outcomes. Simulation results reveal that network structure significantly alters epidemic metrics such as final epidemic size, peak infection, and epidemic duration, highlighting the importance of heterogeneity in contact patterns for disease forecasting and control strategies.

\section*{Introduction}
Understanding epidemic propagation in a structured population is essential for accurate forecasting and intervention planning. Traditional compartmental epidemic models such as SIR and SEIR often rely on mean-field, well-mixed assumptions, which may not capture the impact of population heterogeneity and network topology. Recent research demonstrates that network-based models, which explicitly encode contact patterns, can produce qualitatively and quantitatively different epidemic dynamics when compared to their well-mixed analogues \cite{Tian2024, AntulovFantulin2014}. In particular, the choice of network type—such as Erdős-Rényi (ER), scale-free, or small-world—significantly affects critical measures like epidemic threshold, size, and temporal evolution \cite{Tian2024}.

This work models an epidemic with SIR dynamics on an ER static contact network that mimics respiratory disease spread, parameterized for COVID-19-like transmission. Our goal is to assess how key epidemiological outcomes are shaped by network structure and disease parameters. We leverage recent advances in stochastic, network-driven simulations informed by empirical and theoretical literature \cite{Tian2024}. The study’s findings provide a basis for understanding the interplay between disease transmission and contact network organization, thereby supporting public health response and forecasting efforts.

\section*{Methodology}
\subsection*{Network Construction}
We constructed a static Erdős-Rényi (ER) network with $N=1000$ nodes and an average degree $\langle k \rangle \approx 8$, reflecting typical contact rates in human social networks for airborne diseases \cite{Tian2024}. The network was generated by assigning an independent connection probability $p=\langle k \rangle/(N-1)$ between all node-pairs. Network properties were validated by calculating mean degree ($8.03$) and second moment of degree ($72.48$), and by visualizing the degree distribution (see Figure~\ref{fig:degree-distribution}).

{\centering
\includegraphics[width=0.7\textwidth]{degree_dist.png}
\captionof{figure}{Degree Distribution of the Generated Erdős-Rényi Network.}\label{fig:degree-distribution}
}

\subsection*{Mechanistic Model and Parameterization}
We implemented the standard SIR compartmental model, where individuals are either Susceptible, Infectious, or Removed (Recovered). Model transitions are:
\begin{itemize}
    \item $S + I \xrightarrow{\beta}$ $2I$ (infection via contact)
    \item $I \xrightarrow{\gamma}$ $R$ (recovery/removal)
\end{itemize}

Parameterization leveraged real-world values: basic reproduction number $R_0=2.5$ and recovery rate $\gamma=1/7$ days$^{-1}$ (approximate for COVID-19 infectious period). Transmission rate $\beta$ was calculated as $\beta = R_0 \gamma/q$, where $q=({\langle k^2 \rangle} - {\langle k \rangle})/{\langle k \rangle}$ for the constructed network. Final values used: $\beta = 0.045$, $\gamma = 0.143$. The initial state comprised 5 infectious individuals (randomly assigned) and the remainder susceptible.

\subsection*{Simulation}
Stochastic network-based simulations were conducted using fastGEMF, with $nsim=10$ runs over one year (365 days). Population evolution in each compartment was recorded at each time step. Results were saved for subsequent quantitative and visual analysis.

\section*{Results}
The epidemic outbreak demonstrated classic SIR behavior but was quantitatively shaped by network topology. Key outcomes:
\begin{itemize}
    \item \textbf{Final Epidemic Size:} Approximately 78\% of the population was ever infected ($\approx$780/1000 individuals).
    \item \textbf{Peak Infection Rate:} The fraction of infectious individuals peaked at 20.8\%, corresponding to $$208 people infectious simultaneously.
    \item \textbf{Peak Time:} Peak infections occurred around day 40.
    \item \textbf{Epidemic Duration:} The epidemic persisted for approximately 104 days (defined as the period with $I>1$).
    \item \textbf{Doubling Time:} Estimated as 16.2 days early in the outbreak.
\end{itemize}
Final compartment sizes are detailed in Table~\ref{tab:metrics-table} and visualized in Figure~\ref{fig:final-compartement-states}. The simulated population evolution is shown in Figure~\ref{fig:simulation-evolution}.

{\centering
\includegraphics[width=0.7\textwidth]{results-11.png}
\captionof{figure}{SIR Dynamics: Time Evolution of Each Compartment on ER Network, $N=1000$.}\label{fig:simulation-evolution}
}

{\centering
\includegraphics[width=0.6\textwidth]{metrics_table.png}
\captionof{table}{Final State Distribution by Compartment}\label{tab:metrics-table}
}

\section*{Discussion}
Our results reaffirm that network structure can substantially modulate epidemic outcomes, especially compared to well-mixed models that often overestimate or miss key dynamics \cite{Tian2024, AntulovFantulin2014}. Even with a homogeneous ER network, the stochastic nature of transmission and discrete contacts influences epidemic trajectory: roughly 22\% of the network never experienced infection, in contrast to well-mixed SIR predictions where near-total susceptible depletion typically occurs for similar $R_0$.

The relatively high final epidemic size and peak infection indicate rapid spread, consistent with respiratory pathogens in semi-random contact populations. Doubling time and duration reflect significant transmission but also a network-limited reach. Limitations include the assumptions of homogeneity in contact rates and constant network structure, neither of which holds for all real-world diseases. Extensions to consider heterogeneous networks (e.g., scale-free, multilayer) or intervention strategies are natural next directions.

This study complements recent works demonstrating that even modest network structural details or multilayer architectures, as well as individual-level heterogeneity, can markedly shift epidemic thresholds and size \cite{Tian2024}.

\section*{Conclusion}
In conclusion, explicitly accounting for population contact structure in simulating disease spread yields substantially different epidemic outcomes than traditional mean-field models. Our SIR-ER simulation of a COVID-19-like disease showed a large but not universal epidemic, sharp peak, and well-defined duration. Accurate modeling of networked populations is therefore critical for epidemic forecasting, mitigation planning, and interpreting intervention efficacy.

\section*{References}

\begin{thebibliography}{99}
\bibitem{Tian2024} Y. Tian and O. Yağan, "Spreading Processes With Layer-Dependent Population Heterogeneity Over Multilayer Networks," IEEE Transactions on Network Science and Engineering, vol. 11, pp. 4106-4119, 2024.
\bibitem{AntulovFantulin2014} N. Antulov-Fantulin, A. Lancić, T. Šmuc, et al., "Detectability limits of epidemic sources in networks," Phys. Rev. Lett., vol. 114, p. 248701, 2014.
\end{thebibliography}

\section*{Appendix}
\subsection*{Network Structure and Compartment Distribution}
\begin{itemize}
    \item \textbf{Network Parameters:} ER, $N=1000$, $\langle k \rangle=8$, Second moment $\langle k^2 \rangle=72.48$.
    \item \textbf{Initial Conditions:} 995 Susceptible, 5 Infectious, 0 Removed, randomly assigned.
    \item \textbf{Simulation Artifacts:}
        \begin{itemize}
            \item \texttt{degree-dist.png}: ER network degree distribution
            \item \texttt{metrics_table.png}: Final compartment distribution
            \item \texttt{results-11.png}: Compartment time evolution
        \end{itemize}
\end{itemize}

