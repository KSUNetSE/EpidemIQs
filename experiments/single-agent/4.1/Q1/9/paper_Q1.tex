\documentclass[conference]{IEEEtran}
\usepackage{graphicx}
\usepackage{booktabs}
% No underscores in section titles or labels!
\title{Comparative Dynamics of SIR Epidemics on Static Erdős-Rényi and Barabási-Albert Networks}

\begin{document}
\maketitle

\begin{abstract}
This study conducts a comparative analysis of Susceptible-Infected-Recovered (SIR) epidemic processes on two prominent static network models: Erdős-Rényi (ER) random graphs and Barabási-Albert (BA) scale-free networks. Using mechanistic simulations with identical mean degree and initial seeding fraction, we evaluate how network topology impacts key epidemic outcomes including peak infection rate, epidemic duration, and final epidemic size. Our results demonstrate that, despite being parameterized for the same reproduction number ($R_0=2.5$), the ER topology produces markedly higher and later infection peaks, as well as a larger cumulative epidemic size, compared to the BA network under identical conditions. These findings highlight the substantial role of structural heterogeneity and node centrality distributions in shaping emergent epidemic trajectories. We visualize degree distributions and epidemic curves to support the quantitative results and discuss implications for interpretation of $R_0$ and intervention planning in diverse host populations.
\end{abstract}

\section{Introduction}
Epidemic spread, as observed in diseases ranging from seasonal influenza to COVID-19, is fundamentally shaped by the connectivity patterns of the host population. While classical SIR (Susceptible-Infected-Recovered) models assume homogeneity in mixing or regular lattice structure, mounting evidence shows that most real-world contact networks exhibit significant heterogeneity in node degree, connectivity clustering, and the presence of influential hubs \cite{Newman2002, KeelingEames2005}. Static network models such as Erdős-Rényi (ER) graphs, which feature a random and approximately Poisson degree distribution, and Barabási-Albert (BA) scale-free networks, which display a power-law degree distribution with a minority of highly-connected hub nodes, are widely used to abstract different aspects of real social or technological contact structure \cite{BarabasiAlbert1999, PastorSatorras2001}.

Disease transmission on such static networks can yield dynamics starkly different from mean-field models with identical average connectivity, especially for pathogens whose basic reproduction number $R_0$ is only modestly above the epidemic threshold \cite{PastorSatorras2001, AndersonMay1992}. Heterogeneity in connectivity, particularly in scale-free topologies, has been theoretically and empirically linked to lowered epidemic thresholds, smaller peaks, and complex patterns of persistence or local extinction \cite{PastorSatorras2001, Moreno2002}. However, systematic quantitative comparisons with controlled initial conditions and infection parameters remain vital for both theory and data-informed epidemic planning.

In this work, we deploy stochastic SIR simulations on large ER and BA networks designed with identical mean degree. We seek to isolate the effect of degree variance and hub structure on five key epidemic metrics: peak infection fraction, time to peak, total epidemic size, duration, and early doubling time. In doing so, we aim to provide clear pedagogical and practical guidance for interpreting network-sensitive epidemic risk and prioritizing interventions under network uncertainty.

\section{Methodology}
\subsection{Network Construction}
We constructed two network models, each with $N=1000$ nodes and average degree $\langle k \rangle \approx 8$.

\begin{itemize}
  \item \textbf{Erdős-Rényi (ER) Network:} Each possible edge between the $N$ nodes is formed independently with probability $p = \langle k \rangle / (N-1)$. This yields a Poisson-like degree distribution with low variance. Empirical statistics: mean degree $8.04$, degree second moment $72.5$.
  \item \textbf{Barabási-Albert (BA) Network:} Built by preferential attachment, each new node connects to $m=4$ existing nodes, favoring those already well-connected. This process creates a power-law degree distribution with a few hubs. Empirical statistics: mean degree $7.97$, degree second moment $138.0$.
\end{itemize}

Figure~\ref{fig:degree-distributions} visualizes the contrasting degree distributions for both topologies, as realized in this study.

\begin{figure}[!ht]
  \centering
  \includegraphics[width=0.45\textwidth]{degree-distributions.png}
  \caption{Empirical degree distributions for the Erdős-Rényi (ER) and Barabási-Albert (BA) networks with $N=1000$ and mean degree $8$. The BA network exhibits a heavy-tailed distribution, reflecting hub nodes.}
  \label{fig:degree-distributions}
\end{figure}

\subsection{Epidemic Model and Parameters}
We employ the discrete-state SIR model. The state transitions are:
\begin{align*}
\text{S} + \text{I} &\xrightarrow{\beta} \text{I} + \text{I} \\
\text{I} &\xrightarrow{\gamma} \text{R}
\end{align*}
Transmission (\(\beta\)) and recovery (\(\gamma\)) rates are calibrated so that $R_0 = 2.5$ according to the network structure. For each run, initial conditions assign $1\%$ of individuals (10 nodes) as infected, all others susceptible, none recovered.

\subsection{Simulation Protocol}
Stochastic simulations are performed (nsim=10) on both network topologies for 120 days, recording the mean time-series of each compartment ($S$, $I$, $R$). All parameters, initializations, and scripts for network/model construction and simulation are available in the supplementary material.

\section{Results}
Figure~\ref{fig:epidemic-curves} presents epidemic curves for both network types. Key metric outcomes are summarized in Table~\ref{tab:metrics}.

\begin{figure}[!ht]
  \centering
  \includegraphics[width=0.45\textwidth]{results-1-ER.png}\\
  \includegraphics[width=0.45\textwidth]{results-1-BA.png}
  \caption{SIR epidemic time-courses for ER (top) and BA (bottom) networks under identical parameters/$R_0$. The ER curve reaches a higher and later peak, and maintains a larger infected population over time.}
  \label{fig:epidemic-curves}
\end{figure}

\begin{table}[!ht]
  \centering
  \caption{Key epidemic metrics (mean of 10 runs).}
  \begin{tabular}{lcc}
    \toprule
    Metric & ER (Random) & BA (Scale-Free) \\
    \midrule
    Peak infection fraction & 0.187 & 0.091 \\
    Time to peak (d)       & 26.5  & 19.5 \\
    Final epidemic size    & 0.772 & 0.430 \\
    Duration (days)        & 1534  & 850   \\
    Max infected (#)       & 187   & 91    \\
    Doubling time (early)  & N/A   & N/A   \\
    \bottomrule
  \end{tabular}
  \label{tab:metrics}
\end{table}

The ER network produces a substantially higher and more protracted infection peak compared to BA, which displays a lower, earlier peak and reduced overall attack rate. This contrast emerges despite identical average degree and $R_0$ calibration, reflecting the profound effect of hub-driven topology in the BA model, consistent with theoretical expectations.

\section{Discussion}
Our comparative study underscores the impact of network topology on the trajectory of epidemic outbreaks, independent of $R_0$ and mean degree. In the BA network, distribution of connectivity into hubs and a majority of low-degree nodes suppresses explosive population-level transmission; infection persists locally, but the outbreak is less synchronously widespread. By contrast, in the homogenous ER network, random connectivity enhances synchrony and the epidemic propagates swiftly through the population, maximizing peak demand for healthcare and intervention measures.

These quantitative findings echo prior theoretical and numerical research on the reduced epidemic threshold and altered attack rates in scale-free networks \cite{Moreno2002, PastorSatorras2001}. They highlight practical risks in interpreting $R_0$ or attack rate estimates without considering host population structure, an issue of ongoing relevance for both real-time forecasting and post-epidemic analysis. We note limitations, including static structure, no demographic or behavioral variation, and fixed recovery time, but the results provide a clean illustration of topology-linked epidemic variability.

\section{Conclusion}
Structural heterogeneity, such as that encoded in scale-free (Barabási-Albert) networks, fundamentally alters epidemic risk and outcomes under SIR dynamics, even with identical mean degree and reproduction number as a random (Erdős-Rényi) network. Accurate assessment of epidemic potential and intervention strategy thus demands explicit attention to population contact structure beyond summary statistics like $R_0$.

\section*{References}

\begin{thebibliography}{99}
  \bibitem{Newman2002} M. E. J. Newman, Spread of epidemic disease on networks, Physical Review E, vol. 66, no. 1, p. 016128, 2002.
  \bibitem{KeelingEames2005} M. J. Keeling and K. T. D. Eames, Networks and epidemic models, Journal of the Royal Society Interface, vol. 2, no. 4, pp. 295-307, 2005.
  \bibitem{BarabasiAlbert1999} A.-L. Barabási and R. Albert, Emergence of scaling in random networks, Science, vol. 286, no. 5439, pp. 509-512, 1999.
  \bibitem{PastorSatorras2001} R. Pastor-Satorras and A. Vespignani, Epidemic spreading in scale-free networks, Physical Review Letters, vol. 86, no. 14, pp. 3200-3203, 2001.
  \bibitem{AndersonMay1992} R. M. Anderson and R. M. May, Infectious Diseases of Humans: Dynamics and Control, Oxford University Press, 1992.
  \bibitem{Moreno2002} Y. Moreno, R. Pastor-Satorras, and A. Vespignani, Epidemic outbreaks in complex heterogeneous networks, European Physical Journal B, vol. 26, pp. 521–529, 2002.
\end{thebibliography}

% Supplementary code, result files, and simulation settings available on request or as appendices.

\end{document}
