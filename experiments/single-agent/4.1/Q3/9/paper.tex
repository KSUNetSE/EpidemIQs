
\title{Impact of Temporal Network Structure on SIR Epidemic Spread: \newline A Comparative Study of Activity-Driven and Aggregated Static Networks}

\begin{document}
\maketitle

%====================
%   ABSTRACT
%====================
\begin{abstract}
This paper investigates the influence of temporal contact patterns on epidemic spread by comparing the dynamics of the Susceptible-Infectious-Recovered (SIR) model on an activity-driven temporal network and its corresponding time-aggregated static network. We focus on a population of $N=1000$ nodes, each activating with probability $\alpha=0.1$ per time step to form $m=2$ transient connections. To model infection dynamics, the SIR process is parameterized to have a basic reproduction number $R_0=3$. We conduct detailed stochastic simulations for (a) the original temporal activity-driven network and (b) its static, time-aggregated counterpart, where edge weights encode the frequency of observed contacts. Our analysis reveals that the temporal structure significantly slows the epidemic spread, lowers the epidemic peak, and—despite some cumulative catch-up—prolongs the outbreak duration and alters the final epidemic size relative to the aggregated static network. These findings underscore the necessity of accounting for temporal structure in network-based epidemic modeling, with implications for forecasting, prevention, and intervention design.
\end{abstract}

%====================
%   INTRODUCTION
%====================
\section{Introduction}
The accurate modeling and prediction of infectious disease outbreaks is a central challenge in network science and epidemiology. Conventional approaches often model populations as static graphs, topologically encoding contacts that could potentially transmit infection within the time window considered\cite{HolmeMasuda2015,Valdano2015}. However, real-world contact patterns are inherently time-varying: individuals activate intermittently, and connections are transient, shaped by behavioral or social rhythms\cite{Nadini2020,Perra2012}. The temporal structure—the sequencing and burstiness of interactions—can fundamentally affect epidemic trajectories, with possible consequences for outbreak severity, duration, and optimal intervention timing\cite{Nadini2020,Riad2019a,Riad2019b}.

Activity-driven models provide a minimal yet powerful framework for generating temporal networks representative of these realistic patterns\cite{Perra2012}. At each discrete time step, nodes activate with probability $\alpha$ and establish $m$ links to randomly chosen peers, generating a rapidly fluctuating network whose aggregate over time resembles a weighted static graph. Infectious diseases propagate over these networks via mechanistic models such as SIR, where the interplay of infection, recovery, and changing connectivity yields complex outcomes\cite{Nadini2020,Riad2019a}.

A key question is: How does the temporal structure of activity-driven networks modulate epidemic outcomes, compared to the predictions of their time-aggregated static counterparts? Existing work suggests that temporality often raises the effective epidemic threshold and diminishes the final epidemic size, as infections must traverse the actual sequence of contacts, frequently breaking transmission chains or introducing delays\cite{HolmeMasuda2015,Valdano2015}. In contrast, time aggregation ignores the ordering and duration of links, yielding an ``instantaneously dense'' network that may overestimate spreading potential and accelerate outbreaks\cite{PMC8169215}.

In this study, we focus on the comparative analysis of SIR epidemic dynamics on an activity-driven temporal network with $N=1000$, $\alpha=0.1$, $m=2$, and $R_0=3$. We simulate both the temporal and aggregated networks, systematically measuring epidemic thresholds, peak prevalence, final size, and duration. Our results illuminate critical differences—delayed, reduced peaks and prolonged outbreaks in the temporal case—which are highly relevant for policy and real-time forecasting. Section II describes the network construction, SIR modeling, and simulation setup. Section III presents results with quantitative metrics and epidemic trajectory comparisons. Sections IV and V provide discussion, implications, and conclusions for epidemic modeling and control.

%====================
%   REFERENCES (partial for intro)
%====================
\begin{thebibliography}{99}

\bibitem{HolmeMasuda2015} Petter Holme and Naoki Masuda, "The basic reproduction number as a predictor for epidemic outbreaks in temporal networks," PLoS ONE, vol. 10, no. 3, p. e0120567, 2015.

\bibitem{Valdano2015} Eugenio Valdano, Luca Ferreri, Chiara Poletto, and Vittoria Colizza, "Analytical computation of the epidemic threshold on temporal networks," Phys. Rev. X, vol. 5, 021005, 2015.

\bibitem{Nadini2020} Matthieu Nadini, Alessandro Rizzo, Maurizio Porfiri, "Epidemic spreading in temporal and adaptive networks with static backbone," IEEE Trans. Netw. Sci. Eng., 2020.

\bibitem{Riad2019a} Mahbubul H. Riad, M. Sekamatte, F. Ocom, et al., "Risk assessment of Ebola virus disease spreading in Uganda using a two-layer temporal network," Scientific Reports, vol. 9, p. 12260, 2019.

\bibitem{Riad2019b} Mahbubul H. Riad et al., "Risk assessment of Ebola virus disease spreading in Uganda using a multilayer temporal network," bioRxiv, 2019.

\bibitem{PMC8169215} Petter Holme, Naoki Masuda, "Concurrency measures in the era of temporal network epidemiology," PLoS Comput Biol, 2021.

\bibitem{Perra2012} Nicola Perra, Bruno Gonçalves, Romualdo Pastor-Satorras, and Alessandro Vespignani, "Activity driven modeling of time varying networks," Scientific Reports, vol. 2, p. 469, 2012.

\end{thebibliography}

\end{document}