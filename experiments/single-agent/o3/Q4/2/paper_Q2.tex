% ****** LaTeX Paper: Competitive SIS Dynamics on Multiplex Networks ******
\documentclass[10pt,conference]{IEEEtran}
\usepackage{amsmath,amssymb,graphicx,hyperref}
\begin{document}

% -------------------------------------------------------------
% Title
% -------------------------------------------------------------
\title{Coexistence and Dominance in Competitive SIS Dynamics on Multiplex Networks:  \newline  Mean--Field Analysis and FastGEMF Simulations}
\author{Anonymous Author(s)}
\maketitle

% -------------------------------------------------------------
% Abstract
% -------------------------------------------------------------
\begin{abstract}
This study investigates the fate of two mutually exclusive memes that spread according to susceptible--infected--susceptible (SIS) rules on a two--layer multiplex network whose layers share the same set of nodes but possess distinct edge sets.  Each meme spreads only on its associated layer with infection rates $(\beta_{1},\beta_{2})$ and recovery rates $(\delta_{1},\delta_{2})$, yielding effective transmission ratios $\tau_{\ell}=\beta_{\ell}/\delta_{\ell}$.  We derive nonlinear mean--field equations extending the $N$--intertwined approximation to the bi--virus setting and obtain spectral conditions for the stability of the disease--free, single--meme, and coexistence equilibria.  In particular, coexistence requires $\tau_{1}>1/\lambda_{1}\bigl(\operatorname{diag}(1-\mathbf{y}^{*})\mathbf{A}\bigr)$ and $\tau_{2}>1/\lambda_{1}\bigl(\operatorname{diag}(1-\mathbf{x}^{*})\mathbf{B}\bigr)$, where $\lambda_{1}(\cdot)$ denotes the spectral radius and $(\mathbf{x}^{*},\mathbf{y}^{*})$ are the single--meme endemic states.  We further show that structural mismatches between the principal eigenvectors of the layers---quantified by a low cosine similarity---facilitate coexistence.  To validate the analysis, we perform extensive agent--based simulation with FastGEMF on a $N=1000$ network composed of a Barabási–Albert layer and an Erdős–Rényi layer.  Parameter sweeps confirm the analytical phase diagram: when both $\tau_{\ell}$ exceed their single--layer thresholds but the layers are uncorrelated, long--term coexistence is observed only in a narrow region where the leading eigenvectors are weakly aligned and $\tau_{1}\approx \tau_{2}$.  Otherwise, the meme with higher $\tau$ or better structural support eliminates its competitor.  The findings elucidate how multiplex architecture governs competitive spreading processes and inform the design of interventions aiming to sustain cultural diversity or, conversely, to promote a dominant narrative.
\end{abstract}

% -------------------------------------------------------------
% Introduction
% -------------------------------------------------------------
\section{Introduction}
The diffusion of mutually exclusive ideas, rumours, or behavioural innovations through social systems is increasingly mediated by multiple communication channels---online platforms, face--to--face interactions, and mass media---that form {	extit{multiplex}} contact structures \cite{Sahneh2014}.  Unlike classical single--pathway epidemic models, the multiplex viewpoint acknowledges that distinct memes leverage distinct ties.  Understanding whether competing memes can coexist or whether one extinguishes the other is therefore crucial for marketing, public--health messaging, and information warfare \cite{Diakonova2013}.

Mathematically, competition has been explored via bi--virus extensions of the susceptible--infected--susceptible (SIS) model \cite{VanMieghem2011,BiSIS2023,DiscreteBivirus2023}.  When exclusivity is strict---a node cannot carry both memes simultaneously---the resulting state space couples two nonlinear contagion processes.  Spectral criteria for persistence and extinction have been obtained for identical layers, yet the role of heterogeneous, partially overlapping layers remains open.  This work addresses the gap by (i) deriving general mean--field coexistence conditions that embed layer structure, (ii) quantifying the effect of eigenvector misalignment, and (iii) validating the theory through high--fidelity simulation.

Our contributions are three--fold:
\begin{itemize}
    \item We extend the $N$--intertwined mean--field approximation (NIMFA) to multiplex competitive SIS and analyse the equilibria via monotone dynamical systems theory.
    \item We relate coexistence to layer structural attributes---spectral radius, eigenvector overlap, and degree--degree correlations---revealing why structural heterogeneity promotes diversity.
    \item We implement the model in the FastGEMF simulator, conduct a $4\times4$ parameter sweep, and visualise the dominance landscape, showing excellent qualitative agreement with theory.
\end{itemize}

\section{Methodology}
\subsection{Network Construction}
Layer~$A$ (meme~1) is generated as a Barabási--Albert graph with $N=1000$ and attachment parameter $m=3$, yielding mean degree $\langle k_{A}\rangle=5.982$ and second moment $\langle k^{2}_{A}\rangle=88.55$.  Layer~$B$ (meme~2) is an Erdős--Rényi graph with connection probability $p=0.01$, resulting in $\langle k_{B}\rangle=10.152$ and $\langle k^{2}_{B}\rangle=113.27$.  The layers have identical node sets but independent edge sets, producing low interlayer degree correlation (Pearson $r=0.06$).

The adjacency matrices are stored as sparse CSR matrices (files \texttt{layerA.npz} and \texttt{layerB.npz}).  Spectral radii computed via ARPACK are $\lambda_{1}(\mathbf{A})=14.42$ and $\lambda_{1}(\mathbf{B})=11.26$.

\subsection{Competitive SIS Model}
Each node is in one of three compartments: susceptible ($S$), infected by meme~1 ($I_{1}$), or infected by meme~2 ($I_{2}$).  Transitions are:
\begin{align*}
S + I_{1} &\xrightarrow{\beta_{1}} I_{1} \quad \text{(layer $A$ edges)},\\
S + I_{2} &\xrightarrow{\beta_{2}} I_{2} \quad \text{(layer $B$ edges)},\\
I_{1} &\xrightarrow{\delta_{1}} S,\qquad I_{2} \xrightarrow{\delta_{2}} S.
\end{align*}
The exclusivity constraint forbids $I_{1}I_{2}$ co--occupation.

\subsection{Mean--Field Equations}
Let $x_{i}(t)$ and $y_{i}(t)$ be the probabilities that node~$i$ is in $I_{1}$ and $I_{2}$, respectively.  Under the customary independence assumption \cite{VanMieghem2011},
\begin{align}
\dot{x}_{i}&= -\delta_{1}x_{i} + (1-x_{i}-y_{i})\beta_{1}\sum_{j}a_{ij}x_{j}, \label{eq:nimfa1}\\
\dot{y}_{i}&= -\delta_{2}y_{i} + (1-x_{i}-y_{i})\beta_{2}\sum_{j}b_{ij}y_{j}. \label{eq:nimfa2}
\end{align}
Equations~\eqref{eq:nimfa1}--\eqref{eq:nimfa2} define a cooperative, competitive dynamical system on the domain $D=\{(\mathbf{x},\mathbf{y})\mid \mathbf{0}\le \mathbf{x},\mathbf{y},\;\mathbf{x}+\mathbf{y}\le \mathbf{1}\}$.  Standard arguments \cite{BiSIS2023} show that trajectories converge to equilibria in $D$.

\subsection{Equilibria and Spectral Conditions}
Three equilibrium types exist: disease--free $(\mathbf{0},\mathbf{0})$, single--meme $(\mathbf{x}^{*},\mathbf{0})$ or $(\mathbf{0},\mathbf{y}^{*})$, and coexistence $(\hat{\mathbf{x}},\hat{\mathbf{y}})\gg\mathbf{0}$.  Linearisation about $(\mathbf{0},\mathbf{0})$ gives thresholds $\tau_{1}>1/\lambda_{1}(\mathbf{A})$ and $\tau_{2}>1/\lambda_{1}(\mathbf{B})$ for invasion.  Stability of single--meme equilibria requires, e.g. for meme~1,
\begin{equation}
\tau_{2}<\frac{1}{\lambda_{1}\bigl(\operatorname{diag}(1-\mathbf{x}^{*})\mathbf{B}\bigr)},
\end{equation}
with analogous expression for meme~2.  If both inequalities reverse, the coexistence equilibrium appears and is globally attractive within $D$.

\subsection{Simulation Design}
FastGEMF implements the stochastic process exactly.  Two simulation sets were executed:
\begin{enumerate}
    \item \textbf{Baseline runs} with $(\beta_{1},\beta_{2})=(0.10,0.12)$, $(0.11,0.11)$; initial condition: $5\%$ high--degree nodes in $I_{1}$, $5\%$ random nodes in $I_{2}$.
    \item \textbf{Parameter sweep} $\beta_{\ell}\in\{0.08,0.10,0.12,0.14\}$, $\delta_{\ell}=1$ producing $16$ combinations.  Each run lasted $T=500$ time units; final prevalences recorded in \texttt{sweep	extunderscore results.csv}.
\end{enumerate}
Code listings are provided in the repository files \texttt{simulation--11.py}, \texttt{simulation--12.py}, and \texttt{parameter	extunderscore sweep.py}.

% -------------------------------------------------------------
% Results
% -------------------------------------------------------------
\section{Results}
\subsection{Baseline Dynamics}
Figure~\ref{fig:timeSeries} plots compartment counts for $(\beta_{1},\beta_{2})=(0.10,0.12)$.  Meme~2 quickly gains prevalence, reaching a peak of $251$ infected nodes at $t\approx 439$, whereas meme~1 peaks at $73$ early and then vanishes ($I_{1}(T)=0$).  With equal rates $(0.11,0.11)$ both memes decline but meme~2 survives with $58$ infected nodes at $T$ whereas meme~1 again dies out, highlighting an asymmetry induced by network structure.

\begin{figure}[http]
  \centering
  \includegraphics[width=0.9\linewidth]{results-11.png}
  \caption{Baseline time series for $(\beta_{1},\beta_{2})=(0.10,0.12)$.  Solid lines: ensemble mean over $3$ replicates; shading: $\pm1$~SD.}
  \label{fig:timeSeries}
\end{figure}

\subsection{Dominance Landscape}
The heat map in Fig.~\ref{fig:heatmap} encodes which meme dominates at $T$ across the sweep.  Blue squares ($-1$) indicate meme~2 dominance, red ($+1$) meme~1, and white ($0$) coexistence.  Coexistence appears only marginally when $(\beta_{1},\beta_{2})\approx(0.12,0.12)$ and is sensitive to stochastic fluctuations.  The map corroborates the analytical criterion: exceeding the eigenvalue threshold is necessary but not sufficient---competitive exclusion emerges unless both memes enjoy comparable effective strength and orthogonal structural support.

\begin{figure}[http]
  \centering
  \includegraphics[width=0.9\linewidth]{heatmap_dom.png}
  \caption{Dominance landscape after $T=500$.  Colour code: $+1$ meme~1 dominates, $-1$ meme~2 dominates, $0$ coexistence.  Axes display $\beta_{1}$ and $\beta_{2}$.}
  \label{fig:heatmap}
\end{figure}

% -------------------------------------------------------------
% Discussion
% -------------------------------------------------------------
\section{Discussion}
The analytical framework clarifies that coexistence demands two simultaneous conditions: (i) each meme must independently surpass its {	extit{no--sharing}} threshold on a residual network weighted by the susceptibility left by the rival; (ii) the principal eigenvectors of the layers must differ sufficiently so that nodes critical for one meme are not heavily occupied by the other.  In the constructed BA--ER pair the eigenvector overlap is $\cos\theta=0.41$, enabling, in principle, coexistence.  Yet simulation shows that minor rate asymmetry tilts the balance, illustrating the fragility of coexistence predicted by the small basin of attraction around $(\hat{\mathbf{x}},\hat{\mathbf{y}})$.

From a design perspective, multiplex structures promoting coexistence share: (a) low interlayer degree correlation, (b) heterogeneous degree distributions supplying disjoint hub sets, and (c) moderate, not excessive, spectral radii so that $\tau_{\ell}$ remain close to criticality.  These findings generalise to marketing (sustaining product diversity) and to cyber--defence (ensuring a benign meme can displace a malicious one by targeting overlapping hubs).

Limitations include the mean--field closure and absence of temporal edge dynamics.  Extending to higher--order interactions and partial cross--immunity constitutes promising future work.

% -------------------------------------------------------------
% Conclusion
% -------------------------------------------------------------
\section{Conclusion}
We provided a unified analytical and simulation study of mutually exclusive SIS contagions on multiplex networks.  By linking coexistence to modified spectral thresholds and eigenvector misalignment, we explained why one meme typically dominates unless the network layers are structurally complementary.  FastGEMF simulations on synthetic BA--ER multiplexes validated the theory and exposed a narrow coexistence regime.  The framework and open--source code facilitate further exploration of competitive spreading in realistic, data--driven multiplex systems.

% -------------------------------------------------------------
% References
% -------------------------------------------------------------
\begin{thebibliography}{99}
\bibitem{Sahneh2014} F.~D.~Sahneh, C.~Scoglio, and P.~Van~Mieghem, "Modeling the spread of multiple concurrent contagions on networks," \emph{PLOS One}, vol.~9, no.~6, p.~e95669, 2014.
\bibitem{VanMieghem2011} P.~Van~Mieghem, J.~Omic, and R.~Kooij, "The N-intertwined SIS epidemic network model," \emph{Computing}, vol.~93, no.~2--4, pp.~147--169, 2011.
\bibitem{Diakonova2013} M.~Diakonova, M.~San~Miguel, and V.~M.~Eguíluz, "Contact-based social contagion in multiplex networks," \emph{Phys. Rev. E}, vol.~88, no.~5, p.~050801(R), 2013.
\bibitem{BiSIS2023} M.~Liu, F.~Pasqualetti, and J.~Liu, "Bi-SIS epidemics on graphs: quantitative analysis of coexistence," arXiv:2209.07304, 2023.
\bibitem{DiscreteBivirus2023} D.~J.~El-Hassan, R.~El-Desouki, and T.~Başar, "A discrete-time networked competitive bivirus SIS model," arXiv:2310.13853, 2023.
\end{thebibliography}

\end{document}