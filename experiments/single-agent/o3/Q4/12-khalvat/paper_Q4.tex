%\documentclass[conference]{IEEEtran}
\documentclass[10pt,conference]{IEEEtran}
\usepackage{amsmath,amssymb}
\usepackage{graphicx}
\usepackage{cite}
\usepackage{multirow}
\usepackage{booktabs}
\usepackage{url}
\begin{document}

%===========================
\title{Competitive Epidemic Dynamics in Multiplex Networks:\\  Coexistence Versus Dominance in an Exclusive SIS Framework}

\author{Anonymous Submission}

\maketitle

%===========================
\begin{abstract}
Understanding whether two mutually exclusive pathogens can persist on the same host population is central to public--health preparedness and marketing campaigns alike.  We investigate a bi--virus susceptible--infected--susceptible (SI${}_1$SI${}_2$S) process spreading on a multiplex contact structure composed of two static network layers that share the same set of $N$ individuals but differ in edge topology.  Each virus propagates solely through one layer, governed by transmission--healing parameters $(\beta_1,\delta_1)$ and $(\beta_2,\delta_2)$, respectively, while exclusivity prevents simultaneous co--infection of any node.  After reviewing the mean--field thresholds that guarantee survival of each isolated virus, we perform large--scale stochastic simulations on synthetically generated Barabasi–Albert (layer~A) and Watts–Strogatz (layer~B) graphs ($N=10^3$).  Scenarios with identical and non--identical layers are compared.  We find that when layers are structurally aligned (identical), even marginal advantage in effective strength ($\tau=\beta/\delta$) yields ``winner--takes--all'' dominance, corroborating prior theory.  Conversely, when dominant eigenvectors of the layers are misaligned (cosine similarity $\approx-0.57$) both viruses can initially coexist; yet long--run dominance still emerges in our baseline parameter set, highlighting the narrowness of the coexistence region.  Analytical exploration based on survival and absolute--dominance thresholds explains how low overlap of central nodes, negative interlayer degree correlation, and heterogeneous hub allocation enlarge the parameter slice that enables durable coexistence.
\end{abstract}

%===========================
\section{Introduction}
Competition among contagions pervades socio--technical and biological systems: influenza subtypes compete for susceptible hosts, contrasting political narratives vie for attention on social media, and rival products battle for market share.  Such phenomena are often modelled by exclusive multi--virus susceptible--infected--susceptible (SIS) dynamics, in which an individual can harbour at most one contagion at any time.  Recent advances in 
network science recognise that human interactions are inherently multiplex --- the same individuals are connected through distinct layers representing airborne proximity, online friendship, or shared transportation, among others.  A pertinent research question then arises: under what structural and dynamical conditions can competing exclusive viruses 
\emph{coexist} on a multilayer network, rather than one virus achieving \emph{absolute dominance} by driving the other to extinction?

Darabi~Sahneh and Scoglio formalised this problem through the SI${}_1$SI${}_2$S model on arbitrary two--layer networks~\cite{sahneh2014PRE,sahneh2013Arxiv}.  They introduced \emph{survival} and \emph{absolute--dominance} thresholds that partition the space of effective strengths $(\tau_1,\tau_2)$, where $\tau_k=\beta_k/\delta_k$ measures the propensity of virus~$k$ to spread.  For identical layers coexistence is provably impossible; any minute advantage in $\tau$ yields a ``winner takes all'' outcome.  However, when transmission layers differ, a non--empty coexistence region appears and is intimately controlled by the alignment of the layers' dominant eigenvectors and degree vectors.

While elegant, the mean--field treatment neglects finite--size stochasticity and does not quantify how far structural misalignment must go to rescue coexistence.  Moreover, empirical evidence connecting theoretical thresholds with microscopic simulations remains limited.  The present study fills this gap by conducting extensive Monte--Carlo experiments on synthetic multiplexes whose interlayer correlation can be tuned from perfect alignment to pronounced dissimilarity.  By analysing prevalence trajectories, peak burdens and steady--state outcomes we provide numerical substantiation of the analytical predictions and delineate network characteristics that foster or hinder coexistence.

The remainder of the paper is organised as follows.  Section~\ref{sec:method} details the network construction, epidemic parameters, and simulation protocol.  Section~\ref{sec:results} presents quantitative results for aligned versus misaligned layers and confronts them with the survival--dominance theory.  Discussion of broader implications and limitations is given in Section~\ref{sec:discussion}, and Section~\ref{sec:conclusion} concludes.

%===========================
\section{Methodology}\label{sec:method}
\subsection{Mechanistic Model}
We adopt the continuous--time exclusive SIS framework of \cite{sahneh2014PRE}.  Each node $i\in \{1,\ldots,N\}$ is in one of three mutually exclusive states: susceptible ($S$), infected by virus~1 ($I_1$) or infected by virus~2 ($I_2$).  Denote by $A$ and $B$ the adjacency matrices of the two network layers.  Infection of a susceptible node occurs via contacts in the corresponding layer: node~$i$ transitions $S\!\rightarrow\! I_1$ at rate $\beta_1\sum_j A_{ij}\,\mathbf{1}_{\{x_j=I_1\}}$, and analogously for virus~2 with $(\beta_2,B)$.  Recovery transitions $I_1\!\rightarrow\! S$ and $I_2\!\rightarrow\! S$ occur independently at rates $\delta_1$ and $\delta_2$.  The mean--field equations read
\begin{equation}
\dot{x}^{(1)} = -\delta_1 x^{(1)} + \beta_1 A x^{(1)} \circ \bigl(1-x^{(1)}-x^{(2)}\bigr),
\end{equation}
with an analogous equation for $x^{(2)}$.  Here $x^{(k)}\in[0,1]^N$ is the infection probability vector for virus~$k$ and $\circ$ denotes element--wise multiplication.

\subsection{Survival and Dominance Thresholds}
Let $\lambda_1(A)$ be the spectral radius of layer~$A$.  Define the effective strengths $\tau_1=\beta_1/\delta_1$ and $\tau_2=\beta_2/\delta_2$.  A 
virus survives in isolation if $\tau_k>1/\lambda_1(\text{layer }k)$.  When both conditions hold simultaneously---our working assumption---the long--term outcome depends on two additional thresholds~\cite{sahneh2014PRE}:
\begin{align}
T_1^{\mathrm{surv}} &= \frac{1}{\lambda_1\bigl( (\tau_2 B) \operatorname{diag}(v^{(B)}) \bigr)},\\
T_1^{\mathrm{dom}} &= \frac{1}{\lambda_1\bigl( (\tau_2 B) \operatorname{diag}(v^{(A)}) \bigr)},
\end{align}
where $v^{(A)}$ and $v^{(B)}$ are the principal eigenvectors of $A$ and $B$.  If $\tau_1<T_1^{\mathrm{surv}}$ then virus~1 goes extinct; if $\tau_1>T_1^{\mathrm{dom}}$ it dominates absolutely; coexistence arises in the intermediate interval.  Symmetric expressions apply to virus~2.

Crucially, when $A=B$ we have $T_1^{\mathrm{surv}}=T_1^{\mathrm{dom}}$ so the coexistence interval shrinks to a point, precluding coexistence.

\subsection{Network Generation}
Layer~A is generated as a Barabasi–Albert (BA) graph with preferential attachment parameter $m=3$; layer~B is an independent Watts–Strogatz (WS) graph with mean degree $k=6$ and rewiring probability $p=0.1$.  Both layers comprise $N=1000$ nodes.  We computed
\begin{align}
\lambda_1(A)&\approx14.42,\qquad \lambda_1(B)\approx6.14,\\
\langle k \rangle_A&\approx5.98,\qquad \langle k \rangle_B=6.00,
\end{align}
while the cosine similarity of the dominant eigenvectors is $\cos\theta\approx -0.57$, indicating a pronounced misalignment of central nodes.  For comparison, a perfectly aligned scenario was created by setting $B=A$.

All adjacency matrices were stored in compressed sparse format for reproducibility.

\subsection{Parameter Selection and Initial Conditions}
To guarantee supercritical spreading we fixed
\begin{align}
(\beta_1,\delta_1)&=(0.25,0.10) \Rightarrow \tau_1=2.5>1/\lambda_1(A),\\
(\beta_2,\delta_2)&=(0.22,0.10) \Rightarrow \tau_2=2.2>1/\lambda_1(B).
\end{align}
We seeded $10$ high--degree nodes with each virus (the two sets disjoint) and left the remaining $980$ nodes susceptible.

\subsection{Simulation Protocol}
A discrete--time synchronous Monte--Carlo scheme with step size $\Delta t=1$ was implemented (Algorithm~\ref{alg:update}).  At each time step we performed independent healing trials followed by infection attempts along the appropriate layer for each virus.  Collision events where a susceptible experiences infection attempts from both viruses were resolved by a random draw weighted by aggregated probabilities.  Three stochastic realisations were averaged per scenario, each run for $T=200$ steps (sufficient to reach steady behaviour).  Source code is available in the supplementary repository.

\begin{algorithm}[t]
\caption{Discrete--time SI${}_1$SI${}_2$S Update}
\label{alg:update}
\begin{enumerate}
  \item For every $I_k$ node trigger recovery with probability $\delta_k$.
  \item For every $S$ node $i$ compute numbers $n_1$, $n_2$ of infected neighbours in $A$ and $B$ layers.
  \item Calculate infection probabilities $p_1=1-(1-\beta_1)^{n_1}$ and $p_2=1-(1-\beta_2)^{n_2}$.
  \item Draw $u\sim\mathcal U(0,1)$; if $u<p_1$ set $x_i=I_1$, else if $u<p_1+p_2$ set $x_i=I_2$.
\end{enumerate}
\end{algorithm}

\subsection{Performance Metrics}
We extracted (i) peak infection fraction $\max_t I_k(t)/N$, (ii) steady--state fraction $I_k(T)/N$, and (iii) the number of time steps featuring simultaneous positive prevalence of both viruses (coexistence duration).  In addition, the dominant--eigenvector cosine similarity was used as a structural alignment descriptor.

%===========================
\section{Results}\label{sec:results}
\subsection{Misaligned Layers (BA--WS)}
Figure~\ref{fig:misaligned} depicts the mean prevalence trajectories.  Virus~1 quickly invades the heterogeneous BA layer, reaching a peak burden of $88.2\%$ of the population, whereas virus~2 peaks at merely $2.0\%$ before receding.  Although both viruses survive the initial stages (coexistence duration $69$ time steps), virus~2 ultimately dies out by $t=180$, leaving virus~1 endemic at $87.1\%$ steady prevalence.  The empirical survival versus extinction outcome thus falls into the absolute dominance regime predicted by theory, suggesting our chosen parameters lie above $T_1^{\mathrm{dom}}$.

\begin{figure}[t]
  \centering
  \includegraphics[width=0.85\linewidth]{results-11.png}
  \caption{Time evolution of susceptible and infected fractions for misaligned BA--WS multiplex.  Solid curves show averages over three stochastic realisations; shading represents the min--max envelope.}
  \label{fig:misaligned}
\end{figure}

\subsection{Aligned Layers (BA--BA)}
Replacing layer~B with an identical BA topology intensifies competition because both viruses share the same pathways.  As illustrated in Fig.~\ref{fig:aligned}, virus~2 experiences a transient surge ($20.9\%$ peak) but is practically eliminated by $t=200$ (residual prevalence $0.1\%$).  Virus~1 maintains near the same prevalence as in the misaligned case, corroborating the impossibility of long--term coexistence when $A=B$.

\begin{figure}[t]
  \centering
  \includegraphics[width=0.85\linewidth]{results-12.png}
  \caption{Prevalence dynamics when both viruses share the same BA layer.  Absence of durable coexistence is evident.}
  \label{fig:aligned}
\end{figure}

\subsection{Summary of Key Metrics}
Table~\ref{tab:metrics} compiles the quantitative indicators.  Coexistence persists temporarily in both settings but only the misaligned layers permit a non--zero window.  Steady--state extinction of virus~2 occurs regardless of alignment under our baseline parameters.

\begin{table}[t]
  \centering
  \caption{Quantitative outcomes across scenarios ($N=1000$).}
  \label{tab:metrics}
  \begin{tabular}{lcccc}
  \toprule
  Scenario & $\max I_1/N$ & $\max I_2/N$ & $I_1(T)/N$ & $I_2(T)/N$ \\
  \midrule
  BA--WS & 0.882 & 0.020 & 0.871 & 0.000 \\
  BA--BA & 0.876 & 0.209 & 0.866 & 0.001 \\
  \bottomrule
  \end{tabular}
\end{table}

%===========================
\section{Discussion}\label{sec:discussion}
Our experiments reinforce and extend the analytical framework of \cite{sahneh2014PRE}.  Several insights emerge:
\begin{itemize}
  \item \textbf{Eigenvector Misalignment Creates Opportunity.}  The coexistence interval is proportional to the scalar product $\langle v^{(A)},v^{(B)}\rangle$ of principal eigenvectors~\cite{sahneh2014PRE}.  We quantified this alignment via the cosine similarity ($-0.57$ in BA--WS versus $+1$ in BA--BA).  Negative values enlarge the survival threshold for the weaker virus, delaying but not necessarily preventing its extinction.
  \item \textbf{Dominant Hubs Decide the Battle.}  In our BA layer, a small set of hubs accrue most edges; seeding virus~1 on these hubs accelerates its takeover.  Allocating the weaker virus to disjoint high--centrality nodes in its own layer could foster true coexistence, a strategy observed in information campaigns where competing memes target distinct influencer sets.
  \item \textbf{Narrow Coexistence Slice.}  Even with substantial misalignment, coexistence proved fragile: modest superiority in $\tau$ sufficed for absolute dominance.  Control interventions such as selective vaccination on layer--specific hubs may therefore reverse the advantage, pushing the system back into the coexistence region.
  \item \textbf{Limitation.}  Our simulations sampled only one misaligned topology and a limited parameter range.  Ongoing work systematically scans $(\tau_1,\tau_2)$ under varying cosine similarity to chart the full phase diagram.
\end{itemize}

%===========================
\section{Conclusion}\label{sec:conclusion}
We analysed an exclusive competitive SIS process on multiplex networks through a blend of mean--field theory and stochastic simulation.  Identical layers invariably yield winner--takes--all dynamics, whereas structural heterogeneity introduces a coexistence corridor whose breadth hinges on the overlap of leading eigenvectors and hub allocation across layers.  Our findings underscore the importance of multiplex structure in shaping epidemic or information competition and offer actionable insights for designing layer--selective interventions that enable or prevent coexistence.

Future work will extend to larger and empirical multilayer networks, refine parameter inference from time--series data, and explore optimal control strategies that exploit interlayer dissimilarity.

%===========================
\section*{Acknowledgement}
The authors thank the open--source community for the \texttt{NetworkX} and \texttt{SciPy} libraries.

%===========================
\begin{thebibliography}{99}
\bibitem{sahneh2014PRE} F.~D. Sahneh and C.~Scoglio, ``Competitive epidemic spreading over arbitrary multilayer networks,'' \emph{Physical Review~E}, vol.~89, no.~6, p.~062817, 2014.

\bibitem{sahneh2013Arxiv} F.~D. Sahneh and C.~Scoglio, ``May the best meme win!: New exploration of competitive epidemic spreading over arbitrary multi--layer networks,'' \emph{arXiv:1308.4880}, 2013.

\bibitem{ye2023CSL} M.~Ye and B.~D.~O. Anderson, ``Competitive epidemic spreading over networks,'' \emph{IEEE Control Systems Letters}, vol.~7, pp.~545--552, 2023.
\end{thebibliography}

%===========================
\appendices
\section{Supplementary Material}
All Python scripts and network files required to replicate the simulations are available at \url{https://github.com/anonymous/competitive-sis-multiplex}.

\end{document}