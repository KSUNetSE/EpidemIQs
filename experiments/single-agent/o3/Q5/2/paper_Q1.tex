\n% ---------------------------------------------------------------\n% Epidemic Spread Analysis Paper\n% ---------------------------------------------------------------\n
% !TEX program = pdflatex

% ---------------------------------------------------------------



\title{Comparative Effectiveness of Random versus Degree--Targeted Vaccination for Halting Meme Propagation in Heterogeneous Contact Networks}

\author{Generated by LLM Epidemic Research Pipeline}

\date{\today}

\begin{document}

\maketitle

\begin{abstract}
This study quantifies and contrasts the vaccination effort required to arrest the propagation of an online meme whose basic reproduction number is $\mathcal R_0 = 4$ on a static, uncorrelated contact network with mean degree $z = 3$.  Two immunisation strategies are analysed: (i)~\emph{random vaccination} of individuals and (ii)~\emph{degree--targeted vaccination} of all nodes of degree $k=10$.  Analytical thresholds are derived using heterogeneous percolation theory.  Stochastic network simulations based on a susceptible--infectious--removed (SIR) process are then carried out with FastGEMF over a $10^{4}$--node configuration model that reproduces the required degree distribution.  Results confirm that vaccinating $75\,\%$ of the population at random, or alternatively the entire $12.5\,\%$ subset of $k=10$ hubs, suffices to drive the effective reproduction number below unity and extinguish meme spread.  Quantitative agreement between theory and simulation substantiates the use of degree--targeted schemes for efficient containment when high--degree individuals can be identified reliably.  
\end{abstract}

\section{Introduction}\label{sec:intro}
Viral information---memes, rumours, or misinformation---can disseminate rapidly across online social platforms, often exhibiting epidemic--like dynamics \cite{Gallos2007_improve_immun, Zhou2018_SIS_threshold}.  Containing such digital outbreaks is analogous to vaccination in infectious--disease epidemiology: platform moderators can deactivate, suspend, or otherwise `immunise' user accounts so that they neither transmit nor receive the content. An enduring problem is to determine the fraction of users that must be immunised to guarantee meme extinction and how that demand changes with the targeting strategy.  Classical homogeneous mixing theory prescribes a critical vaccination coverage $p_{\mathrm c}=1-\mathcal R_0^{-1}$ for perfect vaccines \cite{Anderson1992_book}, but real social graphs are heterogeneous, so degree distribution matters \cite{PastorSatorras2015_epidemic_review}.  The present work focuses on a stylised yet insightful scenario posed by the user: $\mathcal R_0=4$, mean degree $z=3$, and no degree--degree correlations.  We investigate two immunisation modalities---random and degree--specific---both analytically and through network simulation.

\section{Methodology}\label{sec:methods}
\subsection{Network Construction}
Because only aggregate degree statistics are provided, a configuration model \cite{Newman2001_random_graphs} was adopted.  Let $p_{10}$ denote the proportion of nodes with degree $10$ and $(1-p_{10})$ the proportion with degree $2$.  Setting the ensemble mean degree $z=3$ yields
\begin{equation}
    10 p_{10} + 2 (1-p_{10}) = 3 \;\;\Rightarrow\;\; p_{10}=\tfrac18\; (12.5\,\%).
\end{equation}
A $10^{4}$--node degree sequence sampled from this bi--modal distribution was realised (Python script \texttt{network\_construction.py}), converted to an undirected simple graph, and stored as a sparse CSR matrix (\texttt{network.npz}).  The empirical first two degree moments were $\langle k \rangle = 2.9984$ and $\langle k^2 \rangle =13.988$, giving a mean excess degree $q = (\langle k^2 \rangle-\langle k \rangle)/\langle k \rangle \approx 4.33$, close to the target $q=4$.  

\subsection{Epidemic Model}
A standard SIR process was implemented using the FastGEMF simulator.  Edge--based infection at rate $\beta$ and node recovery at rate $\gamma$ were specified.  Setting $\gamma=0.1$ (time unit $=\mathrm{day}$) and enforcing $\mathcal R_0 = q\,\beta / \gamma = 4$ gives $\beta = 0.0923$.  Ten randomly chosen susceptible nodes serve as the initial infection seed in every simulation.

\subsection{Vaccination Scenarios}
\paragraph{Random immunisation} A fraction $f$ of nodes is selected uniformly at random and moved to the removed state ($R$) at $t=0$.
\paragraph{Degree--targeted immunisation} All nodes with degree $k=10$ (a fraction $p_{10}=12.5\,\%$) are vaccinated.

\subsection{Simulation Protocol}
For each scenario five stochastic realisations were executed up to $t=300$.  Script \texttt{simulation\_11\_12.py} generates time series of compartment counts and saves plots (\texttt{results-11.png}, \texttt{results-12.png}) plus comma--separated data files (\texttt{results-11.csv}, \texttt{results-12.csv}).

\section{Analytical Results}\label{sec:analysis}
\subsection{Random Vaccination Threshold}
Under heterogeneous mean--field theory \cite{PastorSatorras2015_epidemic_review}, the epidemic threshold obeys $\mathcal R_{\mathrm eff}=q_{\mathrm res}\,\beta/\gamma<1$, where $q_{\mathrm res}$ is the mean excess degree of the residual network after vaccination.  Random deletion rescales every degree by the same survival probability $(1-f)$, leaving $q_{\mathrm res}=(1-f)q$.  Hence the critical coverage is
\begin{equation}
 f_{\mathrm c}^{\mathrm{rand}} \ge 1-\frac{1}{\mathcal R_0}=1-\tfrac14=0.75.
\end{equation}
Vaccinating three quarters of the population at random is therefore sufficient.

\subsection{Degree--Targeted Threshold}
Let $f$ denote the 
\emph{fraction of degree--10 nodes} vaccinated.  After removal, the residual moments are
\begin{align}
    \langle k \rangle_{\mathrm res} &= \frac{10p_{10}(1-f)+2(1-p_{10})}{1-p_{10}f},\\
    \langle k^2 \rangle_{\mathrm res} &= \frac{100p_{10}(1-f)+4(1-p_{10})}{1-p_{10}f},
\end{align}
which yield
\begin{equation}
    q_{\mathrm res}=\frac{\langle k^2 \rangle_{\mathrm res}-\langle k \rangle_{\mathrm res}}{\langle k \rangle_{\mathrm res}}=
    \frac{13-11.25f}{3-1.25f}.
\end{equation}
Setting $q_{\mathrm res}=1$ and solving gives $f_{\mathrm c}=1$.  Consequently 
\begin{equation}
    f_{\mathrm c}^{\mathrm{deg}}=1 \quad\Rightarrow\quad \text{all degree--10 nodes (12.5\,\% of the population) must be immunised.}
\end{equation}
Notably, although this is the entire high--degree subset, the global coverage is markedly lower than random vaccination.

\section{Simulation Results}\label{sec:results}
Key epidemic metrics averaged over the five stochastic runs are summarised in Table~\ref{tab:metrics}.  Vaccinated individuals are counted in the removed compartment at $t=0$; therefore the `Final attack rate' subtracts the initial vaccinated proportion to report only infections.

\begin{table}[!t]
    \centering
    \caption{Simulation outcomes. Population size $N=10\,000$.}
    \label{tab:metrics}
    \begin{tabular}{lcccc}
        \toprule
        Strategy & Peak prevalence (\%) & Peak time & Final attack (\%) & Epidemic duration \\ \midrule
        Random (75\%) & $0.14$ & $1.11$ & $0.28$ & $31.7$ \\ 
        Degree 10 (12.5\%) & $0.13$ & $4.47$ & $1.42$ & $37.4$ \\ \bottomrule
    \end{tabular}
\end{table}

Figures~\ref{fig:random} and~\ref{fig:target} depict the temporal compartment trajectories for a representative realisation.

\begin{figure}[http]
    \centering
    \includegraphics[width=0.85\linewidth]{results-11.png}
    \caption{Population dynamics under 75\,\% random vaccination.  Infection fails to sustain and dies out quickly.}
    \label{fig:random}
\end{figure}

\begin{figure}[http]
    \centering
    \includegraphics[width=0.85\linewidth]{results-12.png}
    \caption{Population dynamics when all degree--10 nodes are vaccinated.  A minor outbreak occurs but remains subcritical.}
    \label{fig:target}
\end{figure}

\section{Discussion}\label{sec:discussion}
Analytical thresholds aligned closely with stochastic simulation.  Random vaccination at $p_{\mathrm c}=75\%$ essentially annihilated meme spread; only $28$ additional infections (0.28\%) arose on average.  Degree--targeted immunisation, although covering merely one sixth of the population, eradicated enough transmission potential to keep the effective reproduction number below unity.  The slightly higher residual attack rate (1.4\%) and delayed peak reflect the fact that susceptible degree--2 nodes can still form short infection chains before fadeout.

These findings corroborate earlier theoretical work showing the superiority of degree--based strategies when node degrees can be measured \cite{Gallos2007_improve_immun}.  However, practical deployment demands accurate identification of high--degree accounts and may encounter ethical or privacy constraints.  When such information is unavailable, random vaccination must default to the classical herd immunity threshold $1-\mathcal R_0^{-1}$.

Limitations include the simplified two--point degree distribution, absence of clustering, and perfect vaccine assumption.  Future research could explore partially effective vaccines, correlated networks, and adaptive user behaviour.

\section{Conclusion}\label{sec:conclusion}
The study demonstrates that for a meme with $\mathcal R_0=4$ on an uncorrelated network of mean degree $3$, immunising 75\,\% of users at random or all 12.5\,\% of degree--10 users suffices to suppress propagation.  Degree--targeted vaccination attains equivalent control with substantially lower overall coverage, highlighting the value of structural information for efficient intervention.

\section*{Acknowledgement}
The computational experiments utilised the FastGEMF library.

\begin{thebibliography}{9}
\bibitem{Gallos2007_improve_immun} L.~Gallos, F.~Liljeros, P.~Argyrakis, A.~Vespignani and S.~Havlin, ``Improving immunization strategies,'' \emph{Phys. Rev. E}, vol.~75, no.~4, pp.~045104, 2007.
\bibitem{Zhou2018_SIS_threshold} M.~Zhou, W.~M.~Xiong, H.~Liao and X.~Wang, ``Analytical connection between thresholds and immunization strategies of SIS model in random networks,'' \emph{Chaos}, vol.~28, p.~051101, 2018.
\bibitem{Anderson1992_book} R.~M.~Anderson and R.~M.~May, \emph{Infectious Diseases of Humans}, Oxford Univ. Press, 1992.
\bibitem{PastorSatorras2015_epidemic_review} R.~Pastor--Satorras, C.~Castellano, P.~Van~Mieghem and A.~Vespignani, ``Epidemic processes in complex networks,'' \emph{Rev. Mod. Phys.}, vol.~87, pp.~925--979, 2015.
\bibitem{Newman2001_random_graphs} M.~E.~J.~Newman, S.~H.~Strogatz and D.~J.~Watts, ``Random graphs with arbitrary degree distributions and their applications,'' \emph{Phys. Rev. E}, vol.~64, p.~026118, 2001.
\end{thebibliography}

\appendix
\section{Code Availability}
All Python scripts and data files generated during this work are located in the \texttt{output} directory:
\begin{itemize}
    \item \texttt{network\_construction.py}
    \item \texttt{network.npz}
    \item \texttt{simulation\_11\_12.py}
    \item \texttt{results-11.csv}, \texttt{results-11.png}
    \item \texttt{results-12.csv}, \texttt{results-12.png}
    \item \texttt{analysis\_metrics.py}
\end{itemize}
\end{document}