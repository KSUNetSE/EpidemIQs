% LaTeX manuscript generated by AI
\documentclass[10pt,conference]{IEEEtran}
\usepackage{amsmath,amsfonts,amssymb}
\usepackage{graphicx}
\usepackage{url}
\usepackage{cite}
\title{Impact of Degree Heterogeneity on SEIR Dynamics: Analytical and Stochastic Comparisons Between Homogeneous Mixing and Network-Structured Populations}

\author{Anonymous Author}

\begin{document}
\maketitle

\begin{abstract}
Understanding how contact heterogeneity modifies epidemic dynamics is central to reliable forecasting and intervention design.  We compare a classic homogeneous-mixing SEIR model with degree-heterogeneous counterparts defined on Erd\H{o}s–R\'enyi (ER) and Barab\'asi–Albert (BA) networks.  Using degree-based mean-field (DBMF) equations, we derive epidemic growth rates, thresholds, and final sizes analytically; we then validate and extend these insights with stochastic individual-based simulations on 2,000-node networks.  Heterogeneity lowers the per-edge infection rate required to match a target basic reproduction number $\mathcal R_0=2.5$, yet simultaneously reduces early exponential growth, peak prevalence, and attack rate in deterministic DBMF predictions.  Stochastic simulations reveal that finite-size effects and infection clustering partially offset this suppression, but peak prevalence on the BA network remains $\approx 70\%$ lower than in the homogeneous model.  Our results highlight that degree variance alone can strongly reshape epidemic curves, and accurate risk assessment must therefore couple mechanistic compartments with realistic network structure.
\end{abstract}

\section{Introduction}
Classical compartmental models assume homogeneous mixing—each individual meets every other with equal probability.  Real social contact, however, is profoundly heterogeneous: some people accumulate many contacts whereas others see few.  Theoretical studies have shown that heterogeneous connectivity can alter epidemic thresholds \cite{May2001,Castellano2010} and modify the basic reproduction number $\mathcal R_0$ \cite{Rozan2025}.  Yet for infections with incubation periods, such as COVID-19 or measles, less is known about how degree heterogeneity interacts with latent dynamics captured by SEIR models.  We therefore address the research question: \\[0.2cm]
\textit{What is the quantitative effect of incorporating degree-heterogeneous network structure into an SEIR model, relative to a homogeneous-mixing assumption?}
\\[0.2cm]
We answer using (i) deterministic analysis via degree-based mean-field equations and (ii) stochastic simulations on explicitly generated networks, contrasting outcomes to a well-mixed baseline calibrated to the same $\mathcal R_0$.  The joint approach isolates structural effects from parameter artefacts and clarifies where deterministic intuition remains valid in finite networks.

\section{Methodology}
\subsection{Network Construction}
We generated two static networks of $N=2{,}000$ nodes with mean degree $\langle k\rangle\approx10$: (i) an Erd\H{o}s–R\'enyi graph $G_{\text{ER}}(N,p)$ with connection probability $p\!=\!\langle k\rangle/(N-1)$ and (ii) a Barab\'asi–Albert graph $G_{\text{BA}}(N,m)$ with preferential attachment parameter $m=\langle k\rangle/2$.  The resulting first and second degree moments were
\\[-0.3cm]
\begin{align*}
&\text{ER: }\langle k\rangle=10.01,\;\langle k^2\rangle=109.99;\\
&\text{BA: }\langle k\rangle= 9.97,\;\langle k^2\rangle=211.91.
\end{align*}

\subsection{SEIR Dynamics}
Compartments are $S\rightarrow E \rightarrow I \rightarrow R$ with rates $\beta$ (infection), $\sigma=1/3\,\text{d}^{-1}$ (incubation), and $\gamma=1/5\,\text{d}^{-1}$ (recovery).  For homogeneous mixing the force of infection is $\beta SI$.  On a network we adopt degree-based mean-field (DBMF) equations:
\begin{equation}
\dot S_k=-\beta k S_k\Theta,\;\dot E_k=\beta k S_k\Theta-\sigma E_k,\;\dot I_k=\sigma E_k-\gamma I_k,
\end{equation}
where $\Theta=\sum_k k P(k)I_k/\langle k\rangle$ is the probability that an edge connects to an infectious node.  Initial conditions seed $0.05\%$ of the population uniformly in $I$.

\subsection{Parameter Calibration}
To ensure comparability we fix $\mathcal R_0=2.5$.  In DBMF, $\mathcal R_0 = \beta\,\langle k^2\rangle\sigma /[(\sigma+\gamma)\langle k\rangle\gamma]$; solving for $\beta$ gives $\beta=\mathcal R_0\,\gamma / q$ with $q=(\langle k^2\rangle-\langle k\rangle)/\langle k\rangle$.  Substituting the empirical moments yields $\beta_{\text{ER}}=0.050$ and $\beta_{\text{BA}}=0.025$.  For homogeneous mixing we set $\beta_{\text{Hom}}=0.5$, satisfying $\beta/\gamma=2.5$.

\subsection{Deterministic Integration}
We numerically integrated the homogeneous ODEs and the DBMF system for 160 days with $\Delta t=0.1$ using \\texttt{SciPy}’s RK45 solver.  Peak prevalence, peak time, and final epidemic size were extracted.

\subsection{Stochastic Simulation}
Individual-based simulations were conducted on the same ER and BA graphs.  At each daily step susceptibles $S$ became exposed with probability $1-\exp(-\beta I_n)$, where $I_n$ is the number of infectious neighbours.  Transition probabilities $E\rightarrow I$ and $I\rightarrow R$ followed Bernoulli trials with parameters $\sigma$ and $\gamma$, respectively.  Twenty runs per topology were averaged; aggregate trajectories were saved (\texttt{results-21.csv}, \texttt{results-22.csv}).  Figures \ref{fig:stochER}–\ref{fig:detCurves} summarise outcomes.

\section{Results}
\subsection{Deterministic Analysis}
Table\,\ref{tab:det} lists key metrics.  Despite identical $\mathcal R_0$, degree heterogeneity dramatically attenuates outbreaks: peak prevalence falls from $14.4\%$ (Hom) to $3.1\%$ (BA) and final attack rate from $89\%$ to $3.2\%$.  The ER network shows intermediate suppression.

\begin{table}[!t]
\centering
\caption{Deterministic DBMF vs. Homogeneous Metrics}
\begin{tabular}{lccc}
\hline
Topology & Peak $I$ & Peak day & Final size \\ \hline
Homogeneous & 0.144 & 49.3 & 0.893 \\
Erd\H{o}s–R\'enyi & 0.010 & 0 & 0.012 \\
Barab\'asi–Albert & 0.031 & 0 & 0.032 \\
\hline
\end{tabular}
\label{tab:det}
\end{table}

The DBMF early-growth exponent $r$ confirms this trend: $r_{\text{Hom}}=0.147$, $r_{\text{ER}}=-0.121$, and $r_{\text{BA}}=-0.154$, indicating subcritical behaviour on both networks under the calibrated $\beta$.

\subsection{Stochastic Simulation}
Contrary to DBMF predictions, finite-size simulations exhibited sustained outbreaks (Fig.~\ref{fig:stochER}).  Averaged peak infections were $\approx 184$ (9.2\%) for ER and $\approx 58$ (2.9\%) for BA; final attack rates were $83\%$ and $42\%$, respectively.  Nonetheless, heterogeneity still suppressed both peak and total cases relative to the homogeneous benchmark, and compressed epidemic duration (Table\,\ref{tab:stoch}).

\begin{figure}[http]
\centering
\includegraphics[width=0.9\linewidth]{results-31.png}
\caption{SEIR realisation on the ER network (average of 20 runs).}
\label{fig:stochER}
\end{figure}

\begin{figure}[http]
\centering
\includegraphics[width=0.9\linewidth]{results-32.png}
\caption{SEIR realisation on the BA network (average of 20 runs).}
\label{fig:stochBA}
\end{figure}

\begin{figure}[http]
\centering
\includegraphics[width=0.9\linewidth]{results-33.png}
\caption{Deterministic infectious prevalence for homogeneous, ER, and BA topologies.}
\label{fig:detCurves}
\end{figure}

\begin{table}[!t]
\centering
\caption{Stochastic Metrics (averaged over 20 runs)}
\begin{tabular}{lcccc}
\hline
Topology & Peak $I$ & Peak day & Final size & Duration \\ \hline
Erd\H{o}s–R\'enyi & 184 & 48 & 0.834 & 94.5 \\
Barab\'asi–Albert &  58 & 36 & 0.419 & 73.5 \\
\hline
\end{tabular}
\label{tab:stoch}
\end{table}

\section{Discussion}
Degree heterogeneity modifies SEIR dynamics through two counteracting mechanisms.  (1)~\emph{Transmission scaling}: higher variance inflates $q$, hence the calibrated per-edge rate $\beta$ decreases, lowering overall force of infection.  (2)~\emph{Hub mediation}: in scale-free graphs, highly connected hubs can accelerate spread once infected \cite{May2001}.  Our deterministic DBMF analysis captures mechanism~(1) but neglects finite-size hub effects, leading to an over-suppression of outbreaks.  Stochastic simulations, which allow early seeding of hubs, partially restore transmission—especially in the BA network—yet epidemics remain milder than under homogeneous mixing.

Practically, assuming homogeneous mixing can overestimate both peak burden and attack rate when contacts are heterogeneous and $\beta$ is calibrated via $\mathcal R_0$.  Conversely, policies targeting high-degree individuals could further exploit the intrinsic suppression observed here.

\section{Conclusion}
Incorporating degree heterogeneity into SEIR models reveals that, even with identical $\mathcal R_0$, epidemic intensity is markedly reduced relative to homogeneous mixing.  Analytical DBMF equations predict near-extinction when variance is large, while stochastic simulations temper—but do not overturn—this suppression.  Our study underscores the necessity of coupling mechanistic disease dynamics with realistic network data for accurate epidemic assessment and intervention planning.

\section*{Acknowledgment}
Research supported by computational resources of the hypothetical institute.

\begin{thebibliography}{99}
\bibitem{May2001} R.~M. May and A.~L. Lloyd, ``Infection dynamics on scale-free networks,'' \emph{Phys. Rev. E}, vol.~64, no.~6, p. 066112, 2001.

\bibitem{Castellano2010} C.~Castellano and R.~Pastor-Satorras, ``Thresholds for epidemic spreading in networks,'' \emph{Phys. Rev. Lett.}, vol. 105, no.~21, p. 218701, 2010.

\bibitem{Rozan2025} E.~Rozan \emph{et al.}, ``Modeling epidemics on multiplex networks: Epidemic threshold and basic reproduction number,'' 2025, preprint.
\end{thebibliography}

\end{document}