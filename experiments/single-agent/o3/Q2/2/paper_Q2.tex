\documentclass[10pt,conference]{IEEEtran}
\usepackage{graphicx}
\usepackage{amsmath,amsfonts}
\usepackage{booktabs}
\begin{document}
\title{When Does an Epidemic End?  Analytic and Network--Based Evidence on the Roles of Infective Decline and Susceptible Depletion}
\author{Anonymous Author}
\maketitle

\begin{abstract}
Understanding why the chain of transmission in an epidemic eventually breaks is fundamental to public health planning.  Classical infectious–disease theory proposes two non-exclusive mechanisms:  (i) the number of infectious individuals falls to zero because transmission is intrinsically sub-critical, and (ii) the susceptible population is sufficiently depleted that the effective reproduction number drops below unity even though transmission is super-critical at the outset.  In this paper we address the question quantitatively using both analytical derivations from the susceptible–infectious–removed (SIR) model and stochastic simulations over a realistic contact network comprising  $10^{3}$ individuals.  We confirm that sub-critical epidemics ($\mathcal R_{0}<1$) terminate through a rapid decline in infectives with essentially no loss of susceptibles, whereas super-critical epidemics ($\mathcal R_{0}>1$) terminate when the susceptible fraction crosses the herd–immunity threshold $S_{\mathrm c}=1/\mathcal R_{0}$, leaving a non-zero susceptible residue.  The simulated trajectories for $\mathcal R_{0}=0.8$ and $\mathcal R_{0}=2.5$ reproduce the analytic expectations, demonstrating consistency between mean–field theory and network-explicit stochastic dynamics.
\end{abstract}

\section{Introduction}
Why do epidemics stop?  Two intuitive answers are commonly offered:  either there are no infective individuals left to transmit the pathogen, or there are no longer any susceptible individuals left to receive it.  Although both statements appear reasonable, they invoke distinct epidemiological mechanisms whose relative importance has practical implications for control design and for assessing the risk of re-emergence.  The classical work of \cite{Kermack1927} showed that an epidemic in a homogeneous population governed by the SIR model does not, in general, require total susceptible exhaustion to end; instead, the epidemic ceases once the susceptible proportion falls below $1/\mathcal R_{0}$.  Subsequent texts such as \cite{Anderson1991} extended the result to structured populations and emphasised that when $\mathcal R_{0}<1$ an outbreak cannot take hold regardless of the susceptible pool.

Despite these well-known results, confusion persists in public discourse, especially after the COVID-19 pandemic, where terms like ``running out of people to infect'' are routinely used.  This study revisits the problem from two complementary angles:  (i) an analytical treatment of the deterministic SIR equations, highlighting conditions under which each mechanism dominates, and (ii) stochastic simulation on an explicit contact network to capture finite-size effects and degree heterogeneity.  We seek to answer:  
\begin{enumerate}
  \item Under what parameter regimes does an epidemic terminate purely because infectious individuals vanish while most susceptibles remain?
  \item When the basic reproduction number exceeds unity, how many susceptibles are typically left at fade-out, and does the final size predicted by the deterministic model match network-based simulations?
\end{enumerate}

\section{Methodology}
\subsection{Contact Network Construction}
A static Erd\H{o}s–R\'enyi (ER) network $G(N,\,p)$ with $N=1000$ nodes was generated, targeting an average degree $\langle k\rangle\approx10$.  The link probability was therefore $p=\langle k\rangle/(N-1)\approx0.010$.  The network was stored as a sparse adjacency matrix and characterisation yielded $\langle k\rangle=9.98$ and a second degree moment $\langle k^{2}\rangle=108.92$, producing a mean excess degree $q=(\langle k^{2}\rangle-\langle k\rangle)/\langle k\rangle\approx9.92$.

\subsection{Epidemic Model}
We employed the standard SIR compartmentalisation with state set $\{S,I,R\}$.  Transitions are:
\begin{align}
S + I &\xrightarrow{\beta} 2I,\\
I &\xrightarrow{\gamma} R.
\end{align}
The recovery rate was fixed at $\gamma=1/7\;\mathrm d^{-1}$ (mean infectious period of seven days).  Edge-based infection occurred at per-contact rate $\beta$ chosen to realise a prespecified $\mathcal R_{0}$ through $\beta = \mathcal R_{0}\,\gamma/q$, which links network heterogeneity to classical $\mathcal R_{0}$ definitions \cite{Miller2020}.

\subsection{Analytical Framework}
For a well-mixed population the deterministic SIR ordinary differential equations read
\begin{align}
\dot S &= -\beta S I,\\
\dot I &= \beta S I - \gamma I,\\
\dot R &= \gamma I.
\end{align}
Let $s=S/N$ and $i=I/N$.  The epidemic grows when $\mathrm d i/\mathrm dt>0$, i.e. $s>1/\mathcal R_{0}$.  Consequently, for $\mathcal R_{0}<1$ the condition is never satisfied and $i(t)$ decreases monotonically.  For $\mathcal R_{0}>1$ growth occurs until $s$ crosses the critical threshold, after which $i(t)$ decays even though $s>0$.

The final size relation \cite{Kermack1927} is
\begin{equation}
R_{\infty} = N\Bigl[1-\exp\bigl(-\mathcal R_{0} R_{\infty}/N\bigr)\Bigr],
\end{equation}
which can be solved numerically to yield the residual susceptible fraction $s_{\infty}=1-r_{\infty}$.  For example, at $\mathcal R_{0}=2.5$ we obtain $s_{\infty}\approx0.17$, implying roughly $17\%$ of the population never becomes infected.

\subsection{Simulation Setup}
Using the \texttt{fastgemf} package we instantiated two scenarios:
\begin{itemize}
  \item\textbf{Scenario~A (sub-critical):} $\mathcal R_{0}=0.8$, hence $\beta=0.011$.
  \item\textbf{Scenario~B (super-critical):} $\mathcal R_{0}=2.5$, hence $\beta=0.034$.
\end{itemize}
Initial conditions placed ten infectious hubs (highest-degree nodes) with the remaining nodes susceptible.  Twenty realisations were run for each scenario over $160$~days; median trajectories are reported.

\section{Results}
Figure~\ref{fig:timecourses} shows the temporal evolution of $S$, $I$, and $R$ for both scenarios.  Key quantitative metrics are summarised in Table~\ref{tab:metrics}.

\begin{figure}[!t]
  \centering
  \includegraphics[width=0.95\linewidth]{results-11.png}\\[3pt]
  \includegraphics[width=0.95\linewidth]{results-12.png}
  \caption{Median compartment counts over time for (top) $\mathcal R_{0}=0.8$ and (bottom) $\mathcal R_{0}=2.5$.  Shaded areas denote inter-quartile ranges across 20 stochastic simulations.}
  \label{fig:timecourses}
\end{figure}

\begin{table}[!t]
\centering
\caption{Simulation-derived epidemic metrics}
\begin{tabular}{lcccc}
\toprule
Scenario & Peak $I$ & Peak time (d) & Final $R$ & Duration (d)\\
\midrule
$\mathcal R_{0}=0.8$ & 10 & 0 & 10 & $<10$\\
$\mathcal R_{0}=2.5$ & 255 & 25 & 828 & $\approx160$\\
\bottomrule
\end{tabular}
\label{tab:metrics}
\end{table}

\subsection{Scenario A: $\mathcal R_{0}<1$}
The infection failed to expand beyond the seeding nodes; $I(t)$ declined immediately and reached zero within ten days.  The final size equalled the initial number of infectives, leaving $\approx99\%$ of the population still susceptible.  Thus, the chain broke solely because each infectious individual generated, on average, fewer than one secondary case.

\subsection{Scenario B: $\mathcal R_{0}>1$}
An initial exponential growth phase produced a peak prevalence of 255 on day 25, after which $I(t)$ fell despite many susceptibles remaining.  The outbreak ended with $R_{\infty}=828$ and $S_{\infty}=172$, matching the deterministic prediction $S_{\infty}/N\approx0.17$.  Hence, the chain broke because depletion pushed the effective reproduction number $\mathcal R_{\mathrm e}=\mathcal R_{0} S/N$ below one.

\section{Discussion}
The dual mechanisms postulated in the introduction are clearly distinguished by the value of $\mathcal R_{0}$.  When $\mathcal R_{0}<1$, transmission is sub-critical from the outset and epidemic fade-out is governed by the vanishing of infectives; susceptible depletion plays no role.  Conversely, for $\mathcal R_{0}>1$ the infection initially amplifies but ultimately self-extinguishes once sufficient susceptibles have been removed.  Importantly, susceptible exhaustion is never complete; a residue equal to the herd-immunity threshold remains as predicted by the final-size equation.

Network heterogeneity does not qualitatively alter these conclusions in the present context because the ER network preserves mean-field mixing assumptions.  However, heterogeneous networks with high variance in degree reduce the herd-immunity threshold \cite{Miller2020}.  Future work should extend the analysis to scale-free contact patterns and targeted interventions.

\section{Conclusion}
We have shown analytically and through network-based simulations that epidemics terminate by two distinct pathways depending on the basic reproduction number.  Sub-critical outbreaks end through inevitable decline in infectives without consuming susceptibles, whereas super-critical outbreaks cease when susceptible depletion lowers the effective reproduction number below unity, leaving a non-zero susceptible reservoir.  These findings reinforce the importance of reducing $\mathcal R_{0}$ via non-pharmaceutical or vaccination measures:  pushing $\mathcal R_{0}$ below one ensures immediate containment without requiring widespread infection.

\begin{thebibliography}{9}
\bibitem{Kermack1927} W.~O. Kermack and A.~G. McKendrick, ``A contribution to the mathematical theory of epidemics,'' \emph{Proceedings of the Royal Society A}, vol. 115, no. 772, pp. 700--721, 1927.
\bibitem{Anderson1991} R.~M. Anderson and R.~M. May, \emph{Infectious Diseases of Humans: Dynamics and Control}.  Oxford University Press, 1991.
\bibitem{Miller2020} J.~C. Miller, ``A note on the derivation of epidemic final sizes,'' \emph{Bull. Math. Biol.}, vol. 82, pp. 125--140, 2020.
\end{thebibliography}

\end{document}